\section{Pracovní úkol}
\begin{enumerate}
\item DÚ: Odvoďte vztah (15), spočtěte $\beta$ pro U = 100 V (dosazujte energii v jednotkách
keV!) a diskutujte, zda je korektní považovat elektrony v této úloze za
nerelativistické.
\item DÚ: Odvoďte vztahy (17) a (19) (stačí ponechat v domácí přípravě).
\item Změřte měrný náboj elektronu působením podélného magnetického pole. Měření proveďte
pro různé hodnoty urychlovacího napětí U v rozmezí 750 až 1250 V. Pomocné napětí na
A1 (Obr. 7) volte 140 V. Hodnotu e/me určete fitováním závislosti (17) s errorbary.
\item Změřte měrný náboj elektronu působením kolmého magnetického pole. Naměřte několik
dvojic urychlovacích napětí U (v rozsahu do 300 V) a magnetizačního proudu I (v rozsahu
do 4 A). Hodnotu e/me určete fitováním závislosti (19) s errorbary.

\end{enumerate}


\begin{enumerate}
\item DÚ: Odvoďte vztah (10) pro výpočet náboje kapky.
\item Proveďte Millikanův experiment pro alespoň deset kapiček oleje. Výsledky zpracujte formou
grafu Q na r a určete elementární náboj.
\item Z výsledků úlohy 3a (Měrný náboj elektronu) a této stanovte hmotnost elektronu, vyjádřete
v jednotkách $\jd{keV/c^2}$

\end{enumerate}

