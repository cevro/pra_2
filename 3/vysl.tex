\section{Výsledky merania}
\subsection{%
Meranie náboja pozdĺžnym magnetickým poľom
}
Namerané hodnoty napätie $U$ a prúdu $I$ sú v tabuľke Tab. \ref{T_1} a 
vynesené do grafu Obr. \ref{G_1}.
Z fitu z obrázku Obr. \ref{G_1} dostávame hodnotu 
$U(I^2) = "\(17.7\pm8.0\)I^2+4.9\pm95.4"$, z ktorej za použitia 
vzťahu \ref{R_1}, \ref{R_1-1} a \ref{SCH_2} dostávame hodnotu $e/m_e="6.8\pm3.3 C\cdot kg^{-1}"$, pričom 
\eq{
\mu = "4\pi10^{-7} Wb\cdot A^{-1}\cdot m^{-1}"\,,\\
N = "174" \cite{C_1} \,,\\
z = "0.381 m" \cite{C_1} \,,\\
z = "0.249 m" \cite{C_1} \,.\\
}

\begin{table}[h]
\begin{center}
\begin{tabular}{| c | c |}
\hline
\popi{U}{V} & \popi{I}{A}\\
\hline
$"135"$ & $"3"$\\
$"200"$ & $"3.1"$\\
$"275"$ & $"3.3"$\\
$"300"$ & $"3.6"$\\
$"225"$ & $"3.7"$\\
$"275"$ & $"4"$\\
$"150"$ & $"3.5"$\\
$"125"$ & $"2.9"$\\
$"100"$ & $-$\\
\hline
\end{tabular}
\caption{Namerané hodnoty napätia $U$ a prúdu $I$, pri ktorom nastala jedna otáčka zväzku.
} \label{T_1}
\end{center}
\end{table}


\begin{figure}
% GNUPLOT: LaTeX picture
\setlength{\unitlength}{0.240900pt}
\ifx\plotpoint\undefined\newsavebox{\plotpoint}\fi
\begin{picture}(1500,900)(0,0)
\sbox{\plotpoint}{\rule[-0.200pt]{0.400pt}{0.400pt}}%
\put(171.0,131.0){\rule[-0.200pt]{4.818pt}{0.400pt}}
\put(151,131){\makebox(0,0)[r]{ 0.5}}
\put(1419.0,131.0){\rule[-0.200pt]{4.818pt}{0.400pt}}
\put(171.0,235.0){\rule[-0.200pt]{4.818pt}{0.400pt}}
\put(151,235){\makebox(0,0)[r]{ 1}}
\put(1419.0,235.0){\rule[-0.200pt]{4.818pt}{0.400pt}}
\put(171.0,339.0){\rule[-0.200pt]{4.818pt}{0.400pt}}
\put(151,339){\makebox(0,0)[r]{ 1.5}}
\put(1419.0,339.0){\rule[-0.200pt]{4.818pt}{0.400pt}}
\put(171.0,443.0){\rule[-0.200pt]{4.818pt}{0.400pt}}
\put(151,443){\makebox(0,0)[r]{ 2}}
\put(1419.0,443.0){\rule[-0.200pt]{4.818pt}{0.400pt}}
\put(171.0,547.0){\rule[-0.200pt]{4.818pt}{0.400pt}}
\put(151,547){\makebox(0,0)[r]{ 2.5}}
\put(1419.0,547.0){\rule[-0.200pt]{4.818pt}{0.400pt}}
\put(171.0,651.0){\rule[-0.200pt]{4.818pt}{0.400pt}}
\put(151,651){\makebox(0,0)[r]{ 3}}
\put(1419.0,651.0){\rule[-0.200pt]{4.818pt}{0.400pt}}
\put(171.0,755.0){\rule[-0.200pt]{4.818pt}{0.400pt}}
\put(151,755){\makebox(0,0)[r]{ 3.5}}
\put(1419.0,755.0){\rule[-0.200pt]{4.818pt}{0.400pt}}
\put(171.0,859.0){\rule[-0.200pt]{4.818pt}{0.400pt}}
\put(151,859){\makebox(0,0)[r]{ 4}}
\put(1419.0,859.0){\rule[-0.200pt]{4.818pt}{0.400pt}}
\put(171.0,131.0){\rule[-0.200pt]{0.400pt}{4.818pt}}
\put(171,90){\makebox(0,0){ 0}}
\put(171.0,839.0){\rule[-0.200pt]{0.400pt}{4.818pt}}
\put(382.0,131.0){\rule[-0.200pt]{0.400pt}{4.818pt}}
\put(382,90){\makebox(0,0){ 2}}
\put(382.0,839.0){\rule[-0.200pt]{0.400pt}{4.818pt}}
\put(594.0,131.0){\rule[-0.200pt]{0.400pt}{4.818pt}}
\put(594,90){\makebox(0,0){ 4}}
\put(594.0,839.0){\rule[-0.200pt]{0.400pt}{4.818pt}}
\put(805.0,131.0){\rule[-0.200pt]{0.400pt}{4.818pt}}
\put(805,90){\makebox(0,0){ 6}}
\put(805.0,839.0){\rule[-0.200pt]{0.400pt}{4.818pt}}
\put(1016.0,131.0){\rule[-0.200pt]{0.400pt}{4.818pt}}
\put(1016,90){\makebox(0,0){ 8}}
\put(1016.0,839.0){\rule[-0.200pt]{0.400pt}{4.818pt}}
\put(1228.0,131.0){\rule[-0.200pt]{0.400pt}{4.818pt}}
\put(1228,90){\makebox(0,0){ 10}}
\put(1228.0,839.0){\rule[-0.200pt]{0.400pt}{4.818pt}}
\put(1439.0,131.0){\rule[-0.200pt]{0.400pt}{4.818pt}}
\put(1439,90){\makebox(0,0){ 12}}
\put(1439.0,839.0){\rule[-0.200pt]{0.400pt}{4.818pt}}
\put(171.0,131.0){\rule[-0.200pt]{0.400pt}{175.375pt}}
\put(171.0,131.0){\rule[-0.200pt]{305.461pt}{0.400pt}}
\put(1439.0,131.0){\rule[-0.200pt]{0.400pt}{175.375pt}}
\put(171.0,859.0){\rule[-0.200pt]{305.461pt}{0.400pt}}
\put(30,495){\makebox(0,0){\popi{I}{A}}}
\put(805,29){\makebox(0,0){\popi{U}{V}}}
\put(1279,172){\makebox(0,0)[r]{Namerané hodnoty}}
\put(1397,792){\makebox(0,0){$+$}}
\put(1192,772){\makebox(0,0){$+$}}
\put(1068,721){\makebox(0,0){$+$}}
\put(653,519){\makebox(0,0){$+$}}
\put(541,454){\makebox(0,0){$+$}}
\put(382,345){\makebox(0,0){$+$}}
\put(296,266){\makebox(0,0){$+$}}
\put(248,218){\makebox(0,0){$+$}}
\put(748,573){\makebox(0,0){$+$}}
\put(837,618){\makebox(0,0){$+$}}
\put(916,655){\makebox(0,0){$+$}}
\put(1047,713){\makebox(0,0){$+$}}
\put(1110,740){\makebox(0,0){$+$}}
\put(1349,172){\makebox(0,0){$+$}}
\put(171.0,131.0){\rule[-0.200pt]{0.400pt}{175.375pt}}
\put(171.0,131.0){\rule[-0.200pt]{305.461pt}{0.400pt}}
\put(1439.0,131.0){\rule[-0.200pt]{0.400pt}{175.375pt}}
\put(171.0,859.0){\rule[-0.200pt]{305.461pt}{0.400pt}}
\end{picture}

\caption{Závislosť napätia $U$ na prúde $I^2$ preložená funkciou $U(I^2) = "\(17.7\pm8.0\)I^2+4.9\pm95.4"$
}  \label{G_1}
\end{figure}

\subsection{
Meranie náboja kolmým magnetickým poľom
}
Namerané hodnoty priemeru $d$ od napätia $U$ a prúdu $I$ boli vynesené do tabuľky \ref{T_2}. 
Následne boli vynesené do grafu Obr. \ref{G_2} a z hodnoty fitu $U(I^2)="\(94.0\pm3.2\)I^2+17.1\pm6.7"$ a pomocou vzťahov \ref{R_2}, \ref{R_2-1}a \ref{SCH_2} bola vypočítaná hodnota $e/m_e$,
\eq{
\frac{e}{m_e}="\(1.12\pm0.41\)\cdot10^11 C\cdot kg^{-1}"\,,
}
pričom
\eq{
N = "130" \cite{C_1} \,,\\
R = "15 cm" \cite{C_1} \,.\\
}

\begin{table}[h]
\begin{center}
\begin{tabular}{| c | c | c |}
\hline
\popi{U}{V} & \popi{I}{A} & \popi{d}{cm}\\
\hline
$"100"$ & $"1"$ & $"10.5\pm0.1"$\\
$"125"$ & $"1.05"$ & $"10.5\pm0.1"$\\
$"150"$ & $"1.15"$ & $"10.5\pm0.1"$\\
$"175"$ & $"1.3"$ & $"10.5\pm0.1"$\\
$"200"$ & $"1.4"$ & $"10.5\pm0.1"$\\
$"225"$ & $"1.5"$ & $"10.5\pm0.1"$\\
$"250"$ & $"1.55"$ & $"10.5\pm0.1"$\\
$"275"$ & $"1.65"$ & $"10.5\pm0.1"$\\
$"300"$ & $"1.75"$ & $"10.5\pm0.1"$\\
\hline
\end{tabular}
\caption{Namerané hodnoty napätia $U$, prúdu $I$ a priemere dráhy elektrónu $d$.
} \label{T_2}
\end{center}
\end{table}

\begin{figure}
% GNUPLOT: LaTeX picture
\setlength{\unitlength}{0.240900pt}
\ifx\plotpoint\undefined\newsavebox{\plotpoint}\fi
\begin{picture}(1500,900)(0,0)
\sbox{\plotpoint}{\rule[-0.200pt]{0.400pt}{0.400pt}}%
\put(191.0,131.0){\rule[-0.200pt]{4.818pt}{0.400pt}}
\put(171,131){\makebox(0,0)[r]{ 0}}
\put(1419.0,131.0){\rule[-0.200pt]{4.818pt}{0.400pt}}
\put(191.0,212.0){\rule[-0.200pt]{4.818pt}{0.400pt}}
\put(171,212){\makebox(0,0)[r]{ 0.02}}
\put(1419.0,212.0){\rule[-0.200pt]{4.818pt}{0.400pt}}
\put(191.0,293.0){\rule[-0.200pt]{4.818pt}{0.400pt}}
\put(171,293){\makebox(0,0)[r]{ 0.04}}
\put(1419.0,293.0){\rule[-0.200pt]{4.818pt}{0.400pt}}
\put(191.0,374.0){\rule[-0.200pt]{4.818pt}{0.400pt}}
\put(171,374){\makebox(0,0)[r]{ 0.06}}
\put(1419.0,374.0){\rule[-0.200pt]{4.818pt}{0.400pt}}
\put(191.0,455.0){\rule[-0.200pt]{4.818pt}{0.400pt}}
\put(171,455){\makebox(0,0)[r]{ 0.08}}
\put(1419.0,455.0){\rule[-0.200pt]{4.818pt}{0.400pt}}
\put(191.0,535.0){\rule[-0.200pt]{4.818pt}{0.400pt}}
\put(171,535){\makebox(0,0)[r]{ 0.1}}
\put(1419.0,535.0){\rule[-0.200pt]{4.818pt}{0.400pt}}
\put(191.0,616.0){\rule[-0.200pt]{4.818pt}{0.400pt}}
\put(171,616){\makebox(0,0)[r]{ 0.12}}
\put(1419.0,616.0){\rule[-0.200pt]{4.818pt}{0.400pt}}
\put(191.0,697.0){\rule[-0.200pt]{4.818pt}{0.400pt}}
\put(171,697){\makebox(0,0)[r]{ 0.14}}
\put(1419.0,697.0){\rule[-0.200pt]{4.818pt}{0.400pt}}
\put(191.0,778.0){\rule[-0.200pt]{4.818pt}{0.400pt}}
\put(171,778){\makebox(0,0)[r]{ 0.16}}
\put(1419.0,778.0){\rule[-0.200pt]{4.818pt}{0.400pt}}
\put(191.0,859.0){\rule[-0.200pt]{4.818pt}{0.400pt}}
\put(171,859){\makebox(0,0)[r]{ 0.18}}
\put(1419.0,859.0){\rule[-0.200pt]{4.818pt}{0.400pt}}
\put(191.0,131.0){\rule[-0.200pt]{0.400pt}{4.818pt}}
\put(191,90){\makebox(0,0){ 0}}
\put(191.0,839.0){\rule[-0.200pt]{0.400pt}{4.818pt}}
\put(330.0,131.0){\rule[-0.200pt]{0.400pt}{4.818pt}}
\put(330,90){\makebox(0,0){ 10}}
\put(330.0,839.0){\rule[-0.200pt]{0.400pt}{4.818pt}}
\put(468.0,131.0){\rule[-0.200pt]{0.400pt}{4.818pt}}
\put(468,90){\makebox(0,0){ 20}}
\put(468.0,839.0){\rule[-0.200pt]{0.400pt}{4.818pt}}
\put(607.0,131.0){\rule[-0.200pt]{0.400pt}{4.818pt}}
\put(607,90){\makebox(0,0){ 30}}
\put(607.0,839.0){\rule[-0.200pt]{0.400pt}{4.818pt}}
\put(746.0,131.0){\rule[-0.200pt]{0.400pt}{4.818pt}}
\put(746,90){\makebox(0,0){ 40}}
\put(746.0,839.0){\rule[-0.200pt]{0.400pt}{4.818pt}}
\put(884.0,131.0){\rule[-0.200pt]{0.400pt}{4.818pt}}
\put(884,90){\makebox(0,0){ 50}}
\put(884.0,839.0){\rule[-0.200pt]{0.400pt}{4.818pt}}
\put(1023.0,131.0){\rule[-0.200pt]{0.400pt}{4.818pt}}
\put(1023,90){\makebox(0,0){ 60}}
\put(1023.0,839.0){\rule[-0.200pt]{0.400pt}{4.818pt}}
\put(1162.0,131.0){\rule[-0.200pt]{0.400pt}{4.818pt}}
\put(1162,90){\makebox(0,0){ 70}}
\put(1162.0,839.0){\rule[-0.200pt]{0.400pt}{4.818pt}}
\put(1300.0,131.0){\rule[-0.200pt]{0.400pt}{4.818pt}}
\put(1300,90){\makebox(0,0){ 80}}
\put(1300.0,839.0){\rule[-0.200pt]{0.400pt}{4.818pt}}
\put(1439.0,131.0){\rule[-0.200pt]{0.400pt}{4.818pt}}
\put(1439,90){\makebox(0,0){ 90}}
\put(1439.0,839.0){\rule[-0.200pt]{0.400pt}{4.818pt}}
\put(191.0,131.0){\rule[-0.200pt]{0.400pt}{175.375pt}}
\put(191.0,131.0){\rule[-0.200pt]{300.643pt}{0.400pt}}
\put(1439.0,131.0){\rule[-0.200pt]{0.400pt}{175.375pt}}
\put(191.0,859.0){\rule[-0.200pt]{300.643pt}{0.400pt}}
\put(30,495){\makebox(0,0){\popi{U}{V}}}
\put(815,29){\makebox(0,0){\popi{\phi}{\dg}}}
\put(1131,213){\makebox(0,0)[r]{namerané hodnoty}}
\put(1151.0,213.0){\rule[-0.200pt]{24.090pt}{0.400pt}}
\put(1151.0,203.0){\rule[-0.200pt]{0.400pt}{4.818pt}}
\put(1251.0,203.0){\rule[-0.200pt]{0.400pt}{4.818pt}}
\put(1439.0,147.0){\rule[-0.200pt]{0.400pt}{19.513pt}}
\put(1429.0,147.0){\rule[-0.200pt]{2.409pt}{0.400pt}}
\put(1429.0,228.0){\rule[-0.200pt]{2.409pt}{0.400pt}}
\put(1370.0,224.0){\rule[-0.200pt]{0.400pt}{19.513pt}}
\put(1360.0,224.0){\rule[-0.200pt]{4.818pt}{0.400pt}}
\put(1360.0,305.0){\rule[-0.200pt]{4.818pt}{0.400pt}}
\put(1300.0,317.0){\rule[-0.200pt]{0.400pt}{19.513pt}}
\put(1290.0,317.0){\rule[-0.200pt]{4.818pt}{0.400pt}}
\put(1290.0,398.0){\rule[-0.200pt]{4.818pt}{0.400pt}}
\put(1231.0,398.0){\rule[-0.200pt]{0.400pt}{19.513pt}}
\put(1221.0,398.0){\rule[-0.200pt]{4.818pt}{0.400pt}}
\put(1221.0,479.0){\rule[-0.200pt]{4.818pt}{0.400pt}}
\put(1162.0,467.0){\rule[-0.200pt]{0.400pt}{19.513pt}}
\put(1152.0,467.0){\rule[-0.200pt]{4.818pt}{0.400pt}}
\put(1152.0,548.0){\rule[-0.200pt]{4.818pt}{0.400pt}}
\put(1092.0,519.0){\rule[-0.200pt]{0.400pt}{19.513pt}}
\put(1082.0,519.0){\rule[-0.200pt]{4.818pt}{0.400pt}}
\put(1082.0,600.0){\rule[-0.200pt]{4.818pt}{0.400pt}}
\put(1023.0,572.0){\rule[-0.200pt]{0.400pt}{19.513pt}}
\put(1013.0,572.0){\rule[-0.200pt]{4.818pt}{0.400pt}}
\put(1013.0,653.0){\rule[-0.200pt]{4.818pt}{0.400pt}}
\put(954.0,608.0){\rule[-0.200pt]{0.400pt}{19.513pt}}
\put(944.0,608.0){\rule[-0.200pt]{4.818pt}{0.400pt}}
\put(944.0,689.0){\rule[-0.200pt]{4.818pt}{0.400pt}}
\put(884.0,641.0){\rule[-0.200pt]{0.400pt}{19.272pt}}
\put(874.0,641.0){\rule[-0.200pt]{4.818pt}{0.400pt}}
\put(874.0,721.0){\rule[-0.200pt]{4.818pt}{0.400pt}}
\put(815.0,665.0){\rule[-0.200pt]{0.400pt}{19.513pt}}
\put(805.0,665.0){\rule[-0.200pt]{4.818pt}{0.400pt}}
\put(805.0,746.0){\rule[-0.200pt]{4.818pt}{0.400pt}}
\put(746.0,689.0){\rule[-0.200pt]{0.400pt}{19.513pt}}
\put(736.0,689.0){\rule[-0.200pt]{4.818pt}{0.400pt}}
\put(736.0,770.0){\rule[-0.200pt]{4.818pt}{0.400pt}}
\put(676.0,709.0){\rule[-0.200pt]{0.400pt}{19.513pt}}
\put(666.0,709.0){\rule[-0.200pt]{4.818pt}{0.400pt}}
\put(666.0,790.0){\rule[-0.200pt]{4.818pt}{0.400pt}}
\put(607.0,726.0){\rule[-0.200pt]{0.400pt}{19.272pt}}
\put(597.0,726.0){\rule[-0.200pt]{4.818pt}{0.400pt}}
\put(597.0,806.0){\rule[-0.200pt]{4.818pt}{0.400pt}}
\put(538.0,734.0){\rule[-0.200pt]{0.400pt}{19.513pt}}
\put(528.0,734.0){\rule[-0.200pt]{4.818pt}{0.400pt}}
\put(528.0,815.0){\rule[-0.200pt]{4.818pt}{0.400pt}}
\put(468.0,746.0){\rule[-0.200pt]{0.400pt}{19.513pt}}
\put(458.0,746.0){\rule[-0.200pt]{4.818pt}{0.400pt}}
\put(458.0,827.0){\rule[-0.200pt]{4.818pt}{0.400pt}}
\put(399.0,750.0){\rule[-0.200pt]{0.400pt}{19.513pt}}
\put(389.0,750.0){\rule[-0.200pt]{4.818pt}{0.400pt}}
\put(389.0,831.0){\rule[-0.200pt]{4.818pt}{0.400pt}}
\put(330.0,754.0){\rule[-0.200pt]{0.400pt}{19.513pt}}
\put(320.0,754.0){\rule[-0.200pt]{4.818pt}{0.400pt}}
\put(320.0,835.0){\rule[-0.200pt]{4.818pt}{0.400pt}}
\put(260.0,758.0){\rule[-0.200pt]{0.400pt}{19.513pt}}
\put(250.0,758.0){\rule[-0.200pt]{4.818pt}{0.400pt}}
\put(250.0,839.0){\rule[-0.200pt]{4.818pt}{0.400pt}}
\put(191.0,758.0){\rule[-0.200pt]{0.400pt}{19.513pt}}
\put(191.0,758.0){\rule[-0.200pt]{2.409pt}{0.400pt}}
\put(191.0,839.0){\rule[-0.200pt]{2.409pt}{0.400pt}}
\put(1425.0,188.0){\rule[-0.200pt]{3.373pt}{0.400pt}}
\put(1425.0,178.0){\rule[-0.200pt]{0.400pt}{4.818pt}}
\put(1439.0,178.0){\rule[-0.200pt]{0.400pt}{4.818pt}}
\put(1356.0,264.0){\rule[-0.200pt]{6.745pt}{0.400pt}}
\put(1356.0,254.0){\rule[-0.200pt]{0.400pt}{4.818pt}}
\put(1384.0,254.0){\rule[-0.200pt]{0.400pt}{4.818pt}}
\put(1286.0,357.0){\rule[-0.200pt]{6.745pt}{0.400pt}}
\put(1286.0,347.0){\rule[-0.200pt]{0.400pt}{4.818pt}}
\put(1314.0,347.0){\rule[-0.200pt]{0.400pt}{4.818pt}}
\put(1217.0,438.0){\rule[-0.200pt]{6.745pt}{0.400pt}}
\put(1217.0,428.0){\rule[-0.200pt]{0.400pt}{4.818pt}}
\put(1245.0,428.0){\rule[-0.200pt]{0.400pt}{4.818pt}}
\put(1148.0,507.0){\rule[-0.200pt]{6.745pt}{0.400pt}}
\put(1148.0,497.0){\rule[-0.200pt]{0.400pt}{4.818pt}}
\put(1176.0,497.0){\rule[-0.200pt]{0.400pt}{4.818pt}}
\put(1078.0,560.0){\rule[-0.200pt]{6.745pt}{0.400pt}}
\put(1078.0,550.0){\rule[-0.200pt]{0.400pt}{4.818pt}}
\put(1106.0,550.0){\rule[-0.200pt]{0.400pt}{4.818pt}}
\put(1009.0,612.0){\rule[-0.200pt]{6.745pt}{0.400pt}}
\put(1009.0,602.0){\rule[-0.200pt]{0.400pt}{4.818pt}}
\put(1037.0,602.0){\rule[-0.200pt]{0.400pt}{4.818pt}}
\put(940.0,649.0){\rule[-0.200pt]{6.745pt}{0.400pt}}
\put(940.0,639.0){\rule[-0.200pt]{0.400pt}{4.818pt}}
\put(968.0,639.0){\rule[-0.200pt]{0.400pt}{4.818pt}}
\put(870.0,681.0){\rule[-0.200pt]{6.745pt}{0.400pt}}
\put(870.0,671.0){\rule[-0.200pt]{0.400pt}{4.818pt}}
\put(898.0,671.0){\rule[-0.200pt]{0.400pt}{4.818pt}}
\put(801.0,705.0){\rule[-0.200pt]{6.745pt}{0.400pt}}
\put(801.0,695.0){\rule[-0.200pt]{0.400pt}{4.818pt}}
\put(829.0,695.0){\rule[-0.200pt]{0.400pt}{4.818pt}}
\put(732.0,730.0){\rule[-0.200pt]{6.745pt}{0.400pt}}
\put(732.0,720.0){\rule[-0.200pt]{0.400pt}{4.818pt}}
\put(760.0,720.0){\rule[-0.200pt]{0.400pt}{4.818pt}}
\put(662.0,750.0){\rule[-0.200pt]{6.745pt}{0.400pt}}
\put(662.0,740.0){\rule[-0.200pt]{0.400pt}{4.818pt}}
\put(690.0,740.0){\rule[-0.200pt]{0.400pt}{4.818pt}}
\put(593.0,766.0){\rule[-0.200pt]{6.745pt}{0.400pt}}
\put(593.0,756.0){\rule[-0.200pt]{0.400pt}{4.818pt}}
\put(621.0,756.0){\rule[-0.200pt]{0.400pt}{4.818pt}}
\put(524.0,774.0){\rule[-0.200pt]{6.745pt}{0.400pt}}
\put(524.0,764.0){\rule[-0.200pt]{0.400pt}{4.818pt}}
\put(552.0,764.0){\rule[-0.200pt]{0.400pt}{4.818pt}}
\put(454.0,786.0){\rule[-0.200pt]{6.745pt}{0.400pt}}
\put(454.0,776.0){\rule[-0.200pt]{0.400pt}{4.818pt}}
\put(482.0,776.0){\rule[-0.200pt]{0.400pt}{4.818pt}}
\put(385.0,790.0){\rule[-0.200pt]{6.745pt}{0.400pt}}
\put(385.0,780.0){\rule[-0.200pt]{0.400pt}{4.818pt}}
\put(413.0,780.0){\rule[-0.200pt]{0.400pt}{4.818pt}}
\put(316.0,794.0){\rule[-0.200pt]{6.745pt}{0.400pt}}
\put(316.0,784.0){\rule[-0.200pt]{0.400pt}{4.818pt}}
\put(344.0,784.0){\rule[-0.200pt]{0.400pt}{4.818pt}}
\put(246.0,798.0){\rule[-0.200pt]{6.745pt}{0.400pt}}
\put(246.0,788.0){\rule[-0.200pt]{0.400pt}{4.818pt}}
\put(274.0,788.0){\rule[-0.200pt]{0.400pt}{4.818pt}}
\put(191.0,798.0){\rule[-0.200pt]{3.373pt}{0.400pt}}
\put(191.0,788.0){\rule[-0.200pt]{0.400pt}{4.818pt}}
\put(1439,188){\makebox(0,0){$+$}}
\put(1370,264){\makebox(0,0){$+$}}
\put(1300,357){\makebox(0,0){$+$}}
\put(1231,438){\makebox(0,0){$+$}}
\put(1162,507){\makebox(0,0){$+$}}
\put(1092,560){\makebox(0,0){$+$}}
\put(1023,612){\makebox(0,0){$+$}}
\put(954,649){\makebox(0,0){$+$}}
\put(884,681){\makebox(0,0){$+$}}
\put(815,705){\makebox(0,0){$+$}}
\put(746,730){\makebox(0,0){$+$}}
\put(676,750){\makebox(0,0){$+$}}
\put(607,766){\makebox(0,0){$+$}}
\put(538,774){\makebox(0,0){$+$}}
\put(468,786){\makebox(0,0){$+$}}
\put(399,790){\makebox(0,0){$+$}}
\put(330,794){\makebox(0,0){$+$}}
\put(260,798){\makebox(0,0){$+$}}
\put(191,798){\makebox(0,0){$+$}}
\put(1201,213){\makebox(0,0){$+$}}
\put(205.0,788.0){\rule[-0.200pt]{0.400pt}{4.818pt}}
\put(1131,172){\makebox(0,0)[r]{$f\(\phi\)=\(0.14\pm0.07\)\mathrm{cos}^2\(\phi\)+\(0.034\pm0.005\)$}}
\multiput(1151,172)(20.756,0.000){5}{\usebox{\plotpoint}}
\put(1251,172){\usebox{\plotpoint}}
\put(191,839){\usebox{\plotpoint}}
\put(191.00,839.00){\usebox{\plotpoint}}
\put(211.76,839.00){\usebox{\plotpoint}}
\put(232.47,838.00){\usebox{\plotpoint}}
\put(253.19,837.06){\usebox{\plotpoint}}
\put(273.92,836.42){\usebox{\plotpoint}}
\put(294.59,834.57){\usebox{\plotpoint}}
\put(315.18,832.14){\usebox{\plotpoint}}
\put(335.75,829.52){\usebox{\plotpoint}}
\put(356.32,826.78){\usebox{\plotpoint}}
\put(376.81,823.49){\usebox{\plotpoint}}
\put(397.32,820.28){\usebox{\plotpoint}}
\put(417.63,816.08){\usebox{\plotpoint}}
\put(438.01,812.25){\usebox{\plotpoint}}
\put(458.20,807.45){\usebox{\plotpoint}}
\put(478.38,802.60){\usebox{\plotpoint}}
\put(498.34,796.92){\usebox{\plotpoint}}
\put(518.53,792.11){\usebox{\plotpoint}}
\put(538.28,785.76){\usebox{\plotpoint}}
\put(558.11,779.63){\usebox{\plotpoint}}
\put(577.87,773.27){\usebox{\plotpoint}}
\put(597.53,766.64){\usebox{\plotpoint}}
\put(617.14,759.88){\usebox{\plotpoint}}
\put(636.45,752.29){\usebox{\plotpoint}}
\put(655.70,744.54){\usebox{\plotpoint}}
\put(675.06,737.05){\usebox{\plotpoint}}
\put(694.30,729.29){\usebox{\plotpoint}}
\put(713.25,720.81){\usebox{\plotpoint}}
\put(732.20,712.37){\usebox{\plotpoint}}
\put(751.32,704.34){\usebox{\plotpoint}}
\put(770.07,695.43){\usebox{\plotpoint}}
\put(788.73,686.36){\usebox{\plotpoint}}
\put(807.22,676.96){\usebox{\plotpoint}}
\put(825.69,667.48){\usebox{\plotpoint}}
\put(844.27,658.26){\usebox{\plotpoint}}
\put(862.50,648.38){\usebox{\plotpoint}}
\put(880.89,638.81){\usebox{\plotpoint}}
\put(899.11,628.87){\usebox{\plotpoint}}
\put(917.24,618.78){\usebox{\plotpoint}}
\put(934.99,608.01){\usebox{\plotpoint}}
\put(953.03,597.75){\usebox{\plotpoint}}
\put(970.93,587.27){\usebox{\plotpoint}}
\put(988.78,576.68){\usebox{\plotpoint}}
\put(1006.57,566.00){\usebox{\plotpoint}}
\put(1024.30,555.20){\usebox{\plotpoint}}
\put(1041.84,544.11){\usebox{\plotpoint}}
\put(1059.37,533.00){\usebox{\plotpoint}}
\put(1076.76,521.68){\usebox{\plotpoint}}
\put(1094.44,510.81){\usebox{\plotpoint}}
\put(1111.80,499.44){\usebox{\plotpoint}}
\put(1128.93,487.71){\usebox{\plotpoint}}
\put(1146.44,476.58){\usebox{\plotpoint}}
\put(1163.61,464.93){\usebox{\plotpoint}}
\put(1180.80,453.29){\usebox{\plotpoint}}
\put(1197.99,441.67){\usebox{\plotpoint}}
\put(1215.07,429.87){\usebox{\plotpoint}}
\put(1232.22,418.19){\usebox{\plotpoint}}
\put(1249.34,406.45){\usebox{\plotpoint}}
\put(1266.45,394.70){\usebox{\plotpoint}}
\put(1283.62,383.04){\usebox{\plotpoint}}
\put(1300.35,370.76){\usebox{\plotpoint}}
\put(1317.41,358.95){\usebox{\plotpoint}}
\put(1334.25,346.81){\usebox{\plotpoint}}
\put(1351.20,334.85){\usebox{\plotpoint}}
\put(1367.94,322.58){\usebox{\plotpoint}}
\put(1385.01,310.76){\usebox{\plotpoint}}
\put(1401.74,298.49){\usebox{\plotpoint}}
\put(1418.67,286.49){\usebox{\plotpoint}}
\put(1435.54,274.40){\usebox{\plotpoint}}
\put(1439,272){\usebox{\plotpoint}}
\put(191.0,131.0){\rule[-0.200pt]{0.400pt}{175.375pt}}
\put(191.0,131.0){\rule[-0.200pt]{300.643pt}{0.400pt}}
\put(1439.0,131.0){\rule[-0.200pt]{0.400pt}{175.375pt}}
\put(191.0,859.0){\rule[-0.200pt]{300.643pt}{0.400pt}}
\end{picture}

\caption{Závislosť napätia $U$ na prúde $I^2$ preložená funkciou $U(I^2)="\(94.0\pm3.2\)I^2+17.1\pm6.7"$.
}  \label{G_2}
\end{figure}

\subsection{%
Millikanov experiment%
}

Namerané hodnoty časov stúpania $t_s$ a klesaní $t_k$ pre jednotlivé kvapky oleja sú v tabuľke \ref{T_3}. Z nich za pomoci vzťahov \ref{R_3}, \ref{R_4} a \ref{SCH_2} boli dopočítané hodnoty veľkosti náboja $Q$.
Pričom boli hodnoty 
\eq[m]{
U &="500\pm1 V"\,,\\
\eta &= "0.000018 Pa\cdot s^{-1}"\cite{C_3}\,,\\
s &= "0.0001 m"\cite{C_1}\,,\\
\rho\_{olej} &= "874 kg\cdot m^{-3}" \cite{C_1}\,,\\
\rho\_{vzd} &= "1.204 kg\cdot m^{-3}" \cite{C_3}\,,\\
g &="9.81 m\cdot s^{-2}"\,,\\
d &="0.006 m"\cite{C_1}\,,\\
E &= \frac{d}{U}="0.000012 V\cdot m^{-1}"\,,\\
p &="750 torr"\,.
}

Pomer náboja k tabuľkovému $e="1.602\cdot10^{-19} C"$ \cite{C_2} bol vynesené ho histogrami Obr. \ref{G_3}.

\begin{table}[h]
\begin{center}
\begin{tabular}{| c | c | c | c | c | c | c | c |}
\hline
\popi{t\_{s1}}{s} & \popi{t\_{k1}}{s} & \popi{t\_{s1}}{s} &\popi{t\_{k1}}{s} & \popi{\mean{t\_{s1}}}{s} & \popi{\mean{t\_{k1}}}{s} & \popi{Q}{C} &\popi{Q/e}{-}\\
\hline
$"18.0\pm0.3"$ & $"11.0\pm0.3"$ & $"19.0\pm0.3"$ & $"11.5\pm0.3"$ & $"18.5\pm0.63"$ & $"11.3\pm0.6"$ & $"\(4.69\pm0.51\)\cdot 10^{-19}"$ & $"2.93"$\\
$"10.0\pm0.3"$ & $"15.0\pm0.3"$ & $"10.0\pm0.3"$ & $"11.0\pm0.3"$ & $"10.0\pm0.5" $ & $"13.0\pm0.6"$ & $"\(5.35\pm0.49\)\cdot 10^{-19}"$ & $"3.34"$\\
$ "8.0\pm0.3"$ & $"15.0\pm0.3"$ & $ "8.0\pm0.3"$ & $"24.0\pm0.3"$ & $ "8.0\pm0.5" $ & $"19.5\pm4.6"$ & $"\(4.23\pm0.38\)\cdot 10^{-19}"$ & $"2.64"$\\
$ "5.0\pm0.3"$ & $ "9.0\pm0.3"$ & $ "6.0\pm0.3"$ & $ "9.0\pm0.3"$ & $ "5.5\pm0.63"$ & $ "9.0\pm0.6"$ & $"\(1.09\pm0.13\)\cdot 10^{-18}"$ & $"6.80"$\\
$ "7.0\pm0.3"$ & $ "8.0\pm0.3"$ & $ "9.0\pm0.3"$ & $ "8.5\pm0.3"$ & $ "8.0\pm0.63"$ & $ "8.2\pm0.6"$ & $"\(9.60\pm1.01\)\cdot 10^{-19}"$ & $"6.00"$\\
$"10.5\pm0.3"$ & $"15.0\pm0.3"$ & $ "9.0\pm0.3"$ & $ "5.5\pm0.3"$ & $ "9.8\pm0.57"$ & $"10.3\pm0.6"$ & $"\(6.92\pm0.72\)\cdot 10^{-19}"$ & $"4.32"$\\
$"16.0\pm0.3"$ & $"12.0\pm0.3"$ & $"14.0\pm0.3"$ & $"10.5\pm0.3"$ & $"15.0\pm0.63"$ & $"11.3\pm0.6"$ & $"\(5.10\pm0.52\)\cdot 10^{-19}"$ & $"3.19"$\\
$ "2.0\pm0.3"$ & $"17.8\pm0.3"$ & -              & -              & $ "2.0\pm0.5" $ & $"17.8\pm0.5"$ & $"\(1.41\pm0.13\)\cdot 10^{-18}"$ & $"8.78"$\\
$ "7.0\pm0.3"$ & $ "9.0\pm0.3"$ & $ "7.0\pm0.3"$ & $ "8.5\pm0.3"$ & $ "7.0\pm0.63"$ & $ "8.8\pm0.6"$ & $"\(9.71\pm0.98\)\cdot 10^{-19}"$ & $"6.06"$\\
$"13.0\pm0.3"$ & $"15.0\pm0.3"$ & $"12.0\pm0.3"$ & $"12.0\pm0.3"$ & $"12.5\pm0.63"$ & $"13.5\pm0.6"$ & $"\(4.56\pm0.48\)\cdot 10^{-19}"$ & $"2.85"$\\
\hline
\end{tabular}
\caption{Namerané hodnoty času vzostupu $t_{s}$ a zostupu $t_{k}$, ich priemery $\mean{t_{k}}$ a $\mean{t_{s}}$, vypočítané náboje na kvapke $Q$, pomer $Q/e$, kde $e$ je elementárny náboj.
} \label{T_3}
\end{center}
\end{table}

\begin{figure}
% GNUPLOT: LaTeX picture
\setlength{\unitlength}{0.240900pt}
\ifx\plotpoint\undefined\newsavebox{\plotpoint}\fi
\begin{picture}(1500,900)(0,0)
\sbox{\plotpoint}{\rule[-0.200pt]{0.400pt}{0.400pt}}%
\put(151.0,131.0){\rule[-0.200pt]{4.818pt}{0.400pt}}
\put(131,131){\makebox(0,0)[r]{ 0}}
\put(1419.0,131.0){\rule[-0.200pt]{4.818pt}{0.400pt}}
\put(151.0,252.0){\rule[-0.200pt]{4.818pt}{0.400pt}}
\put(131,252){\makebox(0,0)[r]{ 10}}
\put(1419.0,252.0){\rule[-0.200pt]{4.818pt}{0.400pt}}
\put(151.0,374.0){\rule[-0.200pt]{4.818pt}{0.400pt}}
\put(131,374){\makebox(0,0)[r]{ 20}}
\put(1419.0,374.0){\rule[-0.200pt]{4.818pt}{0.400pt}}
\put(151.0,495.0){\rule[-0.200pt]{4.818pt}{0.400pt}}
\put(131,495){\makebox(0,0)[r]{ 30}}
\put(1419.0,495.0){\rule[-0.200pt]{4.818pt}{0.400pt}}
\put(151.0,616.0){\rule[-0.200pt]{4.818pt}{0.400pt}}
\put(131,616){\makebox(0,0)[r]{ 40}}
\put(1419.0,616.0){\rule[-0.200pt]{4.818pt}{0.400pt}}
\put(151.0,738.0){\rule[-0.200pt]{4.818pt}{0.400pt}}
\put(131,738){\makebox(0,0)[r]{ 50}}
\put(1419.0,738.0){\rule[-0.200pt]{4.818pt}{0.400pt}}
\put(151.0,859.0){\rule[-0.200pt]{4.818pt}{0.400pt}}
\put(131,859){\makebox(0,0)[r]{ 60}}
\put(1419.0,859.0){\rule[-0.200pt]{4.818pt}{0.400pt}}
\put(151.0,131.0){\rule[-0.200pt]{0.400pt}{4.818pt}}
\put(151,90){\makebox(0,0){ 480}}
\put(151.0,839.0){\rule[-0.200pt]{0.400pt}{4.818pt}}
\put(312.0,131.0){\rule[-0.200pt]{0.400pt}{4.818pt}}
\put(312,90){\makebox(0,0){ 500}}
\put(312.0,839.0){\rule[-0.200pt]{0.400pt}{4.818pt}}
\put(473.0,131.0){\rule[-0.200pt]{0.400pt}{4.818pt}}
\put(473,90){\makebox(0,0){ 520}}
\put(473.0,839.0){\rule[-0.200pt]{0.400pt}{4.818pt}}
\put(634.0,131.0){\rule[-0.200pt]{0.400pt}{4.818pt}}
\put(634,90){\makebox(0,0){ 540}}
\put(634.0,839.0){\rule[-0.200pt]{0.400pt}{4.818pt}}
\put(795.0,131.0){\rule[-0.200pt]{0.400pt}{4.818pt}}
\put(795,90){\makebox(0,0){ 560}}
\put(795.0,839.0){\rule[-0.200pt]{0.400pt}{4.818pt}}
\put(956.0,131.0){\rule[-0.200pt]{0.400pt}{4.818pt}}
\put(956,90){\makebox(0,0){ 580}}
\put(956.0,839.0){\rule[-0.200pt]{0.400pt}{4.818pt}}
\put(1117.0,131.0){\rule[-0.200pt]{0.400pt}{4.818pt}}
\put(1117,90){\makebox(0,0){ 600}}
\put(1117.0,839.0){\rule[-0.200pt]{0.400pt}{4.818pt}}
\put(1278.0,131.0){\rule[-0.200pt]{0.400pt}{4.818pt}}
\put(1278,90){\makebox(0,0){ 620}}
\put(1278.0,839.0){\rule[-0.200pt]{0.400pt}{4.818pt}}
\put(1439.0,131.0){\rule[-0.200pt]{0.400pt}{4.818pt}}
\put(1439,90){\makebox(0,0){ 640}}
\put(1439.0,839.0){\rule[-0.200pt]{0.400pt}{4.818pt}}
\put(151.0,131.0){\rule[-0.200pt]{0.400pt}{175.375pt}}
\put(151.0,131.0){\rule[-0.200pt]{310.279pt}{0.400pt}}
\put(1439.0,131.0){\rule[-0.200pt]{0.400pt}{175.375pt}}
\put(151.0,859.0){\rule[-0.200pt]{310.279pt}{0.400pt}}
\put(30,495){\makebox(0,0){\popi{\alpha}{\dg}}}
\put(795,29){\makebox(0,0){\popi{\lambda}{nm}}}
\put(1279,819){\makebox(0,0)[r]{namerané hodnoty}}
\put(1299.0,819.0){\rule[-0.200pt]{24.090pt}{0.400pt}}
\put(1299.0,809.0){\rule[-0.200pt]{0.400pt}{4.818pt}}
\put(1399.0,809.0){\rule[-0.200pt]{0.400pt}{4.818pt}}
\put(240.0,131.0){\rule[-0.200pt]{0.400pt}{36.617pt}}
\put(230.0,131.0){\rule[-0.200pt]{4.818pt}{0.400pt}}
\put(230.0,283.0){\rule[-0.200pt]{4.818pt}{0.400pt}}
\put(393.0,131.0){\rule[-0.200pt]{0.400pt}{162.848pt}}
\put(383.0,131.0){\rule[-0.200pt]{4.818pt}{0.400pt}}
\put(383.0,807.0){\rule[-0.200pt]{4.818pt}{0.400pt}}
\put(1037.0,131.0){\rule[-0.200pt]{0.400pt}{49.866pt}}
\put(1027.0,131.0){\rule[-0.200pt]{4.818pt}{0.400pt}}
\put(1027.0,338.0){\rule[-0.200pt]{4.818pt}{0.400pt}}
\put(1359.0,131.0){\rule[-0.200pt]{0.400pt}{99.733pt}}
\put(1349.0,131.0){\rule[-0.200pt]{4.818pt}{0.400pt}}
\put(240,207){\makebox(0,0){$+$}}
\put(393,469){\makebox(0,0){$+$}}
\put(1037,234){\makebox(0,0){$+$}}
\put(1359,338){\makebox(0,0){$+$}}
\put(1349,819){\makebox(0,0){$+$}}
\put(1349.0,545.0){\rule[-0.200pt]{4.818pt}{0.400pt}}
\put(1279,778){\makebox(0,0)[r]{$f(x)=1.7\cdot\(5.64\pm1.10\)\cdot10^{-12}\lambda^{-2}$}}
\multiput(1299,778)(20.756,0.000){5}{\usebox{\plotpoint}}
\put(1399,778){\usebox{\plotpoint}}
\put(240,614){\usebox{\plotpoint}}
\put(240.00,614.00){\usebox{\plotpoint}}
\put(260.02,608.54){\usebox{\plotpoint}}
\put(280.30,604.17){\usebox{\plotpoint}}
\put(300.44,599.19){\usebox{\plotpoint}}
\put(320.69,594.69){\usebox{\plotpoint}}
\put(340.89,590.03){\usebox{\plotpoint}}
\put(361.14,585.52){\usebox{\plotpoint}}
\put(381.35,580.85){\usebox{\plotpoint}}
\put(401.60,576.35){\usebox{\plotpoint}}
\put(421.99,572.50){\usebox{\plotpoint}}
\put(442.27,568.13){\usebox{\plotpoint}}
\put(462.49,563.58){\usebox{\plotpoint}}
\put(482.81,559.42){\usebox{\plotpoint}}
\put(503.16,555.43){\usebox{\plotpoint}}
\put(523.55,551.58){\usebox{\plotpoint}}
\put(543.81,547.20){\usebox{\plotpoint}}
\put(564.23,543.50){\usebox{\plotpoint}}
\put(584.57,539.48){\usebox{\plotpoint}}
\put(604.90,535.35){\usebox{\plotpoint}}
\put(625.34,531.76){\usebox{\plotpoint}}
\put(645.76,528.04){\usebox{\plotpoint}}
\put(666.21,524.51){\usebox{\plotpoint}}
\put(686.65,520.89){\usebox{\plotpoint}}
\put(707.08,517.26){\usebox{\plotpoint}}
\put(727.53,513.72){\usebox{\plotpoint}}
\put(747.95,510.01){\usebox{\plotpoint}}
\put(768.40,506.43){\usebox{\plotpoint}}
\put(788.83,502.76){\usebox{\plotpoint}}
\put(809.40,500.20){\usebox{\plotpoint}}
\put(829.83,496.53){\usebox{\plotpoint}}
\put(850.27,492.98){\usebox{\plotpoint}}
\put(870.83,490.21){\usebox{\plotpoint}}
\put(891.28,486.68){\usebox{\plotpoint}}
\put(911.85,484.03){\usebox{\plotpoint}}
\put(932.28,480.40){\usebox{\plotpoint}}
\put(952.85,477.84){\usebox{\plotpoint}}
\put(973.28,474.13){\usebox{\plotpoint}}
\put(993.85,471.57){\usebox{\plotpoint}}
\put(1014.35,468.42){\usebox{\plotpoint}}
\put(1034.90,465.65){\usebox{\plotpoint}}
\put(1055.41,462.60){\usebox{\plotpoint}}
\put(1075.99,460.00){\usebox{\plotpoint}}
\put(1096.53,457.21){\usebox{\plotpoint}}
\put(1117.07,454.36){\usebox{\plotpoint}}
\put(1137.61,451.53){\usebox{\plotpoint}}
\put(1158.15,448.71){\usebox{\plotpoint}}
\put(1178.71,445.94){\usebox{\plotpoint}}
\put(1199.25,443.14){\usebox{\plotpoint}}
\put(1219.92,441.28){\usebox{\plotpoint}}
\put(1240.46,438.41){\usebox{\plotpoint}}
\put(1261.01,435.64){\usebox{\plotpoint}}
\put(1281.65,433.56){\usebox{\plotpoint}}
\put(1302.23,430.98){\usebox{\plotpoint}}
\put(1322.81,428.36){\usebox{\plotpoint}}
\put(1343.46,426.32){\usebox{\plotpoint}}
\put(1359,424){\usebox{\plotpoint}}
\put(151.0,131.0){\rule[-0.200pt]{0.400pt}{175.375pt}}
\put(151.0,131.0){\rule[-0.200pt]{310.279pt}{0.400pt}}
\put(1439.0,131.0){\rule[-0.200pt]{0.400pt}{175.375pt}}
\put(151.0,859.0){\rule[-0.200pt]{310.279pt}{0.400pt}}
\end{picture}

\caption{Vynesená závislosť výskytu jednotlivých nábojov na ich pomere voči elementárnemu náboju.
}  \label{G_3}
\end{figure}

Hmotnosť elektrónu bola stanovená na 
\eq{
m_e = \frac{m_e e}{e} = "4.07\cdot10^{-31} kg" = "2.24 \frac{MeV}{c^2}"\,.
}

