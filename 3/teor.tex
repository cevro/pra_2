\section{Teória}
\subsection{Meranie náboja pozdĺžnym magnetickým poľom}
Závislosť urýchľovacieho napätia $U$ na magnetizačnom prúd $I$ môžeme vyjadriť ako
\eq[m]{
U\(I^2\) &= \frac{z^2 e \mu_0^2 N^2}{8 m_e\pi^2 l^2} I^2 \,,\lbl{R_1}\\
U\(I^2\) &= A I^2\,,\\
\frac{e}{m_e} &= 8 A \(\frac{\pi^2 l^2}{z^2 \mu_0^2 N^2}\)\,, \lbl{R_1-1}
} 
kde $"\mu_0 = 4\pi\cdot10^{-7} Wb\cdot A^{-1} m^{-1}"$ je permeabilita vákua, $N$ je počet závitov, $l$ je dĺžka solenoidu a $A$ je smernica fitu.

\subsection{Meranie náboja kolmým magnetickým poľom}

V tomto prípade závislosť urýchľovacieho napätia $U$ na magnetizačnom prúd $I$ môžeme vyjadriť ako
\eq[m]{
U\(I^2\) &= \frac{e 8 d^2 \mu_0^2 N^2}{125 m_e R^2} I^2 \,,\lbl{R_2}\\
\frac{e}{m_e} &=  A \(\frac{125 R^2}{8 d^2 \mu_0^2 N^2}\)\,, \lbl{R_2-1}
}
kde $"\mu_0 = 4\pi\cdot10^{-7} Wb\cdot A^{-1} m^{-1}"$ je permeabilita vákua, $N$ je počet závitov, $d$ je priemer dráhy trajektórie, $R$ je polomer cievok a $A$ je smernica fitu.

\subsection{Millikanov experiment}
Polomer častice $r$ od jed zostupnej rýchlosti určíme zo vzťahu 
\eq{
r=\sqrt{\frac{9\eta v\_k}{2g\(\rho\_{olej}-\rho\_{vzd}\)}}\,,\lbl{R_3}
}
kde $\rho\_{olej}$ a $\rho\_{vzd}$ sú hustoty, $\eta$ je dynamická viskozita vzduchu, $v\_k$ je rýchlosť poklesu danej kvapky a $g$ je tiažové zrýchlenie.

Po zahrnutí všetkých síl pôsobiacich na kvapku a Cunninghamovej korekcie dostávame vzťah pre výpočet náboja kvapky oleja
\eq{
Q=E \(\frac{6 \pi \eta}{f_c^{3/2}}\(v\_k+v\_s\)r\) \,, \lbl{R_4}
}
kde $v\_k$ a $v\_s$ sú rýchlosti pre zostupný a vzostupný pohyb, $\eta$ je dynamická viskozita vzduchu ,$f_c$ je Cunninghamov korekční faktor, pre ktorý platí
\eq{
f_c = 1+ \frac{6.18\cdot 10^{-5}}{r p}\,,
}
kde $p$ je atmosferický tlak v torroch a $r$ je polomer kvapky.


%%%%%%%%%%%%%%%%%%%%%%%%%%%%%%%%%%%%%%%
\subsubsection{Spracovanie chýb merania}

Označme $\mean{t}$ aritmetický priemer nameraných hodnôt $t_i$, a $\Delta t$ hodnotu $\mean{t}-t$, pričom 
\eq{
\mean{t} = \frac{1}{n}\sum_{i=1}^n t_i \,, \lbl{SCH_1}
}  
a chybu aritmetického priemeru 
\eq{
  \sigma_0=\sqrt{\frac{\sum_{i=1}^n \(t_i - \mean{t}\)^2}{n\(n-1\)}}\,, \lbl{SCH_2}
}
pričom $n$ je počet meraní.



