%\documentclass[a4paper,10pt]{article}
\documentclass[10pt]{scrartcl}
%\usepackage[IL2]{fontenc}
\usepackage[utf8x]{inputenc}
\usepackage[czech]{babel}
\usepackage{listings}  
\usepackage{amsfonts,amsmath,amssymb,graphicx,color}
%\usepackage[total={17cm,27cm}, top=2cm, left=2cm, includefoot]{geometry}
%\usepackage{fancyhdr}
\usepackage{fkssugar}
\usepackage{hyperref}

%\usepackage{caption}

%  Umožňuje rozdělovat obsah na více sloupců
\usepackage{multicol}
\usepackage{booktabs}
\usepackage{pgffor}


% FJFI Types of popi
\renewcommand{\popi}[2]{$\frac{#1}{[\jd{#2}]}$}

\renewcommand{\figurename}{Obr.}
\addto\captionsczech{\renewcommand{\figurename}{Obr.}}
\addto\captionsczech{\renewcommand{\tablename}{Tab.}}
\def\mean#1{\left< #1 \right>}


\newcommand{\MakeFJFIHead}{
\setlength{\parindent}{0cm}
\begin{multicols}{2}
\textsf{\textbf{\FJFISubject \hspace{6.25cm} \FJFIInstitute}\\
\textbf{\large{\FJFITitle}}}

\begin{tabular}{rlrl}
	 \textsf{Jméno:} & \textbf{\textsf\FJFIAuthor}    &      \textsf{Kolega:} & \textsf{\FJFICoauthor} \\[1.5pt]
	  \textsf{Kruh:} & \textbf{\textsf\FJFIGroup}     & \textsf{Číslo skup.:} & \textsf{\FJFICircle}   \\[1.5pt]
	\textsf{Měřeno:} & \textbf{\textsf{\FJFILabdate}} &  \textsf{Zpracování}: & \textsf{\FJFIWorktime}
\end{tabular}

\begin{flushright}

\includegraphics[scale=0.2]{../img/fjfi.pdf}
%\hspace{0.4cm}
%\includegraphics[scale=0.2]{../img/cvut.pdf}


\textsf{Klasifikace:} \hspace{3.2cm} 

\end{flushright}
\end{multicols}

\hrule
}




%  Nastaví autora, název, datum, skupinu měření apod. 
\newcommand{\FJFIInstitute}{FJFI~ČVUT~v~Praze}
%\newcommand{\Subject}{Základy fyzikálních měření}
\newcommand{\FJFISubject}{Fyzikální praktikum II}  %odkomentujte dle potřeby

%  Máte-li více spoluměřících než jednoho, vložte jen jejich příjmení
\newcommand{\FJFIAuthor}{Michal Červeňák}
\newcommand{\FJFICoauthor}{Ondřej Glac} 
\newcommand{\FJFIGroup}{Útorok} %den, kdy chodíte na praktika, nikoli obor
\newcommand{\FJFICircle}{1} %číslo skupiny v rámci praktika, nikoli kruh

%  Tato část bude v každém protokolu jiná, nezapomeňte upravit!
\newcommand{\FJFITitle}{9. Polarizace světla}
\newcommand{\FJFILabdate}{25.5.2017} %datum měření, nikoli datum odevzdání
\newcommand{\FJFIWorktime}{4 h} %jak dlouho vám trvalo vypracování protokolu


\begin{document}

\MakeFJFIHead{}

\section{Pracovní úkol}
\begin{enumerate}
\item DÚ: Odvoďte vztah (15), spočtěte $\beta$ pro U = 100 V (dosazujte energii v jednotkách
keV!) a diskutujte, zda je korektní považovat elektrony v této úloze za
nerelativistické.
\item DÚ: Odvoďte vztahy (17) a (19) (stačí ponechat v domácí přípravě).
\item Změřte měrný náboj elektronu působením podélného magnetického pole. Měření proveďte
pro různé hodnoty urychlovacího napětí U v rozmezí 750 až 1250 V. Pomocné napětí na
A1 (Obr. 7) volte 140 V. Hodnotu e/me určete fitováním závislosti (17) s errorbary.
\item Změřte měrný náboj elektronu působením kolmého magnetického pole. Naměřte několik
dvojic urychlovacích napětí U (v rozsahu do 300 V) a magnetizačního proudu I (v rozsahu
do 4 A). Hodnotu e/me určete fitováním závislosti (19) s errorbary.

\end{enumerate}


\begin{enumerate}
\item DÚ: Odvoďte vztah (10) pro výpočet náboje kapky.
\item Proveďte Millikanův experiment pro alespoň deset kapiček oleje. Výsledky zpracujte formou
grafu Q na r a určete elementární náboj.
\item Z výsledků úlohy 3a (Měrný náboj elektronu) a této stanovte hmotnost elektronu, vyjádřete
v jednotkách $\jd{keV/c^2}$

\end{enumerate}



\section{Teória}
Lámavý uhol $\sigma$ určíme z nameraných uhlov $d_{1,2}$ podľa vzťahu
\eq{
\sigma = \frac{d_1-d_2}{2}\,.\lbl{R_1}
}

Uhol minimálnej deviacie $\epsilon_0$ určíme z nameraných uhlov $d_{1,2}$ podľa vzťahu
\eq{
\epsilon_0 = \frac{d_1-d_2}{2}\,.\lbl{R_2}
}

Index lomu $n$ vypočítame z lámavého uhlu $\sigma$ a minimálnej deviacie $\epsilon_0$ ako
\eq{
 n = \frac{\sin{\frac{\epsilon_0+\sigma}{2}}}{\sin{\sigma}} \,. \lbl{R_3}
}

Závislosť vlnovej dĺžky $\lambda$ indexu lomu $n$ na konštantách $n_n$, $C$ a $\lambda_n$ udáva vzťah
\eq{
n = n_n + \frac{C}{\lambda - \lambda_n}\,. \lbl{R_4}
}

%%%%%%%%%%%%%%%%%%%%%%%%%%%%%%%%%%%%%%%
\subsubsection{Spracovanie chýb merania}

Označme $\mean{t}$ aritmetický priemer nameraných hodnôt $t_i$, a $\Delta t$ hodnotu $\mean{t}-t$, pričom 
\eq{
\mean{t} = \frac{1}{n}\sum_{i=1}^n t_i \,, \lbl{SCH_1}
}  
a chybu aritmetického priemeru 
\eq{
  \sigma_0=\sqrt{\frac{\sum_{i=1}^n \(t_i - \mean{t}\)^2}{n\(n-1\)}}\,, \lbl{SCH_2}
}
pričom $n$ je počet meraní.






\section{Pomôcky}
Regulovatelné zdroje napětí: 300 V a 2 kV, regulovatelný zdroj proudu 10 A, zdroj
střídavého napětí 6,3 V, ampérmetr, voltmetr, obrazovka se solenoidem, katodová trubice, Helmholtzovy
cívky, aparatura na měření průměru elektronového svazku s dřevěnými posuvnými
měřidly a zástěnou, tyčový a podkovovitý permanentní magnet.
Millikanův přístroj HELAGO 559 412, napájecí jednotka HELAGO 559 421, 2x
elektronické stopky HELAGO 313 033, vodiče, olej.


\section{Postup merania}

\subsection{Meranie náboja pozdĺžnym magnetickým poľom}

\begin{enumerate}
\item Podľa obrázku Obr. 7 \cite{C_1} bol zostavený obvod.
\item Žhaviace napätie boli nastavené na $"1 kV"$
\item postupne pre hodnoty z urýchlovacieho napätia z rozsahu $"100-300 V"$ bol nájdený taký prúd aby sa obraz oproti stavu keď neprechádza žiadne prúd zrotoval. Hodnoty boli zapísané
\end{enumerate}

\subsection{Meranie náboja kolmým magnetickým poľom}
\begin{enumerate}
\item Obvod bol zostavený podľa Obr. 8 \cite{C_1}
\item Bolo zapojené žhaviace napätie a privedený prúd do Helmholtzových cievok
\item Bol nastavený magnetizačný prúd a urýchľovacie napätie tak aby bol zväzok viditeľný a merateľný jeho polomer.
\item postupnou zmenou napätia a prúdu sme udržiavali jeho polomer(priemer) a pritom zaznamenávali $I$ a $U$
\end{enumerate}

\subsection{Millikanov experiment}
\begin{enumerate}
\item Podľa obrázku Obr. 1 a Obr. 2 \cite{C_4} bol zostavený obvod.
\item Napätie bolo nevolené na $U = "500 V"$
\item Do aparatúry bola vháňané nabité olejové kvapky
\item Vždy bola vybraná jedna častica a pre ňu boli pozorované (stopované) 2 prechody hor a 2 dole
\end{enumerate}


\section{Výsledky merania}
\subsection{Lámavého uhol}
Namerané hodnoty lámavého uhla sú v tabuľke Tab. \ref{T_1}.
Z nich bola pomocou vzťahu \ref{SCH_1} vypočítaná hodnota\eq{
\sigma="60.065\pm0.028 \deg"\,.
}

\begin{table}[h]
\begin{center}
\begin{tabular}{| c | c | c |}
\hline
\popi{d_1}{\deg} & \popi{d_1}{\deg} & \popi{d_1-d_2 = 2\sigma}{\deg} \\
\hline
$"281.894\pm0.001"$ & $"161.677\pm0.001"$ & $"120.217\pm0.001"$\\
$"281.866\pm0.001"$ & $"161.769\pm0.001"$ & $"120.097\pm0.001"$\\
$"281.869\pm0.001"$ & $"161.761\pm0.001"$ & $"120.108\pm0.001"$\\
$"281.888\pm0.001"$ & $"161.769\pm0.001"$ & $"120.119\pm0.001"$\\
$"281.869\pm0.001"$ & $"161.764\pm0.001"$ & $"120.106\pm0.001"$\\
\hline
\end{tabular}
\caption{
Namerané uhly $d_1$ a $d_2$ a vypočítaná hodnota $\sigma$ podľa \ref{R_1}
} \label{T_1}
\end{center}
\end{table}

\subsection{Ortuťové spektrum}
Namerané hodnoty ortuťového ortuťového spektra sú v tabulke Tab. \ref{T_2}.


\begin{table}[h]
\begin{center}
\begin{tabular}{| c | c | c | c | c | c |}
\hline
farba & \popi{d_1}{\deg} & \popi{d_1}{\deg} & \popi{\epsilon_0}{\deg} & \popi{n}{-} & \popi{\lambda}{nm} \\
\hline
oranžová     & $"268.942\pm0.001"$ & $"171.214\pm0.001"$ & $"48.864\pm0.002"$ & $"1.625\pm0.003"$ & $"579.065"$\\
žltá         & $"268.964\pm0.001"$ & $"171.192\pm0.001"$ & $"48.886\pm0.002"$ & $"1.626\pm0.003"$ & $"576.074"$\\
zelená       & $"269.242\pm0.001"$ & $"170.900\pm0.001"$ & $"49.171\pm0.002"$ & $"1.629\pm0.003"$ & $"546.074"$\\
azurová      & $"269.883\pm0.001"$ & $"170.244\pm0.001"$ & $"49.819\pm0.002"$ & $"1.636\pm0.003"$ & $"435.835"$\\
fialová      & $"270.858\pm0.001"$ & $"169.261\pm0.001"$ & $"50.799\pm0.002"$ & $"1.645\pm0.003"$ & $"407.781"$\\
ultrafialová & $"271.519\pm0.001"$ & $"168.439\pm0.001"$ & $"51.540\pm0.002"$ & $"1.652\pm0.003"$ & $"404.656"$\\
\hline
\end{tabular}
\caption{
Namerané uhly $d_1$ a $d_2$ a vypočítaná hodnota $\epsilon_0$ podľa \ref{R_2} a index lomu $n$ podľa \ref{R_3} a tabuľková vlnová dĺžka $\lambda$.
} \label{T_2}
\end{center}
\end{table}

Namerané hodnoty boli vynesené do grafu Obr. \ref{G_1} a z fitu dostávame vzťah pre závislosť vlnovej dĺžky na indexe lomu\eq{
n = 1.620\pm0.001 + \frac{0.49\pm0.24}{\lambda-386.2\pm8.1} \,. \lbl{R_V_1}
}

\begin{figure}
% GNUPLOT: LaTeX picture
\setlength{\unitlength}{0.240900pt}
\ifx\plotpoint\undefined\newsavebox{\plotpoint}\fi
\begin{picture}(1500,900)(0,0)
\sbox{\plotpoint}{\rule[-0.200pt]{0.400pt}{0.400pt}}%
\put(171.0,131.0){\rule[-0.200pt]{4.818pt}{0.400pt}}
\put(151,131){\makebox(0,0)[r]{ 0.5}}
\put(1419.0,131.0){\rule[-0.200pt]{4.818pt}{0.400pt}}
\put(171.0,235.0){\rule[-0.200pt]{4.818pt}{0.400pt}}
\put(151,235){\makebox(0,0)[r]{ 1}}
\put(1419.0,235.0){\rule[-0.200pt]{4.818pt}{0.400pt}}
\put(171.0,339.0){\rule[-0.200pt]{4.818pt}{0.400pt}}
\put(151,339){\makebox(0,0)[r]{ 1.5}}
\put(1419.0,339.0){\rule[-0.200pt]{4.818pt}{0.400pt}}
\put(171.0,443.0){\rule[-0.200pt]{4.818pt}{0.400pt}}
\put(151,443){\makebox(0,0)[r]{ 2}}
\put(1419.0,443.0){\rule[-0.200pt]{4.818pt}{0.400pt}}
\put(171.0,547.0){\rule[-0.200pt]{4.818pt}{0.400pt}}
\put(151,547){\makebox(0,0)[r]{ 2.5}}
\put(1419.0,547.0){\rule[-0.200pt]{4.818pt}{0.400pt}}
\put(171.0,651.0){\rule[-0.200pt]{4.818pt}{0.400pt}}
\put(151,651){\makebox(0,0)[r]{ 3}}
\put(1419.0,651.0){\rule[-0.200pt]{4.818pt}{0.400pt}}
\put(171.0,755.0){\rule[-0.200pt]{4.818pt}{0.400pt}}
\put(151,755){\makebox(0,0)[r]{ 3.5}}
\put(1419.0,755.0){\rule[-0.200pt]{4.818pt}{0.400pt}}
\put(171.0,859.0){\rule[-0.200pt]{4.818pt}{0.400pt}}
\put(151,859){\makebox(0,0)[r]{ 4}}
\put(1419.0,859.0){\rule[-0.200pt]{4.818pt}{0.400pt}}
\put(171.0,131.0){\rule[-0.200pt]{0.400pt}{4.818pt}}
\put(171,90){\makebox(0,0){ 0}}
\put(171.0,839.0){\rule[-0.200pt]{0.400pt}{4.818pt}}
\put(382.0,131.0){\rule[-0.200pt]{0.400pt}{4.818pt}}
\put(382,90){\makebox(0,0){ 2}}
\put(382.0,839.0){\rule[-0.200pt]{0.400pt}{4.818pt}}
\put(594.0,131.0){\rule[-0.200pt]{0.400pt}{4.818pt}}
\put(594,90){\makebox(0,0){ 4}}
\put(594.0,839.0){\rule[-0.200pt]{0.400pt}{4.818pt}}
\put(805.0,131.0){\rule[-0.200pt]{0.400pt}{4.818pt}}
\put(805,90){\makebox(0,0){ 6}}
\put(805.0,839.0){\rule[-0.200pt]{0.400pt}{4.818pt}}
\put(1016.0,131.0){\rule[-0.200pt]{0.400pt}{4.818pt}}
\put(1016,90){\makebox(0,0){ 8}}
\put(1016.0,839.0){\rule[-0.200pt]{0.400pt}{4.818pt}}
\put(1228.0,131.0){\rule[-0.200pt]{0.400pt}{4.818pt}}
\put(1228,90){\makebox(0,0){ 10}}
\put(1228.0,839.0){\rule[-0.200pt]{0.400pt}{4.818pt}}
\put(1439.0,131.0){\rule[-0.200pt]{0.400pt}{4.818pt}}
\put(1439,90){\makebox(0,0){ 12}}
\put(1439.0,839.0){\rule[-0.200pt]{0.400pt}{4.818pt}}
\put(171.0,131.0){\rule[-0.200pt]{0.400pt}{175.375pt}}
\put(171.0,131.0){\rule[-0.200pt]{305.461pt}{0.400pt}}
\put(1439.0,131.0){\rule[-0.200pt]{0.400pt}{175.375pt}}
\put(171.0,859.0){\rule[-0.200pt]{305.461pt}{0.400pt}}
\put(30,495){\makebox(0,0){\popi{I}{A}}}
\put(805,29){\makebox(0,0){\popi{U}{V}}}
\put(1279,172){\makebox(0,0)[r]{Namerané hodnoty}}
\put(1397,792){\makebox(0,0){$+$}}
\put(1192,772){\makebox(0,0){$+$}}
\put(1068,721){\makebox(0,0){$+$}}
\put(653,519){\makebox(0,0){$+$}}
\put(541,454){\makebox(0,0){$+$}}
\put(382,345){\makebox(0,0){$+$}}
\put(296,266){\makebox(0,0){$+$}}
\put(248,218){\makebox(0,0){$+$}}
\put(748,573){\makebox(0,0){$+$}}
\put(837,618){\makebox(0,0){$+$}}
\put(916,655){\makebox(0,0){$+$}}
\put(1047,713){\makebox(0,0){$+$}}
\put(1110,740){\makebox(0,0){$+$}}
\put(1349,172){\makebox(0,0){$+$}}
\put(171.0,131.0){\rule[-0.200pt]{0.400pt}{175.375pt}}
\put(171.0,131.0){\rule[-0.200pt]{305.461pt}{0.400pt}}
\put(1439.0,131.0){\rule[-0.200pt]{0.400pt}{175.375pt}}
\put(171.0,859.0){\rule[-0.200pt]{305.461pt}{0.400pt}}
\end{picture}

\caption{
Závislosť indexu lomu $n$ na tabuľkovej vlnovej dĺžke $\lambda$ preložená závislosťou $y = 1.620\pm0.001 + \frac{0.49\pm0.24}{\lambda-386.2\pm8.1}$.
}  \label{G_1}
\end{figure}

\subsection{Spektrum zinku}
Bohužiaľ zinková výboja bola v čase merania pokazená, z tohoto dôvodu nebola nameraná.

\subsection{Vodíkové spektrum}
Namerané hodnoty vodíkového spektra sú v tabuľke \ref{T_3}.
\begin{table}[h]
\begin{center}
\begin{tabular}{| c | c | c | c | c | c |}
\hline
farba & \popi{d_1}{\deg} & \popi{d_1}{\deg} & \popi{\epsilon_0}{\deg} & \popi{n}{-} & \popi{\lambda}{nm} \\
\hline
červená & $"278.727\pm0.001"$ & $"171.833\pm0.001"$ & $"53.447\pm0.002"$ & $"1.670\pm0.003"$ & $"376.484"$\\
modrá   & $"270.069\pm0.001"$ & $"170.258\pm0.001"$ & $"50.331\pm0.002"$ & $"1.640\pm0.003"$ & $"361.719"$\\
fialová & $"270.008\pm0.001"$ & $"169.344\pm0.001"$ & $"49.906\pm0.002"$ & $"1.635\pm0.003"$ & $"355.105"$\\
\hline
\end{tabular}
\caption{Namerané uhly $d_1$ a $d_2$ a vypočítaná hodnota $\epsilon_0$ 
podľa \ref{R_2} a index lomu $n$ podľa \ref{R_3} a vypočítaná hodnota 
vlnovej dĺžky $\lambda$ podľa vzťahu \ref{R_V_1}.
} \label{T_3}
\end{center}
\end{table}

\subsection{Výpočet Rydbergovy konštanty}
Pre jednotlivé farby bola postupne spočítaná Rydbergovu konštanta $R$
\eq[m]{
R_c &= "19.1 nm^{-1}" \,,\\
R_m &= "14.8 nm^{-1}" \,,\\
R_f &= "13.3 nm^{-1}" \,,\\
}
kde spodný index označuje farbu $c$ červenú, $m$ modrú a $f$ fialovú.


\subsection{Sodíkové spektrum}

Namerané hodnoty sodíkového spektra sú v tabuľke Tab. \ref{T_4}

\begin{table}[h]
\begin{center}
\begin{tabular}{| c | c | c | c | c | c |}
\hline
farba & \popi{d_1}{\deg} & \popi{d_1}{\deg} & \popi{\epsilon_0}{\deg} & \popi{n}{-} & \popi{\lambda}{nm} \\
\hline
červená & $"269.002\pm0.001"$ & $"171.391\pm0.001"$ & $"48.805\pm0.002"$ & $"1.627\pm0.003"$ & $"385.12"$\\

\hline
\end{tabular}
\caption{Namerané uhly $d_1$ a $d_2$ a vypočítaná hodnota $\epsilon_0$ 
podľa \ref{R_2} a index lomu $n$ podľa \ref{R_3} a vypočítaná hodnota 
vlnovej dĺžky $\lambda$ podľa vzťahu \ref{R_V_1}.
} \label{T_4}
\end{center}
\end{table}




\section{Diskusia}

V prvej časti pri meraní náboja elektrónu v pozdĺžnom poli sme sa dopúšťali asi najzávažnejších chýb merania.
V prvej rade bolo veľmi náročné presne určiť kedy nastala jedna otáčka zväzku, jeho \uv{trajektória} na clone bola špirálovitá a s pripadajúcim magnetizačným prúdom sa zväzok viac fokusoval až v niektorých prípadoch do jedného bodu. V takomto prípade sme zaznačili údaje pre tento bod. Špirálovitá dráha spôsobovala aj v prípade nie úplne fokusovaného zväzku problém pri určení jeho rotácie. Aj z tohoto dôvodu uvažujem pri meraní prídu chybu až $"1 A"$.

Druhá časť merania sa ukázala presnejšia, predovšetkým s veľkou presnosťou sa nám darili držať dráhu zväzku v dráhe s daným polomerom.
o čom svedčí aj menšia chyba merania v tomto prípade pre $\Delta I = "0.2 A"$.

V poslednej časti (Millikanov experiment) sa síce podarilo namerať náboj pomerne presne (vrámci chyby) ale náboj len z nameraných dát ťažko vyčítame. Dôvod je ten, že náboje nespĺňajú naše predpokladané chovanie a nemajú pomer v celočíselných násobkoch. Teda je nemožné určiť elementárny náboj.

Pre ďalšie výpočty bol použitý druhý najmenší, prvý najmenší bol vyradený, pretože mal veľmi vysokú chybu merania času.

K všetkým meraniam môžeme pristupovať nerelativisticky, pretože chyba merania je rádovo väčšia ako chyba STR.




\section{Záver}

Meraním náboja pozdĺžnym magnetickým poľom sme zistili pomer náboja k hmotnosti 
\eq{
\frac{e}{m_e} ="6.8\pm3.3 C\cdot kg^{-1}"\,,
}
kolmým nábojom
\eq{
\frac{e}{m_e}="\(1.12\pm0.41\)\cdot10^11 C\cdot kg^{-1}"\,.
}

Z Millikaveho experimentu bol určený náboj $"\(4.56\pm0.48\)\cdot 10^{-19}"$

Hmotnosť elektrónu bola stanovená na 
\eq{
m_e = "2.24 \frac{MeV}{c^2}"\,.
}



\begin{thebibliography}{2}
\bibitem{C_1}
Měrný náboj elektronu [cit. 20.03.2017]Dostupné po prihlásení z Kurz: Fyzikální praktikum II:\url{https://praktikum.fjfi.cvut.cz/pluginfile.php/425/mod_resource/content/6/3a_Naboj_170218.pdf}
\bibitem{C_4}
Millikanův experiment [cit. 20.03.2017]Dostupné po prihlásení z Kurz: Fyzikální praktikum II:\url{https://praktikum.fjfi.cvut.cz/pluginfile.php/6566/mod_resource/content/3/3b_Millikan_170218.pdf}

\bibitem{C_2}
Elektron [cit. 20.03.2017]Dostupné na:\url{https://cs.wikipedia.org/wiki/Elektron}

\bibitem{C_3}
Vzduch [cit. 20.03.2017]Dostupné na:\url{http://www.converter.cz/tabulky/vzduch.htm}

\end{thebibliography}

\end{document}

