\section{Teória}
Cievkou s $n_1$ závitami navinutej na toroide o polomere $r$, prechádza prúd $I_M$ 
Potom pre túto cievku môžeme intenzitu magnetického poľa $H$ vyjadriť ako
\eq{
H=\frac{n_1 I_M}{2\pi r}\,. \lbl{R_1}
}

Pre zmenu magnetickej indukcie $B$ v závislosti na výchylke balistického galvanometru $s$ platí vzťah 
\eq{
\Delta B = \frac{B K_b \lambda s_1^{*}}{n_2 S} \,, \lbl{R_2}
}
pričom $K_b$ je balistická konštanta, $R$ je odpor na odporovej dekáde, $s_1^{*}$ je výchylka galvanometra a $n_2$ počet meracích závitov. 

pre balistickú konštantu platí 
\eq{
R K_b \lambda = \frac{2L_{12} I_i}{s_1}\,,
}
kde $L_{12}$ je normálová cievka so známou indukčnosťou, a $s_1$ je výchylka pri zmene prúdu o $I_i$

%%%%%%%%%%%%%%%%%%%%%%%%%%%%%%%%%%%%%%%=
\subsubsection{Spracovanie chýb merania}

Označme $\mean{t}$ aritmetický priemer nameraných hodnôt $t_i$, a $\Delta t$ hodnotu $\mean{t}-t$, pričom 
\eq{
\mean{t} = \frac{1}{n}\sum_{i=1}^n t_i \,, \lbl{SCH_1}
}  
a chybu aritmetického priemeru 
\eq{
  \sigma_0=\sqrt{\frac{\sum_{i=1}^n \(t_i - \mean{t}\)^2}{n\(n-1\)}}\,, \lbl{SCH_2}
}
pričom $n$ je počet meraní.



