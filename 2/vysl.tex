\section{Výsledky merania}
Pre výpočty boli použité nasledujúce konštanty
\eq[m]{
r &= "17.1\cdot10^{3} m" \,,\\
n_1 &= 62 \,,\\
n_2 &=400\,,\\
S &="24.3\cdot10^{-6} m" \,,\\
L_{12} &= "7.27 mH"\,,
}

K výpočtu balistickej konštanty sme použili namerané dáta $R="20 k\Omega"$, $s="10.35 cm"$ a $I="0.206 A"$.

V tabuľke \ref{T_1} sú namerané hodnoty, pre oba smery slučky.

Hodnoty sú vynesené do grafu Obr. \ref{G_1}

\begin{table}[h]
\begin{center}
\begin{tabular}{| c | c | c | c | c | c | c | c |}
\hline
\popi{I}{mA} & \popi{s}{cm} & \popi{I^\prime}{mA} & \popi{s^\prime}{cm} &  \popi{H}{A/m} & \popi{\Delta B}{T} & \popi{H^\prime}{A/m} & \popi{\Delta B^\prime}{T} \\
\hline
$ 0.08  $&$  2.70 $&$ 0.08   $&$ 14.4 $&$ 43.2  $&$ 0.081 $&$ 46.2  $&$ 0.43$\\
$ 0.09  $&$  2.00 $&$ 0.09   $&$ 14.5 $&$ 50.7  $&$ 0.060 $&$ 51.9  $&$ 0.43$\\
$ 0.11  $&$  1.20 $&$ 0.01   $&$ 4.5  $&$ 60.5  $&$ 0.036 $&$ 5.8   $&$ 0.13$\\
$ 0.12  $&$  1.00 $& -        &   -    &$ 69.8  $&$ 0.030 $&  -      &  -    \\
$ 0.131 $&$  0.9  $&$ 0.141  $&$ 15.5 $&$ 75.5  $&$ 0.027 $&$ 81.4  $&$ 0.46$\\
$ 0.156 $&$  0.6  $& -        &   -    &$ 90.0  $&$ 0.018 $&  -      &  -    \\
$ 0.174 $&$  0.8  $&$ 0.172  $&$ 17   $&$ 100.4 $&$ 0.024 $&$ 99.3  $&$ 0.51$\\
$ 0.195 $&$  0.2  $&$ 0.065  $&$ 13.5 $&$ 112.5 $&$ 0.006 $&$ 37.5  $&$ 0.40$\\
$ 0.051 $&$  3    $&$ 0.051  $&$ 14.5 $&$ 29.4  $&$ 0.090 $&$ 29.4  $&$ 0.43$\\
$ 0.031 $&$  4.4  $&$ 0.031  $&$ 13.3 $&$ 17.9  $&$ 0.13  $&$ 17.9  $&$ 0.40$\\
$ 0.025 $&$  4.6  $&$ 0.025  $&$ 6.6  $&$ 14.4  $&$ 0.14  $&$ 14.4  $&$ 0.20$\\
$ 0.015 $&$  4.6  $&  -       &   -    &$ 8.7   $&$ 0.14  $&  -      &  -    \\
$ 0     $&$  4.6  $&$ 0      $&$ 3.6  $&$ 0     $&$ 0.14  $&$ 0     $&$ 0.10$\\
$-0.031 $&$  8.4  $& -        &   -    &$ -17.9 $&$ 0.25  $&  -      &  -    \\
$-0.038 $&$  8.5  $&$ -0.038 $&$ 2    $&$ -21.9 $&$ 0.25  $&$ -21.9 $&$ 0.06$\\
$-0.051 $&$ 13.6  $&$ -0.051 $&$ 2    $&$ -29.4 $&$ 0.41  $&$ -29.4 $&$ 0.06$\\
$-0.06  $&$ 14.1  $&$ -0.06  $&$ 1.1  $&$ -34.6 $&$ 0.42  $&$ -34.6 $&$ 0.03$\\
$-0.075 $&$ 14.7  $&$ -0.074 $&$ 1.5  $&$ -43.3 $&$ 0.44  $&$ -42.7 $&$ 0.04$\\
$-0.098 $&$ 14.8  $&$ -0.098 $&$ 1.3  $&$ -56.6 $&$ 0.44  $&$ -56.6 $&$ 0.04$\\
$-0.121 $&$ 15.6  $& -        &  -     &$ -69.8 $&$ 0.47  $&  -      & -     \\
$-0.143 $&$ 15.5  $& -        &  -     &$ -82.5 $&$ 0.46  $&  -      & -     \\
$-0.173 $&$ 15.9  $& -        &  -     &$ -99.8 $&$ 0.48  $&  -      & -     \\
$-0.198 $&$ 16.5  $&$ -0.197 $&$ 0.9  $&$-114.3 $&$ 0.49  $&$ -113.7$&$ 0.03$\\
\hline
\end{tabular}
\caption{Namerané dáta prúdu $I$ a amplitúdy kyvadla $s$ a 
z nich vypočítané pomocou \ref{R_1} a \ref{R_2} hodnoty intenzity magnetického poľa $H$ a 
magnetická indukcie $B$ , pre kladné a záporné prúdy
kde opačný sme je označený pomocou $~^\prime$. 
} \label{T_1}
\end{center}
\end{table}


\begin{figure}
% GNUPLOT: LaTeX picture
\setlength{\unitlength}{0.240900pt}
\ifx\plotpoint\undefined\newsavebox{\plotpoint}\fi
\begin{picture}(1500,900)(0,0)
\sbox{\plotpoint}{\rule[-0.200pt]{0.400pt}{0.400pt}}%
\put(171.0,131.0){\rule[-0.200pt]{4.818pt}{0.400pt}}
\put(151,131){\makebox(0,0)[r]{ 0.5}}
\put(1419.0,131.0){\rule[-0.200pt]{4.818pt}{0.400pt}}
\put(171.0,235.0){\rule[-0.200pt]{4.818pt}{0.400pt}}
\put(151,235){\makebox(0,0)[r]{ 1}}
\put(1419.0,235.0){\rule[-0.200pt]{4.818pt}{0.400pt}}
\put(171.0,339.0){\rule[-0.200pt]{4.818pt}{0.400pt}}
\put(151,339){\makebox(0,0)[r]{ 1.5}}
\put(1419.0,339.0){\rule[-0.200pt]{4.818pt}{0.400pt}}
\put(171.0,443.0){\rule[-0.200pt]{4.818pt}{0.400pt}}
\put(151,443){\makebox(0,0)[r]{ 2}}
\put(1419.0,443.0){\rule[-0.200pt]{4.818pt}{0.400pt}}
\put(171.0,547.0){\rule[-0.200pt]{4.818pt}{0.400pt}}
\put(151,547){\makebox(0,0)[r]{ 2.5}}
\put(1419.0,547.0){\rule[-0.200pt]{4.818pt}{0.400pt}}
\put(171.0,651.0){\rule[-0.200pt]{4.818pt}{0.400pt}}
\put(151,651){\makebox(0,0)[r]{ 3}}
\put(1419.0,651.0){\rule[-0.200pt]{4.818pt}{0.400pt}}
\put(171.0,755.0){\rule[-0.200pt]{4.818pt}{0.400pt}}
\put(151,755){\makebox(0,0)[r]{ 3.5}}
\put(1419.0,755.0){\rule[-0.200pt]{4.818pt}{0.400pt}}
\put(171.0,859.0){\rule[-0.200pt]{4.818pt}{0.400pt}}
\put(151,859){\makebox(0,0)[r]{ 4}}
\put(1419.0,859.0){\rule[-0.200pt]{4.818pt}{0.400pt}}
\put(171.0,131.0){\rule[-0.200pt]{0.400pt}{4.818pt}}
\put(171,90){\makebox(0,0){ 0}}
\put(171.0,839.0){\rule[-0.200pt]{0.400pt}{4.818pt}}
\put(382.0,131.0){\rule[-0.200pt]{0.400pt}{4.818pt}}
\put(382,90){\makebox(0,0){ 2}}
\put(382.0,839.0){\rule[-0.200pt]{0.400pt}{4.818pt}}
\put(594.0,131.0){\rule[-0.200pt]{0.400pt}{4.818pt}}
\put(594,90){\makebox(0,0){ 4}}
\put(594.0,839.0){\rule[-0.200pt]{0.400pt}{4.818pt}}
\put(805.0,131.0){\rule[-0.200pt]{0.400pt}{4.818pt}}
\put(805,90){\makebox(0,0){ 6}}
\put(805.0,839.0){\rule[-0.200pt]{0.400pt}{4.818pt}}
\put(1016.0,131.0){\rule[-0.200pt]{0.400pt}{4.818pt}}
\put(1016,90){\makebox(0,0){ 8}}
\put(1016.0,839.0){\rule[-0.200pt]{0.400pt}{4.818pt}}
\put(1228.0,131.0){\rule[-0.200pt]{0.400pt}{4.818pt}}
\put(1228,90){\makebox(0,0){ 10}}
\put(1228.0,839.0){\rule[-0.200pt]{0.400pt}{4.818pt}}
\put(1439.0,131.0){\rule[-0.200pt]{0.400pt}{4.818pt}}
\put(1439,90){\makebox(0,0){ 12}}
\put(1439.0,839.0){\rule[-0.200pt]{0.400pt}{4.818pt}}
\put(171.0,131.0){\rule[-0.200pt]{0.400pt}{175.375pt}}
\put(171.0,131.0){\rule[-0.200pt]{305.461pt}{0.400pt}}
\put(1439.0,131.0){\rule[-0.200pt]{0.400pt}{175.375pt}}
\put(171.0,859.0){\rule[-0.200pt]{305.461pt}{0.400pt}}
\put(30,495){\makebox(0,0){\popi{I}{A}}}
\put(805,29){\makebox(0,0){\popi{U}{V}}}
\put(1279,172){\makebox(0,0)[r]{Namerané hodnoty}}
\put(1397,792){\makebox(0,0){$+$}}
\put(1192,772){\makebox(0,0){$+$}}
\put(1068,721){\makebox(0,0){$+$}}
\put(653,519){\makebox(0,0){$+$}}
\put(541,454){\makebox(0,0){$+$}}
\put(382,345){\makebox(0,0){$+$}}
\put(296,266){\makebox(0,0){$+$}}
\put(248,218){\makebox(0,0){$+$}}
\put(748,573){\makebox(0,0){$+$}}
\put(837,618){\makebox(0,0){$+$}}
\put(916,655){\makebox(0,0){$+$}}
\put(1047,713){\makebox(0,0){$+$}}
\put(1110,740){\makebox(0,0){$+$}}
\put(1349,172){\makebox(0,0){$+$}}
\put(171.0,131.0){\rule[-0.200pt]{0.400pt}{175.375pt}}
\put(171.0,131.0){\rule[-0.200pt]{305.461pt}{0.400pt}}
\put(1439.0,131.0){\rule[-0.200pt]{0.400pt}{175.375pt}}
\put(171.0,859.0){\rule[-0.200pt]{305.461pt}{0.400pt}}
\end{picture}

\caption{Hysterzná slučka z nameraných dát, preložená $erf(x)$, 
kde $H$ je intenzita magnetického poľa a $B$ je magnetická indukcia. 
}  \label{G_1}
\end{figure}

Z nameraných dát dostávame $B_r="0.9\pm0.1 H"$ a z preloženia grafu funkciami 
\eq[m]{
g(x) &= \("0.22\pm0.01"\)\cdot \mathrm{erf}\(\("0.023\pm0.002"\)\cdot\(x+\("10.8\pm2.0"\)\)\)\,,\\
f(x) &= \("0.22\pm0.01"\)\cdot \mathrm{erf}\(\("0.023\pm0.002"\)\cdot\(x+\("13.2\pm2.1"\)\)\)\,,
}
odkiaľ dostávame $H_k=" 12.05\pm1.2 A/m"$.

Magnetické straty boli pomocou určené z grafu ako $x = "70.9 A\cdot m^{-1}\cdot T"$.



%\begin{table}[h]
%\begin{center}
%\begin{tabular}{ | c | c |}
%\hline
%\popi{f}{kHz} & \popi{\mean{\Delta d_i}}{m} \\
%\hline
%$2$&$0.080\pm0.004$\\
%$2.5$&$0.070\pm0.001$\\
%$3$&$0.058\pm0.002$\\
%$3.5$&$0.050\pm0.001$\\
%$4$&$0.041\pm0.005$\\
%$4.5$&$0.039\pm0.001$\\
%$5$&$0.034\pm0.008$\\
%$5.5$&$0.031\pm0.001$\\
%$6$&$0.029\pm0.001$\\
%$4.25$& $0.040\pm0.001$\\
%\hline
%\end{tabular}
%\caption{Vypočítané hodnoty priemernej dĺžky vzdialenosti minim $\Delta d_i$ od frekvencie $f$
%} \label{T_2}
%\end{center}
%\end{table}

