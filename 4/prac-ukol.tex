\section{Pracovní úkol}
\begin{enumerate}
\item DU: V přípravě odvoďťe vzorec (11) pro případ, kdy je splněna podmínka úhlu
nejmenší deviace $\alpha_1 = \alpha_2$.
\item  Metodou dělených svazků změřte lámavý úhel hranolu. Měření opakujte 5×.
\item  Změřte index lomu hranolu v závislosti na vlnové délce pro čáry rtut’ového spektra, vyneste
do grafu a fitováním nelineární funkcí (13) určete disperzní vztah $n = n(\lambda)$.
\item  Změřte vlnové délky spektrálních čar zinkové výbojky a porovnejte je s tabulkovými hodnotami.
\item  Změřte spektrum vodíkové výbojky, porovnejte s tabulkovými hodnotami, ověřte platnost
vztahu (5) a určete hodnotu Rydbergovy konstanty.
\item  Určete charakteristickou disperzi $d n/d \lambda $
v okolí vlnové délky 589 nm (žlutá čára v sodíkovém
spektru). Poté spočítejte minimální velikost základny hranolu, vyrobeného ze stejného materiálu
jako hranol, se kterým měříte, který je ještě schopný sodíkový dublet rozlišit.
\end{enumerate}

