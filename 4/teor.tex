\section{Teória}
Lámavý uhol $\sigma$ určíme z nameraných uhlov $d_{1,2}$ podľa vzťahu
\eq{
\sigma = \frac{d_1-d_2}{2}\,.\lbl{R_1}
}

Uhol minimálnej deviacie $\epsilon_0$ určíme z nameraných uhlov $d_{1,2}$ podľa vzťahu
\eq{
\epsilon_0 = \frac{d_1-d_2}{2}\,.\lbl{R_2}
}

Index lomu $n$ vypočítame z lámavého uhlu $\sigma$ a minimálnej deviacie $\epsilon_0$ ako
\eq{
 n = \frac{\sin{\frac{\epsilon_0+\sigma}{2}}}{\sin{\sigma}} \,. \lbl{R_3}
}

Závislosť vlnovej dĺžky $\lambda$ indexu lomu $n$ na konštantách $n_n$, $C$ a $\lambda_n$ udáva vzťah
\eq{
n = n_n + \frac{C}{\lambda - \lambda_n}\,. \lbl{R_4}
}

%%%%%%%%%%%%%%%%%%%%%%%%%%%%%%%%%%%%%%%
\subsubsection{Spracovanie chýb merania}

Označme $\mean{t}$ aritmetický priemer nameraných hodnôt $t_i$, a $\Delta t$ hodnotu $\mean{t}-t$, pričom 
\eq{
\mean{t} = \frac{1}{n}\sum_{i=1}^n t_i \,, \lbl{SCH_1}
}  
a chybu aritmetického priemeru 
\eq{
  \sigma_0=\sqrt{\frac{\sum_{i=1}^n \(t_i - \mean{t}\)^2}{n\(n-1\)}}\,, \lbl{SCH_2}
}
pričom $n$ je počet meraní.



