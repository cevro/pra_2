\section{Výsledky merania}
\subsection{Lámavého uhol}
Namerané hodnoty lámavého uhla sú v tabuľke Tab. \ref{T_1}.
Z nich bola pomocou vzťahu \ref{SCH_1} vypočítaná hodnota\eq{
\sigma="60.065\pm0.028 \deg"\,.
}

\begin{table}[h]
\begin{center}
\begin{tabular}{| c | c | c |}
\hline
\popi{d_1}{\deg} & \popi{d_1}{\deg} & \popi{d_1-d_2 = 2\sigma}{\deg} \\
\hline
$"281.894\pm0.001"$ & $"161.677\pm0.001"$ & $"120.217\pm0.001"$\\
$"281.866\pm0.001"$ & $"161.769\pm0.001"$ & $"120.097\pm0.001"$\\
$"281.869\pm0.001"$ & $"161.761\pm0.001"$ & $"120.108\pm0.001"$\\
$"281.888\pm0.001"$ & $"161.769\pm0.001"$ & $"120.119\pm0.001"$\\
$"281.869\pm0.001"$ & $"161.764\pm0.001"$ & $"120.106\pm0.001"$\\
\hline
\end{tabular}
\caption{
Namerané uhly $d_1$ a $d_2$ a vypočítaná hodnota $\sigma$ podľa \ref{R_1}
} \label{T_1}
\end{center}
\end{table}

\subsection{Ortuťové spektrum}
Namerané hodnoty ortuťového ortuťového spektra sú v tabulke Tab. \ref{T_2}.


\begin{table}[h]
\begin{center}
\begin{tabular}{| c | c | c | c | c | c |}
\hline
farba & \popi{d_1}{\deg} & \popi{d_1}{\deg} & \popi{\epsilon_0}{\deg} & \popi{n}{-} & \popi{\lambda}{nm} \\
\hline
oranžová     & $"268.942\pm0.001"$ & $"171.214\pm0.001"$ & $"48.864\pm0.002"$ & $"1.625\pm0.003"$ & $"579.065"$\\
žltá         & $"268.964\pm0.001"$ & $"171.192\pm0.001"$ & $"48.886\pm0.002"$ & $"1.626\pm0.003"$ & $"576.074"$\\
zelená       & $"269.242\pm0.001"$ & $"170.900\pm0.001"$ & $"49.171\pm0.002"$ & $"1.629\pm0.003"$ & $"546.074"$\\
azurová      & $"269.883\pm0.001"$ & $"170.244\pm0.001"$ & $"49.819\pm0.002"$ & $"1.636\pm0.003"$ & $"435.835"$\\
fialová      & $"270.858\pm0.001"$ & $"169.261\pm0.001"$ & $"50.799\pm0.002"$ & $"1.645\pm0.003"$ & $"407.781"$\\
ultrafialová & $"271.519\pm0.001"$ & $"168.439\pm0.001"$ & $"51.540\pm0.002"$ & $"1.652\pm0.003"$ & $"404.656"$\\
\hline
\end{tabular}
\caption{
Namerané uhly $d_1$ a $d_2$ a vypočítaná hodnota $\epsilon_0$ podľa \ref{R_2} a index lomu $n$ podľa \ref{R_3} a tabuľková vlnová dĺžka $\lambda$.
} \label{T_2}
\end{center}
\end{table}

Namerané hodnoty boli vynesené do grafu Obr. \ref{G_1} a z fitu dostávame vzťah pre závislosť vlnovej dĺžky na indexe lomu\eq{
n = 1.620\pm0.001 + \frac{0.49\pm0.24}{\lambda-386.2\pm8.1} \,. \lbl{R_V_1}
}

\begin{figure}
% GNUPLOT: LaTeX picture
\setlength{\unitlength}{0.240900pt}
\ifx\plotpoint\undefined\newsavebox{\plotpoint}\fi
\sbox{\plotpoint}{\rule[-0.200pt]{0.400pt}{0.400pt}}%
\begin{picture}(1500,900)(0,0)
\sbox{\plotpoint}{\rule[-0.200pt]{0.400pt}{0.400pt}}%
\put(171.0,131.0){\rule[-0.200pt]{4.818pt}{0.400pt}}
\put(151,131){\makebox(0,0)[r]{-0.3}}
\put(1419.0,131.0){\rule[-0.200pt]{4.818pt}{0.400pt}}
\put(171.0,252.0){\rule[-0.200pt]{4.818pt}{0.400pt}}
\put(151,252){\makebox(0,0)[r]{-0.2}}
\put(1419.0,252.0){\rule[-0.200pt]{4.818pt}{0.400pt}}
\put(171.0,374.0){\rule[-0.200pt]{4.818pt}{0.400pt}}
\put(151,374){\makebox(0,0)[r]{-0.1}}
\put(1419.0,374.0){\rule[-0.200pt]{4.818pt}{0.400pt}}
\put(171.0,495.0){\rule[-0.200pt]{4.818pt}{0.400pt}}
\put(151,495){\makebox(0,0)[r]{ 0}}
\put(1419.0,495.0){\rule[-0.200pt]{4.818pt}{0.400pt}}
\put(171.0,616.0){\rule[-0.200pt]{4.818pt}{0.400pt}}
\put(151,616){\makebox(0,0)[r]{ 0.1}}
\put(1419.0,616.0){\rule[-0.200pt]{4.818pt}{0.400pt}}
\put(171.0,738.0){\rule[-0.200pt]{4.818pt}{0.400pt}}
\put(151,738){\makebox(0,0)[r]{ 0.2}}
\put(1419.0,738.0){\rule[-0.200pt]{4.818pt}{0.400pt}}
\put(171.0,859.0){\rule[-0.200pt]{4.818pt}{0.400pt}}
\put(151,859){\makebox(0,0)[r]{ 0.3}}
\put(1419.0,859.0){\rule[-0.200pt]{4.818pt}{0.400pt}}
\put(171.0,131.0){\rule[-0.200pt]{0.400pt}{4.818pt}}
\put(171,90){\makebox(0,0){-150}}
\put(171.0,839.0){\rule[-0.200pt]{0.400pt}{4.818pt}}
\put(382.0,131.0){\rule[-0.200pt]{0.400pt}{4.818pt}}
\put(382,90){\makebox(0,0){-100}}
\put(382.0,839.0){\rule[-0.200pt]{0.400pt}{4.818pt}}
\put(594.0,131.0){\rule[-0.200pt]{0.400pt}{4.818pt}}
\put(594,90){\makebox(0,0){-50}}
\put(594.0,839.0){\rule[-0.200pt]{0.400pt}{4.818pt}}
\put(805.0,131.0){\rule[-0.200pt]{0.400pt}{4.818pt}}
\put(805,90){\makebox(0,0){ 0}}
\put(805.0,839.0){\rule[-0.200pt]{0.400pt}{4.818pt}}
\put(1016.0,131.0){\rule[-0.200pt]{0.400pt}{4.818pt}}
\put(1016,90){\makebox(0,0){ 50}}
\put(1016.0,839.0){\rule[-0.200pt]{0.400pt}{4.818pt}}
\put(1228.0,131.0){\rule[-0.200pt]{0.400pt}{4.818pt}}
\put(1228,90){\makebox(0,0){ 100}}
\put(1228.0,839.0){\rule[-0.200pt]{0.400pt}{4.818pt}}
\put(1439.0,131.0){\rule[-0.200pt]{0.400pt}{4.818pt}}
\put(1439,90){\makebox(0,0){ 150}}
\put(1439.0,839.0){\rule[-0.200pt]{0.400pt}{4.818pt}}
\put(171.0,495.0){\rule[-0.200pt]{305.461pt}{0.400pt}}
\put(805.0,131.0){\rule[-0.200pt]{0.400pt}{175.375pt}}
\put(171.0,131.0){\rule[-0.200pt]{0.400pt}{175.375pt}}
\put(171.0,131.0){\rule[-0.200pt]{305.461pt}{0.400pt}}
\put(1439.0,131.0){\rule[-0.200pt]{0.400pt}{175.375pt}}
\put(171.0,859.0){\rule[-0.200pt]{305.461pt}{0.400pt}}
\put(30,495){\makebox(0,0){\popi{B}{H}}}
\put(805,29){\makebox(0,0){\popi{H}{A\cdot m^{-1}}}}
\put(1279,295){\makebox(0,0)[r]{namerané dáta}}
\put(988,669){\makebox(0,0){$+$}}
\put(1020,694){\makebox(0,0){$+$}}
\put(1061,723){\makebox(0,0){$+$}}
\put(1100,730){\makebox(0,0){$+$}}
\put(1125,734){\makebox(0,0){$+$}}
\put(1185,745){\makebox(0,0){$+$}}
\put(1229,738){\makebox(0,0){$+$}}
\put(1281,759){\makebox(0,0){$+$}}
\put(929,658){\makebox(0,0){$+$}}
\put(881,607){\makebox(0,0){$+$}}
\put(866,600){\makebox(0,0){$+$}}
\put(842,600){\makebox(0,0){$+$}}
\put(805,600){\makebox(0,0){$+$}}
\put(729,462){\makebox(0,0){$+$}}
\put(712,458){\makebox(0,0){$+$}}
\put(681,273){\makebox(0,0){$+$}}
\put(659,255){\makebox(0,0){$+$}}
\put(622,234){\makebox(0,0){$+$}}
\put(566,230){\makebox(0,0){$+$}}
\put(510,201){\makebox(0,0){$+$}}
\put(456,205){\makebox(0,0){$+$}}
\put(383,190){\makebox(0,0){$+$}}
\put(322,168){\makebox(0,0){$+$}}
\put(1349,295){\makebox(0,0){$+$}}
\put(1279,254){\makebox(0,0)[r]{namerané dáta}}
\put(1000,746){\makebox(0,0){$\times$}}
\put(1025,749){\makebox(0,0){$\times$}}
\put(829,387){\makebox(0,0){$\times$}}
\put(1149,785){\makebox(0,0){$\times$}}
\put(1225,840){\makebox(0,0){$\times$}}
\put(964,713){\makebox(0,0){$\times$}}
\put(929,749){\makebox(0,0){$\times$}}
\put(881,706){\makebox(0,0){$\times$}}
\put(866,463){\makebox(0,0){$\times$}}
\put(805,354){\makebox(0,0){$\times$}}
\put(712,296){\makebox(0,0){$\times$}}
\put(681,296){\makebox(0,0){$\times$}}
\put(659,263){\makebox(0,0){$\times$}}
\put(625,278){\makebox(0,0){$\times$}}
\put(566,270){\makebox(0,0){$\times$}}
\put(325,256){\makebox(0,0){$\times$}}
\put(1349,254){\makebox(0,0){$\times$}}
\sbox{\plotpoint}{\rule[-0.400pt]{0.800pt}{0.800pt}}%
\sbox{\plotpoint}{\rule[-0.200pt]{0.400pt}{0.400pt}}%
\put(1279,213){\makebox(0,0)[r]{Aproximácia dát pomocou $erf(x)$}}
\sbox{\plotpoint}{\rule[-0.400pt]{0.800pt}{0.800pt}}%
\put(1299.0,213.0){\rule[-0.400pt]{24.090pt}{0.800pt}}
\put(322,224){\usebox{\plotpoint}}
\put(390,222.84){\rule{2.409pt}{0.800pt}}
\multiput(390.00,222.34)(5.000,1.000){2}{\rule{1.204pt}{0.800pt}}
\put(322.0,224.0){\rule[-0.400pt]{16.381pt}{0.800pt}}
\put(409,223.84){\rule{2.409pt}{0.800pt}}
\multiput(409.00,223.34)(5.000,1.000){2}{\rule{1.204pt}{0.800pt}}
\put(400.0,225.0){\rule[-0.400pt]{2.168pt}{0.800pt}}
\put(429,224.84){\rule{2.168pt}{0.800pt}}
\multiput(429.00,224.34)(4.500,1.000){2}{\rule{1.084pt}{0.800pt}}
\put(438,225.84){\rule{2.409pt}{0.800pt}}
\multiput(438.00,225.34)(5.000,1.000){2}{\rule{1.204pt}{0.800pt}}
\put(448,226.84){\rule{2.409pt}{0.800pt}}
\multiput(448.00,226.34)(5.000,1.000){2}{\rule{1.204pt}{0.800pt}}
\put(458,227.84){\rule{2.168pt}{0.800pt}}
\multiput(458.00,227.34)(4.500,1.000){2}{\rule{1.084pt}{0.800pt}}
\put(467,228.84){\rule{2.409pt}{0.800pt}}
\multiput(467.00,228.34)(5.000,1.000){2}{\rule{1.204pt}{0.800pt}}
\put(477,230.34){\rule{2.409pt}{0.800pt}}
\multiput(477.00,229.34)(5.000,2.000){2}{\rule{1.204pt}{0.800pt}}
\put(487,232.34){\rule{2.168pt}{0.800pt}}
\multiput(487.00,231.34)(4.500,2.000){2}{\rule{1.084pt}{0.800pt}}
\put(496,234.34){\rule{2.409pt}{0.800pt}}
\multiput(496.00,233.34)(5.000,2.000){2}{\rule{1.204pt}{0.800pt}}
\put(506,236.34){\rule{2.409pt}{0.800pt}}
\multiput(506.00,235.34)(5.000,2.000){2}{\rule{1.204pt}{0.800pt}}
\put(516,238.84){\rule{2.168pt}{0.800pt}}
\multiput(516.00,237.34)(4.500,3.000){2}{\rule{1.084pt}{0.800pt}}
\put(525,242.34){\rule{2.200pt}{0.800pt}}
\multiput(525.00,240.34)(5.434,4.000){2}{\rule{1.100pt}{0.800pt}}
\put(535,245.84){\rule{2.409pt}{0.800pt}}
\multiput(535.00,244.34)(5.000,3.000){2}{\rule{1.204pt}{0.800pt}}
\multiput(545.00,250.38)(1.096,0.560){3}{\rule{1.640pt}{0.135pt}}
\multiput(545.00,247.34)(5.596,5.000){2}{\rule{0.820pt}{0.800pt}}
\multiput(554.00,255.38)(1.264,0.560){3}{\rule{1.800pt}{0.135pt}}
\multiput(554.00,252.34)(6.264,5.000){2}{\rule{0.900pt}{0.800pt}}
\multiput(564.00,260.38)(1.264,0.560){3}{\rule{1.800pt}{0.135pt}}
\multiput(564.00,257.34)(6.264,5.000){2}{\rule{0.900pt}{0.800pt}}
\multiput(574.00,265.39)(0.797,0.536){5}{\rule{1.400pt}{0.129pt}}
\multiput(574.00,262.34)(6.094,6.000){2}{\rule{0.700pt}{0.800pt}}
\multiput(583.00,271.40)(0.738,0.526){7}{\rule{1.343pt}{0.127pt}}
\multiput(583.00,268.34)(7.213,7.000){2}{\rule{0.671pt}{0.800pt}}
\multiput(593.00,278.40)(0.738,0.526){7}{\rule{1.343pt}{0.127pt}}
\multiput(593.00,275.34)(7.213,7.000){2}{\rule{0.671pt}{0.800pt}}
\multiput(603.00,285.40)(0.548,0.516){11}{\rule{1.089pt}{0.124pt}}
\multiput(603.00,282.34)(7.740,9.000){2}{\rule{0.544pt}{0.800pt}}
\multiput(613.00,294.40)(0.554,0.520){9}{\rule{1.100pt}{0.125pt}}
\multiput(613.00,291.34)(6.717,8.000){2}{\rule{0.550pt}{0.800pt}}
\multiput(622.00,302.40)(0.487,0.514){13}{\rule{1.000pt}{0.124pt}}
\multiput(622.00,299.34)(7.924,10.000){2}{\rule{0.500pt}{0.800pt}}
\multiput(632.00,312.40)(0.487,0.514){13}{\rule{1.000pt}{0.124pt}}
\multiput(632.00,309.34)(7.924,10.000){2}{\rule{0.500pt}{0.800pt}}
\multiput(643.40,321.00)(0.516,0.674){11}{\rule{0.124pt}{1.267pt}}
\multiput(640.34,321.00)(9.000,9.371){2}{\rule{0.800pt}{0.633pt}}
\multiput(652.40,333.00)(0.514,0.543){13}{\rule{0.124pt}{1.080pt}}
\multiput(649.34,333.00)(10.000,8.758){2}{\rule{0.800pt}{0.540pt}}
\multiput(662.40,344.00)(0.514,0.654){13}{\rule{0.124pt}{1.240pt}}
\multiput(659.34,344.00)(10.000,10.426){2}{\rule{0.800pt}{0.620pt}}
\multiput(672.40,357.00)(0.516,0.737){11}{\rule{0.124pt}{1.356pt}}
\multiput(669.34,357.00)(9.000,10.186){2}{\rule{0.800pt}{0.678pt}}
\multiput(681.40,370.00)(0.514,0.710){13}{\rule{0.124pt}{1.320pt}}
\multiput(678.34,370.00)(10.000,11.260){2}{\rule{0.800pt}{0.660pt}}
\multiput(691.40,384.00)(0.514,0.710){13}{\rule{0.124pt}{1.320pt}}
\multiput(688.34,384.00)(10.000,11.260){2}{\rule{0.800pt}{0.660pt}}
\multiput(701.40,398.00)(0.516,0.863){11}{\rule{0.124pt}{1.533pt}}
\multiput(698.34,398.00)(9.000,11.817){2}{\rule{0.800pt}{0.767pt}}
\multiput(710.40,413.00)(0.514,0.766){13}{\rule{0.124pt}{1.400pt}}
\multiput(707.34,413.00)(10.000,12.094){2}{\rule{0.800pt}{0.700pt}}
\multiput(720.40,428.00)(0.514,0.821){13}{\rule{0.124pt}{1.480pt}}
\multiput(717.34,428.00)(10.000,12.928){2}{\rule{0.800pt}{0.740pt}}
\multiput(730.40,444.00)(0.516,0.927){11}{\rule{0.124pt}{1.622pt}}
\multiput(727.34,444.00)(9.000,12.633){2}{\rule{0.800pt}{0.811pt}}
\multiput(739.40,460.00)(0.514,0.821){13}{\rule{0.124pt}{1.480pt}}
\multiput(736.34,460.00)(10.000,12.928){2}{\rule{0.800pt}{0.740pt}}
\multiput(749.40,476.00)(0.514,0.821){13}{\rule{0.124pt}{1.480pt}}
\multiput(746.34,476.00)(10.000,12.928){2}{\rule{0.800pt}{0.740pt}}
\multiput(759.40,492.00)(0.516,0.990){11}{\rule{0.124pt}{1.711pt}}
\multiput(756.34,492.00)(9.000,13.449){2}{\rule{0.800pt}{0.856pt}}
\multiput(768.40,509.00)(0.514,0.821){13}{\rule{0.124pt}{1.480pt}}
\multiput(765.34,509.00)(10.000,12.928){2}{\rule{0.800pt}{0.740pt}}
\multiput(778.40,525.00)(0.514,0.821){13}{\rule{0.124pt}{1.480pt}}
\multiput(775.34,525.00)(10.000,12.928){2}{\rule{0.800pt}{0.740pt}}
\multiput(788.40,541.00)(0.514,0.821){13}{\rule{0.124pt}{1.480pt}}
\multiput(785.34,541.00)(10.000,12.928){2}{\rule{0.800pt}{0.740pt}}
\multiput(798.40,557.00)(0.516,0.863){11}{\rule{0.124pt}{1.533pt}}
\multiput(795.34,557.00)(9.000,11.817){2}{\rule{0.800pt}{0.767pt}}
\multiput(807.40,572.00)(0.514,0.766){13}{\rule{0.124pt}{1.400pt}}
\multiput(804.34,572.00)(10.000,12.094){2}{\rule{0.800pt}{0.700pt}}
\multiput(817.40,587.00)(0.514,0.766){13}{\rule{0.124pt}{1.400pt}}
\multiput(814.34,587.00)(10.000,12.094){2}{\rule{0.800pt}{0.700pt}}
\multiput(827.40,602.00)(0.516,0.800){11}{\rule{0.124pt}{1.444pt}}
\multiput(824.34,602.00)(9.000,11.002){2}{\rule{0.800pt}{0.722pt}}
\multiput(836.40,616.00)(0.514,0.654){13}{\rule{0.124pt}{1.240pt}}
\multiput(833.34,616.00)(10.000,10.426){2}{\rule{0.800pt}{0.620pt}}
\multiput(846.40,629.00)(0.514,0.654){13}{\rule{0.124pt}{1.240pt}}
\multiput(843.34,629.00)(10.000,10.426){2}{\rule{0.800pt}{0.620pt}}
\multiput(856.40,642.00)(0.516,0.674){11}{\rule{0.124pt}{1.267pt}}
\multiput(853.34,642.00)(9.000,9.371){2}{\rule{0.800pt}{0.633pt}}
\multiput(865.40,654.00)(0.514,0.543){13}{\rule{0.124pt}{1.080pt}}
\multiput(862.34,654.00)(10.000,8.758){2}{\rule{0.800pt}{0.540pt}}
\multiput(875.40,665.00)(0.514,0.543){13}{\rule{0.124pt}{1.080pt}}
\multiput(872.34,665.00)(10.000,8.758){2}{\rule{0.800pt}{0.540pt}}
\multiput(885.40,676.00)(0.516,0.548){11}{\rule{0.124pt}{1.089pt}}
\multiput(882.34,676.00)(9.000,7.740){2}{\rule{0.800pt}{0.544pt}}
\multiput(893.00,687.40)(0.548,0.516){11}{\rule{1.089pt}{0.124pt}}
\multiput(893.00,684.34)(7.740,9.000){2}{\rule{0.544pt}{0.800pt}}
\multiput(903.00,696.40)(0.627,0.520){9}{\rule{1.200pt}{0.125pt}}
\multiput(903.00,693.34)(7.509,8.000){2}{\rule{0.600pt}{0.800pt}}
\multiput(913.00,704.40)(0.554,0.520){9}{\rule{1.100pt}{0.125pt}}
\multiput(913.00,701.34)(6.717,8.000){2}{\rule{0.550pt}{0.800pt}}
\multiput(922.00,712.40)(0.738,0.526){7}{\rule{1.343pt}{0.127pt}}
\multiput(922.00,709.34)(7.213,7.000){2}{\rule{0.671pt}{0.800pt}}
\multiput(932.00,719.39)(0.909,0.536){5}{\rule{1.533pt}{0.129pt}}
\multiput(932.00,716.34)(6.817,6.000){2}{\rule{0.767pt}{0.800pt}}
\multiput(942.00,725.39)(0.797,0.536){5}{\rule{1.400pt}{0.129pt}}
\multiput(942.00,722.34)(6.094,6.000){2}{\rule{0.700pt}{0.800pt}}
\multiput(951.00,731.38)(1.264,0.560){3}{\rule{1.800pt}{0.135pt}}
\multiput(951.00,728.34)(6.264,5.000){2}{\rule{0.900pt}{0.800pt}}
\put(961,735.34){\rule{2.200pt}{0.800pt}}
\multiput(961.00,733.34)(5.434,4.000){2}{\rule{1.100pt}{0.800pt}}
\put(971,739.34){\rule{2.000pt}{0.800pt}}
\multiput(971.00,737.34)(4.849,4.000){2}{\rule{1.000pt}{0.800pt}}
\put(980,743.34){\rule{2.200pt}{0.800pt}}
\multiput(980.00,741.34)(5.434,4.000){2}{\rule{1.100pt}{0.800pt}}
\put(990,746.84){\rule{2.409pt}{0.800pt}}
\multiput(990.00,745.34)(5.000,3.000){2}{\rule{1.204pt}{0.800pt}}
\put(1000,749.34){\rule{2.409pt}{0.800pt}}
\multiput(1000.00,748.34)(5.000,2.000){2}{\rule{1.204pt}{0.800pt}}
\put(1010,751.84){\rule{2.168pt}{0.800pt}}
\multiput(1010.00,750.34)(4.500,3.000){2}{\rule{1.084pt}{0.800pt}}
\put(1019,754.34){\rule{2.409pt}{0.800pt}}
\multiput(1019.00,753.34)(5.000,2.000){2}{\rule{1.204pt}{0.800pt}}
\put(1029,755.84){\rule{2.409pt}{0.800pt}}
\multiput(1029.00,755.34)(5.000,1.000){2}{\rule{1.204pt}{0.800pt}}
\put(1039,757.34){\rule{2.168pt}{0.800pt}}
\multiput(1039.00,756.34)(4.500,2.000){2}{\rule{1.084pt}{0.800pt}}
\put(1048,758.84){\rule{2.409pt}{0.800pt}}
\multiput(1048.00,758.34)(5.000,1.000){2}{\rule{1.204pt}{0.800pt}}
\put(1058,759.84){\rule{2.409pt}{0.800pt}}
\multiput(1058.00,759.34)(5.000,1.000){2}{\rule{1.204pt}{0.800pt}}
\put(1068,760.84){\rule{2.168pt}{0.800pt}}
\multiput(1068.00,760.34)(4.500,1.000){2}{\rule{1.084pt}{0.800pt}}
\put(1077,761.84){\rule{2.409pt}{0.800pt}}
\multiput(1077.00,761.34)(5.000,1.000){2}{\rule{1.204pt}{0.800pt}}
\put(419.0,226.0){\rule[-0.400pt]{2.409pt}{0.800pt}}
\put(1097,762.84){\rule{2.168pt}{0.800pt}}
\multiput(1097.00,762.34)(4.500,1.000){2}{\rule{1.084pt}{0.800pt}}
\put(1087.0,764.0){\rule[-0.400pt]{2.409pt}{0.800pt}}
\put(1126,763.84){\rule{2.168pt}{0.800pt}}
\multiput(1126.00,763.34)(4.500,1.000){2}{\rule{1.084pt}{0.800pt}}
\put(1106.0,765.0){\rule[-0.400pt]{4.818pt}{0.800pt}}
\put(1193,764.84){\rule{2.409pt}{0.800pt}}
\multiput(1193.00,764.34)(5.000,1.000){2}{\rule{1.204pt}{0.800pt}}
\put(1135.0,766.0){\rule[-0.400pt]{13.972pt}{0.800pt}}
\put(1203.0,767.0){\rule[-0.400pt]{18.790pt}{0.800pt}}
\sbox{\plotpoint}{\rule[-0.500pt]{1.000pt}{1.000pt}}%
\sbox{\plotpoint}{\rule[-0.200pt]{0.400pt}{0.400pt}}%
\put(1279,172){\makebox(0,0)[r]{Aproximácia dát pomocou $erf(x)$}}
\sbox{\plotpoint}{\rule[-0.500pt]{1.000pt}{1.000pt}}%
\multiput(1299,172)(20.756,0.000){5}{\usebox{\plotpoint}}
\put(1399,172){\usebox{\plotpoint}}
\put(322,223){\usebox{\plotpoint}}
\put(322.00,223.00){\usebox{\plotpoint}}
\put(342.76,223.00){\usebox{\plotpoint}}
\put(363.51,223.00){\usebox{\plotpoint}}
\put(384.27,223.00){\usebox{\plotpoint}}
\put(405.02,223.00){\usebox{\plotpoint}}
\put(425.73,224.00){\usebox{\plotpoint}}
\put(446.48,224.00){\usebox{\plotpoint}}
\put(467.24,224.00){\usebox{\plotpoint}}
\put(487.99,224.11){\usebox{\plotpoint}}
\put(508.69,225.00){\usebox{\plotpoint}}
\put(529.39,226.00){\usebox{\plotpoint}}
\put(550.07,227.56){\usebox{\plotpoint}}
\put(570.72,229.67){\usebox{\plotpoint}}
\put(591.08,233.62){\usebox{\plotpoint}}
\put(611.43,237.69){\usebox{\plotpoint}}
\put(631.26,243.78){\usebox{\plotpoint}}
\put(650.72,250.88){\usebox{\plotpoint}}
\put(669.29,260.14){\usebox{\plotpoint}}
\put(686.88,271.13){\usebox{\plotpoint}}
\put(703.67,283.26){\usebox{\plotpoint}}
\put(719.13,297.12){\usebox{\plotpoint}}
\put(734.00,311.56){\usebox{\plotpoint}}
\put(747.94,326.93){\usebox{\plotpoint}}
\put(761.48,342.64){\usebox{\plotpoint}}
\put(774.05,359.16){\usebox{\plotpoint}}
\put(786.25,375.95){\usebox{\plotpoint}}
\put(798.22,392.90){\usebox{\plotpoint}}
\put(809.53,410.30){\usebox{\plotpoint}}
\put(820.82,427.71){\usebox{\plotpoint}}
\put(831.38,445.57){\usebox{\plotpoint}}
\put(842.09,463.35){\usebox{\plotpoint}}
\put(853.09,480.95){\usebox{\plotpoint}}
\put(863.41,498.95){\usebox{\plotpoint}}
\put(874.35,516.59){\usebox{\plotpoint}}
\put(884.84,534.50){\usebox{\plotpoint}}
\put(895.28,552.43){\usebox{\plotpoint}}
\put(906.80,569.70){\usebox{\plotpoint}}
\put(917.93,587.21){\usebox{\plotpoint}}
\put(929.46,604.44){\usebox{\plotpoint}}
\put(941.52,621.33){\usebox{\plotpoint}}
\put(953.52,638.27){\usebox{\plotpoint}}
\put(966.70,654.27){\usebox{\plotpoint}}
\put(980.11,670.12){\usebox{\plotpoint}}
\put(994.49,685.04){\usebox{\plotpoint}}
\put(1009.92,698.93){\usebox{\plotpoint}}
\put(1026.05,711.94){\usebox{\plotpoint}}
\put(1043.12,723.75){\usebox{\plotpoint}}
\put(1061.32,733.66){\usebox{\plotpoint}}
\put(1080.19,742.28){\usebox{\plotpoint}}
\put(1099.83,748.94){\usebox{\plotpoint}}
\put(1119.73,754.75){\usebox{\plotpoint}}
\put(1140.12,758.51){\usebox{\plotpoint}}
\put(1160.66,761.26){\usebox{\plotpoint}}
\put(1181.29,763.00){\usebox{\plotpoint}}
\put(1201.99,764.00){\usebox{\plotpoint}}
\put(1222.69,765.00){\usebox{\plotpoint}}
\put(1243.39,766.00){\usebox{\plotpoint}}
\put(1264.15,766.00){\usebox{\plotpoint}}
\put(1281,766){\usebox{\plotpoint}}
\sbox{\plotpoint}{\rule[-0.200pt]{0.400pt}{0.400pt}}%
\put(171.0,131.0){\rule[-0.200pt]{0.400pt}{175.375pt}}
\put(171.0,131.0){\rule[-0.200pt]{305.461pt}{0.400pt}}
\put(1439.0,131.0){\rule[-0.200pt]{0.400pt}{175.375pt}}
\put(171.0,859.0){\rule[-0.200pt]{305.461pt}{0.400pt}}
\end{picture}

\caption{
Závislosť indexu lomu $n$ na tabuľkovej vlnovej dĺžke $\lambda$ preložená závislosťou $y = 1.620\pm0.001 + \frac{0.49\pm0.24}{\lambda-386.2\pm8.1}$.
}  \label{G_1}
\end{figure}

\subsection{Spektrum zinku}
Bohužiaľ zinková výboja bola v čase merania pokazená, z tohoto dôvodu nebola nameraná.

\subsection{Vodíkové spektrum}
Namerané hodnoty vodíkového spektra sú v tabuľke \ref{T_3}.
\begin{table}[h]
\begin{center}
\begin{tabular}{| c | c | c | c | c | c |}
\hline
farba & \popi{d_1}{\deg} & \popi{d_1}{\deg} & \popi{\epsilon_0}{\deg} & \popi{n}{-} & \popi{\lambda}{nm} \\
\hline
červená & $"278.727\pm0.001"$ & $"171.833\pm0.001"$ & $"53.447\pm0.002"$ & $"1.670\pm0.003"$ & $"376.484"$\\
modrá   & $"270.069\pm0.001"$ & $"170.258\pm0.001"$ & $"50.331\pm0.002"$ & $"1.640\pm0.003"$ & $"361.719"$\\
fialová & $"270.008\pm0.001"$ & $"169.344\pm0.001"$ & $"49.906\pm0.002"$ & $"1.635\pm0.003"$ & $"355.105"$\\
\hline
\end{tabular}
\caption{Namerané uhly $d_1$ a $d_2$ a vypočítaná hodnota $\epsilon_0$ 
podľa \ref{R_2} a index lomu $n$ podľa \ref{R_3} a vypočítaná hodnota 
vlnovej dĺžky $\lambda$ podľa vzťahu \ref{R_V_1}.
} \label{T_3}
\end{center}
\end{table}

\subsection{Výpočet Rydbergovy konštanty}
Pre jednotlivé farby bola postupne spočítaná Rydbergovu konštanta $R$
\eq[m]{
R_c &= "19.1 nm^{-1}" \,,\\
R_m &= "14.8 nm^{-1}" \,,\\
R_f &= "13.3 nm^{-1}" \,,\\
}
kde spodný index označuje farbu $c$ červenú, $m$ modrú a $f$ fialovú.


\subsection{Sodíkové spektrum}

Namerané hodnoty sodíkového spektra sú v tabuľke Tab. \ref{T_4}

\begin{table}[h]
\begin{center}
\begin{tabular}{| c | c | c | c | c | c |}
\hline
farba & \popi{d_1}{\deg} & \popi{d_1}{\deg} & \popi{\epsilon_0}{\deg} & \popi{n}{-} & \popi{\lambda}{nm} \\
\hline
červená & $"269.002\pm0.001"$ & $"171.391\pm0.001"$ & $"48.805\pm0.002"$ & $"1.627\pm0.003"$ & $"385.12"$\\

\hline
\end{tabular}
\caption{Namerané uhly $d_1$ a $d_2$ a vypočítaná hodnota $\epsilon_0$ 
podľa \ref{R_2} a index lomu $n$ podľa \ref{R_3} a vypočítaná hodnota 
vlnovej dĺžky $\lambda$ podľa vzťahu \ref{R_V_1}.
} \label{T_4}
\end{center}
\end{table}


