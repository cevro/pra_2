\section{Výsledky merania}
\subsection{Lámavého uhol}
Namerané hodnoty lámavého uhla sú v tabuľke Tab. \ref{T_1}.
Z nich bola pomocou vzťahu \ref{SCH_1} vypočítaná hodnota\eq{
\sigma="60.065\pm0.028 \deg"\,.
}

\begin{table}[h]
\begin{center}
\begin{tabular}{| c | c | c |}
\hline
\popi{d_1}{\deg} & \popi{d_1}{\deg} & \popi{d_1-d_2 = 2\sigma}{\deg} \\
\hline
$"281.894\pm0.001"$ & $"161.677\pm0.001"$ & $"120.217\pm0.001"$\\
$"281.866\pm0.001"$ & $"161.769\pm0.001"$ & $"120.097\pm0.001"$\\
$"281.869\pm0.001"$ & $"161.761\pm0.001"$ & $"120.108\pm0.001"$\\
$"281.888\pm0.001"$ & $"161.769\pm0.001"$ & $"120.119\pm0.001"$\\
$"281.869\pm0.001"$ & $"161.764\pm0.001"$ & $"120.106\pm0.001"$\\
\hline
\end{tabular}
\caption{
Namerané uhly $d_1$ a $d_2$ a vypočítaná hodnota $\sigma$ podľa \ref{R_1}
} \label{T_1}
\end{center}
\end{table}

\subsection{Ortuťové spektrum}
Namerané hodnoty ortuťového ortuťového spektra sú v tabulke Tab. \ref{T_2}.


\begin{table}[h]
\begin{center}
\begin{tabular}{| c | c | c | c | c | c |}
\hline
farba & \popi{d_1}{\deg} & \popi{d_1}{\deg} & \popi{\epsilon_0}{\deg} & \popi{n}{-} & \popi{\lambda}{nm} \\
\hline
oranžová     & $"268.942\pm0.001"$ & $"171.214\pm0.001"$ & $"48.864\pm0.002"$ & $"1.625\pm0.003"$ & $"579.065"$\\
žltá         & $"268.964\pm0.001"$ & $"171.192\pm0.001"$ & $"48.886\pm0.002"$ & $"1.626\pm0.003"$ & $"576.074"$\\
zelená       & $"269.242\pm0.001"$ & $"170.900\pm0.001"$ & $"49.171\pm0.002"$ & $"1.629\pm0.003"$ & $"546.074"$\\
azurová      & $"269.883\pm0.001"$ & $"170.244\pm0.001"$ & $"49.819\pm0.002"$ & $"1.636\pm0.003"$ & $"435.835"$\\
fialová      & $"270.858\pm0.001"$ & $"169.261\pm0.001"$ & $"50.799\pm0.002"$ & $"1.645\pm0.003"$ & $"407.781"$\\
ultrafialová & $"271.519\pm0.001"$ & $"168.439\pm0.001"$ & $"51.540\pm0.002"$ & $"1.652\pm0.003"$ & $"404.656"$\\
\hline
\end{tabular}
\caption{
Namerané uhly $d_1$ a $d_2$ a vypočítaná hodnota $\epsilon_0$ podľa \ref{R_2} a index lomu $n$ podľa \ref{R_3} a tabuľková vlnová dĺžka $\lambda$.
} \label{T_2}
\end{center}
\end{table}

Namerané hodnoty boli vynesené do grafu Obr. \ref{G_1} a z fitu dostávame vzťah pre závislosť vlnovej dĺžky na indexe lomu\eq{
n = 1.620\pm0.001 + \frac{0.49\pm0.24}{\lambda-386.2\pm8.1} \,. \lbl{R_V_1}
}

\begin{figure}
% GNUPLOT: LaTeX picture
\setlength{\unitlength}{0.240900pt}
\ifx\plotpoint\undefined\newsavebox{\plotpoint}\fi
\begin{picture}(1500,900)(0,0)
\sbox{\plotpoint}{\rule[-0.200pt]{0.400pt}{0.400pt}}%
\put(171.0,131.0){\rule[-0.200pt]{4.818pt}{0.400pt}}
\put(151,131){\makebox(0,0)[r]{ 0.5}}
\put(1419.0,131.0){\rule[-0.200pt]{4.818pt}{0.400pt}}
\put(171.0,235.0){\rule[-0.200pt]{4.818pt}{0.400pt}}
\put(151,235){\makebox(0,0)[r]{ 1}}
\put(1419.0,235.0){\rule[-0.200pt]{4.818pt}{0.400pt}}
\put(171.0,339.0){\rule[-0.200pt]{4.818pt}{0.400pt}}
\put(151,339){\makebox(0,0)[r]{ 1.5}}
\put(1419.0,339.0){\rule[-0.200pt]{4.818pt}{0.400pt}}
\put(171.0,443.0){\rule[-0.200pt]{4.818pt}{0.400pt}}
\put(151,443){\makebox(0,0)[r]{ 2}}
\put(1419.0,443.0){\rule[-0.200pt]{4.818pt}{0.400pt}}
\put(171.0,547.0){\rule[-0.200pt]{4.818pt}{0.400pt}}
\put(151,547){\makebox(0,0)[r]{ 2.5}}
\put(1419.0,547.0){\rule[-0.200pt]{4.818pt}{0.400pt}}
\put(171.0,651.0){\rule[-0.200pt]{4.818pt}{0.400pt}}
\put(151,651){\makebox(0,0)[r]{ 3}}
\put(1419.0,651.0){\rule[-0.200pt]{4.818pt}{0.400pt}}
\put(171.0,755.0){\rule[-0.200pt]{4.818pt}{0.400pt}}
\put(151,755){\makebox(0,0)[r]{ 3.5}}
\put(1419.0,755.0){\rule[-0.200pt]{4.818pt}{0.400pt}}
\put(171.0,859.0){\rule[-0.200pt]{4.818pt}{0.400pt}}
\put(151,859){\makebox(0,0)[r]{ 4}}
\put(1419.0,859.0){\rule[-0.200pt]{4.818pt}{0.400pt}}
\put(171.0,131.0){\rule[-0.200pt]{0.400pt}{4.818pt}}
\put(171,90){\makebox(0,0){ 0}}
\put(171.0,839.0){\rule[-0.200pt]{0.400pt}{4.818pt}}
\put(382.0,131.0){\rule[-0.200pt]{0.400pt}{4.818pt}}
\put(382,90){\makebox(0,0){ 2}}
\put(382.0,839.0){\rule[-0.200pt]{0.400pt}{4.818pt}}
\put(594.0,131.0){\rule[-0.200pt]{0.400pt}{4.818pt}}
\put(594,90){\makebox(0,0){ 4}}
\put(594.0,839.0){\rule[-0.200pt]{0.400pt}{4.818pt}}
\put(805.0,131.0){\rule[-0.200pt]{0.400pt}{4.818pt}}
\put(805,90){\makebox(0,0){ 6}}
\put(805.0,839.0){\rule[-0.200pt]{0.400pt}{4.818pt}}
\put(1016.0,131.0){\rule[-0.200pt]{0.400pt}{4.818pt}}
\put(1016,90){\makebox(0,0){ 8}}
\put(1016.0,839.0){\rule[-0.200pt]{0.400pt}{4.818pt}}
\put(1228.0,131.0){\rule[-0.200pt]{0.400pt}{4.818pt}}
\put(1228,90){\makebox(0,0){ 10}}
\put(1228.0,839.0){\rule[-0.200pt]{0.400pt}{4.818pt}}
\put(1439.0,131.0){\rule[-0.200pt]{0.400pt}{4.818pt}}
\put(1439,90){\makebox(0,0){ 12}}
\put(1439.0,839.0){\rule[-0.200pt]{0.400pt}{4.818pt}}
\put(171.0,131.0){\rule[-0.200pt]{0.400pt}{175.375pt}}
\put(171.0,131.0){\rule[-0.200pt]{305.461pt}{0.400pt}}
\put(1439.0,131.0){\rule[-0.200pt]{0.400pt}{175.375pt}}
\put(171.0,859.0){\rule[-0.200pt]{305.461pt}{0.400pt}}
\put(30,495){\makebox(0,0){\popi{I}{A}}}
\put(805,29){\makebox(0,0){\popi{U}{V}}}
\put(1279,172){\makebox(0,0)[r]{Namerané hodnoty}}
\put(1397,792){\makebox(0,0){$+$}}
\put(1192,772){\makebox(0,0){$+$}}
\put(1068,721){\makebox(0,0){$+$}}
\put(653,519){\makebox(0,0){$+$}}
\put(541,454){\makebox(0,0){$+$}}
\put(382,345){\makebox(0,0){$+$}}
\put(296,266){\makebox(0,0){$+$}}
\put(248,218){\makebox(0,0){$+$}}
\put(748,573){\makebox(0,0){$+$}}
\put(837,618){\makebox(0,0){$+$}}
\put(916,655){\makebox(0,0){$+$}}
\put(1047,713){\makebox(0,0){$+$}}
\put(1110,740){\makebox(0,0){$+$}}
\put(1349,172){\makebox(0,0){$+$}}
\put(171.0,131.0){\rule[-0.200pt]{0.400pt}{175.375pt}}
\put(171.0,131.0){\rule[-0.200pt]{305.461pt}{0.400pt}}
\put(1439.0,131.0){\rule[-0.200pt]{0.400pt}{175.375pt}}
\put(171.0,859.0){\rule[-0.200pt]{305.461pt}{0.400pt}}
\end{picture}

\caption{
Závislosť indexu lomu $n$ na tabuľkovej vlnovej dĺžke $\lambda$ preložená závislosťou $y = 1.620\pm0.001 + \frac{0.49\pm0.24}{\lambda-386.2\pm8.1}$.
}  \label{G_1}
\end{figure}

\subsection{Spektrum zinku}
Bohužiaľ zinková výboja bola v čase merania pokazená, z tohoto dôvodu nebola nameraná.

\subsection{Vodíkové spektrum}
Namerané hodnoty vodíkového spektra sú v tabuľke \ref{T_3}.
\begin{table}[h]
\begin{center}
\begin{tabular}{| c | c | c | c | c | c |}
\hline
farba & \popi{d_1}{\deg} & \popi{d_1}{\deg} & \popi{\epsilon_0}{\deg} & \popi{n}{-} & \popi{\lambda}{nm} \\
\hline
červená & $"278.727\pm0.001"$ & $"171.833\pm0.001"$ & $"53.447\pm0.002"$ & $"1.670\pm0.003"$ & $"376.484"$\\
modrá   & $"270.069\pm0.001"$ & $"170.258\pm0.001"$ & $"50.331\pm0.002"$ & $"1.640\pm0.003"$ & $"361.719"$\\
fialová & $"270.008\pm0.001"$ & $"169.344\pm0.001"$ & $"49.906\pm0.002"$ & $"1.635\pm0.003"$ & $"355.105"$\\
\hline
\end{tabular}
\caption{Namerané uhly $d_1$ a $d_2$ a vypočítaná hodnota $\epsilon_0$ 
podľa \ref{R_2} a index lomu $n$ podľa \ref{R_3} a vypočítaná hodnota 
vlnovej dĺžky $\lambda$ podľa vzťahu \ref{R_V_1}.
} \label{T_3}
\end{center}
\end{table}

\subsection{Výpočet Rydbergovy konštanty}
Pre jednotlivé farby bola postupne spočítaná Rydbergovu konštanta $R$
\eq[m]{
R_c &= "19.1 nm^{-1}" \,,\\
R_m &= "14.8 nm^{-1}" \,,\\
R_f &= "13.3 nm^{-1}" \,,\\
}
kde spodný index označuje farbu $c$ červenú, $m$ modrú a $f$ fialovú.


\subsection{Sodíkové spektrum}

Namerané hodnoty sodíkového spektra sú v tabuľke Tab. \ref{T_4}

\begin{table}[h]
\begin{center}
\begin{tabular}{| c | c | c | c | c | c |}
\hline
farba & \popi{d_1}{\deg} & \popi{d_1}{\deg} & \popi{\epsilon_0}{\deg} & \popi{n}{-} & \popi{\lambda}{nm} \\
\hline
červená & $"269.002\pm0.001"$ & $"171.391\pm0.001"$ & $"48.805\pm0.002"$ & $"1.627\pm0.003"$ & $"385.12"$\\

\hline
\end{tabular}
\caption{Namerané uhly $d_1$ a $d_2$ a vypočítaná hodnota $\epsilon_0$ 
podľa \ref{R_2} a index lomu $n$ podľa \ref{R_3} a vypočítaná hodnota 
vlnovej dĺžky $\lambda$ podľa vzťahu \ref{R_V_1}.
} \label{T_4}
\end{center}
\end{table}


