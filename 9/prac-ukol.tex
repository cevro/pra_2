\section{Pracovní úkol}
\begin{enumerate}
\item DU: Odvo ďte vzorec (1) pro Brewsterův úhel úplné polarizace. Vycházejte z Obr. 1 a ze zákona lomu světla na rozhraní dvou optických prostředí. Spočtěte
Brewsterův úhel pro rozhraní vzduch - skleněné zrcadlo. Při měření Brewsterova
úhlu se doporučuje mít připravenou tabulku v Excelu pro výpočet stupně polarizace.
\item Při polarizaci bílého světla odrazem na černé skleněné desce proměřte závislost stupně polarizace
na sklonu desky a určete optimální hodnotu Brewsterova úhlu. Výsledky zaneste do grafu
a porovnejte s vypočtenou hodnotou z domácího úkolu.
\item Cernou otočnou desku nahraďte polarizačním filtrem a proměřte závislost intenzity polarizo-vaného světla na úhlu otočení analyzátoru (Malusův zákon). Výsledek srovnejte s teoretickou
předpovědí, znázorněte graficky a výsledek diskutujte.
\item Na optické lavici prozkoumejte vliv čtyř celofánových dvojlomných filtrů, způsobujících interferenci.
Vyzkoušejte vliv otáčení analyzátoru vůči polarizátoru a vliv otáčení dvojlomného filtru
mezi zkříženými i rovnoběžnými polarizátory v bílém světle. Pozorováním zjistěte, které vlnové
délky (barvy) se interferencí zvýrazní. Výsledky pozorování popište.
\item Pomocí dvou polarizačních filtrů, fotočlánku a barevných filtrů změřte měrnou otáčivost
křemene s tloušťkou 1 mm pro 4 vlnové délky světla. Jakou závislost pozorujete mezi vlnovou
délkou světla a měrnou otáčivostí? Naměřené hodnoty porovnejte s tabulkovými. Jak se změní
výsledek když použijete křemenný vzorek s větší tloušťkou? Diskutujte naměřené výsledky.
\end{enumerate}

