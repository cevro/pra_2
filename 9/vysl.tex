\section{Výsledky merania}
\subsection{Určenie Brewsterového uhlu}
Pomocou vzťahu \ref{R_P_1}, \ref{R_P_2}, \ref{R_P_3} a \ref{R_P_4} z \ref{R_P} bola spočítaná polarizácia $P$ pre jednotlivé uhly dopadu doštičky, namarané a spočítané dáta sú v tabuľke \ref{T_1}

\begin{table}[h]
\begin{center}
\begin{tabular}{ |  c | c | c | c | c | c | }
\hline
\popi{\phi}{\dg} & \popi{U_1}{mV}& \popi{U_2}{mV} & \popi{U_3}{mV} & \popi{U_4}{mV} & \popi{|P|}{Cm^{-2}}\\
\hline
$"30\pm1"$ & $" 5.0\pm1.0"$ & $" 2.2\pm1.0"$ & $"15.0\pm1.0"$ & $" 4.0\pm1.0"$ & $"3.19\pm2.39"$\\
$"40\pm1"$ & $"10.3\pm1.0"$ & $" 1.0\pm1.0"$ & $"10.0\pm1.0"$ & $" 5.3\pm1.0"$ & $"1.13\pm0.95"$\\
$"50\pm1"$ & $"12.4\pm1.0"$ & $" 0.1\pm1.0"$ & $" 7.2\pm1.0"$ & $" 8.8\pm1.0"$ & $"1.02\pm0.94"$\\
$"60\pm1"$ & $"18.5\pm1.0"$ & $"-1.3\pm1.0"$ & $" 5.5\pm1.0"$ & $"15.4\pm1.0"$ & $"1.27\pm0.98"$\\
$"70\pm1"$ & $"28.0\pm1.0"$ & $" 4.0\pm1.0"$ & $" 6.0\pm1.0"$ & $"23.1\pm1.0"$ & $"1.00\pm0.94"$\\
\hline
\end{tabular}
\caption{Namerané hodnoty napätí $U_i$ v závislosti na uhle dopadu $\phi$ a vypočítaná hodnota polarizácie $|P|$.
} \label{T_1}
\end{center}
\end{table}

\subsection{Malusov zákon}
V grafe Obr. \ref{G_1} je vynesená závislosť uhlu analyzačného filtru $\phi$ na intenzite svetelného toku reprezentovaného napätím na dióde $U$, preložená závislosťou $f\(\phi\)=\(0.14\pm0.07\)\mathrm{cos}^2\(\phi\)+\(0.034\pm0.005\)$. Pri fitovaní funkciou $f\(\phi\)=A\mathrm{cos}^2\(B\phi+C\)+D$ boli zafixované parametre $B=1$ a $C=0$, ktoré vychádzajú definície polarizácie.

\begin{figure}
% GNUPLOT: LaTeX picture
\setlength{\unitlength}{0.240900pt}
\ifx\plotpoint\undefined\newsavebox{\plotpoint}\fi
\begin{picture}(1500,900)(0,0)
\sbox{\plotpoint}{\rule[-0.200pt]{0.400pt}{0.400pt}}%
\put(171.0,131.0){\rule[-0.200pt]{4.818pt}{0.400pt}}
\put(151,131){\makebox(0,0)[r]{ 50}}
\put(1419.0,131.0){\rule[-0.200pt]{4.818pt}{0.400pt}}
\put(171.0,252.0){\rule[-0.200pt]{4.818pt}{0.400pt}}
\put(151,252){\makebox(0,0)[r]{ 100}}
\put(1419.0,252.0){\rule[-0.200pt]{4.818pt}{0.400pt}}
\put(171.0,374.0){\rule[-0.200pt]{4.818pt}{0.400pt}}
\put(151,374){\makebox(0,0)[r]{ 150}}
\put(1419.0,374.0){\rule[-0.200pt]{4.818pt}{0.400pt}}
\put(171.0,495.0){\rule[-0.200pt]{4.818pt}{0.400pt}}
\put(151,495){\makebox(0,0)[r]{ 200}}
\put(1419.0,495.0){\rule[-0.200pt]{4.818pt}{0.400pt}}
\put(171.0,616.0){\rule[-0.200pt]{4.818pt}{0.400pt}}
\put(151,616){\makebox(0,0)[r]{ 250}}
\put(1419.0,616.0){\rule[-0.200pt]{4.818pt}{0.400pt}}
\put(171.0,738.0){\rule[-0.200pt]{4.818pt}{0.400pt}}
\put(151,738){\makebox(0,0)[r]{ 300}}
\put(1419.0,738.0){\rule[-0.200pt]{4.818pt}{0.400pt}}
\put(171.0,859.0){\rule[-0.200pt]{4.818pt}{0.400pt}}
\put(151,859){\makebox(0,0)[r]{ 350}}
\put(1419.0,859.0){\rule[-0.200pt]{4.818pt}{0.400pt}}
\put(171.0,131.0){\rule[-0.200pt]{0.400pt}{4.818pt}}
\put(171,90){\makebox(0,0){ 0.5}}
\put(171.0,839.0){\rule[-0.200pt]{0.400pt}{4.818pt}}
\put(382.0,131.0){\rule[-0.200pt]{0.400pt}{4.818pt}}
\put(382,90){\makebox(0,0){ 1}}
\put(382.0,839.0){\rule[-0.200pt]{0.400pt}{4.818pt}}
\put(594.0,131.0){\rule[-0.200pt]{0.400pt}{4.818pt}}
\put(594,90){\makebox(0,0){ 1.5}}
\put(594.0,839.0){\rule[-0.200pt]{0.400pt}{4.818pt}}
\put(805.0,131.0){\rule[-0.200pt]{0.400pt}{4.818pt}}
\put(805,90){\makebox(0,0){ 2}}
\put(805.0,839.0){\rule[-0.200pt]{0.400pt}{4.818pt}}
\put(1016.0,131.0){\rule[-0.200pt]{0.400pt}{4.818pt}}
\put(1016,90){\makebox(0,0){ 2.5}}
\put(1016.0,839.0){\rule[-0.200pt]{0.400pt}{4.818pt}}
\put(1228.0,131.0){\rule[-0.200pt]{0.400pt}{4.818pt}}
\put(1228,90){\makebox(0,0){ 3}}
\put(1228.0,839.0){\rule[-0.200pt]{0.400pt}{4.818pt}}
\put(1439.0,131.0){\rule[-0.200pt]{0.400pt}{4.818pt}}
\put(1439,90){\makebox(0,0){ 3.5}}
\put(1439.0,839.0){\rule[-0.200pt]{0.400pt}{4.818pt}}
\put(171.0,131.0){\rule[-0.200pt]{0.400pt}{175.375pt}}
\put(171.0,131.0){\rule[-0.200pt]{305.461pt}{0.400pt}}
\put(1439.0,131.0){\rule[-0.200pt]{0.400pt}{175.375pt}}
\put(171.0,859.0){\rule[-0.200pt]{305.461pt}{0.400pt}}
\put(30,495){\makebox(0,0){\popi{U}{V}}}
\put(805,29){\makebox(0,0){\popi{I^2}{A^2}}}
\put(1279,213){\makebox(0,0)[r]{namerané dáta}}
\put(1299.0,213.0){\rule[-0.200pt]{24.090pt}{0.400pt}}
\put(1299.0,203.0){\rule[-0.200pt]{0.400pt}{4.818pt}}
\put(1399.0,203.0){\rule[-0.200pt]{0.400pt}{4.818pt}}
\put(382.0,204.0){\rule[-0.200pt]{0.400pt}{23.367pt}}
\put(372.0,204.0){\rule[-0.200pt]{4.818pt}{0.400pt}}
\put(372.0,301.0){\rule[-0.200pt]{4.818pt}{0.400pt}}
\put(426.0,264.0){\rule[-0.200pt]{0.400pt}{23.608pt}}
\put(416.0,264.0){\rule[-0.200pt]{4.818pt}{0.400pt}}
\put(416.0,362.0){\rule[-0.200pt]{4.818pt}{0.400pt}}
\put(519.0,325.0){\rule[-0.200pt]{0.400pt}{23.367pt}}
\put(509.0,325.0){\rule[-0.200pt]{4.818pt}{0.400pt}}
\put(509.0,422.0){\rule[-0.200pt]{4.818pt}{0.400pt}}
\put(674.0,386.0){\rule[-0.200pt]{0.400pt}{23.367pt}}
\put(664.0,386.0){\rule[-0.200pt]{4.818pt}{0.400pt}}
\put(664.0,483.0){\rule[-0.200pt]{4.818pt}{0.400pt}}
\put(788.0,446.0){\rule[-0.200pt]{0.400pt}{23.608pt}}
\put(778.0,446.0){\rule[-0.200pt]{4.818pt}{0.400pt}}
\put(778.0,544.0){\rule[-0.200pt]{4.818pt}{0.400pt}}
\put(911.0,507.0){\rule[-0.200pt]{0.400pt}{23.367pt}}
\put(901.0,507.0){\rule[-0.200pt]{4.818pt}{0.400pt}}
\put(901.0,604.0){\rule[-0.200pt]{4.818pt}{0.400pt}}
\put(975.0,568.0){\rule[-0.200pt]{0.400pt}{23.367pt}}
\put(965.0,568.0){\rule[-0.200pt]{4.818pt}{0.400pt}}
\put(965.0,665.0){\rule[-0.200pt]{4.818pt}{0.400pt}}
\put(1110.0,628.0){\rule[-0.200pt]{0.400pt}{23.608pt}}
\put(1100.0,628.0){\rule[-0.200pt]{4.818pt}{0.400pt}}
\put(1100.0,726.0){\rule[-0.200pt]{4.818pt}{0.400pt}}
\put(1254.0,689.0){\rule[-0.200pt]{0.400pt}{23.367pt}}
\put(1244.0,689.0){\rule[-0.200pt]{4.818pt}{0.400pt}}
\put(1244.0,786.0){\rule[-0.200pt]{4.818pt}{0.400pt}}
\put(357.0,252.0){\rule[-0.200pt]{12.286pt}{0.400pt}}
\put(357.0,242.0){\rule[-0.200pt]{0.400pt}{4.818pt}}
\put(408.0,242.0){\rule[-0.200pt]{0.400pt}{4.818pt}}
\put(399.0,313.0){\rule[-0.200pt]{12.768pt}{0.400pt}}
\put(399.0,303.0){\rule[-0.200pt]{0.400pt}{4.818pt}}
\put(452.0,303.0){\rule[-0.200pt]{0.400pt}{4.818pt}}
\put(489.0,374.0){\rule[-0.200pt]{14.213pt}{0.400pt}}
\put(489.0,364.0){\rule[-0.200pt]{0.400pt}{4.818pt}}
\put(548.0,364.0){\rule[-0.200pt]{0.400pt}{4.818pt}}
\put(641.0,434.0){\rule[-0.200pt]{15.899pt}{0.400pt}}
\put(641.0,424.0){\rule[-0.200pt]{0.400pt}{4.818pt}}
\put(707.0,424.0){\rule[-0.200pt]{0.400pt}{4.818pt}}
\put(753.0,495.0){\rule[-0.200pt]{17.104pt}{0.400pt}}
\put(753.0,485.0){\rule[-0.200pt]{0.400pt}{4.818pt}}
\put(824.0,485.0){\rule[-0.200pt]{0.400pt}{4.818pt}}
\put(873.0,556.0){\rule[-0.200pt]{18.308pt}{0.400pt}}
\put(873.0,546.0){\rule[-0.200pt]{0.400pt}{4.818pt}}
\put(949.0,546.0){\rule[-0.200pt]{0.400pt}{4.818pt}}
\put(936.0,616.0){\rule[-0.200pt]{18.790pt}{0.400pt}}
\put(936.0,606.0){\rule[-0.200pt]{0.400pt}{4.818pt}}
\put(1014.0,606.0){\rule[-0.200pt]{0.400pt}{4.818pt}}
\put(1069.0,677.0){\rule[-0.200pt]{19.995pt}{0.400pt}}
\put(1069.0,667.0){\rule[-0.200pt]{0.400pt}{4.818pt}}
\put(1152.0,667.0){\rule[-0.200pt]{0.400pt}{4.818pt}}
\put(1210.0,738.0){\rule[-0.200pt]{21.199pt}{0.400pt}}
\put(1210.0,728.0){\rule[-0.200pt]{0.400pt}{4.818pt}}
\put(382,252){\makebox(0,0){$+$}}
\put(426,313){\makebox(0,0){$+$}}
\put(519,374){\makebox(0,0){$+$}}
\put(674,434){\makebox(0,0){$+$}}
\put(788,495){\makebox(0,0){$+$}}
\put(911,556){\makebox(0,0){$+$}}
\put(975,616){\makebox(0,0){$+$}}
\put(1110,677){\makebox(0,0){$+$}}
\put(1254,738){\makebox(0,0){$+$}}
\put(1349,213){\makebox(0,0){$+$}}
\put(1298.0,728.0){\rule[-0.200pt]{0.400pt}{4.818pt}}
\put(1279,172){\makebox(0,0)[r]{fit $y(x)="\(94.0\pm3.2\)x+17.1\pm6.7"$}}
\multiput(1299,172)(20.756,0.000){5}{\usebox{\plotpoint}}
\put(1399,172){\usebox{\plotpoint}}
\put(357,266){\usebox{\plotpoint}}
\put(357.00,266.00){\usebox{\plotpoint}}
\put(375.36,275.68){\usebox{\plotpoint}}
\put(393.74,285.30){\usebox{\plotpoint}}
\put(412.11,294.95){\usebox{\plotpoint}}
\put(430.06,305.37){\usebox{\plotpoint}}
\put(448.43,315.02){\usebox{\plotpoint}}
\put(466.80,324.67){\usebox{\plotpoint}}
\put(485.17,334.32){\usebox{\plotpoint}}
\put(503.12,344.73){\usebox{\plotpoint}}
\put(521.49,354.38){\usebox{\plotpoint}}
\put(539.86,364.03){\usebox{\plotpoint}}
\put(558.17,373.78){\usebox{\plotpoint}}
\put(576.14,384.08){\usebox{\plotpoint}}
\put(594.50,393.75){\usebox{\plotpoint}}
\put(612.86,403.43){\usebox{\plotpoint}}
\put(631.22,413.11){\usebox{\plotpoint}}
\put(649.10,423.55){\usebox{\plotpoint}}
\put(667.46,433.23){\usebox{\plotpoint}}
\put(685.82,442.91){\usebox{\plotpoint}}
\put(704.17,452.59){\usebox{\plotpoint}}
\put(722.06,463.03){\usebox{\plotpoint}}
\put(740.42,472.71){\usebox{\plotpoint}}
\put(758.77,482.39){\usebox{\plotpoint}}
\put(777.04,492.22){\usebox{\plotpoint}}
\put(795.05,502.53){\usebox{\plotpoint}}
\put(813.41,512.20){\usebox{\plotpoint}}
\put(831.77,521.87){\usebox{\plotpoint}}
\put(850.14,531.52){\usebox{\plotpoint}}
\put(868.25,541.63){\usebox{\plotpoint}}
\put(886.61,551.30){\usebox{\plotpoint}}
\put(904.96,560.98){\usebox{\plotpoint}}
\put(923.32,570.66){\usebox{\plotpoint}}
\put(941.21,581.10){\usebox{\plotpoint}}
\put(959.56,590.78){\usebox{\plotpoint}}
\put(977.92,600.46){\usebox{\plotpoint}}
\put(996.18,610.31){\usebox{\plotpoint}}
\put(1014.20,620.60){\usebox{\plotpoint}}
\put(1032.56,630.28){\usebox{\plotpoint}}
\put(1050.91,639.95){\usebox{\plotpoint}}
\put(1069.28,649.60){\usebox{\plotpoint}}
\put(1087.23,660.02){\usebox{\plotpoint}}
\put(1105.60,669.67){\usebox{\plotpoint}}
\put(1123.97,679.32){\usebox{\plotpoint}}
\put(1142.34,688.97){\usebox{\plotpoint}}
\put(1160.29,699.39){\usebox{\plotpoint}}
\put(1178.66,709.04){\usebox{\plotpoint}}
\put(1197.03,718.69){\usebox{\plotpoint}}
\put(1215.29,728.53){\usebox{\plotpoint}}
\put(1233.32,738.73){\usebox{\plotpoint}}
\put(1251.69,748.38){\usebox{\plotpoint}}
\put(1270.06,758.03){\usebox{\plotpoint}}
\put(1288.42,767.71){\usebox{\plotpoint}}
\put(1298,774){\usebox{\plotpoint}}
\put(171.0,131.0){\rule[-0.200pt]{0.400pt}{175.375pt}}
\put(171.0,131.0){\rule[-0.200pt]{305.461pt}{0.400pt}}
\put(1439.0,131.0){\rule[-0.200pt]{0.400pt}{175.375pt}}
\put(171.0,859.0){\rule[-0.200pt]{305.461pt}{0.400pt}}
\end{picture}

\caption{Namerané hodnoty intenzity reprezentovanej napätím $U$ na uhle otočenia analyzačného polarizačného filtra $\phi$, preložená závislosť $f\(\phi\)=\(0.14\pm0.07\)\mathrm{cos}^2\(\phi\)+\(0.034\pm0.005\)$}\label{G_1}
\end{figure}

\subsection{Optická aktivita kremeňa}

V grafoch Obr. \ref{G_2-0} až Obr. \ref{G_2-3} sú vynesená namerané hodnoty intenzity reprezentovanej napätím $U$ v okolí jej maxima a preložené funkciou $f\(\phi\)=A\mathrm{cos}^2\(B\phi+C\)+D$, kde boli parametre $B=1$ a $D=0$ fixované.
Z fitu boli uhly otočenia pre jednotlivé filtre vynesená v závislosti na vlnovej dĺžke svetla do grafu Obr. \ref{G_3} a preložené funkciu 
\eq{
\alpha\(\lambda\)=\frac{d \(5.64\pm1.10\)\cdot10^{6}}{\lambda^2} [\dg,\jd{mm},\jd{nm}]\,,\lbl{R_V_3}
}
kde $d$ je hruba kremeňa v našom pripade $d="1.7 mm"$.


\begin{figure}
% GNUPLOT: LaTeX picture
\setlength{\unitlength}{0.240900pt}
\ifx\plotpoint\undefined\newsavebox{\plotpoint}\fi
\begin{picture}(1500,900)(0,0)
\sbox{\plotpoint}{\rule[-0.200pt]{0.400pt}{0.400pt}}%
\put(131.0,131.0){\rule[-0.200pt]{4.818pt}{0.400pt}}
\put(111,131){\makebox(0,0)[r]{-2}}
\put(1419.0,131.0){\rule[-0.200pt]{4.818pt}{0.400pt}}
\put(131.0,222.0){\rule[-0.200pt]{4.818pt}{0.400pt}}
\put(111,222){\makebox(0,0)[r]{-1}}
\put(1419.0,222.0){\rule[-0.200pt]{4.818pt}{0.400pt}}
\put(131.0,313.0){\rule[-0.200pt]{4.818pt}{0.400pt}}
\put(111,313){\makebox(0,0)[r]{ 0}}
\put(1419.0,313.0){\rule[-0.200pt]{4.818pt}{0.400pt}}
\put(131.0,404.0){\rule[-0.200pt]{4.818pt}{0.400pt}}
\put(111,404){\makebox(0,0)[r]{ 1}}
\put(1419.0,404.0){\rule[-0.200pt]{4.818pt}{0.400pt}}
\put(131.0,495.0){\rule[-0.200pt]{4.818pt}{0.400pt}}
\put(111,495){\makebox(0,0)[r]{ 2}}
\put(1419.0,495.0){\rule[-0.200pt]{4.818pt}{0.400pt}}
\put(131.0,586.0){\rule[-0.200pt]{4.818pt}{0.400pt}}
\put(111,586){\makebox(0,0)[r]{ 3}}
\put(1419.0,586.0){\rule[-0.200pt]{4.818pt}{0.400pt}}
\put(131.0,677.0){\rule[-0.200pt]{4.818pt}{0.400pt}}
\put(111,677){\makebox(0,0)[r]{ 4}}
\put(1419.0,677.0){\rule[-0.200pt]{4.818pt}{0.400pt}}
\put(131.0,768.0){\rule[-0.200pt]{4.818pt}{0.400pt}}
\put(111,768){\makebox(0,0)[r]{ 5}}
\put(1419.0,768.0){\rule[-0.200pt]{4.818pt}{0.400pt}}
\put(131.0,859.0){\rule[-0.200pt]{4.818pt}{0.400pt}}
\put(111,859){\makebox(0,0)[r]{ 6}}
\put(1419.0,859.0){\rule[-0.200pt]{4.818pt}{0.400pt}}
\put(131.0,131.0){\rule[-0.200pt]{0.400pt}{4.818pt}}
\put(131,90){\makebox(0,0){ 0}}
\put(131.0,839.0){\rule[-0.200pt]{0.400pt}{4.818pt}}
\put(276.0,131.0){\rule[-0.200pt]{0.400pt}{4.818pt}}
\put(276,90){\makebox(0,0){ 10}}
\put(276.0,839.0){\rule[-0.200pt]{0.400pt}{4.818pt}}
\put(422.0,131.0){\rule[-0.200pt]{0.400pt}{4.818pt}}
\put(422,90){\makebox(0,0){ 20}}
\put(422.0,839.0){\rule[-0.200pt]{0.400pt}{4.818pt}}
\put(567.0,131.0){\rule[-0.200pt]{0.400pt}{4.818pt}}
\put(567,90){\makebox(0,0){ 30}}
\put(567.0,839.0){\rule[-0.200pt]{0.400pt}{4.818pt}}
\put(712.0,131.0){\rule[-0.200pt]{0.400pt}{4.818pt}}
\put(712,90){\makebox(0,0){ 40}}
\put(712.0,839.0){\rule[-0.200pt]{0.400pt}{4.818pt}}
\put(858.0,131.0){\rule[-0.200pt]{0.400pt}{4.818pt}}
\put(858,90){\makebox(0,0){ 50}}
\put(858.0,839.0){\rule[-0.200pt]{0.400pt}{4.818pt}}
\put(1003.0,131.0){\rule[-0.200pt]{0.400pt}{4.818pt}}
\put(1003,90){\makebox(0,0){ 60}}
\put(1003.0,839.0){\rule[-0.200pt]{0.400pt}{4.818pt}}
\put(1148.0,131.0){\rule[-0.200pt]{0.400pt}{4.818pt}}
\put(1148,90){\makebox(0,0){ 70}}
\put(1148.0,839.0){\rule[-0.200pt]{0.400pt}{4.818pt}}
\put(1294.0,131.0){\rule[-0.200pt]{0.400pt}{4.818pt}}
\put(1294,90){\makebox(0,0){ 80}}
\put(1294.0,839.0){\rule[-0.200pt]{0.400pt}{4.818pt}}
\put(1439.0,131.0){\rule[-0.200pt]{0.400pt}{4.818pt}}
\put(1439,90){\makebox(0,0){ 90}}
\put(1439.0,839.0){\rule[-0.200pt]{0.400pt}{4.818pt}}
\put(131.0,131.0){\rule[-0.200pt]{0.400pt}{175.375pt}}
\put(131.0,131.0){\rule[-0.200pt]{315.097pt}{0.400pt}}
\put(1439.0,131.0){\rule[-0.200pt]{0.400pt}{175.375pt}}
\put(131.0,859.0){\rule[-0.200pt]{315.097pt}{0.400pt}}
\put(30,495){\makebox(0,0){\popi{U}{V}}}
\put(785,29){\makebox(0,0){\popi{\phi}{\dg}}}
\put(931,213){\makebox(0,0)[r]{namerané hodnoty}}
\put(951.0,213.0){\rule[-0.200pt]{24.090pt}{0.400pt}}
\put(951.0,203.0){\rule[-0.200pt]{0.400pt}{4.818pt}}
\put(1051.0,203.0){\rule[-0.200pt]{0.400pt}{4.818pt}}
\put(1439.0,450.0){\rule[-0.200pt]{0.400pt}{43.844pt}}
\put(1429.0,450.0){\rule[-0.200pt]{2.409pt}{0.400pt}}
\put(1429.0,632.0){\rule[-0.200pt]{2.409pt}{0.400pt}}
\put(1366.0,286.0){\rule[-0.200pt]{0.400pt}{43.844pt}}
\put(1356.0,286.0){\rule[-0.200pt]{4.818pt}{0.400pt}}
\put(1356.0,468.0){\rule[-0.200pt]{4.818pt}{0.400pt}}
\put(1294.0,313.0){\rule[-0.200pt]{0.400pt}{43.844pt}}
\put(1284.0,313.0){\rule[-0.200pt]{4.818pt}{0.400pt}}
\put(1284.0,495.0){\rule[-0.200pt]{4.818pt}{0.400pt}}
\put(1221.0,204.0){\rule[-0.200pt]{0.400pt}{43.844pt}}
\put(1211.0,204.0){\rule[-0.200pt]{4.818pt}{0.400pt}}
\put(1211.0,386.0){\rule[-0.200pt]{4.818pt}{0.400pt}}
\put(1148.0,268.0){\rule[-0.200pt]{0.400pt}{43.844pt}}
\put(1138.0,268.0){\rule[-0.200pt]{4.818pt}{0.400pt}}
\put(1138.0,450.0){\rule[-0.200pt]{4.818pt}{0.400pt}}
\put(1076.0,359.0){\rule[-0.200pt]{0.400pt}{43.844pt}}
\put(1066.0,359.0){\rule[-0.200pt]{4.818pt}{0.400pt}}
\put(1066.0,541.0){\rule[-0.200pt]{4.818pt}{0.400pt}}
\put(1003.0,368.0){\rule[-0.200pt]{0.400pt}{43.844pt}}
\put(993.0,368.0){\rule[-0.200pt]{4.818pt}{0.400pt}}
\put(993.0,550.0){\rule[-0.200pt]{4.818pt}{0.400pt}}
\put(930.0,295.0){\rule[-0.200pt]{0.400pt}{43.844pt}}
\put(920.0,295.0){\rule[-0.200pt]{4.818pt}{0.400pt}}
\put(920.0,477.0){\rule[-0.200pt]{4.818pt}{0.400pt}}
\put(858.0,340.0){\rule[-0.200pt]{0.400pt}{43.844pt}}
\put(848.0,340.0){\rule[-0.200pt]{4.818pt}{0.400pt}}
\put(848.0,522.0){\rule[-0.200pt]{4.818pt}{0.400pt}}
\put(785.0,468.0){\rule[-0.200pt]{0.400pt}{43.844pt}}
\put(775.0,468.0){\rule[-0.200pt]{4.818pt}{0.400pt}}
\put(775.0,650.0){\rule[-0.200pt]{4.818pt}{0.400pt}}
\put(712.0,431.0){\rule[-0.200pt]{0.400pt}{43.844pt}}
\put(702.0,431.0){\rule[-0.200pt]{4.818pt}{0.400pt}}
\put(702.0,613.0){\rule[-0.200pt]{4.818pt}{0.400pt}}
\put(640.0,349.0){\rule[-0.200pt]{0.400pt}{43.844pt}}
\put(630.0,349.0){\rule[-0.200pt]{4.818pt}{0.400pt}}
\put(630.0,531.0){\rule[-0.200pt]{4.818pt}{0.400pt}}
\put(567.0,404.0){\rule[-0.200pt]{0.400pt}{43.844pt}}
\put(557.0,404.0){\rule[-0.200pt]{4.818pt}{0.400pt}}
\put(557.0,586.0){\rule[-0.200pt]{4.818pt}{0.400pt}}
\put(494.0,413.0){\rule[-0.200pt]{0.400pt}{43.844pt}}
\put(484.0,413.0){\rule[-0.200pt]{4.818pt}{0.400pt}}
\put(484.0,595.0){\rule[-0.200pt]{4.818pt}{0.400pt}}
\put(422.0,304.0){\rule[-0.200pt]{0.400pt}{43.844pt}}
\put(412.0,304.0){\rule[-0.200pt]{4.818pt}{0.400pt}}
\put(412.0,486.0){\rule[-0.200pt]{4.818pt}{0.400pt}}
\put(349.0,340.0){\rule[-0.200pt]{0.400pt}{43.844pt}}
\put(339.0,340.0){\rule[-0.200pt]{4.818pt}{0.400pt}}
\put(339.0,522.0){\rule[-0.200pt]{4.818pt}{0.400pt}}
\put(276.0,331.0){\rule[-0.200pt]{0.400pt}{43.844pt}}
\put(266.0,331.0){\rule[-0.200pt]{4.818pt}{0.400pt}}
\put(266.0,513.0){\rule[-0.200pt]{4.818pt}{0.400pt}}
\put(204.0,322.0){\rule[-0.200pt]{0.400pt}{43.844pt}}
\put(194.0,322.0){\rule[-0.200pt]{4.818pt}{0.400pt}}
\put(194.0,504.0){\rule[-0.200pt]{4.818pt}{0.400pt}}
\put(131.0,377.0){\rule[-0.200pt]{0.400pt}{43.844pt}}
\put(131.0,377.0){\rule[-0.200pt]{2.409pt}{0.400pt}}
\put(1439,541){\makebox(0,0){$+$}}
\put(1366,377){\makebox(0,0){$+$}}
\put(1294,404){\makebox(0,0){$+$}}
\put(1221,295){\makebox(0,0){$+$}}
\put(1148,359){\makebox(0,0){$+$}}
\put(1076,450){\makebox(0,0){$+$}}
\put(1003,459){\makebox(0,0){$+$}}
\put(930,386){\makebox(0,0){$+$}}
\put(858,431){\makebox(0,0){$+$}}
\put(785,559){\makebox(0,0){$+$}}
\put(712,522){\makebox(0,0){$+$}}
\put(640,440){\makebox(0,0){$+$}}
\put(567,495){\makebox(0,0){$+$}}
\put(494,504){\makebox(0,0){$+$}}
\put(422,395){\makebox(0,0){$+$}}
\put(349,431){\makebox(0,0){$+$}}
\put(276,422){\makebox(0,0){$+$}}
\put(204,413){\makebox(0,0){$+$}}
\put(131,468){\makebox(0,0){$+$}}
\put(1001,213){\makebox(0,0){$+$}}
\put(131.0,559.0){\rule[-0.200pt]{2.409pt}{0.400pt}}
\put(931,172){\makebox(0,0)[r]{$f(x)=1.90\pm0.24\mathrm{cos}^2\(x-\(33.01\pm6.27\)\)$}}
\multiput(951,172)(20.756,0.000){5}{\usebox{\plotpoint}}
\put(1051,172){\usebox{\plotpoint}}
\put(131,435){\usebox{\plotpoint}}
\put(131.00,435.00){\usebox{\plotpoint}}
\put(151.41,438.71){\usebox{\plotpoint}}
\put(171.86,442.13){\usebox{\plotpoint}}
\put(192.38,445.29){\usebox{\plotpoint}}
\put(212.71,449.42){\usebox{\plotpoint}}
\put(233.24,452.46){\usebox{\plotpoint}}
\put(253.76,455.58){\usebox{\plotpoint}}
\put(274.27,458.73){\usebox{\plotpoint}}
\put(294.81,461.74){\usebox{\plotpoint}}
\put(315.43,463.96){\usebox{\plotpoint}}
\put(335.95,467.07){\usebox{\plotpoint}}
\put(356.59,469.09){\usebox{\plotpoint}}
\put(377.18,471.63){\usebox{\plotpoint}}
\put(397.76,474.21){\usebox{\plotpoint}}
\put(418.45,475.75){\usebox{\plotpoint}}
\put(439.15,477.32){\usebox{\plotpoint}}
\put(459.84,478.91){\usebox{\plotpoint}}
\put(480.54,480.43){\usebox{\plotpoint}}
\put(501.24,482.00){\usebox{\plotpoint}}
\put(521.97,482.61){\usebox{\plotpoint}}
\put(542.71,483.13){\usebox{\plotpoint}}
\put(563.43,484.00){\usebox{\plotpoint}}
\put(584.17,484.32){\usebox{\plotpoint}}
\put(604.90,485.00){\usebox{\plotpoint}}
\put(625.64,484.57){\usebox{\plotpoint}}
\put(646.37,484.00){\usebox{\plotpoint}}
\put(667.13,484.00){\usebox{\plotpoint}}
\put(687.85,483.00){\usebox{\plotpoint}}
\put(708.57,482.26){\usebox{\plotpoint}}
\put(729.28,481.00){\usebox{\plotpoint}}
\put(750.01,480.15){\usebox{\plotpoint}}
\put(770.70,478.56){\usebox{\plotpoint}}
\put(791.40,477.04){\usebox{\plotpoint}}
\put(812.03,474.92){\usebox{\plotpoint}}
\put(832.68,472.87){\usebox{\plotpoint}}
\put(853.30,470.67){\usebox{\plotpoint}}
\put(873.94,468.55){\usebox{\plotpoint}}
\put(894.45,465.39){\usebox{\plotpoint}}
\put(915.09,463.37){\usebox{\plotpoint}}
\put(935.61,460.21){\usebox{\plotpoint}}
\put(956.12,457.06){\usebox{\plotpoint}}
\put(976.66,454.05){\usebox{\plotpoint}}
\put(997.17,450.90){\usebox{\plotpoint}}
\put(1017.66,447.62){\usebox{\plotpoint}}
\put(1038.03,443.71){\usebox{\plotpoint}}
\put(1058.52,440.42){\usebox{\plotpoint}}
\put(1078.88,436.48){\usebox{\plotpoint}}
\put(1099.35,433.07){\usebox{\plotpoint}}
\put(1119.76,429.35){\usebox{\plotpoint}}
\put(1140.08,425.22){\usebox{\plotpoint}}
\put(1160.46,421.33){\usebox{\plotpoint}}
\put(1180.77,417.11){\usebox{\plotpoint}}
\put(1201.10,412.98){\usebox{\plotpoint}}
\put(1221.44,408.94){\usebox{\plotpoint}}
\put(1241.77,404.82){\usebox{\plotpoint}}
\put(1262.11,400.75){\usebox{\plotpoint}}
\put(1282.41,396.48){\usebox{\plotpoint}}
\put(1302.80,392.65){\usebox{\plotpoint}}
\put(1323.08,388.29){\usebox{\plotpoint}}
\put(1343.47,384.50){\usebox{\plotpoint}}
\put(1363.80,380.42){\usebox{\plotpoint}}
\put(1384.16,376.43){\usebox{\plotpoint}}
\put(1404.58,372.80){\usebox{\plotpoint}}
\put(1425.01,369.15){\usebox{\plotpoint}}
\put(1439,366){\usebox{\plotpoint}}
\put(131.0,131.0){\rule[-0.200pt]{0.400pt}{175.375pt}}
\put(131.0,131.0){\rule[-0.200pt]{315.097pt}{0.400pt}}
\put(1439.0,131.0){\rule[-0.200pt]{0.400pt}{175.375pt}}
\put(131.0,859.0){\rule[-0.200pt]{315.097pt}{0.400pt}}
\end{picture}

\caption{Namerané hodnoty intezty reprentovanje napätim $U$ na úhle otočenia polarizačného filtra $\phi$ pre modrý filter}\label{G_2-0}
\end{figure}

\begin{figure}
% GNUPLOT: LaTeX picture
\setlength{\unitlength}{0.240900pt}
\ifx\plotpoint\undefined\newsavebox{\plotpoint}\fi
\begin{picture}(1500,900)(0,0)
\sbox{\plotpoint}{\rule[-0.200pt]{0.400pt}{0.400pt}}%
\put(131.0,131.0){\rule[-0.200pt]{4.818pt}{0.400pt}}
\put(111,131){\makebox(0,0)[r]{-1}}
\put(1419.0,131.0){\rule[-0.200pt]{4.818pt}{0.400pt}}
\put(131.0,235.0){\rule[-0.200pt]{4.818pt}{0.400pt}}
\put(111,235){\makebox(0,0)[r]{ 0}}
\put(1419.0,235.0){\rule[-0.200pt]{4.818pt}{0.400pt}}
\put(131.0,339.0){\rule[-0.200pt]{4.818pt}{0.400pt}}
\put(111,339){\makebox(0,0)[r]{ 1}}
\put(1419.0,339.0){\rule[-0.200pt]{4.818pt}{0.400pt}}
\put(131.0,443.0){\rule[-0.200pt]{4.818pt}{0.400pt}}
\put(111,443){\makebox(0,0)[r]{ 2}}
\put(1419.0,443.0){\rule[-0.200pt]{4.818pt}{0.400pt}}
\put(131.0,547.0){\rule[-0.200pt]{4.818pt}{0.400pt}}
\put(111,547){\makebox(0,0)[r]{ 3}}
\put(1419.0,547.0){\rule[-0.200pt]{4.818pt}{0.400pt}}
\put(131.0,651.0){\rule[-0.200pt]{4.818pt}{0.400pt}}
\put(111,651){\makebox(0,0)[r]{ 4}}
\put(1419.0,651.0){\rule[-0.200pt]{4.818pt}{0.400pt}}
\put(131.0,755.0){\rule[-0.200pt]{4.818pt}{0.400pt}}
\put(111,755){\makebox(0,0)[r]{ 5}}
\put(1419.0,755.0){\rule[-0.200pt]{4.818pt}{0.400pt}}
\put(131.0,859.0){\rule[-0.200pt]{4.818pt}{0.400pt}}
\put(111,859){\makebox(0,0)[r]{ 6}}
\put(1419.0,859.0){\rule[-0.200pt]{4.818pt}{0.400pt}}
\put(131.0,131.0){\rule[-0.200pt]{0.400pt}{4.818pt}}
\put(131,90){\makebox(0,0){ 0}}
\put(131.0,839.0){\rule[-0.200pt]{0.400pt}{4.818pt}}
\put(276.0,131.0){\rule[-0.200pt]{0.400pt}{4.818pt}}
\put(276,90){\makebox(0,0){ 10}}
\put(276.0,839.0){\rule[-0.200pt]{0.400pt}{4.818pt}}
\put(422.0,131.0){\rule[-0.200pt]{0.400pt}{4.818pt}}
\put(422,90){\makebox(0,0){ 20}}
\put(422.0,839.0){\rule[-0.200pt]{0.400pt}{4.818pt}}
\put(567.0,131.0){\rule[-0.200pt]{0.400pt}{4.818pt}}
\put(567,90){\makebox(0,0){ 30}}
\put(567.0,839.0){\rule[-0.200pt]{0.400pt}{4.818pt}}
\put(712.0,131.0){\rule[-0.200pt]{0.400pt}{4.818pt}}
\put(712,90){\makebox(0,0){ 40}}
\put(712.0,839.0){\rule[-0.200pt]{0.400pt}{4.818pt}}
\put(858.0,131.0){\rule[-0.200pt]{0.400pt}{4.818pt}}
\put(858,90){\makebox(0,0){ 50}}
\put(858.0,839.0){\rule[-0.200pt]{0.400pt}{4.818pt}}
\put(1003.0,131.0){\rule[-0.200pt]{0.400pt}{4.818pt}}
\put(1003,90){\makebox(0,0){ 60}}
\put(1003.0,839.0){\rule[-0.200pt]{0.400pt}{4.818pt}}
\put(1148.0,131.0){\rule[-0.200pt]{0.400pt}{4.818pt}}
\put(1148,90){\makebox(0,0){ 70}}
\put(1148.0,839.0){\rule[-0.200pt]{0.400pt}{4.818pt}}
\put(1294.0,131.0){\rule[-0.200pt]{0.400pt}{4.818pt}}
\put(1294,90){\makebox(0,0){ 80}}
\put(1294.0,839.0){\rule[-0.200pt]{0.400pt}{4.818pt}}
\put(1439.0,131.0){\rule[-0.200pt]{0.400pt}{4.818pt}}
\put(1439,90){\makebox(0,0){ 90}}
\put(1439.0,839.0){\rule[-0.200pt]{0.400pt}{4.818pt}}
\put(131.0,131.0){\rule[-0.200pt]{0.400pt}{175.375pt}}
\put(131.0,131.0){\rule[-0.200pt]{315.097pt}{0.400pt}}
\put(1439.0,131.0){\rule[-0.200pt]{0.400pt}{175.375pt}}
\put(131.0,859.0){\rule[-0.200pt]{315.097pt}{0.400pt}}
\put(30,495){\makebox(0,0){\popi{U}{V}}}
\put(785,29){\makebox(0,0){\popi{\phi}{\dg}}}
\put(1279,819){\makebox(0,0)[r]{namerané hodnoty}}
\put(1299.0,819.0){\rule[-0.200pt]{24.090pt}{0.400pt}}
\put(1299.0,809.0){\rule[-0.200pt]{0.400pt}{4.818pt}}
\put(1399.0,809.0){\rule[-0.200pt]{0.400pt}{4.818pt}}
\put(858.0,193.0){\rule[-0.200pt]{0.400pt}{50.107pt}}
\put(848.0,193.0){\rule[-0.200pt]{4.818pt}{0.400pt}}
\put(848.0,401.0){\rule[-0.200pt]{4.818pt}{0.400pt}}
\put(785.0,349.0){\rule[-0.200pt]{0.400pt}{50.107pt}}
\put(775.0,349.0){\rule[-0.200pt]{4.818pt}{0.400pt}}
\put(775.0,557.0){\rule[-0.200pt]{4.818pt}{0.400pt}}
\put(712.0,277.0){\rule[-0.200pt]{0.400pt}{50.107pt}}
\put(702.0,277.0){\rule[-0.200pt]{4.818pt}{0.400pt}}
\put(702.0,485.0){\rule[-0.200pt]{4.818pt}{0.400pt}}
\put(640.0,349.0){\rule[-0.200pt]{0.400pt}{50.107pt}}
\put(630.0,349.0){\rule[-0.200pt]{4.818pt}{0.400pt}}
\put(630.0,557.0){\rule[-0.200pt]{4.818pt}{0.400pt}}
\put(567.0,152.0){\rule[-0.200pt]{0.400pt}{50.107pt}}
\put(557.0,152.0){\rule[-0.200pt]{4.818pt}{0.400pt}}
\put(557.0,360.0){\rule[-0.200pt]{4.818pt}{0.400pt}}
\put(494.0,183.0){\rule[-0.200pt]{0.400pt}{50.107pt}}
\put(484.0,183.0){\rule[-0.200pt]{4.818pt}{0.400pt}}
\put(484.0,391.0){\rule[-0.200pt]{4.818pt}{0.400pt}}
\put(422.0,349.0){\rule[-0.200pt]{0.400pt}{50.107pt}}
\put(412.0,349.0){\rule[-0.200pt]{4.818pt}{0.400pt}}
\put(412.0,557.0){\rule[-0.200pt]{4.818pt}{0.400pt}}
\put(349.0,214.0){\rule[-0.200pt]{0.400pt}{50.107pt}}
\put(339.0,214.0){\rule[-0.200pt]{4.818pt}{0.400pt}}
\put(339.0,422.0){\rule[-0.200pt]{4.818pt}{0.400pt}}
\put(276.0,339.0){\rule[-0.200pt]{0.400pt}{50.107pt}}
\put(266.0,339.0){\rule[-0.200pt]{4.818pt}{0.400pt}}
\put(858,297){\makebox(0,0){$+$}}
\put(785,453){\makebox(0,0){$+$}}
\put(712,381){\makebox(0,0){$+$}}
\put(640,453){\makebox(0,0){$+$}}
\put(567,256){\makebox(0,0){$+$}}
\put(494,287){\makebox(0,0){$+$}}
\put(422,453){\makebox(0,0){$+$}}
\put(349,318){\makebox(0,0){$+$}}
\put(276,443){\makebox(0,0){$+$}}
\put(1349,819){\makebox(0,0){$+$}}
\put(266.0,547.0){\rule[-0.200pt]{4.818pt}{0.400pt}}
\put(1279,778){\makebox(0,0)[r]{$f(x)=1.38\pm0.32\mathrm{cos}^2\(x-\(24.21\pm27.87\)\)$}}
\multiput(1299,778)(20.756,0.000){5}{\usebox{\plotpoint}}
\put(1399,778){\usebox{\plotpoint}}
\put(131,356){\usebox{\plotpoint}}
\put(131.00,356.00){\usebox{\plotpoint}}
\put(151.63,358.17){\usebox{\plotpoint}}
\put(172.27,360.20){\usebox{\plotpoint}}
\put(192.86,362.68){\usebox{\plotpoint}}
\put(213.44,365.26){\usebox{\plotpoint}}
\put(234.14,366.80){\usebox{\plotpoint}}
\put(254.79,368.74){\usebox{\plotpoint}}
\put(275.42,370.96){\usebox{\plotpoint}}
\put(296.12,372.47){\usebox{\plotpoint}}
\put(316.85,373.07){\usebox{\plotpoint}}
\put(337.54,374.66){\usebox{\plotpoint}}
\put(358.25,376.00){\usebox{\plotpoint}}
\put(378.97,376.77){\usebox{\plotpoint}}
\put(399.71,377.36){\usebox{\plotpoint}}
\put(420.44,378.00){\usebox{\plotpoint}}
\put(441.18,378.48){\usebox{\plotpoint}}
\put(461.91,379.00){\usebox{\plotpoint}}
\put(482.67,379.00){\usebox{\plotpoint}}
\put(503.42,379.00){\usebox{\plotpoint}}
\put(524.15,378.22){\usebox{\plotpoint}}
\put(544.89,378.00){\usebox{\plotpoint}}
\put(565.62,377.11){\usebox{\plotpoint}}
\put(586.35,376.51){\usebox{\plotpoint}}
\put(607.08,375.99){\usebox{\plotpoint}}
\put(627.78,374.40){\usebox{\plotpoint}}
\put(648.47,372.81){\usebox{\plotpoint}}
\put(669.17,371.27){\usebox{\plotpoint}}
\put(689.87,369.70){\usebox{\plotpoint}}
\put(710.56,368.11){\usebox{\plotpoint}}
\put(731.26,366.60){\usebox{\plotpoint}}
\put(751.84,364.02){\usebox{\plotpoint}}
\put(772.54,362.42){\usebox{\plotpoint}}
\put(793.13,359.91){\usebox{\plotpoint}}
\put(813.75,357.65){\usebox{\plotpoint}}
\put(834.29,354.75){\usebox{\plotpoint}}
\put(854.91,352.44){\usebox{\plotpoint}}
\put(875.43,349.32){\usebox{\plotpoint}}
\put(896.04,347.07){\usebox{\plotpoint}}
\put(916.59,344.14){\usebox{\plotpoint}}
\put(937.10,340.98){\usebox{\plotpoint}}
\put(957.62,337.83){\usebox{\plotpoint}}
\put(978.15,334.82){\usebox{\plotpoint}}
\put(998.67,331.67){\usebox{\plotpoint}}
\put(1019.00,327.54){\usebox{\plotpoint}}
\put(1039.53,324.50){\usebox{\plotpoint}}
\put(1060.05,321.38){\usebox{\plotpoint}}
\put(1080.56,318.22){\usebox{\plotpoint}}
\put(1100.90,314.16){\usebox{\plotpoint}}
\put(1121.42,311.09){\usebox{\plotpoint}}
\put(1141.84,307.42){\usebox{\plotpoint}}
\put(1162.29,303.96){\usebox{\plotpoint}}
\put(1182.80,300.80){\usebox{\plotpoint}}
\put(1203.28,297.47){\usebox{\plotpoint}}
\put(1223.66,293.62){\usebox{\plotpoint}}
\put(1244.18,290.51){\usebox{\plotpoint}}
\put(1264.70,287.35){\usebox{\plotpoint}}
\put(1285.22,284.25){\usebox{\plotpoint}}
\put(1305.75,281.19){\usebox{\plotpoint}}
\put(1326.26,278.04){\usebox{\plotpoint}}
\put(1346.80,275.03){\usebox{\plotpoint}}
\put(1367.31,271.88){\usebox{\plotpoint}}
\put(1387.83,268.72){\usebox{\plotpoint}}
\put(1408.35,265.66){\usebox{\plotpoint}}
\put(1428.99,263.54){\usebox{\plotpoint}}
\put(1439,262){\usebox{\plotpoint}}
\put(131.0,131.0){\rule[-0.200pt]{0.400pt}{175.375pt}}
\put(131.0,131.0){\rule[-0.200pt]{315.097pt}{0.400pt}}
\put(1439.0,131.0){\rule[-0.200pt]{0.400pt}{175.375pt}}
\put(131.0,859.0){\rule[-0.200pt]{315.097pt}{0.400pt}}
\end{picture}

\caption{Namerané hodnoty intezty reprentovanje napätim $U$ na úhle otočenia polarizačného filtra $\phi$ pre zelený filter}\label{G_2-1}
\end{figure}

\begin{figure}
% GNUPLOT: LaTeX picture
\setlength{\unitlength}{0.240900pt}
\ifx\plotpoint\undefined\newsavebox{\plotpoint}\fi
\begin{picture}(1500,900)(0,0)
\sbox{\plotpoint}{\rule[-0.200pt]{0.400pt}{0.400pt}}%
\put(131.0,131.0){\rule[-0.200pt]{4.818pt}{0.400pt}}
\put(111,131){\makebox(0,0)[r]{ 0}}
\put(1419.0,131.0){\rule[-0.200pt]{4.818pt}{0.400pt}}
\put(131.0,252.0){\rule[-0.200pt]{4.818pt}{0.400pt}}
\put(111,252){\makebox(0,0)[r]{ 1}}
\put(1419.0,252.0){\rule[-0.200pt]{4.818pt}{0.400pt}}
\put(131.0,374.0){\rule[-0.200pt]{4.818pt}{0.400pt}}
\put(111,374){\makebox(0,0)[r]{ 2}}
\put(1419.0,374.0){\rule[-0.200pt]{4.818pt}{0.400pt}}
\put(131.0,495.0){\rule[-0.200pt]{4.818pt}{0.400pt}}
\put(111,495){\makebox(0,0)[r]{ 3}}
\put(1419.0,495.0){\rule[-0.200pt]{4.818pt}{0.400pt}}
\put(131.0,616.0){\rule[-0.200pt]{4.818pt}{0.400pt}}
\put(111,616){\makebox(0,0)[r]{ 4}}
\put(1419.0,616.0){\rule[-0.200pt]{4.818pt}{0.400pt}}
\put(131.0,738.0){\rule[-0.200pt]{4.818pt}{0.400pt}}
\put(111,738){\makebox(0,0)[r]{ 5}}
\put(1419.0,738.0){\rule[-0.200pt]{4.818pt}{0.400pt}}
\put(131.0,859.0){\rule[-0.200pt]{4.818pt}{0.400pt}}
\put(111,859){\makebox(0,0)[r]{ 6}}
\put(1419.0,859.0){\rule[-0.200pt]{4.818pt}{0.400pt}}
\put(131.0,131.0){\rule[-0.200pt]{0.400pt}{4.818pt}}
\put(131,90){\makebox(0,0){ 0}}
\put(131.0,839.0){\rule[-0.200pt]{0.400pt}{4.818pt}}
\put(276.0,131.0){\rule[-0.200pt]{0.400pt}{4.818pt}}
\put(276,90){\makebox(0,0){ 10}}
\put(276.0,839.0){\rule[-0.200pt]{0.400pt}{4.818pt}}
\put(422.0,131.0){\rule[-0.200pt]{0.400pt}{4.818pt}}
\put(422,90){\makebox(0,0){ 20}}
\put(422.0,839.0){\rule[-0.200pt]{0.400pt}{4.818pt}}
\put(567.0,131.0){\rule[-0.200pt]{0.400pt}{4.818pt}}
\put(567,90){\makebox(0,0){ 30}}
\put(567.0,839.0){\rule[-0.200pt]{0.400pt}{4.818pt}}
\put(712.0,131.0){\rule[-0.200pt]{0.400pt}{4.818pt}}
\put(712,90){\makebox(0,0){ 40}}
\put(712.0,839.0){\rule[-0.200pt]{0.400pt}{4.818pt}}
\put(858.0,131.0){\rule[-0.200pt]{0.400pt}{4.818pt}}
\put(858,90){\makebox(0,0){ 50}}
\put(858.0,839.0){\rule[-0.200pt]{0.400pt}{4.818pt}}
\put(1003.0,131.0){\rule[-0.200pt]{0.400pt}{4.818pt}}
\put(1003,90){\makebox(0,0){ 60}}
\put(1003.0,839.0){\rule[-0.200pt]{0.400pt}{4.818pt}}
\put(1148.0,131.0){\rule[-0.200pt]{0.400pt}{4.818pt}}
\put(1148,90){\makebox(0,0){ 70}}
\put(1148.0,839.0){\rule[-0.200pt]{0.400pt}{4.818pt}}
\put(1294.0,131.0){\rule[-0.200pt]{0.400pt}{4.818pt}}
\put(1294,90){\makebox(0,0){ 80}}
\put(1294.0,839.0){\rule[-0.200pt]{0.400pt}{4.818pt}}
\put(1439.0,131.0){\rule[-0.200pt]{0.400pt}{4.818pt}}
\put(1439,90){\makebox(0,0){ 90}}
\put(1439.0,839.0){\rule[-0.200pt]{0.400pt}{4.818pt}}
\put(131.0,131.0){\rule[-0.200pt]{0.400pt}{175.375pt}}
\put(131.0,131.0){\rule[-0.200pt]{315.097pt}{0.400pt}}
\put(1439.0,131.0){\rule[-0.200pt]{0.400pt}{175.375pt}}
\put(131.0,859.0){\rule[-0.200pt]{315.097pt}{0.400pt}}
\put(30,495){\makebox(0,0){\popi{U}{V}}}
\put(785,29){\makebox(0,0){\popi{\phi}{\dg}}}
\put(1279,819){\makebox(0,0)[r]{namerané hodnoty}}
\put(1299.0,819.0){\rule[-0.200pt]{24.090pt}{0.400pt}}
\put(1299.0,809.0){\rule[-0.200pt]{0.400pt}{4.818pt}}
\put(1399.0,809.0){\rule[-0.200pt]{0.400pt}{4.818pt}}
\put(858.0,240.0){\rule[-0.200pt]{0.400pt}{58.539pt}}
\put(848.0,240.0){\rule[-0.200pt]{4.818pt}{0.400pt}}
\put(848.0,483.0){\rule[-0.200pt]{4.818pt}{0.400pt}}
\put(785.0,277.0){\rule[-0.200pt]{0.400pt}{58.298pt}}
\put(775.0,277.0){\rule[-0.200pt]{4.818pt}{0.400pt}}
\put(775.0,519.0){\rule[-0.200pt]{4.818pt}{0.400pt}}
\put(712.0,325.0){\rule[-0.200pt]{0.400pt}{58.539pt}}
\put(702.0,325.0){\rule[-0.200pt]{4.818pt}{0.400pt}}
\put(702.0,568.0){\rule[-0.200pt]{4.818pt}{0.400pt}}
\put(640.0,264.0){\rule[-0.200pt]{0.400pt}{58.539pt}}
\put(630.0,264.0){\rule[-0.200pt]{4.818pt}{0.400pt}}
\put(630.0,507.0){\rule[-0.200pt]{4.818pt}{0.400pt}}
\put(567.0,337.0){\rule[-0.200pt]{0.400pt}{58.539pt}}
\put(557.0,337.0){\rule[-0.200pt]{4.818pt}{0.400pt}}
\put(557.0,580.0){\rule[-0.200pt]{4.818pt}{0.400pt}}
\put(494.0,167.0){\rule[-0.200pt]{0.400pt}{58.539pt}}
\put(484.0,167.0){\rule[-0.200pt]{4.818pt}{0.400pt}}
\put(484.0,410.0){\rule[-0.200pt]{4.818pt}{0.400pt}}
\put(422.0,204.0){\rule[-0.200pt]{0.400pt}{58.298pt}}
\put(412.0,204.0){\rule[-0.200pt]{4.818pt}{0.400pt}}
\put(412.0,446.0){\rule[-0.200pt]{4.818pt}{0.400pt}}
\put(349.0,192.0){\rule[-0.200pt]{0.400pt}{58.298pt}}
\put(339.0,192.0){\rule[-0.200pt]{4.818pt}{0.400pt}}
\put(339.0,434.0){\rule[-0.200pt]{4.818pt}{0.400pt}}
\put(276.0,216.0){\rule[-0.200pt]{0.400pt}{58.539pt}}
\put(266.0,216.0){\rule[-0.200pt]{4.818pt}{0.400pt}}
\put(858,362){\makebox(0,0){$+$}}
\put(785,398){\makebox(0,0){$+$}}
\put(712,446){\makebox(0,0){$+$}}
\put(640,386){\makebox(0,0){$+$}}
\put(567,459){\makebox(0,0){$+$}}
\put(494,289){\makebox(0,0){$+$}}
\put(422,325){\makebox(0,0){$+$}}
\put(349,313){\makebox(0,0){$+$}}
\put(276,337){\makebox(0,0){$+$}}
\put(1349,819){\makebox(0,0){$+$}}
\put(266.0,459.0){\rule[-0.200pt]{4.818pt}{0.400pt}}
\put(1279,778){\makebox(0,0)[r]{$f(x)=2.22\pm0.22\mathrm{cos}^2\(x-\(45.97\pm8.53\)\)$}}
\multiput(1299,778)(20.756,0.000){5}{\usebox{\plotpoint}}
\put(1399,778){\usebox{\plotpoint}}
\put(131,262){\usebox{\plotpoint}}
\put(131.00,262.00){\usebox{\plotpoint}}
\put(150.84,268.10){\usebox{\plotpoint}}
\put(170.47,274.81){\usebox{\plotpoint}}
\put(190.30,280.94){\usebox{\plotpoint}}
\put(210.14,287.04){\usebox{\plotpoint}}
\put(230.02,293.01){\usebox{\plotpoint}}
\put(249.60,299.85){\usebox{\plotpoint}}
\put(269.43,305.98){\usebox{\plotpoint}}
\put(289.34,311.81){\usebox{\plotpoint}}
\put(309.31,317.46){\usebox{\plotpoint}}
\put(329.27,323.08){\usebox{\plotpoint}}
\put(349.15,329.04){\usebox{\plotpoint}}
\put(369.03,335.01){\usebox{\plotpoint}}
\put(389.12,340.19){\usebox{\plotpoint}}
\put(409.23,345.26){\usebox{\plotpoint}}
\put(429.36,350.26){\usebox{\plotpoint}}
\put(449.47,355.34){\usebox{\plotpoint}}
\put(469.72,359.87){\usebox{\plotpoint}}
\put(489.97,364.45){\usebox{\plotpoint}}
\put(510.19,369.12){\usebox{\plotpoint}}
\put(530.61,372.77){\usebox{\plotpoint}}
\put(551.01,376.54){\usebox{\plotpoint}}
\put(571.34,380.67){\usebox{\plotpoint}}
\put(591.85,383.82){\usebox{\plotpoint}}
\put(612.39,386.83){\usebox{\plotpoint}}
\put(633.02,389.00){\usebox{\plotpoint}}
\put(653.60,391.58){\usebox{\plotpoint}}
\put(674.19,394.09){\usebox{\plotpoint}}
\put(694.89,395.68){\usebox{\plotpoint}}
\put(715.58,397.26){\usebox{\plotpoint}}
\put(736.28,398.79){\usebox{\plotpoint}}
\put(757.01,399.39){\usebox{\plotpoint}}
\put(777.75,400.00){\usebox{\plotpoint}}
\put(798.50,400.00){\usebox{\plotpoint}}
\put(819.26,400.00){\usebox{\plotpoint}}
\put(840.01,400.00){\usebox{\plotpoint}}
\put(860.72,398.79){\usebox{\plotpoint}}
\put(881.45,398.00){\usebox{\plotpoint}}
\put(902.15,396.63){\usebox{\plotpoint}}
\put(922.85,395.09){\usebox{\plotpoint}}
\put(943.43,392.51){\usebox{\plotpoint}}
\put(964.12,390.84){\usebox{\plotpoint}}
\put(984.65,387.82){\usebox{\plotpoint}}
\put(1005.17,384.67){\usebox{\plotpoint}}
\put(1025.68,381.51){\usebox{\plotpoint}}
\put(1046.22,378.51){\usebox{\plotpoint}}
\put(1066.58,374.56){\usebox{\plotpoint}}
\put(1086.99,370.85){\usebox{\plotpoint}}
\put(1107.25,366.37){\usebox{\plotpoint}}
\put(1127.48,361.73){\usebox{\plotpoint}}
\put(1147.71,357.07){\usebox{\plotpoint}}
\put(1167.98,352.62){\usebox{\plotpoint}}
\put(1188.20,347.94){\usebox{\plotpoint}}
\put(1208.18,342.34){\usebox{\plotpoint}}
\put(1228.21,336.94){\usebox{\plotpoint}}
\put(1248.18,331.34){\usebox{\plotpoint}}
\put(1268.13,325.65){\usebox{\plotpoint}}
\put(1288.02,319.71){\usebox{\plotpoint}}
\put(1307.91,313.79){\usebox{\plotpoint}}
\put(1327.98,308.55){\usebox{\plotpoint}}
\put(1347.61,301.81){\usebox{\plotpoint}}
\put(1367.44,295.71){\usebox{\plotpoint}}
\put(1387.28,289.61){\usebox{\plotpoint}}
\put(1407.17,283.67){\usebox{\plotpoint}}
\put(1427.02,277.61){\usebox{\plotpoint}}
\put(1439,273){\usebox{\plotpoint}}
\put(131.0,131.0){\rule[-0.200pt]{0.400pt}{175.375pt}}
\put(131.0,131.0){\rule[-0.200pt]{315.097pt}{0.400pt}}
\put(1439.0,131.0){\rule[-0.200pt]{0.400pt}{175.375pt}}
\put(131.0,859.0){\rule[-0.200pt]{315.097pt}{0.400pt}}
\end{picture}

\caption{Namerané hodnoty intezty reprentovanje napätim $U$ na úhle otočenia polarizačného filtra $\phi$ pre oranžový filter}\label{G_2-2}
\end{figure}

\begin{figure}
% GNUPLOT: LaTeX picture
\setlength{\unitlength}{0.240900pt}
\ifx\plotpoint\undefined\newsavebox{\plotpoint}\fi
\begin{picture}(1500,900)(0,0)
\sbox{\plotpoint}{\rule[-0.200pt]{0.400pt}{0.400pt}}%
\put(131.0,131.0){\rule[-0.200pt]{4.818pt}{0.400pt}}
\put(111,131){\makebox(0,0)[r]{-1}}
\put(1419.0,131.0){\rule[-0.200pt]{4.818pt}{0.400pt}}
\put(131.0,235.0){\rule[-0.200pt]{4.818pt}{0.400pt}}
\put(111,235){\makebox(0,0)[r]{ 0}}
\put(1419.0,235.0){\rule[-0.200pt]{4.818pt}{0.400pt}}
\put(131.0,339.0){\rule[-0.200pt]{4.818pt}{0.400pt}}
\put(111,339){\makebox(0,0)[r]{ 1}}
\put(1419.0,339.0){\rule[-0.200pt]{4.818pt}{0.400pt}}
\put(131.0,443.0){\rule[-0.200pt]{4.818pt}{0.400pt}}
\put(111,443){\makebox(0,0)[r]{ 2}}
\put(1419.0,443.0){\rule[-0.200pt]{4.818pt}{0.400pt}}
\put(131.0,547.0){\rule[-0.200pt]{4.818pt}{0.400pt}}
\put(111,547){\makebox(0,0)[r]{ 3}}
\put(1419.0,547.0){\rule[-0.200pt]{4.818pt}{0.400pt}}
\put(131.0,651.0){\rule[-0.200pt]{4.818pt}{0.400pt}}
\put(111,651){\makebox(0,0)[r]{ 4}}
\put(1419.0,651.0){\rule[-0.200pt]{4.818pt}{0.400pt}}
\put(131.0,755.0){\rule[-0.200pt]{4.818pt}{0.400pt}}
\put(111,755){\makebox(0,0)[r]{ 5}}
\put(1419.0,755.0){\rule[-0.200pt]{4.818pt}{0.400pt}}
\put(131.0,859.0){\rule[-0.200pt]{4.818pt}{0.400pt}}
\put(111,859){\makebox(0,0)[r]{ 6}}
\put(1419.0,859.0){\rule[-0.200pt]{4.818pt}{0.400pt}}
\put(131.0,131.0){\rule[-0.200pt]{0.400pt}{4.818pt}}
\put(131,90){\makebox(0,0){ 0}}
\put(131.0,839.0){\rule[-0.200pt]{0.400pt}{4.818pt}}
\put(276.0,131.0){\rule[-0.200pt]{0.400pt}{4.818pt}}
\put(276,90){\makebox(0,0){ 10}}
\put(276.0,839.0){\rule[-0.200pt]{0.400pt}{4.818pt}}
\put(422.0,131.0){\rule[-0.200pt]{0.400pt}{4.818pt}}
\put(422,90){\makebox(0,0){ 20}}
\put(422.0,839.0){\rule[-0.200pt]{0.400pt}{4.818pt}}
\put(567.0,131.0){\rule[-0.200pt]{0.400pt}{4.818pt}}
\put(567,90){\makebox(0,0){ 30}}
\put(567.0,839.0){\rule[-0.200pt]{0.400pt}{4.818pt}}
\put(712.0,131.0){\rule[-0.200pt]{0.400pt}{4.818pt}}
\put(712,90){\makebox(0,0){ 40}}
\put(712.0,839.0){\rule[-0.200pt]{0.400pt}{4.818pt}}
\put(858.0,131.0){\rule[-0.200pt]{0.400pt}{4.818pt}}
\put(858,90){\makebox(0,0){ 50}}
\put(858.0,839.0){\rule[-0.200pt]{0.400pt}{4.818pt}}
\put(1003.0,131.0){\rule[-0.200pt]{0.400pt}{4.818pt}}
\put(1003,90){\makebox(0,0){ 60}}
\put(1003.0,839.0){\rule[-0.200pt]{0.400pt}{4.818pt}}
\put(1148.0,131.0){\rule[-0.200pt]{0.400pt}{4.818pt}}
\put(1148,90){\makebox(0,0){ 70}}
\put(1148.0,839.0){\rule[-0.200pt]{0.400pt}{4.818pt}}
\put(1294.0,131.0){\rule[-0.200pt]{0.400pt}{4.818pt}}
\put(1294,90){\makebox(0,0){ 80}}
\put(1294.0,839.0){\rule[-0.200pt]{0.400pt}{4.818pt}}
\put(1439.0,131.0){\rule[-0.200pt]{0.400pt}{4.818pt}}
\put(1439,90){\makebox(0,0){ 90}}
\put(1439.0,839.0){\rule[-0.200pt]{0.400pt}{4.818pt}}
\put(131.0,131.0){\rule[-0.200pt]{0.400pt}{175.375pt}}
\put(131.0,131.0){\rule[-0.200pt]{315.097pt}{0.400pt}}
\put(1439.0,131.0){\rule[-0.200pt]{0.400pt}{175.375pt}}
\put(131.0,859.0){\rule[-0.200pt]{315.097pt}{0.400pt}}
\put(30,495){\makebox(0,0){\popi{U}{V}}}
\put(785,29){\makebox(0,0){\popi{\phi}{\dg}}}
\put(1279,819){\makebox(0,0)[r]{namerané hodnoty}}
\put(1299.0,819.0){\rule[-0.200pt]{24.090pt}{0.400pt}}
\put(1299.0,809.0){\rule[-0.200pt]{0.400pt}{4.818pt}}
\put(1399.0,809.0){\rule[-0.200pt]{0.400pt}{4.818pt}}
\put(1148.0,162.0){\rule[-0.200pt]{0.400pt}{50.107pt}}
\put(1138.0,162.0){\rule[-0.200pt]{4.818pt}{0.400pt}}
\put(1138.0,370.0){\rule[-0.200pt]{4.818pt}{0.400pt}}
\put(1076.0,287.0){\rule[-0.200pt]{0.400pt}{50.107pt}}
\put(1066.0,287.0){\rule[-0.200pt]{4.818pt}{0.400pt}}
\put(1066.0,495.0){\rule[-0.200pt]{4.818pt}{0.400pt}}
\put(1003.0,360.0){\rule[-0.200pt]{0.400pt}{50.107pt}}
\put(993.0,360.0){\rule[-0.200pt]{4.818pt}{0.400pt}}
\put(993.0,568.0){\rule[-0.200pt]{4.818pt}{0.400pt}}
\put(930.0,287.0){\rule[-0.200pt]{0.400pt}{50.107pt}}
\put(920.0,287.0){\rule[-0.200pt]{4.818pt}{0.400pt}}
\put(920.0,495.0){\rule[-0.200pt]{4.818pt}{0.400pt}}
\put(858.0,453.0){\rule[-0.200pt]{0.400pt}{50.107pt}}
\put(848.0,453.0){\rule[-0.200pt]{4.818pt}{0.400pt}}
\put(848.0,661.0){\rule[-0.200pt]{4.818pt}{0.400pt}}
\put(785.0,495.0){\rule[-0.200pt]{0.400pt}{50.107pt}}
\put(775.0,495.0){\rule[-0.200pt]{4.818pt}{0.400pt}}
\put(775.0,703.0){\rule[-0.200pt]{4.818pt}{0.400pt}}
\put(712.0,318.0){\rule[-0.200pt]{0.400pt}{50.107pt}}
\put(702.0,318.0){\rule[-0.200pt]{4.818pt}{0.400pt}}
\put(702.0,526.0){\rule[-0.200pt]{4.818pt}{0.400pt}}
\put(640.0,297.0){\rule[-0.200pt]{0.400pt}{50.107pt}}
\put(630.0,297.0){\rule[-0.200pt]{4.818pt}{0.400pt}}
\put(630.0,505.0){\rule[-0.200pt]{4.818pt}{0.400pt}}
\put(567.0,339.0){\rule[-0.200pt]{0.400pt}{50.107pt}}
\put(557.0,339.0){\rule[-0.200pt]{4.818pt}{0.400pt}}
\put(1148,266){\makebox(0,0){$+$}}
\put(1076,391){\makebox(0,0){$+$}}
\put(1003,464){\makebox(0,0){$+$}}
\put(930,391){\makebox(0,0){$+$}}
\put(858,557){\makebox(0,0){$+$}}
\put(785,599){\makebox(0,0){$+$}}
\put(712,422){\makebox(0,0){$+$}}
\put(640,401){\makebox(0,0){$+$}}
\put(567,443){\makebox(0,0){$+$}}
\put(1349,819){\makebox(0,0){$+$}}
\put(557.0,547.0){\rule[-0.200pt]{4.818pt}{0.400pt}}
\put(1279,778){\makebox(0,0)[r]{$f(x)=2.32\pm0.53\mathrm{cos}^2\(x-\(30.18\pm17.04\)\)$}}
\multiput(1299,778)(20.756,0.000){5}{\usebox{\plotpoint}}
\put(1399,778){\usebox{\plotpoint}}
\put(131,418){\usebox{\plotpoint}}
\put(131.00,418.00){\usebox{\plotpoint}}
\put(151.22,422.67){\usebox{\plotpoint}}
\put(171.50,427.11){\usebox{\plotpoint}}
\put(191.72,431.78){\usebox{\plotpoint}}
\put(211.94,436.45){\usebox{\plotpoint}}
\put(232.20,440.97){\usebox{\plotpoint}}
\put(252.63,444.61){\usebox{\plotpoint}}
\put(272.99,448.54){\usebox{\plotpoint}}
\put(293.35,452.52){\usebox{\plotpoint}}
\put(313.87,455.67){\usebox{\plotpoint}}
\put(334.38,458.83){\usebox{\plotpoint}}
\put(354.92,461.85){\usebox{\plotpoint}}
\put(375.43,464.99){\usebox{\plotpoint}}
\put(396.06,467.16){\usebox{\plotpoint}}
\put(416.65,469.62){\usebox{\plotpoint}}
\put(437.35,471.18){\usebox{\plotpoint}}
\put(458.04,472.77){\usebox{\plotpoint}}
\put(478.74,474.29){\usebox{\plotpoint}}
\put(499.44,475.88){\usebox{\plotpoint}}
\put(520.17,476.47){\usebox{\plotpoint}}
\put(540.91,477.00){\usebox{\plotpoint}}
\put(561.66,477.00){\usebox{\plotpoint}}
\put(582.42,477.00){\usebox{\plotpoint}}
\put(603.17,477.00){\usebox{\plotpoint}}
\put(623.89,476.00){\usebox{\plotpoint}}
\put(644.61,475.11){\usebox{\plotpoint}}
\put(665.31,473.55){\usebox{\plotpoint}}
\put(686.01,472.00){\usebox{\plotpoint}}
\put(706.70,470.41){\usebox{\plotpoint}}
\put(727.29,467.90){\usebox{\plotpoint}}
\put(747.91,465.63){\usebox{\plotpoint}}
\put(768.54,463.46){\usebox{\plotpoint}}
\put(789.07,460.42){\usebox{\plotpoint}}
\put(809.59,457.29){\usebox{\plotpoint}}
\put(830.10,454.14){\usebox{\plotpoint}}
\put(850.44,450.08){\usebox{\plotpoint}}
\put(870.78,446.05){\usebox{\plotpoint}}
\put(891.19,442.34){\usebox{\plotpoint}}
\put(911.47,437.93){\usebox{\plotpoint}}
\put(931.87,434.18){\usebox{\plotpoint}}
\put(952.09,429.52){\usebox{\plotpoint}}
\put(972.35,425.00){\usebox{\plotpoint}}
\put(992.34,419.46){\usebox{\plotpoint}}
\put(1012.56,414.79){\usebox{\plotpoint}}
\put(1032.55,409.24){\usebox{\plotpoint}}
\put(1052.81,404.74){\usebox{\plotpoint}}
\put(1072.70,398.86){\usebox{\plotpoint}}
\put(1092.75,393.52){\usebox{\plotpoint}}
\put(1112.78,388.13){\usebox{\plotpoint}}
\put(1132.80,382.68){\usebox{\plotpoint}}
\put(1152.66,376.67){\usebox{\plotpoint}}
\put(1172.76,371.52){\usebox{\plotpoint}}
\put(1192.64,365.57){\usebox{\plotpoint}}
\put(1212.48,359.47){\usebox{\plotpoint}}
\put(1232.63,354.57){\usebox{\plotpoint}}
\put(1252.47,348.47){\usebox{\plotpoint}}
\put(1272.31,342.37){\usebox{\plotpoint}}
\put(1292.43,337.34){\usebox{\plotpoint}}
\put(1312.30,331.37){\usebox{\plotpoint}}
\put(1332.37,326.14){\usebox{\plotpoint}}
\put(1352.31,320.37){\usebox{\plotpoint}}
\put(1372.38,315.14){\usebox{\plotpoint}}
\put(1392.35,309.53){\usebox{\plotpoint}}
\put(1412.62,305.08){\usebox{\plotpoint}}
\put(1432.60,299.48){\usebox{\plotpoint}}
\put(1439,298){\usebox{\plotpoint}}
\put(131.0,131.0){\rule[-0.200pt]{0.400pt}{175.375pt}}
\put(131.0,131.0){\rule[-0.200pt]{315.097pt}{0.400pt}}
\put(1439.0,131.0){\rule[-0.200pt]{0.400pt}{175.375pt}}
\put(131.0,859.0){\rule[-0.200pt]{315.097pt}{0.400pt}}
\end{picture}

\caption{Namerané hodnoty intezty reprentovanje napätim $U$ na úhle otočenia polarizačného filtra $\phi$ pre červený filter}\label{G_2-3}
\end{figure}

\begin{figure}
% GNUPLOT: LaTeX picture
\setlength{\unitlength}{0.240900pt}
\ifx\plotpoint\undefined\newsavebox{\plotpoint}\fi
\begin{picture}(1500,900)(0,0)
\sbox{\plotpoint}{\rule[-0.200pt]{0.400pt}{0.400pt}}%
\put(151,151){\makebox(0,0)[r]{ 0}}
\put(171.0,151.0){\rule[-0.200pt]{4.818pt}{0.400pt}}
\put(151,293){\makebox(0,0)[r]{ 0.5}}
\put(171.0,293.0){\rule[-0.200pt]{4.818pt}{0.400pt}}
\put(151,434){\makebox(0,0)[r]{ 1}}
\put(171.0,434.0){\rule[-0.200pt]{4.818pt}{0.400pt}}
\put(151,576){\makebox(0,0)[r]{ 1.5}}
\put(171.0,576.0){\rule[-0.200pt]{4.818pt}{0.400pt}}
\put(151,717){\makebox(0,0)[r]{ 2}}
\put(171.0,717.0){\rule[-0.200pt]{4.818pt}{0.400pt}}
\put(151,859){\makebox(0,0)[r]{ 2.5}}
\put(171.0,859.0){\rule[-0.200pt]{4.818pt}{0.400pt}}
\put(250,90){\makebox(0,0){ 1}}
\put(250.0,131.0){\rule[-0.200pt]{0.400pt}{4.818pt}}
\put(487,90){\makebox(0,0){ 2.8}}
\put(487.0,131.0){\rule[-0.200pt]{0.400pt}{4.818pt}}
\put(723,90){\makebox(0,0){ 4.6}}
\put(723.0,131.0){\rule[-0.200pt]{0.400pt}{4.818pt}}
\put(960,90){\makebox(0,0){ 6.4}}
\put(960.0,131.0){\rule[-0.200pt]{0.400pt}{4.818pt}}
\put(1196,90){\makebox(0,0){ 8.2}}
\put(1196.0,131.0){\rule[-0.200pt]{0.400pt}{4.818pt}}
\put(1432,90){\makebox(0,0){ 10}}
\put(1432.0,131.0){\rule[-0.200pt]{0.400pt}{4.818pt}}
\put(191.0,151.0){\rule[-0.200pt]{0.400pt}{170.557pt}}
\put(191.0,151.0){\rule[-0.200pt]{300.643pt}{0.400pt}}
\put(1439.0,151.0){\rule[-0.200pt]{0.400pt}{170.557pt}}
\put(191.0,859.0){\rule[-0.200pt]{300.643pt}{0.400pt}}
\put(30,505){\makebox(0,0){\popi{multiplicity}{-}}}
\put(815,29){\makebox(0,0){\popi{Q/e}{-}}}
\put(417,151){\rule{13.0086pt}{68.4156pt}}
\put(417.0,151.0){\rule[-0.200pt]{0.400pt}{68.175pt}}
\put(417.0,434.0){\rule[-0.200pt]{12.768pt}{0.400pt}}
\put(470.0,151.0){\rule[-0.200pt]{0.400pt}{68.175pt}}
\put(476,151){\rule{13.2495pt}{136.59pt}}
\put(417.0,151.0){\rule[-0.200pt]{12.768pt}{0.400pt}}
\put(476.0,151.0){\rule[-0.200pt]{0.400pt}{136.349pt}}
\put(476.0,717.0){\rule[-0.200pt]{13.009pt}{0.400pt}}
\put(530.0,151.0){\rule[-0.200pt]{0.400pt}{136.349pt}}
\put(536,151){\rule{13.0086pt}{136.59pt}}
\put(476.0,151.0){\rule[-0.200pt]{13.009pt}{0.400pt}}
\put(536.0,151.0){\rule[-0.200pt]{0.400pt}{136.349pt}}
\put(536.0,717.0){\rule[-0.200pt]{12.768pt}{0.400pt}}
\put(589.0,151.0){\rule[-0.200pt]{0.400pt}{136.349pt}}
\put(654,151){\rule{13.0086pt}{68.4156pt}}
\put(536.0,151.0){\rule[-0.200pt]{12.768pt}{0.400pt}}
\put(654.0,151.0){\rule[-0.200pt]{0.400pt}{68.175pt}}
\put(654.0,434.0){\rule[-0.200pt]{12.768pt}{0.400pt}}
\put(707.0,151.0){\rule[-0.200pt]{0.400pt}{68.175pt}}
\put(890,151){\rule{13.0086pt}{136.59pt}}
\put(654.0,151.0){\rule[-0.200pt]{12.768pt}{0.400pt}}
\put(890.0,151.0){\rule[-0.200pt]{0.400pt}{136.349pt}}
\put(890.0,717.0){\rule[-0.200pt]{12.768pt}{0.400pt}}
\put(943.0,151.0){\rule[-0.200pt]{0.400pt}{136.349pt}}
\put(1008,151){\rule{13.2495pt}{68.4156pt}}
\put(890.0,151.0){\rule[-0.200pt]{12.768pt}{0.400pt}}
\put(1008.0,151.0){\rule[-0.200pt]{0.400pt}{68.175pt}}
\put(1008.0,434.0){\rule[-0.200pt]{13.009pt}{0.400pt}}
\put(1062.0,151.0){\rule[-0.200pt]{0.400pt}{68.175pt}}
\put(1245,151){\rule{13.0086pt}{68.4156pt}}
\put(1008.0,151.0){\rule[-0.200pt]{13.009pt}{0.400pt}}
\put(1245.0,151.0){\rule[-0.200pt]{0.400pt}{68.175pt}}
\put(1245.0,434.0){\rule[-0.200pt]{12.768pt}{0.400pt}}
\put(1298.0,151.0){\rule[-0.200pt]{0.400pt}{68.175pt}}
\put(1245.0,151.0){\rule[-0.200pt]{12.768pt}{0.400pt}}
\put(191.0,151.0){\rule[-0.200pt]{0.400pt}{170.557pt}}
\put(191.0,151.0){\rule[-0.200pt]{300.643pt}{0.400pt}}
\put(1439.0,151.0){\rule[-0.200pt]{0.400pt}{170.557pt}}
\put(191.0,859.0){\rule[-0.200pt]{300.643pt}{0.400pt}}
\end{picture}

\caption{Namerané hodnoty úhlu stočenia osi polorizácie $\alpha$ na vlnovej dlžke svetla $\lambda$}\label{G_3}
\end{figure}

