\section{Teória}

\subsection{určenie polarizácie na základe súboru intenzít} \label{R_P}
Pre naše meranie boli namerané stredné hodnoty
\begin{enumerate}
\item $\mean{E_x^2}_r$ ako hodnota odpovedajúceho napätia pre natočený polarizačný filter o $"0 \dg"$. \label{R_P_1}
\item $\mean{E_y^2}_r$ ako hodnota odpovedajúceho napätia pre natočený polarizačný filter o $"90 \dg"$. \label{R_P_2}
\item $\mean{E_x E_y}_r + \frac{1}{2}\mean{E_y^2}_r + \frac{1}{2}\mean{E_x^2}_r$ ako hodnota odpovedajúceho napätia pre natočený polarizačný filter o $"45 \dg"$. \label{R_P_3}
\item $\mean{E_x\(\omega t - \pi/2\) E_y\(\omega t\)}_r + \frac{1}{2}\mean{E_y^2}_r + \frac{1}{2}\mean{E_x^2}_r$ ako hodnota odpovedajúceho napätia pre natočený polarizačný filter o $"45 \dg"$ so zaradením štorvlnnej dostičky. \label{R_P_4}
\end{enumerate}

\subsection{merná otáčavosť}
Pre otáčavosť platí vzťah z \cite{2} 
\eq{
\alpha = \frac{B}{\lambda^2}\,,
}
kde $\lambda$ je vlnová dĺžka a $A$ a $B$ sú merné otáčavosť.
V našom prípade musíme otáčavosť ešte vydeliť hrúbkou $d$ vzorku teda dostávame
\eq{
\alpha = \frac{A d}{\lambda^2}\,.
}

%%%%%%%%%%%%%%%%%%%%%%%%%%%%%%%%%%%%%%%
\subsubsection{Spracovanie chýb merania}

Označme $\mean{t}$ aritmetický priemer nameraných hodnôt $t_i$, a $\Delta t$ hodnotu $\mean{t}-t$, pričom 
\eq{
\mean{t} = \frac{1}{n}\sum_{i=1}^n t_i \,, \lbl{SCH_1}
}  
a chybu aritmetického priemeru 
\eq{
  \sigma_0=\sqrt{\frac{\sum_{i=1}^n \(t_i - \mean{t}\)^2}{n\(n-1\)}}\,, \lbl{SCH_2}
}
pričom $n$ je počet meraní.



