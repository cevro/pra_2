\section{Pracovní úkol}
\begin{enumerate}
\item DU: V přípravě odvoď ťe rovnici 5, načrtněte chod paprsků a zdůvodněte nutnost
podmínky e > 4f. Vysvětlete rozďıl mezi Galileovým a Keplerovým dalekohledem.
Zjistěte, co je konvenční zraková vzďalenost.
\item Určete ohniskovou vzďalenost spojné čočky +200 ze znalosti polohy předmětu a jeho obrazu
(pro minimálně pět konfigurací, provedťe téˇz graficky) a Besselovou metodou.
\item Změřte ohniskovou vzďalenost mikroskopického objektivu a Ramsdenova okuláru Besselovou
metodou. V přípravě vysvětlete rozďıl mezi Ramsdenovým a Huygensovým okulárem.
\item Změřte zvětšení lupy při akomodaci oka na konvenční zrakovou vzďalenost. Stanovte z ohniskové
vzďalenosti lupy zvětšení při oku akomodovaném na nekonečno.
\item Určete polohy ohniskových rovin tlustých čoček (mikroskopický objektiv a Ramsdenův okulár)
nutných pro výpočet zvětšení mikroskopu.
\item Z mikroskopického objektivu a Ramsdenova okuláru sestavte na optické lavici mikroskop a
změřte jeho zvětšení.
\item Ze spojky +200 a Ramsdenova okuláru sestavte na optické lavici dalekohled. Změřte jeho
zvětšení přímou metodou.
\item Výsledky měření zvětšení mikroskopu a dalekohledu porovnejte s hodnotami vypočítanými
z ohniskových vzďaleností.
\end{enumerate}

