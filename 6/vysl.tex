\section{Výsledky merania}
\subsection{Šošovka \uv{+200}}

Pre spojku \uv{+200} boli hodnoty zaznamenané v Tab. \ref{T_1}.
Pomocou programu \LaTeX\footnote{balíček metapost, ktorý je súčasťou distribúcie texlive-full.}
boli graficky určené polohy ohniskových vzdialeností $f_i$ pre jednotlivé merania 
\eq[m]{
\vect f_1 &= "[20.30\pm0.05,16.88\pm0.05] cm" \,,\\
\vect f_2 &= "[-,-]"\,,\\
\vect f_3 &= "[18.13\pm0.05,18.55\pm0.05] cm" \,,\\
\vect f_4 &="[17.70\pm0.05,18.92\pm0.05] cm" \,,
}
pričom pre druhé meranie sa nepodarilo určiť priesečník.
Zo zvyšných troch bodov bolo vypočítané ťažisko $\vect f="[18.71\pm0.9,18.12\pm0.8] cm" $ a pomocou vzťahu \ref{SCH_1} určená smerodajná odchylka, z $\vect f$ bola určená $f_g ="184.1\pm11.1 mm"$.

Z tabuľky pomocou vzťahu \ref{SCH_1} a \ref{SCH_2} sa určili mohutnosť šošovky $f\_B="181.76\pm4.87 mm"$.

\begin{table}[h]
\begin{center}
\begin{tabular}{ |  c | c | c | c | c | c | }
\hline
\popi{s_1}{cm}& \popi{s_2}{cm} & \popi{s_z}{cm}& \popi{s_o}{cm} & \popi{f\_B}{mm} & \popi{f_g}{mm} \\
\hline
$"50.20\pm0.05"$ & $"71.50\pm0.05"$ & $"100.0\pm0.1"$ & $"20.0\pm0.1"$ & $"185.82\pm0.1"$ & $"185.8\pm18.0"$\\ 
$"62.70\pm0.05"$ & $"65.50\pm0.05"$ & $"100.0\pm0.1"$ & $"30.0\pm0.1"$ & $"174.68\pm0.1"$ & $"-"$\\ 
$"35.50\pm0.05"$ & $"74.20\pm0.05"$ & $"100.0\pm0.1"$ & $"10.0\pm0.1"$ & $"183.34\pm0.1"$ & $"183.4\pm2.1"$\\ 
$"42.10\pm0.05"$ & $"73.70\pm0.05"$ & $"100.0\pm0.1"$ & $"15.0\pm0.1"$ & $"183.13\pm0.1"$ & $"183.1\pm5.1"$\\ 
\hline
\end{tabular}
\caption{Namerané hodnoty pre spojku \uv{+200}, pričom $s_1$ a $s_2$ sú polohy šošovky na dráhe, $s_o$ je poloha tienidla na dráhe, $s_z$ je poloha o zdroja (predmetu) na dráhe, $f\_B$ je vypočítané ohnisko Besselovou metodóu podľa vzťahu \ref{R_1} a $f_g$ je ohnisko učené grafickou metódou.} \label{T_1}
\end{center}
\end{table}

\subsection{Mikroskopický objektív}

Namerané hodnoty sú zaznamenané v tabuľke Tab. \ref{T_2} a z hodnôt $f\_B$ vypočítané pomocou vzťahu \ref{SCH_1} a \ref{SCH_2} vypočítaná ohnisková vzdialenosť $f = "92.15\pm2.63 mm"$.
Ďalej bola určená poloha ohniskových rovín, 
zdroj svetla bol umiestnený na optickej lavici $s_z="\(93.0\pm0.5\) cm"$, 
objektív $s_o="\(87.8\pm0.1\) cm"$, teda $x="\(52\pm6\) mm"$.
\begin{table}[h]
\begin{center}
\begin{tabular}{ |  c | c | c | c | c |  }
\hline
\popi{s_1}{cm}& \popi{s_2}{cm} & \popi{s_z}{cm}& \popi{s_o}{cm} & \popi{f\_B}{mm} \\
\hline
$"47.50\pm0.05"$ & $"88.70\pm0.05"$ & $"93.0\pm0.5"$ & $"30.0\pm0.1"$ & $"90.1\pm0.6"$ \\ 
$"37.30\pm0.05"$ & $"88.60\pm0.05"$ & $"93.0\pm0.5"$ & $"20.0\pm0.1"$ & $"92.4\pm0.6"$ \\ 
$"27.40\pm0.05"$ & $"88.80\pm0.05"$ & $"93.0\pm0.5"$ & $"10.0\pm0.1"$ & $"93.9\pm0.6"$ \\ 

\hline
\end{tabular}
\caption{Namerané hodnoty pre mikroskopický objektív, pričom $s_1$ a $s_2$ sú polohy šošovky na dráhe, $s_o$ je poloha tienidla na dráhe, $s_z$ je poloha o zdroja (predmetu) na dráhe, $f\_B$ je vypočítané ohnisko Besselovou metódou podľa vzťahu \ref{R_1}.} \label{T_2}
\end{center}
\end{table}

\subsection{Ramsdenový objektív}
Namerané hodnoty sú zaznamenané v tabuľke Tab. \ref{T_3} a z hodnôt $f\_B$ vypočítané pomocou vzťahu \ref{SCH_1} a \ref{SCH_2} vypočítaná ohnisková vzdialenosť $f = "93.27\pm0.73 mm"$.

Ďalej bola určená poloha ohniskových rovín, zdroj svetla bol umiestnený na optickej lavici $s_z="\(93.0\pm0.5\) cm"$, objektív $s_o="\(81.7\pm0.1\) cm"$, teda $x="\(113\pm6\) mm"$.

\begin{table}[h]
\begin{center}
\begin{tabular}{ |  c | c | c | c | c |  }
\hline
\popi{s_1}{cm}& \popi{s_2}{cm} & \popi{s_z}{cm}& \popi{s_o}{cm} & \popi{f\_B}{mm} \\
\hline
$"74.40\pm0.05"$ & $"92.80\pm0.05"$ & $"93.0\pm0.5"$ & $"50.0\pm0.1"$ & $"87.8\pm0.6"$ \\ 
$"64.30\pm0.05"$ & $"92.90\pm0.05"$ & $"93.0\pm0.5"$ & $"40.0\pm0.1"$ & $"93.9\pm0.6"$ \\ 
$"54.30\pm0.05"$ & $"93.00\pm0.05"$ & $"93.0\pm0.5"$ & $"30.0\pm0.1"$ & $"98.0\pm0.6"$ \\ 

\hline
\end{tabular}
\caption{Namerané hodnoty pre Ramsdenový objektív, pričom $s_1$ a $s_2$ sú polohy šošovky na dráhe, $s_o$ je poloha tienidla na dráhe, $s_z$ je poloha o zdroja (predmetu) na dráhe, $f\_B$ je vypočítané ohnisko Besselovou metódou podľa vzťahu \ref{R_1} } \label{T_3}
\end{center}
\end{table}


\subsection{Lupa}
Zväčšenie lupy bolo priamou metódou, pričom referenčná stupnica bola umiestnená do vzdialenosti $l="\(25\pm1\) cm"$ a presnosť určenia počtu dielikov bol $"1 dielik" = "10 \%"$ určené na $Z_l=10\pm1$. 


\subsection{Ďalekohľad}
Zväčšenie ďalekohľadu sme určili na základne porovnania zorných uhlov , pričom sme videli jeden dielik v ďalekohľade ako 20 dielikov \uv{vzoru} teda zväčšenie je $Z = 20\pm1$.

\begin{figure}
\includegraphics{data/graph_a.eps}
\caption{Grafické riešenie čočkovej rovnice pre spojku, kde malé body sú priesečniky $\vect f_i , i\in \{1,3,4\}$, šedý veľký krúžok je zistené ohnisko, a čierny je predpokladané ohnisko, čiarkovanou čiarou je druhé meranie, ktoré sa nepretína.}  \label{G_1}
\end{figure}

