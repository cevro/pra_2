\section{Teória}

Pre zobrazovanie tenkou šošovkov prlatí zobrazovacia rovnica
\eq{
\frac{1}{a}+\frac{1}{a^\prime}=\frac{1}{f}\,,
}
kde $a$ a $a^\prime$ sú vzalenosti obrazu a predmetu od šošovky a $f$ je ohnisková vzdialenosť.
Po jej úprave na tvar
\eq{
\frac{f}{a}+\frac{f}{a^\prime}=1\,,
}
vidíme, že rovnica pripomína úsekový tvar rovnice priamky, teda zvolíme body $[a_1,0]$ a $[0,a_1^\prime]$, a inú dvojcu $[a_2,0]$ a $[0,a_2^\prime]$ pretnú sa v bode $\vect F = [f,f]$.


\subsection{Besselova metóda}

Označme $d$ vzdialenosť oboch polôh šošovky od seba, a $e$ vzdialenosť obrazu od predmetu pre ohnisko $f$ platí
\eq{
f=\frac{e^2-d^2}{4e}\,,
}
v našom prípade $d= |s_1-s_2|$ a $d= |s_p-s_o|$ potom dostávame vzťah
\eq{
f=\frac{\(s_p-s_o\)^2-\(s_1-s_2\)^2}{4|s_p-s_o|}\,. \lbl{R_1}
}

%%%%%%%%%%%%%%%%%%%%%%%%%%%%%%%%%%%%%%%
\subsubsection{Spracovanie chýb merania}

Označme $\mean{t}$ aritmetický priemer nameraných hodnôt $t_i$, a $\Delta t$ hodnotu $\mean{t}-t$, pričom 
\eq{
\mean{t} = \frac{1}{n}\sum_{i=1}^n t_i \,, \lbl{SCH_1}
}  
a chybu aritmetického priemeru 
\eq{
  \sigma_0=\sqrt{\frac{\sum_{i=1}^n \(t_i - \mean{t}\)^2}{n\(n-1\)}}\,, \lbl{SCH_2}
}
pričom $n$ je počet meraní.



