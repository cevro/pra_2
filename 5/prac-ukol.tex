\section{Pracovní úkol}
\begin{enumerate}
\item DU: Z Planckova vyzařovacího zákona odvoďťe Stefan-Boltzmannův zákon a určete
tvar konstanty $\sigma$ pomocí $c$ , $k$ a $\hbar$.
\item Ocejchujte referenční žárovku pomocí měření odporu. Diskutujte, zda $\alpha$ v rovnici (9)\cite{C_1} je konstanta.
Výsledky zpracujte graficky. Ověřte správnost výsledků pomocí závislosti výkonu na čtvrté
mocnině teploty. Pomocí fitu určete konstantu $\beta$.
\item Ověřte Stefan-Boltzmanův zákon (7)\cite{C_1}, výsledky vyneste do grafu a určete konstatu $e$.
\item Zjistěte teplotu žárovky připojené k neznámému zdroji (alespoň 6 měření) pomocí závislosti
transmise na vlnové délce. Graficky zpracujte a teplotu získejte pomocí aritmetického průměru
z fitů závislosti intenzity na vlnové délce $I = I\(\lambda\)$.
\end{enumerate}

