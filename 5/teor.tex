\section{Teória}
Ohmov zákon ná vyjadruje závislosť napätia $U$ na prúde $I$ pomocou odporu $R$
\eq{
U = R I \,. \lbl{R_1}
}

Výkon resp. prikon $P$ spočítame ako súčin napätia $U$ a prúdu $I$ 
\eq{
P = U I \,. \lbl{R_2}
}
Podľa Stefan-Boltzmannovho zákona poznáme vzťah medzi teplotou $T$ a $I$,vyžarovaným výkonom,
\eq{
I = \varepsilon \sigma T\,, \lbl{R_3_X}
}
kde $\sigma = 5.76\cdot 10^{-8} W m^{-2}K^{-4}$ a $\varepsilon$  je emisivita povrchu telesa, túto možeme prespisať pre výkon $P$ a dostávame
\eq{
P=\beta T^4 \,,
}
kde $\beta$ je konštanta.

Pre meraní pomerov transmise využijeme vzťah
\eq{
I = \frac{2 h c }{\lambda^2\(\mathrm{exp}\(\frac{h c}{kT \lambda}\)-1\)}\frac{T_M}{T_R}\,, \lbl{R_4}
}
kde $T_M$ je merená a $T_R$ referenční transmisia, $\lambda$ je vlnová dlžka, $T$ je teplota, $h$ je Planckova konštanta, $k$ je Boltzmanova konštanta, $c$ je rýchlosť svetla.


%%%%%%%%%%%%%%%%%%%%%%%%%%%%%%%%%%%%%%%
\subsubsection{Spracovanie chýb merania}

Označme $\mean{t}$ aritmetický priemer nameraných hodnôt $t_i$, a $\Delta t$ hodnotu $\mean{t}-t$, pričom 
\eq{
\mean{t} = \frac{1}{n}\sum_{i=1}^n t_i \,, \lbl{SCH_1}
}  
a chybu aritmetického priemeru 
\eq{
  \sigma_0=\sqrt{\frac{\sum_{i=1}^n \(t_i - \mean{t}\)^2}{n\(n-1\)}}\,, \lbl{SCH_2}
}
pričom $n$ je počet meraní.



