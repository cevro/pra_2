\section{Výsledky merania}
\subsection{Vákuový výboj}
\subsubsection{Numerická integrácia}
Na začiatku bol spracovaný vákuový výboj(\#23719 podľa \cite{C_shots_DB}). 
Numericky boli vyintegrované hodnoty $\frac{\mathrm{d} B_t\(t\)}{\mathrm{d} t}$ 
a $\frac{\mathrm{d} I^\prime_{tot}\(t\)}{\mathrm{d} t}$, čím sme dostali $B_t\(t\)$ a $I^\prime_{tot}\(t\)$. 
Za integračný krok bola použité samplovacia frekvencia dát z osciloskopu $\Delta t = "1\cdot10^{-7} s"$.

\subsubsection{Kalibracia Rogowského pásku}
Pomocou porovnanie maxím $I^\prime_{tot_{max}}$ s referenčnými dátami $I_{tot_{max}}$\cite{C_shots_DB} , určil kalibračný koeficient Rogowského pásku. 
\eq{
C = \frac{I_{tot_{max}}}{I^\prime_{tot_{max}}} = "2.75\cdot10^{6} "\,. \lbl{R_CAL_C_1}\footnote{Zároveň $1/2 C$ z \cite{C_1}}
}
Touto kalibračnou konštantou boli preškálované všetky hodnoty $I_{tot}\(t\)$ pre všetky následujúce merania.
\subsubsection{Odpor komory}
Zo závislosti napätia $U_l\(t\)$ na prúde $I_{tot}\(t\)$, ktorý je pri vákuovom výboji rovný prúdu komorou, bol zistený odpor komory
\eq{
R = "\(14.308\pm0.001\) m\Omega"\,,
}
kalibračná krivka spolu so závislosťou $U_l\(t\)$ na $I_{tot}\(t\)$ je v grafe Obr. \ref{G_1}.

\subsection{Výboj s pracovným plynom}

Namerané pre všetkých 6 výbojov boli najskôr vyintegrované hodnoty $\frac{\mathrm{d}B_t}{ \mathrm{d}t}$ a $\frac{\mathrm{d} I_{tot}}{\mathrm{d} t}$ rovnako 
ako v predchádzajúcom prípade. 
Prúd $I_{tot}\(t\)$ bol zkalibrovaný pomocou kalibračnej konštanty zo vzťahu \ref{R_CAL_C_1}. 
\subsubsection{Prúd plazmou}
Pre výpočet prúdu plazmou sme využili vzťah (15) z \cite{C_1}
A našli sme si jeho maximálnu hodnotu.

\subsubsection{Teplota plazmy}
Podľa vzťahu (4) z \cite{C_1}
sme určili teplotu plazmy $T\(t\)$.

\subsubsection{Ohmický príkon}
Ohmický príkon sme určili z napätia $U_l\(t\)$ a prúdy plazmou $I_p\(t\)$ ako $P\_{OH}\(t\) = I_p\(t\) U_l\(t\)$.

\subsubsection{Energia plazmy}
Energiu plazmy $W_{th}\(t\)$ sme určili podľa vzťahu (7) z \cite{C_1}, 
kde $n$ sme určili pre každý výstrel zo vzťahu (5) z \cite{C_1}.
Pričom teplotu je potrebné previesť z $[\jd{eV}]$ na $[\jd{K}]$.


\subsubsection{Doba udržania}
Doba udržania $\tau_E$ bola určená v čase $t_0$, kde $W_{th}\(t\)$ nadobúda svoje maximum, teda
\eq{
\frac{\mathrm{d} W_{th}\(t\)}{\mathrm{d} t} = 0\,,
}
potom vzťah (9) z \cite{C_1}
prechádza na $P_{loss}\(t\)=P_{OH}\(t\)$, a teda 
\eq{
\tau_E = \frac{W_{th}\(t_0\)}{P_{OH}\(t_{0}\)}\,.\lbl{R_Tau_E}
}.

\subsubsection{Doba udržania}
Doba udržania bola vypočítaná pomocou dvoch thresholdov z $H_\alpha\(t\)$ radiácie.
Čas začiatku a konca života plazmy.


\begin{table}[h]
\begin{center}
\begin{tabular}{ |  c | c |  c |  c |  c |  c | c | }
\hline
Výstrel ID podľa \cite{C_shots_DB} & \popi{T_{max}}{eV} & \popi{I_{p_max}}{kA} & \popi{W_{max}}{10^{-3} J} & \popi{\tau_E}{\cdot 10^{-3}ms} & \popi{P_{oh_{max}}}{kW} & \popi{t}{ms} \\
\hline
\#23728&$"39.33\pm0.04"$&$"2.901\pm0.03"$&$" 93.69\pm0.09"$&$" 3.96\pm0.01"$&$"30.35\pm0.03"$&$"6.30"$\\
\#23729&$"40.44\pm0.04"$&$"3.453\pm0.03"$&$" 74.66\pm0.07"$&$" 2.30\pm0.01"$&$"44.29\pm0.04"$&$"5.65"$\\
\#23730&$"46.60\pm0.05"$&$"3.837\pm0.04"$&$" 93.99\pm0.09"$&$" 3.40\pm0.01"$&$"49.74\pm0.05"$&$"6.30"$\\
\#23731&$"36.47\pm0.04"$&$"2.325\pm0.02"$&$"144.11\pm0.14"$&$"10.51\pm0.02"$&$"20.63\pm0.02"$&$"6.31"$\\
\#23732&$"36.11\pm0.04"$&$"2.756\pm0.03"$&$"146.32\pm0.15"$&$"15.05\pm0.02"$&$"36.87\pm0.04"$&$"4.90"$\\
\#23733&$"31.44\pm0.03"$&$"2.478\pm0.02"$&$"104.13\pm0.10"$&$" 8.94\pm0.02"$&$"31.45\pm0.03"$&$"4.64"$\\
\hline
\end{tabular}
\caption{ Parametre jednotlivých výstrelov, kde $T_{max}$ je maximálne elektronová teplota plazmy, 
$I_{p_{max}}$ ja maximálny prúd plazmou, $W_{max}$ je maximálna energia plazmy, 
$\tau_E$ je čas udržania plazmy a $P_{OH_{max}}$ je maximálna hodnota ohmického odporu a $t$ doba života plazmy.
} \label{T_1}
\end{center}
\end{table}


\begin{figure}
% GNUPLOT: LaTeX picture
\setlength{\unitlength}{0.240900pt}
\ifx\plotpoint\undefined\newsavebox{\plotpoint}\fi
\begin{picture}(1500,900)(0,0)
\sbox{\plotpoint}{\rule[-0.200pt]{0.400pt}{0.400pt}}%
\put(151.0,131.0){\rule[-0.200pt]{4.818pt}{0.400pt}}
\put(131,131){\makebox(0,0)[r]{-2}}
\put(1419.0,131.0){\rule[-0.200pt]{4.818pt}{0.400pt}}
\put(151.0,235.0){\rule[-0.200pt]{4.818pt}{0.400pt}}
\put(131,235){\makebox(0,0)[r]{ 0}}
\put(1419.0,235.0){\rule[-0.200pt]{4.818pt}{0.400pt}}
\put(151.0,339.0){\rule[-0.200pt]{4.818pt}{0.400pt}}
\put(131,339){\makebox(0,0)[r]{ 2}}
\put(1419.0,339.0){\rule[-0.200pt]{4.818pt}{0.400pt}}
\put(151.0,443.0){\rule[-0.200pt]{4.818pt}{0.400pt}}
\put(131,443){\makebox(0,0)[r]{ 4}}
\put(1419.0,443.0){\rule[-0.200pt]{4.818pt}{0.400pt}}
\put(151.0,547.0){\rule[-0.200pt]{4.818pt}{0.400pt}}
\put(131,547){\makebox(0,0)[r]{ 6}}
\put(1419.0,547.0){\rule[-0.200pt]{4.818pt}{0.400pt}}
\put(151.0,651.0){\rule[-0.200pt]{4.818pt}{0.400pt}}
\put(131,651){\makebox(0,0)[r]{ 8}}
\put(1419.0,651.0){\rule[-0.200pt]{4.818pt}{0.400pt}}
\put(151.0,755.0){\rule[-0.200pt]{4.818pt}{0.400pt}}
\put(131,755){\makebox(0,0)[r]{ 10}}
\put(1419.0,755.0){\rule[-0.200pt]{4.818pt}{0.400pt}}
\put(151.0,859.0){\rule[-0.200pt]{4.818pt}{0.400pt}}
\put(131,859){\makebox(0,0)[r]{ 12}}
\put(1419.0,859.0){\rule[-0.200pt]{4.818pt}{0.400pt}}
\put(151.0,131.0){\rule[-0.200pt]{0.400pt}{4.818pt}}
\put(151,90){\makebox(0,0){-100}}
\put(151.0,839.0){\rule[-0.200pt]{0.400pt}{4.818pt}}
\put(294.0,131.0){\rule[-0.200pt]{0.400pt}{4.818pt}}
\put(294,90){\makebox(0,0){ 0}}
\put(294.0,839.0){\rule[-0.200pt]{0.400pt}{4.818pt}}
\put(437.0,131.0){\rule[-0.200pt]{0.400pt}{4.818pt}}
\put(437,90){\makebox(0,0){ 100}}
\put(437.0,839.0){\rule[-0.200pt]{0.400pt}{4.818pt}}
\put(580.0,131.0){\rule[-0.200pt]{0.400pt}{4.818pt}}
\put(580,90){\makebox(0,0){ 200}}
\put(580.0,839.0){\rule[-0.200pt]{0.400pt}{4.818pt}}
\put(723.0,131.0){\rule[-0.200pt]{0.400pt}{4.818pt}}
\put(723,90){\makebox(0,0){ 300}}
\put(723.0,839.0){\rule[-0.200pt]{0.400pt}{4.818pt}}
\put(867.0,131.0){\rule[-0.200pt]{0.400pt}{4.818pt}}
\put(867,90){\makebox(0,0){ 400}}
\put(867.0,839.0){\rule[-0.200pt]{0.400pt}{4.818pt}}
\put(1010.0,131.0){\rule[-0.200pt]{0.400pt}{4.818pt}}
\put(1010,90){\makebox(0,0){ 500}}
\put(1010.0,839.0){\rule[-0.200pt]{0.400pt}{4.818pt}}
\put(1153.0,131.0){\rule[-0.200pt]{0.400pt}{4.818pt}}
\put(1153,90){\makebox(0,0){ 600}}
\put(1153.0,839.0){\rule[-0.200pt]{0.400pt}{4.818pt}}
\put(1296.0,131.0){\rule[-0.200pt]{0.400pt}{4.818pt}}
\put(1296,90){\makebox(0,0){ 700}}
\put(1296.0,839.0){\rule[-0.200pt]{0.400pt}{4.818pt}}
\put(1439.0,131.0){\rule[-0.200pt]{0.400pt}{4.818pt}}
\put(1439,90){\makebox(0,0){ 800}}
\put(1439.0,839.0){\rule[-0.200pt]{0.400pt}{4.818pt}}
\put(151.0,131.0){\rule[-0.200pt]{0.400pt}{175.375pt}}
\put(151.0,131.0){\rule[-0.200pt]{310.279pt}{0.400pt}}
\put(1439.0,131.0){\rule[-0.200pt]{0.400pt}{175.375pt}}
\put(151.0,859.0){\rule[-0.200pt]{310.279pt}{0.400pt}}
\put(30,495){\makebox(0,0){\popi{U}{V}}}
\put(795,29){\makebox(0,0){\popi{I}{A}}}
\put(711,819){\makebox(0,0)[r]{fit $U=\(14.308\pm0.015\)10^{-3}I$}}
\put(731.0,819.0){\rule[-0.200pt]{24.090pt}{0.400pt}}
\put(252,213){\usebox{\plotpoint}}
\multiput(252.00,213.59)(1.033,0.482){9}{\rule{0.900pt}{0.116pt}}
\multiput(252.00,212.17)(10.132,6.000){2}{\rule{0.450pt}{0.400pt}}
\multiput(264.00,219.59)(0.943,0.482){9}{\rule{0.833pt}{0.116pt}}
\multiput(264.00,218.17)(9.270,6.000){2}{\rule{0.417pt}{0.400pt}}
\multiput(275.00,225.59)(1.033,0.482){9}{\rule{0.900pt}{0.116pt}}
\multiput(275.00,224.17)(10.132,6.000){2}{\rule{0.450pt}{0.400pt}}
\multiput(287.00,231.59)(1.033,0.482){9}{\rule{0.900pt}{0.116pt}}
\multiput(287.00,230.17)(10.132,6.000){2}{\rule{0.450pt}{0.400pt}}
\multiput(299.00,237.59)(0.943,0.482){9}{\rule{0.833pt}{0.116pt}}
\multiput(299.00,236.17)(9.270,6.000){2}{\rule{0.417pt}{0.400pt}}
\multiput(310.00,243.59)(0.874,0.485){11}{\rule{0.786pt}{0.117pt}}
\multiput(310.00,242.17)(10.369,7.000){2}{\rule{0.393pt}{0.400pt}}
\multiput(322.00,250.59)(1.033,0.482){9}{\rule{0.900pt}{0.116pt}}
\multiput(322.00,249.17)(10.132,6.000){2}{\rule{0.450pt}{0.400pt}}
\multiput(334.00,256.59)(0.943,0.482){9}{\rule{0.833pt}{0.116pt}}
\multiput(334.00,255.17)(9.270,6.000){2}{\rule{0.417pt}{0.400pt}}
\multiput(345.00,262.59)(1.033,0.482){9}{\rule{0.900pt}{0.116pt}}
\multiput(345.00,261.17)(10.132,6.000){2}{\rule{0.450pt}{0.400pt}}
\multiput(357.00,268.59)(1.033,0.482){9}{\rule{0.900pt}{0.116pt}}
\multiput(357.00,267.17)(10.132,6.000){2}{\rule{0.450pt}{0.400pt}}
\multiput(369.00,274.59)(0.943,0.482){9}{\rule{0.833pt}{0.116pt}}
\multiput(369.00,273.17)(9.270,6.000){2}{\rule{0.417pt}{0.400pt}}
\multiput(380.00,280.59)(1.033,0.482){9}{\rule{0.900pt}{0.116pt}}
\multiput(380.00,279.17)(10.132,6.000){2}{\rule{0.450pt}{0.400pt}}
\multiput(392.00,286.59)(1.033,0.482){9}{\rule{0.900pt}{0.116pt}}
\multiput(392.00,285.17)(10.132,6.000){2}{\rule{0.450pt}{0.400pt}}
\multiput(404.00,292.59)(1.033,0.482){9}{\rule{0.900pt}{0.116pt}}
\multiput(404.00,291.17)(10.132,6.000){2}{\rule{0.450pt}{0.400pt}}
\multiput(416.00,298.59)(0.943,0.482){9}{\rule{0.833pt}{0.116pt}}
\multiput(416.00,297.17)(9.270,6.000){2}{\rule{0.417pt}{0.400pt}}
\multiput(427.00,304.59)(1.033,0.482){9}{\rule{0.900pt}{0.116pt}}
\multiput(427.00,303.17)(10.132,6.000){2}{\rule{0.450pt}{0.400pt}}
\multiput(439.00,310.59)(1.033,0.482){9}{\rule{0.900pt}{0.116pt}}
\multiput(439.00,309.17)(10.132,6.000){2}{\rule{0.450pt}{0.400pt}}
\multiput(451.00,316.59)(0.943,0.482){9}{\rule{0.833pt}{0.116pt}}
\multiput(451.00,315.17)(9.270,6.000){2}{\rule{0.417pt}{0.400pt}}
\multiput(462.00,322.59)(1.033,0.482){9}{\rule{0.900pt}{0.116pt}}
\multiput(462.00,321.17)(10.132,6.000){2}{\rule{0.450pt}{0.400pt}}
\multiput(474.00,328.59)(0.874,0.485){11}{\rule{0.786pt}{0.117pt}}
\multiput(474.00,327.17)(10.369,7.000){2}{\rule{0.393pt}{0.400pt}}
\multiput(486.00,335.59)(0.943,0.482){9}{\rule{0.833pt}{0.116pt}}
\multiput(486.00,334.17)(9.270,6.000){2}{\rule{0.417pt}{0.400pt}}
\multiput(497.00,341.59)(1.033,0.482){9}{\rule{0.900pt}{0.116pt}}
\multiput(497.00,340.17)(10.132,6.000){2}{\rule{0.450pt}{0.400pt}}
\multiput(509.00,347.59)(1.033,0.482){9}{\rule{0.900pt}{0.116pt}}
\multiput(509.00,346.17)(10.132,6.000){2}{\rule{0.450pt}{0.400pt}}
\multiput(521.00,353.59)(0.943,0.482){9}{\rule{0.833pt}{0.116pt}}
\multiput(521.00,352.17)(9.270,6.000){2}{\rule{0.417pt}{0.400pt}}
\multiput(532.00,359.59)(1.033,0.482){9}{\rule{0.900pt}{0.116pt}}
\multiput(532.00,358.17)(10.132,6.000){2}{\rule{0.450pt}{0.400pt}}
\multiput(544.00,365.59)(1.033,0.482){9}{\rule{0.900pt}{0.116pt}}
\multiput(544.00,364.17)(10.132,6.000){2}{\rule{0.450pt}{0.400pt}}
\multiput(556.00,371.59)(0.943,0.482){9}{\rule{0.833pt}{0.116pt}}
\multiput(556.00,370.17)(9.270,6.000){2}{\rule{0.417pt}{0.400pt}}
\multiput(567.00,377.59)(1.033,0.482){9}{\rule{0.900pt}{0.116pt}}
\multiput(567.00,376.17)(10.132,6.000){2}{\rule{0.450pt}{0.400pt}}
\multiput(579.00,383.59)(1.033,0.482){9}{\rule{0.900pt}{0.116pt}}
\multiput(579.00,382.17)(10.132,6.000){2}{\rule{0.450pt}{0.400pt}}
\multiput(591.00,389.59)(0.943,0.482){9}{\rule{0.833pt}{0.116pt}}
\multiput(591.00,388.17)(9.270,6.000){2}{\rule{0.417pt}{0.400pt}}
\multiput(602.00,395.59)(1.033,0.482){9}{\rule{0.900pt}{0.116pt}}
\multiput(602.00,394.17)(10.132,6.000){2}{\rule{0.450pt}{0.400pt}}
\multiput(614.00,401.59)(1.033,0.482){9}{\rule{0.900pt}{0.116pt}}
\multiput(614.00,400.17)(10.132,6.000){2}{\rule{0.450pt}{0.400pt}}
\multiput(626.00,407.59)(0.874,0.485){11}{\rule{0.786pt}{0.117pt}}
\multiput(626.00,406.17)(10.369,7.000){2}{\rule{0.393pt}{0.400pt}}
\multiput(638.00,414.59)(0.943,0.482){9}{\rule{0.833pt}{0.116pt}}
\multiput(638.00,413.17)(9.270,6.000){2}{\rule{0.417pt}{0.400pt}}
\multiput(649.00,420.59)(1.033,0.482){9}{\rule{0.900pt}{0.116pt}}
\multiput(649.00,419.17)(10.132,6.000){2}{\rule{0.450pt}{0.400pt}}
\multiput(661.00,426.59)(1.033,0.482){9}{\rule{0.900pt}{0.116pt}}
\multiput(661.00,425.17)(10.132,6.000){2}{\rule{0.450pt}{0.400pt}}
\multiput(673.00,432.59)(0.943,0.482){9}{\rule{0.833pt}{0.116pt}}
\multiput(673.00,431.17)(9.270,6.000){2}{\rule{0.417pt}{0.400pt}}
\multiput(684.00,438.59)(1.033,0.482){9}{\rule{0.900pt}{0.116pt}}
\multiput(684.00,437.17)(10.132,6.000){2}{\rule{0.450pt}{0.400pt}}
\multiput(696.00,444.59)(1.033,0.482){9}{\rule{0.900pt}{0.116pt}}
\multiput(696.00,443.17)(10.132,6.000){2}{\rule{0.450pt}{0.400pt}}
\multiput(708.00,450.59)(0.943,0.482){9}{\rule{0.833pt}{0.116pt}}
\multiput(708.00,449.17)(9.270,6.000){2}{\rule{0.417pt}{0.400pt}}
\multiput(719.00,456.59)(1.033,0.482){9}{\rule{0.900pt}{0.116pt}}
\multiput(719.00,455.17)(10.132,6.000){2}{\rule{0.450pt}{0.400pt}}
\multiput(731.00,462.59)(1.033,0.482){9}{\rule{0.900pt}{0.116pt}}
\multiput(731.00,461.17)(10.132,6.000){2}{\rule{0.450pt}{0.400pt}}
\multiput(743.00,468.59)(0.943,0.482){9}{\rule{0.833pt}{0.116pt}}
\multiput(743.00,467.17)(9.270,6.000){2}{\rule{0.417pt}{0.400pt}}
\multiput(754.00,474.59)(1.033,0.482){9}{\rule{0.900pt}{0.116pt}}
\multiput(754.00,473.17)(10.132,6.000){2}{\rule{0.450pt}{0.400pt}}
\multiput(766.00,480.59)(1.033,0.482){9}{\rule{0.900pt}{0.116pt}}
\multiput(766.00,479.17)(10.132,6.000){2}{\rule{0.450pt}{0.400pt}}
\multiput(778.00,486.59)(0.943,0.482){9}{\rule{0.833pt}{0.116pt}}
\multiput(778.00,485.17)(9.270,6.000){2}{\rule{0.417pt}{0.400pt}}
\multiput(789.00,492.59)(0.874,0.485){11}{\rule{0.786pt}{0.117pt}}
\multiput(789.00,491.17)(10.369,7.000){2}{\rule{0.393pt}{0.400pt}}
\multiput(801.00,499.59)(1.033,0.482){9}{\rule{0.900pt}{0.116pt}}
\multiput(801.00,498.17)(10.132,6.000){2}{\rule{0.450pt}{0.400pt}}
\multiput(813.00,505.59)(0.943,0.482){9}{\rule{0.833pt}{0.116pt}}
\multiput(813.00,504.17)(9.270,6.000){2}{\rule{0.417pt}{0.400pt}}
\multiput(824.00,511.59)(1.033,0.482){9}{\rule{0.900pt}{0.116pt}}
\multiput(824.00,510.17)(10.132,6.000){2}{\rule{0.450pt}{0.400pt}}
\multiput(836.00,517.59)(1.033,0.482){9}{\rule{0.900pt}{0.116pt}}
\multiput(836.00,516.17)(10.132,6.000){2}{\rule{0.450pt}{0.400pt}}
\multiput(848.00,523.59)(1.033,0.482){9}{\rule{0.900pt}{0.116pt}}
\multiput(848.00,522.17)(10.132,6.000){2}{\rule{0.450pt}{0.400pt}}
\multiput(860.00,529.59)(0.943,0.482){9}{\rule{0.833pt}{0.116pt}}
\multiput(860.00,528.17)(9.270,6.000){2}{\rule{0.417pt}{0.400pt}}
\multiput(871.00,535.59)(1.033,0.482){9}{\rule{0.900pt}{0.116pt}}
\multiput(871.00,534.17)(10.132,6.000){2}{\rule{0.450pt}{0.400pt}}
\multiput(883.00,541.59)(1.033,0.482){9}{\rule{0.900pt}{0.116pt}}
\multiput(883.00,540.17)(10.132,6.000){2}{\rule{0.450pt}{0.400pt}}
\multiput(895.00,547.59)(0.943,0.482){9}{\rule{0.833pt}{0.116pt}}
\multiput(895.00,546.17)(9.270,6.000){2}{\rule{0.417pt}{0.400pt}}
\multiput(906.00,553.59)(1.033,0.482){9}{\rule{0.900pt}{0.116pt}}
\multiput(906.00,552.17)(10.132,6.000){2}{\rule{0.450pt}{0.400pt}}
\multiput(918.00,559.59)(1.033,0.482){9}{\rule{0.900pt}{0.116pt}}
\multiput(918.00,558.17)(10.132,6.000){2}{\rule{0.450pt}{0.400pt}}
\multiput(930.00,565.59)(0.943,0.482){9}{\rule{0.833pt}{0.116pt}}
\multiput(930.00,564.17)(9.270,6.000){2}{\rule{0.417pt}{0.400pt}}
\multiput(941.00,571.59)(0.874,0.485){11}{\rule{0.786pt}{0.117pt}}
\multiput(941.00,570.17)(10.369,7.000){2}{\rule{0.393pt}{0.400pt}}
\multiput(953.00,578.59)(1.033,0.482){9}{\rule{0.900pt}{0.116pt}}
\multiput(953.00,577.17)(10.132,6.000){2}{\rule{0.450pt}{0.400pt}}
\multiput(965.00,584.59)(0.943,0.482){9}{\rule{0.833pt}{0.116pt}}
\multiput(965.00,583.17)(9.270,6.000){2}{\rule{0.417pt}{0.400pt}}
\multiput(976.00,590.59)(1.033,0.482){9}{\rule{0.900pt}{0.116pt}}
\multiput(976.00,589.17)(10.132,6.000){2}{\rule{0.450pt}{0.400pt}}
\multiput(988.00,596.59)(1.033,0.482){9}{\rule{0.900pt}{0.116pt}}
\multiput(988.00,595.17)(10.132,6.000){2}{\rule{0.450pt}{0.400pt}}
\multiput(1000.00,602.59)(0.943,0.482){9}{\rule{0.833pt}{0.116pt}}
\multiput(1000.00,601.17)(9.270,6.000){2}{\rule{0.417pt}{0.400pt}}
\multiput(1011.00,608.59)(1.033,0.482){9}{\rule{0.900pt}{0.116pt}}
\multiput(1011.00,607.17)(10.132,6.000){2}{\rule{0.450pt}{0.400pt}}
\multiput(1023.00,614.59)(1.033,0.482){9}{\rule{0.900pt}{0.116pt}}
\multiput(1023.00,613.17)(10.132,6.000){2}{\rule{0.450pt}{0.400pt}}
\multiput(1035.00,620.59)(0.943,0.482){9}{\rule{0.833pt}{0.116pt}}
\multiput(1035.00,619.17)(9.270,6.000){2}{\rule{0.417pt}{0.400pt}}
\multiput(1046.00,626.59)(1.033,0.482){9}{\rule{0.900pt}{0.116pt}}
\multiput(1046.00,625.17)(10.132,6.000){2}{\rule{0.450pt}{0.400pt}}
\multiput(1058.00,632.59)(1.033,0.482){9}{\rule{0.900pt}{0.116pt}}
\multiput(1058.00,631.17)(10.132,6.000){2}{\rule{0.450pt}{0.400pt}}
\multiput(1070.00,638.59)(1.033,0.482){9}{\rule{0.900pt}{0.116pt}}
\multiput(1070.00,637.17)(10.132,6.000){2}{\rule{0.450pt}{0.400pt}}
\multiput(1082.00,644.59)(0.943,0.482){9}{\rule{0.833pt}{0.116pt}}
\multiput(1082.00,643.17)(9.270,6.000){2}{\rule{0.417pt}{0.400pt}}
\multiput(1093.00,650.59)(0.874,0.485){11}{\rule{0.786pt}{0.117pt}}
\multiput(1093.00,649.17)(10.369,7.000){2}{\rule{0.393pt}{0.400pt}}
\multiput(1105.00,657.59)(1.033,0.482){9}{\rule{0.900pt}{0.116pt}}
\multiput(1105.00,656.17)(10.132,6.000){2}{\rule{0.450pt}{0.400pt}}
\multiput(1117.00,663.59)(0.943,0.482){9}{\rule{0.833pt}{0.116pt}}
\multiput(1117.00,662.17)(9.270,6.000){2}{\rule{0.417pt}{0.400pt}}
\multiput(1128.00,669.59)(1.033,0.482){9}{\rule{0.900pt}{0.116pt}}
\multiput(1128.00,668.17)(10.132,6.000){2}{\rule{0.450pt}{0.400pt}}
\multiput(1140.00,675.59)(1.033,0.482){9}{\rule{0.900pt}{0.116pt}}
\multiput(1140.00,674.17)(10.132,6.000){2}{\rule{0.450pt}{0.400pt}}
\multiput(1152.00,681.59)(0.943,0.482){9}{\rule{0.833pt}{0.116pt}}
\multiput(1152.00,680.17)(9.270,6.000){2}{\rule{0.417pt}{0.400pt}}
\multiput(1163.00,687.59)(1.033,0.482){9}{\rule{0.900pt}{0.116pt}}
\multiput(1163.00,686.17)(10.132,6.000){2}{\rule{0.450pt}{0.400pt}}
\multiput(1175.00,693.59)(1.033,0.482){9}{\rule{0.900pt}{0.116pt}}
\multiput(1175.00,692.17)(10.132,6.000){2}{\rule{0.450pt}{0.400pt}}
\multiput(1187.00,699.59)(0.943,0.482){9}{\rule{0.833pt}{0.116pt}}
\multiput(1187.00,698.17)(9.270,6.000){2}{\rule{0.417pt}{0.400pt}}
\multiput(1198.00,705.59)(1.033,0.482){9}{\rule{0.900pt}{0.116pt}}
\multiput(1198.00,704.17)(10.132,6.000){2}{\rule{0.450pt}{0.400pt}}
\multiput(1210.00,711.59)(1.033,0.482){9}{\rule{0.900pt}{0.116pt}}
\multiput(1210.00,710.17)(10.132,6.000){2}{\rule{0.450pt}{0.400pt}}
\multiput(1222.00,717.59)(0.943,0.482){9}{\rule{0.833pt}{0.116pt}}
\multiput(1222.00,716.17)(9.270,6.000){2}{\rule{0.417pt}{0.400pt}}
\multiput(1233.00,723.59)(1.033,0.482){9}{\rule{0.900pt}{0.116pt}}
\multiput(1233.00,722.17)(10.132,6.000){2}{\rule{0.450pt}{0.400pt}}
\multiput(1245.00,729.59)(1.033,0.482){9}{\rule{0.900pt}{0.116pt}}
\multiput(1245.00,728.17)(10.132,6.000){2}{\rule{0.450pt}{0.400pt}}
\multiput(1257.00,735.59)(0.798,0.485){11}{\rule{0.729pt}{0.117pt}}
\multiput(1257.00,734.17)(9.488,7.000){2}{\rule{0.364pt}{0.400pt}}
\multiput(1268.00,742.59)(1.033,0.482){9}{\rule{0.900pt}{0.116pt}}
\multiput(1268.00,741.17)(10.132,6.000){2}{\rule{0.450pt}{0.400pt}}
\multiput(1280.00,748.59)(1.033,0.482){9}{\rule{0.900pt}{0.116pt}}
\multiput(1280.00,747.17)(10.132,6.000){2}{\rule{0.450pt}{0.400pt}}
\multiput(1292.00,754.59)(1.033,0.482){9}{\rule{0.900pt}{0.116pt}}
\multiput(1292.00,753.17)(10.132,6.000){2}{\rule{0.450pt}{0.400pt}}
\multiput(1304.00,760.59)(0.943,0.482){9}{\rule{0.833pt}{0.116pt}}
\multiput(1304.00,759.17)(9.270,6.000){2}{\rule{0.417pt}{0.400pt}}
\multiput(1315.00,766.59)(1.033,0.482){9}{\rule{0.900pt}{0.116pt}}
\multiput(1315.00,765.17)(10.132,6.000){2}{\rule{0.450pt}{0.400pt}}
\multiput(1327.00,772.59)(1.033,0.482){9}{\rule{0.900pt}{0.116pt}}
\multiput(1327.00,771.17)(10.132,6.000){2}{\rule{0.450pt}{0.400pt}}
\multiput(1339.00,778.59)(0.943,0.482){9}{\rule{0.833pt}{0.116pt}}
\multiput(1339.00,777.17)(9.270,6.000){2}{\rule{0.417pt}{0.400pt}}
\multiput(1350.00,784.59)(1.033,0.482){9}{\rule{0.900pt}{0.116pt}}
\multiput(1350.00,783.17)(10.132,6.000){2}{\rule{0.450pt}{0.400pt}}
\multiput(1362.00,790.59)(1.033,0.482){9}{\rule{0.900pt}{0.116pt}}
\multiput(1362.00,789.17)(10.132,6.000){2}{\rule{0.450pt}{0.400pt}}
\multiput(1374.00,796.59)(0.943,0.482){9}{\rule{0.833pt}{0.116pt}}
\multiput(1374.00,795.17)(9.270,6.000){2}{\rule{0.417pt}{0.400pt}}
\multiput(1385.00,802.59)(1.033,0.482){9}{\rule{0.900pt}{0.116pt}}
\multiput(1385.00,801.17)(10.132,6.000){2}{\rule{0.450pt}{0.400pt}}
\multiput(1397.00,808.59)(1.033,0.482){9}{\rule{0.900pt}{0.116pt}}
\multiput(1397.00,807.17)(10.132,6.000){2}{\rule{0.450pt}{0.400pt}}
\put(711,778){\makebox(0,0)[r]{data}}
\put(296,245){\rule{1pt}{1pt}}
\put(294,225){\rule{1pt}{1pt}}
\put(296,245){\rule{1pt}{1pt}}
\put(294,225){\rule{1pt}{1pt}}
\put(296,245){\rule{1pt}{1pt}}
\put(294,225){\rule{1pt}{1pt}}
\put(295,225){\rule{1pt}{1pt}}
\put(293,245){\rule{1pt}{1pt}}
\put(295,225){\rule{1pt}{1pt}}
\put(293,245){\rule{1pt}{1pt}}
\put(292,225){\rule{1pt}{1pt}}
\put(293,245){\rule{1pt}{1pt}}
\put(292,245){\rule{1pt}{1pt}}
\put(293,225){\rule{1pt}{1pt}}
\put(292,245){\rule{1pt}{1pt}}
\put(293,225){\rule{1pt}{1pt}}
\put(291,225){\rule{1pt}{1pt}}
\put(293,245){\rule{1pt}{1pt}}
\put(291,245){\rule{1pt}{1pt}}
\put(293,225){\rule{1pt}{1pt}}
\put(295,225){\rule{1pt}{1pt}}
\put(293,245){\rule{1pt}{1pt}}
\put(295,245){\rule{1pt}{1pt}}
\put(293,225){\rule{1pt}{1pt}}
\put(292,225){\rule{1pt}{1pt}}
\put(293,245){\rule{1pt}{1pt}}
\put(292,225){\rule{1pt}{1pt}}
\put(293,245){\rule{1pt}{1pt}}
\put(295,225){\rule{1pt}{1pt}}
\put(293,245){\rule{1pt}{1pt}}
\put(292,245){\rule{1pt}{1pt}}
\put(293,225){\rule{1pt}{1pt}}
\put(292,245){\rule{1pt}{1pt}}
\put(293,225){\rule{1pt}{1pt}}
\put(295,245){\rule{1pt}{1pt}}
\put(293,225){\rule{1pt}{1pt}}
\put(292,245){\rule{1pt}{1pt}}
\put(293,225){\rule{1pt}{1pt}}
\put(295,245){\rule{1pt}{1pt}}
\put(293,225){\rule{1pt}{1pt}}
\put(291,225){\rule{1pt}{1pt}}
\put(293,245){\rule{1pt}{1pt}}
\put(295,225){\rule{1pt}{1pt}}
\put(293,245){\rule{1pt}{1pt}}
\put(295,225){\rule{1pt}{1pt}}
\put(293,245){\rule{1pt}{1pt}}
\put(291,225){\rule{1pt}{1pt}}
\put(293,245){\rule{1pt}{1pt}}
\put(295,245){\rule{1pt}{1pt}}
\put(293,225){\rule{1pt}{1pt}}
\put(295,225){\rule{1pt}{1pt}}
\put(292,245){\rule{1pt}{1pt}}
\put(294,225){\rule{1pt}{1pt}}
\put(292,245){\rule{1pt}{1pt}}
\put(294,245){\rule{1pt}{1pt}}
\put(292,225){\rule{1pt}{1pt}}
\put(290,245){\rule{1pt}{1pt}}
\put(292,225){\rule{1pt}{1pt}}
\put(291,225){\rule{1pt}{1pt}}
\put(293,245){\rule{1pt}{1pt}}
\put(291,225){\rule{1pt}{1pt}}
\put(293,245){\rule{1pt}{1pt}}
\put(291,225){\rule{1pt}{1pt}}
\put(293,245){\rule{1pt}{1pt}}
\put(295,245){\rule{1pt}{1pt}}
\put(293,225){\rule{1pt}{1pt}}
\put(291,245){\rule{1pt}{1pt}}
\put(293,225){\rule{1pt}{1pt}}
\put(291,245){\rule{1pt}{1pt}}
\put(293,225){\rule{1pt}{1pt}}
\put(291,245){\rule{1pt}{1pt}}
\put(293,225){\rule{1pt}{1pt}}
\put(295,245){\rule{1pt}{1pt}}
\put(293,225){\rule{1pt}{1pt}}
\put(291,225){\rule{1pt}{1pt}}
\put(293,245){\rule{1pt}{1pt}}
\put(290,225){\rule{1pt}{1pt}}
\put(293,245){\rule{1pt}{1pt}}
\put(290,245){\rule{1pt}{1pt}}
\put(293,225){\rule{1pt}{1pt}}
\put(295,245){\rule{1pt}{1pt}}
\put(292,225){\rule{1pt}{1pt}}
\put(294,245){\rule{1pt}{1pt}}
\put(292,225){\rule{1pt}{1pt}}
\put(289,245){\rule{1pt}{1pt}}
\put(292,225){\rule{1pt}{1pt}}
\put(293,245){\rule{1pt}{1pt}}
\put(292,225){\rule{1pt}{1pt}}
\put(290,245){\rule{1pt}{1pt}}
\put(292,225){\rule{1pt}{1pt}}
\put(290,245){\rule{1pt}{1pt}}
\put(292,225){\rule{1pt}{1pt}}
\put(293,245){\rule{1pt}{1pt}}
\put(291,225){\rule{1pt}{1pt}}
\put(293,245){\rule{1pt}{1pt}}
\put(291,225){\rule{1pt}{1pt}}
\put(293,225){\rule{1pt}{1pt}}
\put(291,245){\rule{1pt}{1pt}}
\put(290,225){\rule{1pt}{1pt}}
\put(292,245){\rule{1pt}{1pt}}
\put(293,225){\rule{1pt}{1pt}}
\put(291,245){\rule{1pt}{1pt}}
\put(292,225){\rule{1pt}{1pt}}
\put(290,245){\rule{1pt}{1pt}}
\put(292,245){\rule{1pt}{1pt}}
\put(291,225){\rule{1pt}{1pt}}
\put(293,245){\rule{1pt}{1pt}}
\put(291,225){\rule{1pt}{1pt}}
\put(293,245){\rule{1pt}{1pt}}
\put(292,225){\rule{1pt}{1pt}}
\put(293,225){\rule{1pt}{1pt}}
\put(292,245){\rule{1pt}{1pt}}
\put(293,245){\rule{1pt}{1pt}}
\put(292,225){\rule{1pt}{1pt}}
\put(293,245){\rule{1pt}{1pt}}
\put(292,225){\rule{1pt}{1pt}}
\put(293,225){\rule{1pt}{1pt}}
\put(292,245){\rule{1pt}{1pt}}
\put(291,225){\rule{1pt}{1pt}}
\put(293,245){\rule{1pt}{1pt}}
\put(292,225){\rule{1pt}{1pt}}
\put(293,245){\rule{1pt}{1pt}}
\put(292,225){\rule{1pt}{1pt}}
\put(293,245){\rule{1pt}{1pt}}
\put(295,245){\rule{1pt}{1pt}}
\put(293,225){\rule{1pt}{1pt}}
\put(295,245){\rule{1pt}{1pt}}
\put(293,225){\rule{1pt}{1pt}}
\put(292,225){\rule{1pt}{1pt}}
\put(293,245){\rule{1pt}{1pt}}
\put(292,245){\rule{1pt}{1pt}}
\put(293,225){\rule{1pt}{1pt}}
\put(292,225){\rule{1pt}{1pt}}
\put(293,245){\rule{1pt}{1pt}}
\put(375,141){\rule{1pt}{1pt}}
\put(295,381){\rule{1pt}{1pt}}
\put(252,225){\rule{1pt}{1pt}}
\put(288,245){\rule{1pt}{1pt}}
\put(299,245){\rule{1pt}{1pt}}
\put(289,225){\rule{1pt}{1pt}}
\put(291,225){\rule{1pt}{1pt}}
\put(288,245){\rule{1pt}{1pt}}
\put(286,225){\rule{1pt}{1pt}}
\put(288,245){\rule{1pt}{1pt}}
\put(286,225){\rule{1pt}{1pt}}
\put(288,245){\rule{1pt}{1pt}}
\put(286,225){\rule{1pt}{1pt}}
\put(288,245){\rule{1pt}{1pt}}
\put(286,245){\rule{1pt}{1pt}}
\put(288,225){\rule{1pt}{1pt}}
\put(290,225){\rule{1pt}{1pt}}
\put(289,245){\rule{1pt}{1pt}}
\put(291,225){\rule{1pt}{1pt}}
\put(288,245){\rule{1pt}{1pt}}
\put(286,225){\rule{1pt}{1pt}}
\put(288,245){\rule{1pt}{1pt}}
\put(290,225){\rule{1pt}{1pt}}
\put(288,245){\rule{1pt}{1pt}}
\put(290,225){\rule{1pt}{1pt}}
\put(287,245){\rule{1pt}{1pt}}
\put(290,225){\rule{1pt}{1pt}}
\put(287,245){\rule{1pt}{1pt}}
\put(289,245){\rule{1pt}{1pt}}
\put(287,225){\rule{1pt}{1pt}}
\put(289,245){\rule{1pt}{1pt}}
\put(287,225){\rule{1pt}{1pt}}
\put(289,225){\rule{1pt}{1pt}}
\put(287,245){\rule{1pt}{1pt}}
\put(289,225){\rule{1pt}{1pt}}
\put(287,245){\rule{1pt}{1pt}}
\put(285,225){\rule{1pt}{1pt}}
\put(287,245){\rule{1pt}{1pt}}
\put(285,245){\rule{1pt}{1pt}}
\put(287,225){\rule{1pt}{1pt}}
\put(285,225){\rule{1pt}{1pt}}
\put(287,245){\rule{1pt}{1pt}}
\put(285,225){\rule{1pt}{1pt}}
\put(287,245){\rule{1pt}{1pt}}
\put(288,225){\rule{1pt}{1pt}}
\put(286,245){\rule{1pt}{1pt}}
\put(284,245){\rule{1pt}{1pt}}
\put(286,225){\rule{1pt}{1pt}}
\put(288,225){\rule{1pt}{1pt}}
\put(285,245){\rule{1pt}{1pt}}
\put(284,225){\rule{1pt}{1pt}}
\put(313,287){\rule{1pt}{1pt}}
\put(330,277){\rule{1pt}{1pt}}
\put(354,308){\rule{1pt}{1pt}}
\put(373,297){\rule{1pt}{1pt}}
\put(396,329){\rule{1pt}{1pt}}
\put(417,318){\rule{1pt}{1pt}}
\put(435,349){\rule{1pt}{1pt}}
\put(456,339){\rule{1pt}{1pt}}
\put(473,370){\rule{1pt}{1pt}}
\put(490,349){\rule{1pt}{1pt}}
\put(509,381){\rule{1pt}{1pt}}
\put(528,381){\rule{1pt}{1pt}}
\put(544,401){\rule{1pt}{1pt}}
\put(559,391){\rule{1pt}{1pt}}
\put(578,422){\rule{1pt}{1pt}}
\put(596,401){\rule{1pt}{1pt}}
\put(610,433){\rule{1pt}{1pt}}
\put(627,422){\rule{1pt}{1pt}}
\put(640,453){\rule{1pt}{1pt}}
\put(653,443){\rule{1pt}{1pt}}
\put(670,464){\rule{1pt}{1pt}}
\put(683,453){\rule{1pt}{1pt}}
\put(699,485){\rule{1pt}{1pt}}
\put(715,464){\rule{1pt}{1pt}}
\put(727,495){\rule{1pt}{1pt}}
\put(741,474){\rule{1pt}{1pt}}
\put(753,505){\rule{1pt}{1pt}}
\put(767,495){\rule{1pt}{1pt}}
\put(778,516){\rule{1pt}{1pt}}
\put(792,505){\rule{1pt}{1pt}}
\put(802,526){\rule{1pt}{1pt}}
\put(811,516){\rule{1pt}{1pt}}
\put(825,537){\rule{1pt}{1pt}}
\put(838,526){\rule{1pt}{1pt}}
\put(848,547){\rule{1pt}{1pt}}
\put(860,537){\rule{1pt}{1pt}}
\put(869,568){\rule{1pt}{1pt}}
\put(881,557){\rule{1pt}{1pt}}
\put(890,578){\rule{1pt}{1pt}}
\put(898,557){\rule{1pt}{1pt}}
\put(911,589){\rule{1pt}{1pt}}
\put(923,568){\rule{1pt}{1pt}}
\put(931,589){\rule{1pt}{1pt}}
\put(939,578){\rule{1pt}{1pt}}
\put(951,599){\rule{1pt}{1pt}}
\put(962,589){\rule{1pt}{1pt}}
\put(969,620){\rule{1pt}{1pt}}
\put(977,589){\rule{1pt}{1pt}}
\put(987,620){\rule{1pt}{1pt}}
\put(997,609){\rule{1pt}{1pt}}
\put(1004,630){\rule{1pt}{1pt}}
\put(1014,620){\rule{1pt}{1pt}}
\put(1021,641){\rule{1pt}{1pt}}
\put(1030,641){\rule{1pt}{1pt}}
\put(1037,620){\rule{1pt}{1pt}}
\put(1046,630){\rule{1pt}{1pt}}
\put(1052,651){\rule{1pt}{1pt}}
\put(1060,651){\rule{1pt}{1pt}}
\put(1066,630){\rule{1pt}{1pt}}
\put(1071,641){\rule{1pt}{1pt}}
\put(1080,661){\rule{1pt}{1pt}}
\put(1089,661){\rule{1pt}{1pt}}
\put(1094,651){\rule{1pt}{1pt}}
\put(1099,651){\rule{1pt}{1pt}}
\put(1107,682){\rule{1pt}{1pt}}
\put(1115,651){\rule{1pt}{1pt}}
\put(1120,682){\rule{1pt}{1pt}}
\put(1123,672){\rule{1pt}{1pt}}
\put(1133,693){\rule{1pt}{1pt}}
\put(1137,693){\rule{1pt}{1pt}}
\put(1145,672){\rule{1pt}{1pt}}
\put(1149,682){\rule{1pt}{1pt}}
\put(1156,703){\rule{1pt}{1pt}}
\put(1159,703){\rule{1pt}{1pt}}
\put(1168,682){\rule{1pt}{1pt}}
\put(1171,693){\rule{1pt}{1pt}}
\put(1178,713){\rule{1pt}{1pt}}
\put(1181,693){\rule{1pt}{1pt}}
\put(1188,713){\rule{1pt}{1pt}}
\put(1195,693){\rule{1pt}{1pt}}
\put(1198,713){\rule{1pt}{1pt}}
\put(1204,703){\rule{1pt}{1pt}}
\put(1207,724){\rule{1pt}{1pt}}
\put(1210,703){\rule{1pt}{1pt}}
\put(1216,734){\rule{1pt}{1pt}}
\put(1222,734){\rule{1pt}{1pt}}
\put(1225,703){\rule{1pt}{1pt}}
\put(1231,724){\rule{1pt}{1pt}}
\put(1234,745){\rule{1pt}{1pt}}
\put(1239,724){\rule{1pt}{1pt}}
\put(1242,745){\rule{1pt}{1pt}}
\put(1244,745){\rule{1pt}{1pt}}
\put(1250,724){\rule{1pt}{1pt}}
\put(1256,734){\rule{1pt}{1pt}}
\put(1258,745){\rule{1pt}{1pt}}
\put(1260,734){\rule{1pt}{1pt}}
\put(1266,755){\rule{1pt}{1pt}}
\put(1271,734){\rule{1pt}{1pt}}
\put(1273,755){\rule{1pt}{1pt}}
\put(1275,755){\rule{1pt}{1pt}}
\put(1280,734){\rule{1pt}{1pt}}
\put(1285,745){\rule{1pt}{1pt}}
\put(1286,755){\rule{1pt}{1pt}}
\put(1287,745){\rule{1pt}{1pt}}
\put(1293,765){\rule{1pt}{1pt}}
\put(1298,765){\rule{1pt}{1pt}}
\put(1298,745){\rule{1pt}{1pt}}
\put(1304,765){\rule{1pt}{1pt}}
\put(1304,745){\rule{1pt}{1pt}}
\put(1304,755){\rule{1pt}{1pt}}
\put(1309,765){\rule{1pt}{1pt}}
\put(1314,765){\rule{1pt}{1pt}}
\put(1314,755){\rule{1pt}{1pt}}
\put(1315,755){\rule{1pt}{1pt}}
\put(1319,776){\rule{1pt}{1pt}}
\put(1324,776){\rule{1pt}{1pt}}
\put(1324,755){\rule{1pt}{1pt}}
\put(1329,776){\rule{1pt}{1pt}}
\put(1329,755){\rule{1pt}{1pt}}
\put(1330,776){\rule{1pt}{1pt}}
\put(1334,765){\rule{1pt}{1pt}}
\put(1334,776){\rule{1pt}{1pt}}
\put(1338,765){\rule{1pt}{1pt}}
\put(1339,776){\rule{1pt}{1pt}}
\put(1343,765){\rule{1pt}{1pt}}
\put(1343,765){\rule{1pt}{1pt}}
\put(1346,797){\rule{1pt}{1pt}}
\put(1349,765){\rule{1pt}{1pt}}
\put(1350,797){\rule{1pt}{1pt}}
\put(1353,797){\rule{1pt}{1pt}}
\put(1354,765){\rule{1pt}{1pt}}
\put(1355,797){\rule{1pt}{1pt}}
\put(1357,765){\rule{1pt}{1pt}}
\put(1357,786){\rule{1pt}{1pt}}
\put(1360,797){\rule{1pt}{1pt}}
\put(1363,786){\rule{1pt}{1pt}}
\put(1363,797){\rule{1pt}{1pt}}
\put(1363,786){\rule{1pt}{1pt}}
\put(1366,797){\rule{1pt}{1pt}}
\put(1366,786){\rule{1pt}{1pt}}
\put(1370,797){\rule{1pt}{1pt}}
\put(1370,786){\rule{1pt}{1pt}}
\put(1373,807){\rule{1pt}{1pt}}
\put(1375,786){\rule{1pt}{1pt}}
\put(1375,807){\rule{1pt}{1pt}}
\put(1378,786){\rule{1pt}{1pt}}
\put(1378,807){\rule{1pt}{1pt}}
\put(1380,807){\rule{1pt}{1pt}}
\put(1380,786){\rule{1pt}{1pt}}
\put(1380,786){\rule{1pt}{1pt}}
\put(1383,807){\rule{1pt}{1pt}}
\put(1385,807){\rule{1pt}{1pt}}
\put(1385,786){\rule{1pt}{1pt}}
\put(1385,807){\rule{1pt}{1pt}}
\put(1387,786){\rule{1pt}{1pt}}
\put(1390,797){\rule{1pt}{1pt}}
\put(1389,807){\rule{1pt}{1pt}}
\put(1391,797){\rule{1pt}{1pt}}
\put(1390,807){\rule{1pt}{1pt}}
\put(1393,797){\rule{1pt}{1pt}}
\put(1392,807){\rule{1pt}{1pt}}
\put(1394,797){\rule{1pt}{1pt}}
\put(1393,807){\rule{1pt}{1pt}}
\put(1392,807){\rule{1pt}{1pt}}
\put(1394,797){\rule{1pt}{1pt}}
\put(1394,807){\rule{1pt}{1pt}}
\put(1396,797){\rule{1pt}{1pt}}
\put(1398,807){\rule{1pt}{1pt}}
\put(1397,797){\rule{1pt}{1pt}}
\put(1399,807){\rule{1pt}{1pt}}
\put(1398,797){\rule{1pt}{1pt}}
\put(1400,797){\rule{1pt}{1pt}}
\put(1399,807){\rule{1pt}{1pt}}
\put(1401,797){\rule{1pt}{1pt}}
\put(1399,817){\rule{1pt}{1pt}}
\put(1402,797){\rule{1pt}{1pt}}
\put(1400,817){\rule{1pt}{1pt}}
\put(1399,797){\rule{1pt}{1pt}}
\put(1402,817){\rule{1pt}{1pt}}
\put(1404,797){\rule{1pt}{1pt}}
\put(1402,817){\rule{1pt}{1pt}}
\put(1405,797){\rule{1pt}{1pt}}
\put(1404,817){\rule{1pt}{1pt}}
\put(1406,817){\rule{1pt}{1pt}}
\put(1405,797){\rule{1pt}{1pt}}
\put(1407,817){\rule{1pt}{1pt}}
\put(1405,797){\rule{1pt}{1pt}}
\put(1407,797){\rule{1pt}{1pt}}
\put(1405,817){\rule{1pt}{1pt}}
\put(1403,797){\rule{1pt}{1pt}}
\put(1406,817){\rule{1pt}{1pt}}
\put(1407,817){\rule{1pt}{1pt}}
\put(1406,797){\rule{1pt}{1pt}}
\put(1408,797){\rule{1pt}{1pt}}
\put(1406,817){\rule{1pt}{1pt}}
\put(1408,817){\rule{1pt}{1pt}}
\put(1406,797){\rule{1pt}{1pt}}
\put(1408,817){\rule{1pt}{1pt}}
\put(1406,797){\rule{1pt}{1pt}}
\put(1404,817){\rule{1pt}{1pt}}
\put(1407,797){\rule{1pt}{1pt}}
\put(1409,817){\rule{1pt}{1pt}}
\put(1407,797){\rule{1pt}{1pt}}
\put(1409,817){\rule{1pt}{1pt}}
\put(1406,797){\rule{1pt}{1pt}}
\put(1409,817){\rule{1pt}{1pt}}
\put(1406,797){\rule{1pt}{1pt}}
\put(1403,817){\rule{1pt}{1pt}}
\put(1406,797){\rule{1pt}{1pt}}
\put(1407,817){\rule{1pt}{1pt}}
\put(1405,797){\rule{1pt}{1pt}}
\put(1402,797){\rule{1pt}{1pt}}
\put(1404,817){\rule{1pt}{1pt}}
\put(1407,797){\rule{1pt}{1pt}}
\put(1405,817){\rule{1pt}{1pt}}
\put(1403,817){\rule{1pt}{1pt}}
\put(1405,797){\rule{1pt}{1pt}}
\put(1407,817){\rule{1pt}{1pt}}
\put(1405,797){\rule{1pt}{1pt}}
\put(1407,817){\rule{1pt}{1pt}}
\put(1405,797){\rule{1pt}{1pt}}
\put(1402,817){\rule{1pt}{1pt}}
\put(1405,797){\rule{1pt}{1pt}}
\put(1407,817){\rule{1pt}{1pt}}
\put(1404,797){\rule{1pt}{1pt}}
\put(1406,817){\rule{1pt}{1pt}}
\put(1404,797){\rule{1pt}{1pt}}
\put(1406,817){\rule{1pt}{1pt}}
\put(1404,797){\rule{1pt}{1pt}}
\put(1406,817){\rule{1pt}{1pt}}
\put(1404,797){\rule{1pt}{1pt}}
\put(1406,817){\rule{1pt}{1pt}}
\put(1404,797){\rule{1pt}{1pt}}
\put(1406,817){\rule{1pt}{1pt}}
\put(1404,797){\rule{1pt}{1pt}}
\put(1405,797){\rule{1pt}{1pt}}
\put(1402,817){\rule{1pt}{1pt}}
\put(1404,797){\rule{1pt}{1pt}}
\put(1402,817){\rule{1pt}{1pt}}
\put(1400,797){\rule{1pt}{1pt}}
\put(1402,817){\rule{1pt}{1pt}}
\put(1400,797){\rule{1pt}{1pt}}
\put(1401,807){\rule{1pt}{1pt}}
\put(1399,797){\rule{1pt}{1pt}}
\put(1400,807){\rule{1pt}{1pt}}
\put(1402,797){\rule{1pt}{1pt}}
\put(1400,807){\rule{1pt}{1pt}}
\put(1398,807){\rule{1pt}{1pt}}
\put(1400,797){\rule{1pt}{1pt}}
\put(1401,797){\rule{1pt}{1pt}}
\put(1399,807){\rule{1pt}{1pt}}
\put(1400,797){\rule{1pt}{1pt}}
\put(1397,807){\rule{1pt}{1pt}}
\put(1394,797){\rule{1pt}{1pt}}
\put(1396,807){\rule{1pt}{1pt}}
\put(1394,807){\rule{1pt}{1pt}}
\put(1395,797){\rule{1pt}{1pt}}
\put(1393,797){\rule{1pt}{1pt}}
\put(1394,807){\rule{1pt}{1pt}}
\put(1395,807){\rule{1pt}{1pt}}
\put(1394,797){\rule{1pt}{1pt}}
\put(1391,807){\rule{1pt}{1pt}}
\put(1392,797){\rule{1pt}{1pt}}
\put(1394,807){\rule{1pt}{1pt}}
\put(1391,797){\rule{1pt}{1pt}}
\put(1392,797){\rule{1pt}{1pt}}
\put(1390,807){\rule{1pt}{1pt}}
\put(1392,797){\rule{1pt}{1pt}}
\put(1389,807){\rule{1pt}{1pt}}
\put(1390,797){\rule{1pt}{1pt}}
\put(1387,807){\rule{1pt}{1pt}}
\put(1389,807){\rule{1pt}{1pt}}
\put(1386,786){\rule{1pt}{1pt}}
\put(1384,807){\rule{1pt}{1pt}}
\put(1385,786){\rule{1pt}{1pt}}
\put(1383,786){\rule{1pt}{1pt}}
\put(1384,807){\rule{1pt}{1pt}}
\put(1385,786){\rule{1pt}{1pt}}
\put(1383,807){\rule{1pt}{1pt}}
\put(1384,786){\rule{1pt}{1pt}}
\put(1382,807){\rule{1pt}{1pt}}
\put(1380,807){\rule{1pt}{1pt}}
\put(1381,786){\rule{1pt}{1pt}}
\put(1382,786){\rule{1pt}{1pt}}
\put(1380,807){\rule{1pt}{1pt}}
\put(1381,786){\rule{1pt}{1pt}}
\put(1378,807){\rule{1pt}{1pt}}
\put(1379,807){\rule{1pt}{1pt}}
\put(1377,786){\rule{1pt}{1pt}}
\put(1378,786){\rule{1pt}{1pt}}
\put(1375,807){\rule{1pt}{1pt}}
\put(1377,786){\rule{1pt}{1pt}}
\put(1374,807){\rule{1pt}{1pt}}
\put(1372,797){\rule{1pt}{1pt}}
\put(1373,786){\rule{1pt}{1pt}}
\put(1374,786){\rule{1pt}{1pt}}
\put(1372,797){\rule{1pt}{1pt}}
\put(1370,786){\rule{1pt}{1pt}}
\put(1371,797){\rule{1pt}{1pt}}
\put(1368,786){\rule{1pt}{1pt}}
\put(1370,797){\rule{1pt}{1pt}}
\put(1367,786){\rule{1pt}{1pt}}
\put(1368,797){\rule{1pt}{1pt}}
\put(1366,786){\rule{1pt}{1pt}}
\put(1367,797){\rule{1pt}{1pt}}
\put(1368,786){\rule{1pt}{1pt}}
\put(1366,797){\rule{1pt}{1pt}}
\put(1366,786){\rule{1pt}{1pt}}
\put(1364,797){\rule{1pt}{1pt}}
\put(1362,786){\rule{1pt}{1pt}}
\put(1363,797){\rule{1pt}{1pt}}
\put(1361,797){\rule{1pt}{1pt}}
\put(1362,776){\rule{1pt}{1pt}}
\put(1359,797){\rule{1pt}{1pt}}
\put(1360,765){\rule{1pt}{1pt}}
\put(1358,765){\rule{1pt}{1pt}}
\put(1359,797){\rule{1pt}{1pt}}
\put(1360,797){\rule{1pt}{1pt}}
\put(1357,765){\rule{1pt}{1pt}}
\put(1355,797){\rule{1pt}{1pt}}
\put(1356,765){\rule{1pt}{1pt}}
\put(1356,797){\rule{1pt}{1pt}}
\put(1354,765){\rule{1pt}{1pt}}
\put(1355,765){\rule{1pt}{1pt}}
\put(1353,797){\rule{1pt}{1pt}}
\put(1353,765){\rule{1pt}{1pt}}
\put(1350,797){\rule{1pt}{1pt}}
\put(1352,765){\rule{1pt}{1pt}}
\put(1349,797){\rule{1pt}{1pt}}
\put(1346,776){\rule{1pt}{1pt}}
\put(1348,765){\rule{1pt}{1pt}}
\put(1348,776){\rule{1pt}{1pt}}
\put(1346,765){\rule{1pt}{1pt}}
\put(1343,765){\rule{1pt}{1pt}}
\put(1344,776){\rule{1pt}{1pt}}
\put(1343,776){\rule{1pt}{1pt}}
\put(1343,765){\rule{1pt}{1pt}}
\put(1341,776){\rule{1pt}{1pt}}
\put(1341,765){\rule{1pt}{1pt}}
\put(1339,776){\rule{1pt}{1pt}}
\put(1340,765){\rule{1pt}{1pt}}
\put(1341,776){\rule{1pt}{1pt}}
\put(1338,755){\rule{1pt}{1pt}}
\put(1339,776){\rule{1pt}{1pt}}
\put(1337,755){\rule{1pt}{1pt}}
\put(1334,776){\rule{1pt}{1pt}}
\put(1335,755){\rule{1pt}{1pt}}
\put(1336,755){\rule{1pt}{1pt}}
\put(1332,776){\rule{1pt}{1pt}}
\put(1334,776){\rule{1pt}{1pt}}
\put(1331,755){\rule{1pt}{1pt}}
\put(1328,776){\rule{1pt}{1pt}}
\put(1330,755){\rule{1pt}{1pt}}
\put(1327,776){\rule{1pt}{1pt}}
\put(1328,755){\rule{1pt}{1pt}}
\put(1325,776){\rule{1pt}{1pt}}
\put(1327,755){\rule{1pt}{1pt}}
\put(1324,755){\rule{1pt}{1pt}}
\put(1326,776){\rule{1pt}{1pt}}
\put(1327,755){\rule{1pt}{1pt}}
\put(1324,776){\rule{1pt}{1pt}}
\put(1321,765){\rule{1pt}{1pt}}
\put(1322,755){\rule{1pt}{1pt}}
\put(1319,765){\rule{1pt}{1pt}}
\put(1321,755){\rule{1pt}{1pt}}
\put(1317,755){\rule{1pt}{1pt}}
\put(1319,765){\rule{1pt}{1pt}}
\put(1316,755){\rule{1pt}{1pt}}
\put(1317,765){\rule{1pt}{1pt}}
\put(1314,765){\rule{1pt}{1pt}}
\put(1315,755){\rule{1pt}{1pt}}
\put(1316,765){\rule{1pt}{1pt}}
\put(1313,745){\rule{1pt}{1pt}}
\put(1310,765){\rule{1pt}{1pt}}
\put(1311,745){\rule{1pt}{1pt}}
\put(1307,745){\rule{1pt}{1pt}}
\put(1309,765){\rule{1pt}{1pt}}
\put(1305,745){\rule{1pt}{1pt}}
\put(1307,765){\rule{1pt}{1pt}}
\put(1309,765){\rule{1pt}{1pt}}
\put(1305,745){\rule{1pt}{1pt}}
\put(1307,765){\rule{1pt}{1pt}}
\put(1304,745){\rule{1pt}{1pt}}
\put(1305,765){\rule{1pt}{1pt}}
\put(1302,745){\rule{1pt}{1pt}}
\put(1298,745){\rule{1pt}{1pt}}
\put(1300,765){\rule{1pt}{1pt}}
\put(1301,745){\rule{1pt}{1pt}}
\put(1297,765){\rule{1pt}{1pt}}
\put(1293,755){\rule{1pt}{1pt}}
\put(1295,745){\rule{1pt}{1pt}}
\put(1293,755){\rule{1pt}{1pt}}
\put(1293,745){\rule{1pt}{1pt}}
\put(1290,755){\rule{1pt}{1pt}}
\put(1291,745){\rule{1pt}{1pt}}
\put(1292,755){\rule{1pt}{1pt}}
\put(1289,745){\rule{1pt}{1pt}}
\put(1290,755){\rule{1pt}{1pt}}
\put(1287,734){\rule{1pt}{1pt}}
\put(1288,755){\rule{1pt}{1pt}}
\put(1285,734){\rule{1pt}{1pt}}
\put(1282,755){\rule{1pt}{1pt}}
\put(1283,734){\rule{1pt}{1pt}}
\put(1285,755){\rule{1pt}{1pt}}
\put(1281,734){\rule{1pt}{1pt}}
\put(1283,755){\rule{1pt}{1pt}}
\put(1280,734){\rule{1pt}{1pt}}
\put(1276,755){\rule{1pt}{1pt}}
\put(1278,734){\rule{1pt}{1pt}}
\put(1275,734){\rule{1pt}{1pt}}
\put(1275,755){\rule{1pt}{1pt}}
\put(1273,745){\rule{1pt}{1pt}}
\put(1274,734){\rule{1pt}{1pt}}
\put(1271,734){\rule{1pt}{1pt}}
\put(1271,745){\rule{1pt}{1pt}}
\put(1269,734){\rule{1pt}{1pt}}
\put(1270,745){\rule{1pt}{1pt}}
\put(1267,734){\rule{1pt}{1pt}}
\put(1267,745){\rule{1pt}{1pt}}
\put(1268,745){\rule{1pt}{1pt}}
\put(1266,734){\rule{1pt}{1pt}}
\put(1263,745){\rule{1pt}{1pt}}
\put(1264,734){\rule{1pt}{1pt}}
\put(1264,745){\rule{1pt}{1pt}}
\put(1261,724){\rule{1pt}{1pt}}
\put(1259,724){\rule{1pt}{1pt}}
\put(1259,745){\rule{1pt}{1pt}}
\put(1260,724){\rule{1pt}{1pt}}
\put(1258,745){\rule{1pt}{1pt}}
\put(1255,724){\rule{1pt}{1pt}}
\put(1256,745){\rule{1pt}{1pt}}
\put(1256,745){\rule{1pt}{1pt}}
\put(1253,724){\rule{1pt}{1pt}}
\put(1253,724){\rule{1pt}{1pt}}
\put(1250,745){\rule{1pt}{1pt}}
\put(1247,724){\rule{1pt}{1pt}}
\put(1247,734){\rule{1pt}{1pt}}
\put(1245,724){\rule{1pt}{1pt}}
\put(1246,734){\rule{1pt}{1pt}}
\put(1246,724){\rule{1pt}{1pt}}
\put(1243,734){\rule{1pt}{1pt}}
\put(1244,724){\rule{1pt}{1pt}}
\put(1241,734){\rule{1pt}{1pt}}
\put(1241,724){\rule{1pt}{1pt}}
\put(1238,734){\rule{1pt}{1pt}}
\put(1236,734){\rule{1pt}{1pt}}
\put(1236,713){\rule{1pt}{1pt}}
\put(1233,703){\rule{1pt}{1pt}}
\put(1234,734){\rule{1pt}{1pt}}
\put(1232,703){\rule{1pt}{1pt}}
\put(1233,734){\rule{1pt}{1pt}}
\put(1234,703){\rule{1pt}{1pt}}
\put(1230,734){\rule{1pt}{1pt}}
\put(1227,703){\rule{1pt}{1pt}}
\put(1228,734){\rule{1pt}{1pt}}
\put(1225,724){\rule{1pt}{1pt}}
\put(1225,703){\rule{1pt}{1pt}}
\put(1226,713){\rule{1pt}{1pt}}
\put(1223,703){\rule{1pt}{1pt}}
\put(1220,713){\rule{1pt}{1pt}}
\put(1221,703){\rule{1pt}{1pt}}
\put(1218,713){\rule{1pt}{1pt}}
\put(1218,703){\rule{1pt}{1pt}}
\put(1219,713){\rule{1pt}{1pt}}
\put(1215,703){\rule{1pt}{1pt}}
\put(1213,713){\rule{1pt}{1pt}}
\put(1213,703){\rule{1pt}{1pt}}
\put(1213,713){\rule{1pt}{1pt}}
\put(1210,693){\rule{1pt}{1pt}}
\put(1210,713){\rule{1pt}{1pt}}
\put(1207,693){\rule{1pt}{1pt}}
\put(1207,693){\rule{1pt}{1pt}}
\put(1204,713){\rule{1pt}{1pt}}
\put(1205,713){\rule{1pt}{1pt}}
\put(1202,693){\rule{1pt}{1pt}}
\put(1198,713){\rule{1pt}{1pt}}
\put(1198,693){\rule{1pt}{1pt}}
\put(1199,713){\rule{1pt}{1pt}}
\put(1196,693){\rule{1pt}{1pt}}
\put(1196,703){\rule{1pt}{1pt}}
\put(1193,693){\rule{1pt}{1pt}}
\put(1190,693){\rule{1pt}{1pt}}
\put(1191,703){\rule{1pt}{1pt}}
\put(1191,693){\rule{1pt}{1pt}}
\put(1188,703){\rule{1pt}{1pt}}
\put(1185,703){\rule{1pt}{1pt}}
\put(1185,682){\rule{1pt}{1pt}}
\put(1185,682){\rule{1pt}{1pt}}
\put(1182,703){\rule{1pt}{1pt}}
\put(1182,682){\rule{1pt}{1pt}}
\put(1179,703){\rule{1pt}{1pt}}
\put(1179,703){\rule{1pt}{1pt}}
\put(1176,682){\rule{1pt}{1pt}}
\put(1173,703){\rule{1pt}{1pt}}
\put(1174,682){\rule{1pt}{1pt}}
\put(1176,703){\rule{1pt}{1pt}}
\put(1172,682){\rule{1pt}{1pt}}
\put(1172,693){\rule{1pt}{1pt}}
\put(1168,682){\rule{1pt}{1pt}}
\put(1165,682){\rule{1pt}{1pt}}
\put(1165,693){\rule{1pt}{1pt}}
\put(1166,703){\rule{1pt}{1pt}}
\put(1162,682){\rule{1pt}{1pt}}
\put(1159,682){\rule{1pt}{1pt}}
\put(1159,693){\rule{1pt}{1pt}}
\put(1156,682){\rule{1pt}{1pt}}
\put(1157,693){\rule{1pt}{1pt}}
\put(1157,693){\rule{1pt}{1pt}}
\put(1153,672){\rule{1pt}{1pt}}
\put(1149,672){\rule{1pt}{1pt}}
\put(1151,693){\rule{1pt}{1pt}}
\put(1148,672){\rule{1pt}{1pt}}
\put(1150,693){\rule{1pt}{1pt}}
\put(1151,693){\rule{1pt}{1pt}}
\put(1147,672){\rule{1pt}{1pt}}
\put(1147,693){\rule{1pt}{1pt}}
\put(1144,672){\rule{1pt}{1pt}}
\put(1141,672){\rule{1pt}{1pt}}
\put(1142,693){\rule{1pt}{1pt}}
\put(1143,672){\rule{1pt}{1pt}}
\put(1140,682){\rule{1pt}{1pt}}
\put(1137,672){\rule{1pt}{1pt}}
\put(1137,682){\rule{1pt}{1pt}}
\put(1137,672){\rule{1pt}{1pt}}
\put(1134,682){\rule{1pt}{1pt}}
\put(1134,682){\rule{1pt}{1pt}}
\put(1131,651){\rule{1pt}{1pt}}
\put(1128,682){\rule{1pt}{1pt}}
\put(1128,651){\rule{1pt}{1pt}}
\put(1125,682){\rule{1pt}{1pt}}
\put(1126,651){\rule{1pt}{1pt}}
\put(1127,682){\rule{1pt}{1pt}}
\put(1124,651){\rule{1pt}{1pt}}
\put(1124,651){\rule{1pt}{1pt}}
\put(1121,682){\rule{1pt}{1pt}}
\put(1121,651){\rule{1pt}{1pt}}
\put(1118,672){\rule{1pt}{1pt}}
\put(1115,661){\rule{1pt}{1pt}}
\put(1115,651){\rule{1pt}{1pt}}
\put(1112,661){\rule{1pt}{1pt}}
\put(1112,651){\rule{1pt}{1pt}}
\put(1113,661){\rule{1pt}{1pt}}
\put(1110,651){\rule{1pt}{1pt}}
\put(1106,661){\rule{1pt}{1pt}}
\put(1107,651){\rule{1pt}{1pt}}
\put(1106,661){\rule{1pt}{1pt}}
\put(1104,641){\rule{1pt}{1pt}}
\put(1104,661){\rule{1pt}{1pt}}
\put(1101,641){\rule{1pt}{1pt}}
\put(1098,661){\rule{1pt}{1pt}}
\put(1098,641){\rule{1pt}{1pt}}
\put(1094,641){\rule{1pt}{1pt}}
\put(1094,661){\rule{1pt}{1pt}}
\put(1094,661){\rule{1pt}{1pt}}
\put(1091,641){\rule{1pt}{1pt}}
\put(1088,661){\rule{1pt}{1pt}}
\put(1089,641){\rule{1pt}{1pt}}
\put(1089,651){\rule{1pt}{1pt}}
\put(1086,641){\rule{1pt}{1pt}}
\put(1082,641){\rule{1pt}{1pt}}
\put(1083,651){\rule{1pt}{1pt}}
\put(1079,651){\rule{1pt}{1pt}}
\put(1080,641){\rule{1pt}{1pt}}
\put(1076,651){\rule{1pt}{1pt}}
\put(1077,630){\rule{1pt}{1pt}}
\put(1078,651){\rule{1pt}{1pt}}
\put(1074,630){\rule{1pt}{1pt}}
\put(1071,651){\rule{1pt}{1pt}}
\put(1071,630){\rule{1pt}{1pt}}
\put(1067,630){\rule{1pt}{1pt}}
\put(1068,651){\rule{1pt}{1pt}}
\put(1065,651){\rule{1pt}{1pt}}
\put(1066,630){\rule{1pt}{1pt}}
\put(1067,651){\rule{1pt}{1pt}}
\put(1063,630){\rule{1pt}{1pt}}
\put(1060,641){\rule{1pt}{1pt}}
\put(1060,630){\rule{1pt}{1pt}}
\put(1062,630){\rule{1pt}{1pt}}
\put(1058,641){\rule{1pt}{1pt}}
\put(1055,641){\rule{1pt}{1pt}}
\put(1056,630){\rule{1pt}{1pt}}
\put(1057,641){\rule{1pt}{1pt}}
\put(1053,620){\rule{1pt}{1pt}}
\put(1054,641){\rule{1pt}{1pt}}
\put(1050,620){\rule{1pt}{1pt}}
\put(1047,641){\rule{1pt}{1pt}}
\put(1049,620){\rule{1pt}{1pt}}
\put(1050,620){\rule{1pt}{1pt}}
\put(1047,641){\rule{1pt}{1pt}}
\put(1047,620){\rule{1pt}{1pt}}
\put(1043,641){\rule{1pt}{1pt}}
\put(1045,620){\rule{1pt}{1pt}}
\put(1041,630){\rule{1pt}{1pt}}
\put(1037,620){\rule{1pt}{1pt}}
\put(1037,630){\rule{1pt}{1pt}}
\put(1038,630){\rule{1pt}{1pt}}
\put(1035,620){\rule{1pt}{1pt}}
\put(1031,620){\rule{1pt}{1pt}}
\put(1032,630){\rule{1pt}{1pt}}
\put(1032,630){\rule{1pt}{1pt}}
\put(1028,609){\rule{1pt}{1pt}}
\put(1030,609){\rule{1pt}{1pt}}
\put(1026,630){\rule{1pt}{1pt}}
\put(1026,609){\rule{1pt}{1pt}}
\put(1023,630){\rule{1pt}{1pt}}
\put(1019,609){\rule{1pt}{1pt}}
\put(1020,630){\rule{1pt}{1pt}}
\put(1016,630){\rule{1pt}{1pt}}
\put(1016,609){\rule{1pt}{1pt}}
\put(1018,609){\rule{1pt}{1pt}}
\put(1014,630){\rule{1pt}{1pt}}
\put(1011,620){\rule{1pt}{1pt}}
\put(1011,609){\rule{1pt}{1pt}}
\put(1008,609){\rule{1pt}{1pt}}
\put(1009,620){\rule{1pt}{1pt}}
\put(1006,620){\rule{1pt}{1pt}}
\put(1006,589){\rule{1pt}{1pt}}
\put(1003,620){\rule{1pt}{1pt}}
\put(1004,589){\rule{1pt}{1pt}}
\put(1004,589){\rule{1pt}{1pt}}
\put(1001,620){\rule{1pt}{1pt}}
\put(998,589){\rule{1pt}{1pt}}
\put(998,620){\rule{1pt}{1pt}}
\put(995,620){\rule{1pt}{1pt}}
\put(996,589){\rule{1pt}{1pt}}
\put(996,589){\rule{1pt}{1pt}}
\put(992,620){\rule{1pt}{1pt}}
\put(993,589){\rule{1pt}{1pt}}
\put(989,609){\rule{1pt}{1pt}}
\put(989,599){\rule{1pt}{1pt}}
\put(986,589){\rule{1pt}{1pt}}
\put(986,599){\rule{1pt}{1pt}}
\put(982,589){\rule{1pt}{1pt}}
\put(979,599){\rule{1pt}{1pt}}
\put(980,578){\rule{1pt}{1pt}}
\put(977,599){\rule{1pt}{1pt}}
\put(977,578){\rule{1pt}{1pt}}
\put(974,599){\rule{1pt}{1pt}}
\put(975,578){\rule{1pt}{1pt}}
\put(972,578){\rule{1pt}{1pt}}
\put(972,599){\rule{1pt}{1pt}}
\put(972,599){\rule{1pt}{1pt}}
\put(969,578){\rule{1pt}{1pt}}
\put(969,599){\rule{1pt}{1pt}}
\put(965,578){\rule{1pt}{1pt}}
\put(962,578){\rule{1pt}{1pt}}
\put(964,589){\rule{1pt}{1pt}}
\put(964,589){\rule{1pt}{1pt}}
\put(960,578){\rule{1pt}{1pt}}
\put(960,578){\rule{1pt}{1pt}}
\put(957,589){\rule{1pt}{1pt}}
\put(953,589){\rule{1pt}{1pt}}
\put(954,568){\rule{1pt}{1pt}}
\put(951,589){\rule{1pt}{1pt}}
\put(951,568){\rule{1pt}{1pt}}
\put(947,568){\rule{1pt}{1pt}}
\put(948,589){\rule{1pt}{1pt}}
\put(945,568){\rule{1pt}{1pt}}
\put(945,589){\rule{1pt}{1pt}}
\put(946,568){\rule{1pt}{1pt}}
\put(942,589){\rule{1pt}{1pt}}
\put(939,578){\rule{1pt}{1pt}}
\put(939,568){\rule{1pt}{1pt}}
\put(938,578){\rule{1pt}{1pt}}
\put(935,568){\rule{1pt}{1pt}}
\put(931,578){\rule{1pt}{1pt}}
\put(932,557){\rule{1pt}{1pt}}
\put(928,578){\rule{1pt}{1pt}}
\put(929,557){\rule{1pt}{1pt}}
\put(929,557){\rule{1pt}{1pt}}
\put(925,578){\rule{1pt}{1pt}}
\put(926,557){\rule{1pt}{1pt}}
\put(922,578){\rule{1pt}{1pt}}
\put(922,557){\rule{1pt}{1pt}}
\put(919,578){\rule{1pt}{1pt}}
\put(916,557){\rule{1pt}{1pt}}
\put(916,568){\rule{1pt}{1pt}}
\put(915,557){\rule{1pt}{1pt}}
\put(912,568){\rule{1pt}{1pt}}
\put(911,568){\rule{1pt}{1pt}}
\put(908,547){\rule{1pt}{1pt}}
\put(904,547){\rule{1pt}{1pt}}
\put(905,568){\rule{1pt}{1pt}}
\put(902,568){\rule{1pt}{1pt}}
\put(901,537){\rule{1pt}{1pt}}
\put(898,568){\rule{1pt}{1pt}}
\put(897,537){\rule{1pt}{1pt}}
\put(894,568){\rule{1pt}{1pt}}
\put(894,537){\rule{1pt}{1pt}}
\put(890,547){\rule{1pt}{1pt}}
\put(890,537){\rule{1pt}{1pt}}
\put(887,547){\rule{1pt}{1pt}}
\put(888,537){\rule{1pt}{1pt}}
\put(889,537){\rule{1pt}{1pt}}
\put(885,547){\rule{1pt}{1pt}}
\put(881,547){\rule{1pt}{1pt}}
\put(882,526){\rule{1pt}{1pt}}
\put(877,547){\rule{1pt}{1pt}}
\put(879,526){\rule{1pt}{1pt}}
\put(880,526){\rule{1pt}{1pt}}
\put(876,547){\rule{1pt}{1pt}}
\put(877,547){\rule{1pt}{1pt}}
\put(872,526){\rule{1pt}{1pt}}
\put(868,547){\rule{1pt}{1pt}}
\put(869,526){\rule{1pt}{1pt}}
\put(866,537){\rule{1pt}{1pt}}
\put(866,526){\rule{1pt}{1pt}}
\put(866,526){\rule{1pt}{1pt}}
\put(862,537){\rule{1pt}{1pt}}
\put(858,537){\rule{1pt}{1pt}}
\put(858,526){\rule{1pt}{1pt}}
\put(855,537){\rule{1pt}{1pt}}
\put(855,516){\rule{1pt}{1pt}}
\put(855,537){\rule{1pt}{1pt}}
\put(851,516){\rule{1pt}{1pt}}
\put(848,516){\rule{1pt}{1pt}}
\put(848,537){\rule{1pt}{1pt}}
\put(844,537){\rule{1pt}{1pt}}
\put(844,516){\rule{1pt}{1pt}}
\put(844,516){\rule{1pt}{1pt}}
\put(840,537){\rule{1pt}{1pt}}
\put(841,516){\rule{1pt}{1pt}}
\put(838,526){\rule{1pt}{1pt}}
\put(837,526){\rule{1pt}{1pt}}
\put(833,505){\rule{1pt}{1pt}}
\put(833,526){\rule{1pt}{1pt}}
\put(829,505){\rule{1pt}{1pt}}
\put(830,526){\rule{1pt}{1pt}}
\put(826,505){\rule{1pt}{1pt}}
\put(826,505){\rule{1pt}{1pt}}
\put(822,526){\rule{1pt}{1pt}}
\put(818,505){\rule{1pt}{1pt}}
\put(817,526){\rule{1pt}{1pt}}
\put(814,505){\rule{1pt}{1pt}}
\put(814,526){\rule{1pt}{1pt}}
\put(814,505){\rule{1pt}{1pt}}
\put(811,516){\rule{1pt}{1pt}}
\put(807,505){\rule{1pt}{1pt}}
\put(807,516){\rule{1pt}{1pt}}
\put(804,516){\rule{1pt}{1pt}}
\put(803,495){\rule{1pt}{1pt}}
\put(800,495){\rule{1pt}{1pt}}
\put(800,516){\rule{1pt}{1pt}}
\put(796,495){\rule{1pt}{1pt}}
\put(796,516){\rule{1pt}{1pt}}
\put(792,495){\rule{1pt}{1pt}}
\put(792,516){\rule{1pt}{1pt}}
\put(792,495){\rule{1pt}{1pt}}
\put(788,505){\rule{1pt}{1pt}}
\put(785,495){\rule{1pt}{1pt}}
\put(785,505){\rule{1pt}{1pt}}
\put(784,495){\rule{1pt}{1pt}}
\put(781,505){\rule{1pt}{1pt}}
\put(780,474){\rule{1pt}{1pt}}
\put(777,505){\rule{1pt}{1pt}}
\put(776,474){\rule{1pt}{1pt}}
\put(773,505){\rule{1pt}{1pt}}
\put(772,505){\rule{1pt}{1pt}}
\put(768,474){\rule{1pt}{1pt}}
\put(765,505){\rule{1pt}{1pt}}
\put(765,474){\rule{1pt}{1pt}}
\put(764,485){\rule{1pt}{1pt}}
\put(760,474){\rule{1pt}{1pt}}
\put(757,485){\rule{1pt}{1pt}}
\put(756,474){\rule{1pt}{1pt}}
\put(753,474){\rule{1pt}{1pt}}
\put(753,485){\rule{1pt}{1pt}}
\put(752,485){\rule{1pt}{1pt}}
\put(749,464){\rule{1pt}{1pt}}
\put(745,485){\rule{1pt}{1pt}}
\put(745,464){\rule{1pt}{1pt}}
\put(742,485){\rule{1pt}{1pt}}
\put(741,464){\rule{1pt}{1pt}}
\put(741,485){\rule{1pt}{1pt}}
\put(737,464){\rule{1pt}{1pt}}
\put(734,474){\rule{1pt}{1pt}}
\put(734,464){\rule{1pt}{1pt}}
\put(733,474){\rule{1pt}{1pt}}
\put(729,464){\rule{1pt}{1pt}}
\put(726,474){\rule{1pt}{1pt}}
\put(725,464){\rule{1pt}{1pt}}
\put(722,474){\rule{1pt}{1pt}}
\put(722,453){\rule{1pt}{1pt}}
\put(721,474){\rule{1pt}{1pt}}
\put(718,453){\rule{1pt}{1pt}}
\put(715,474){\rule{1pt}{1pt}}
\put(714,453){\rule{1pt}{1pt}}
\put(711,453){\rule{1pt}{1pt}}
\put(710,474){\rule{1pt}{1pt}}
\put(710,453){\rule{1pt}{1pt}}
\put(707,464){\rule{1pt}{1pt}}
\put(706,453){\rule{1pt}{1pt}}
\put(702,464){\rule{1pt}{1pt}}
\put(702,464){\rule{1pt}{1pt}}
\put(698,453){\rule{1pt}{1pt}}
\put(698,464){\rule{1pt}{1pt}}
\put(695,443){\rule{1pt}{1pt}}
\put(694,464){\rule{1pt}{1pt}}
\put(691,443){\rule{1pt}{1pt}}
\put(688,443){\rule{1pt}{1pt}}
\put(687,464){\rule{1pt}{1pt}}
\put(686,443){\rule{1pt}{1pt}}
\put(683,464){\rule{1pt}{1pt}}
\put(683,443){\rule{1pt}{1pt}}
\put(680,453){\rule{1pt}{1pt}}
\put(676,443){\rule{1pt}{1pt}}
\put(676,453){\rule{1pt}{1pt}}
\put(673,443){\rule{1pt}{1pt}}
\put(672,453){\rule{1pt}{1pt}}
\put(669,453){\rule{1pt}{1pt}}
\put(668,422){\rule{1pt}{1pt}}
\put(668,453){\rule{1pt}{1pt}}
\put(664,422){\rule{1pt}{1pt}}
\put(661,422){\rule{1pt}{1pt}}
\put(661,453){\rule{1pt}{1pt}}
\put(660,443){\rule{1pt}{1pt}}
\put(657,422){\rule{1pt}{1pt}}
\put(656,433){\rule{1pt}{1pt}}
\put(653,422){\rule{1pt}{1pt}}
\put(652,433){\rule{1pt}{1pt}}
\put(649,422){\rule{1pt}{1pt}}
\put(646,433){\rule{1pt}{1pt}}
\put(646,422){\rule{1pt}{1pt}}
\put(645,433){\rule{1pt}{1pt}}
\put(642,412){\rule{1pt}{1pt}}
\put(639,433){\rule{1pt}{1pt}}
\put(638,412){\rule{1pt}{1pt}}
\put(637,433){\rule{1pt}{1pt}}
\put(634,412){\rule{1pt}{1pt}}
\put(634,433){\rule{1pt}{1pt}}
\put(630,412){\rule{1pt}{1pt}}
\put(627,412){\rule{1pt}{1pt}}
\put(628,422){\rule{1pt}{1pt}}
\put(625,422){\rule{1pt}{1pt}}
\put(624,412){\rule{1pt}{1pt}}
\put(621,412){\rule{1pt}{1pt}}
\put(620,422){\rule{1pt}{1pt}}
\put(617,422){\rule{1pt}{1pt}}
\put(617,412){\rule{1pt}{1pt}}
\put(617,422){\rule{1pt}{1pt}}
\put(613,401){\rule{1pt}{1pt}}
\put(613,401){\rule{1pt}{1pt}}
\put(610,422){\rule{1pt}{1pt}}
\put(607,422){\rule{1pt}{1pt}}
\put(607,401){\rule{1pt}{1pt}}
\put(608,422){\rule{1pt}{1pt}}
\put(604,401){\rule{1pt}{1pt}}
\put(601,412){\rule{1pt}{1pt}}
\put(601,401){\rule{1pt}{1pt}}
\put(598,401){\rule{1pt}{1pt}}
\put(598,412){\rule{1pt}{1pt}}
\put(598,401){\rule{1pt}{1pt}}
\put(595,412){\rule{1pt}{1pt}}
\put(595,412){\rule{1pt}{1pt}}
\put(592,391){\rule{1pt}{1pt}}
\put(589,412){\rule{1pt}{1pt}}
\put(589,391){\rule{1pt}{1pt}}
\put(589,391){\rule{1pt}{1pt}}
\put(586,412){\rule{1pt}{1pt}}
\put(583,412){\rule{1pt}{1pt}}
\put(583,391){\rule{1pt}{1pt}}
\put(583,391){\rule{1pt}{1pt}}
\put(579,412){\rule{1pt}{1pt}}
\put(576,391){\rule{1pt}{1pt}}
\put(577,401){\rule{1pt}{1pt}}
\put(574,391){\rule{1pt}{1pt}}
\put(574,401){\rule{1pt}{1pt}}
\put(571,391){\rule{1pt}{1pt}}
\put(571,401){\rule{1pt}{1pt}}
\put(567,391){\rule{1pt}{1pt}}
\put(568,401){\rule{1pt}{1pt}}
\put(565,401){\rule{1pt}{1pt}}
\put(566,381){\rule{1pt}{1pt}}
\put(562,381){\rule{1pt}{1pt}}
\put(562,401){\rule{1pt}{1pt}}
\put(559,401){\rule{1pt}{1pt}}
\put(559,381){\rule{1pt}{1pt}}
\put(556,381){\rule{1pt}{1pt}}
\put(557,401){\rule{1pt}{1pt}}
\put(554,381){\rule{1pt}{1pt}}
\put(554,391){\rule{1pt}{1pt}}
\put(551,381){\rule{1pt}{1pt}}
\put(551,391){\rule{1pt}{1pt}}
\put(548,381){\rule{1pt}{1pt}}
\put(549,391){\rule{1pt}{1pt}}
\put(545,381){\rule{1pt}{1pt}}
\put(546,391){\rule{1pt}{1pt}}
\put(543,391){\rule{1pt}{1pt}}
\put(544,360){\rule{1pt}{1pt}}
\put(544,391){\rule{1pt}{1pt}}
\put(540,360){\rule{1pt}{1pt}}
\put(537,360){\rule{1pt}{1pt}}
\put(538,391){\rule{1pt}{1pt}}
\put(535,360){\rule{1pt}{1pt}}
\put(537,391){\rule{1pt}{1pt}}
\put(533,381){\rule{1pt}{1pt}}
\put(534,360){\rule{1pt}{1pt}}
\put(531,360){\rule{1pt}{1pt}}
\put(531,370){\rule{1pt}{1pt}}
\put(532,370){\rule{1pt}{1pt}}
\put(528,360){\rule{1pt}{1pt}}
\put(528,370){\rule{1pt}{1pt}}
\put(525,360){\rule{1pt}{1pt}}
\put(522,360){\rule{1pt}{1pt}}
\put(522,370){\rule{1pt}{1pt}}
\put(523,370){\rule{1pt}{1pt}}
\put(520,349){\rule{1pt}{1pt}}
\put(516,370){\rule{1pt}{1pt}}
\put(516,349){\rule{1pt}{1pt}}
\put(513,349){\rule{1pt}{1pt}}
\put(513,370){\rule{1pt}{1pt}}
\put(510,370){\rule{1pt}{1pt}}
\put(510,349){\rule{1pt}{1pt}}
\put(510,370){\rule{1pt}{1pt}}
\put(507,349){\rule{1pt}{1pt}}
\put(508,349){\rule{1pt}{1pt}}
\put(504,370){\rule{1pt}{1pt}}
\put(504,360){\rule{1pt}{1pt}}
\put(501,349){\rule{1pt}{1pt}}
\put(501,349){\rule{1pt}{1pt}}
\put(498,360){\rule{1pt}{1pt}}
\put(498,360){\rule{1pt}{1pt}}
\put(495,349){\rule{1pt}{1pt}}
\put(496,360){\rule{1pt}{1pt}}
\put(493,349){\rule{1pt}{1pt}}
\put(493,360){\rule{1pt}{1pt}}
\put(490,339){\rule{1pt}{1pt}}
\put(490,360){\rule{1pt}{1pt}}
\put(487,339){\rule{1pt}{1pt}}
\put(485,339){\rule{1pt}{1pt}}
\put(485,360){\rule{1pt}{1pt}}
\put(482,360){\rule{1pt}{1pt}}
\put(483,339){\rule{1pt}{1pt}}
\put(481,360){\rule{1pt}{1pt}}
\put(481,339){\rule{1pt}{1pt}}
\put(478,339){\rule{1pt}{1pt}}
\put(478,349){\rule{1pt}{1pt}}
\put(476,339){\rule{1pt}{1pt}}
\put(476,349){\rule{1pt}{1pt}}
\put(474,339){\rule{1pt}{1pt}}
\put(474,349){\rule{1pt}{1pt}}
\put(474,349){\rule{1pt}{1pt}}
\put(471,339){\rule{1pt}{1pt}}
\put(471,349){\rule{1pt}{1pt}}
\put(469,339){\rule{1pt}{1pt}}
\put(466,349){\rule{1pt}{1pt}}
\put(466,329){\rule{1pt}{1pt}}
\put(464,349){\rule{1pt}{1pt}}
\put(464,329){\rule{1pt}{1pt}}
\put(462,329){\rule{1pt}{1pt}}
\put(462,349){\rule{1pt}{1pt}}
\put(462,329){\rule{1pt}{1pt}}
\put(459,349){\rule{1pt}{1pt}}
\put(457,349){\rule{1pt}{1pt}}
\put(457,329){\rule{1pt}{1pt}}
\put(454,329){\rule{1pt}{1pt}}
\put(455,349){\rule{1pt}{1pt}}
\put(452,329){\rule{1pt}{1pt}}
\put(453,339){\rule{1pt}{1pt}}
\put(450,329){\rule{1pt}{1pt}}
\put(451,339){\rule{1pt}{1pt}}
\put(448,329){\rule{1pt}{1pt}}
\put(449,339){\rule{1pt}{1pt}}
\put(449,329){\rule{1pt}{1pt}}
\put(447,339){\rule{1pt}{1pt}}
\put(444,339){\rule{1pt}{1pt}}
\put(445,329){\rule{1pt}{1pt}}
\put(445,329){\rule{1pt}{1pt}}
\put(443,339){\rule{1pt}{1pt}}
\put(443,339){\rule{1pt}{1pt}}
\put(441,308){\rule{1pt}{1pt}}
\put(438,308){\rule{1pt}{1pt}}
\put(439,339){\rule{1pt}{1pt}}
\put(436,339){\rule{1pt}{1pt}}
\put(437,308){\rule{1pt}{1pt}}
\put(435,308){\rule{1pt}{1pt}}
\put(435,339){\rule{1pt}{1pt}}
\put(433,339){\rule{1pt}{1pt}}
\put(434,308){\rule{1pt}{1pt}}
\put(435,308){\rule{1pt}{1pt}}
\put(431,339){\rule{1pt}{1pt}}
\put(432,318){\rule{1pt}{1pt}}
\put(429,308){\rule{1pt}{1pt}}
\put(426,318){\rule{1pt}{1pt}}
\put(428,308){\rule{1pt}{1pt}}
\put(425,318){\rule{1pt}{1pt}}
\put(426,308){\rule{1pt}{1pt}}
\put(423,318){\rule{1pt}{1pt}}
\put(425,308){\rule{1pt}{1pt}}
\put(425,318){\rule{1pt}{1pt}}
\put(422,308){\rule{1pt}{1pt}}
\put(419,318){\rule{1pt}{1pt}}
\put(420,308){\rule{1pt}{1pt}}
\put(417,318){\rule{1pt}{1pt}}
\put(418,308){\rule{1pt}{1pt}}
\put(414,318){\rule{1pt}{1pt}}
\put(416,297){\rule{1pt}{1pt}}
\put(417,318){\rule{1pt}{1pt}}
\put(414,297){\rule{1pt}{1pt}}
\put(411,297){\rule{1pt}{1pt}}
\put(413,318){\rule{1pt}{1pt}}
\put(410,318){\rule{1pt}{1pt}}
\put(411,297){\rule{1pt}{1pt}}
\put(411,318){\rule{1pt}{1pt}}
\put(408,297){\rule{1pt}{1pt}}
\put(409,297){\rule{1pt}{1pt}}
\put(407,318){\rule{1pt}{1pt}}
\put(404,318){\rule{1pt}{1pt}}
\put(406,297){\rule{1pt}{1pt}}
\put(403,308){\rule{1pt}{1pt}}
\put(404,297){\rule{1pt}{1pt}}
\put(402,297){\rule{1pt}{1pt}}
\put(402,308){\rule{1pt}{1pt}}
\put(400,308){\rule{1pt}{1pt}}
\put(401,297){\rule{1pt}{1pt}}
\put(781,778){\rule{1pt}{1pt}}
\put(151.0,131.0){\rule[-0.200pt]{0.400pt}{175.375pt}}
\put(151.0,131.0){\rule[-0.200pt]{310.279pt}{0.400pt}}
\put(1439.0,131.0){\rule[-0.200pt]{0.400pt}{175.375pt}}
\put(151.0,859.0){\rule[-0.200pt]{310.279pt}{0.400pt}}
\end{picture}

\caption{
Volt-amperová charakteristika komory, preložená fitom $U=\(14.308\pm0.015\)10^{-3}I$, kde smernica predstavuje odpor komory $R$.
}\label{G_1}
\end{figure}


\begin{figure}
% GNUPLOT: LaTeX picture
\setlength{\unitlength}{0.240900pt}
\ifx\plotpoint\undefined\newsavebox{\plotpoint}\fi
\sbox{\plotpoint}{\rule[-0.200pt]{0.400pt}{0.400pt}}%
\begin{picture}(1500,900)(0,0)
\sbox{\plotpoint}{\rule[-0.200pt]{0.400pt}{0.400pt}}%
\put(151.0,131.0){\rule[-0.200pt]{4.818pt}{0.400pt}}
\put(131,131){\makebox(0,0)[r]{-10}}
\put(1419.0,131.0){\rule[-0.200pt]{4.818pt}{0.400pt}}
\put(151.0,222.0){\rule[-0.200pt]{4.818pt}{0.400pt}}
\put(131,222){\makebox(0,0)[r]{-5}}
\put(1419.0,222.0){\rule[-0.200pt]{4.818pt}{0.400pt}}
\put(151.0,313.0){\rule[-0.200pt]{4.818pt}{0.400pt}}
\put(131,313){\makebox(0,0)[r]{ 0}}
\put(1419.0,313.0){\rule[-0.200pt]{4.818pt}{0.400pt}}
\put(151.0,404.0){\rule[-0.200pt]{4.818pt}{0.400pt}}
\put(131,404){\makebox(0,0)[r]{ 5}}
\put(1419.0,404.0){\rule[-0.200pt]{4.818pt}{0.400pt}}
\put(151.0,495.0){\rule[-0.200pt]{4.818pt}{0.400pt}}
\put(131,495){\makebox(0,0)[r]{ 10}}
\put(1419.0,495.0){\rule[-0.200pt]{4.818pt}{0.400pt}}
\put(151.0,586.0){\rule[-0.200pt]{4.818pt}{0.400pt}}
\put(131,586){\makebox(0,0)[r]{ 15}}
\put(1419.0,586.0){\rule[-0.200pt]{4.818pt}{0.400pt}}
\put(151.0,677.0){\rule[-0.200pt]{4.818pt}{0.400pt}}
\put(131,677){\makebox(0,0)[r]{ 20}}
\put(1419.0,677.0){\rule[-0.200pt]{4.818pt}{0.400pt}}
\put(151.0,768.0){\rule[-0.200pt]{4.818pt}{0.400pt}}
\put(131,768){\makebox(0,0)[r]{ 25}}
\put(1419.0,768.0){\rule[-0.200pt]{4.818pt}{0.400pt}}
\put(151.0,859.0){\rule[-0.200pt]{4.818pt}{0.400pt}}
\put(131,859){\makebox(0,0)[r]{ 30}}
\put(1419.0,859.0){\rule[-0.200pt]{4.818pt}{0.400pt}}
\put(151.0,131.0){\rule[-0.200pt]{0.400pt}{4.818pt}}
\put(151,90){\makebox(0,0){ 0}}
\put(151.0,839.0){\rule[-0.200pt]{0.400pt}{4.818pt}}
\put(366.0,131.0){\rule[-0.200pt]{0.400pt}{4.818pt}}
\put(366,90){\makebox(0,0){ 2}}
\put(366.0,839.0){\rule[-0.200pt]{0.400pt}{4.818pt}}
\put(580.0,131.0){\rule[-0.200pt]{0.400pt}{4.818pt}}
\put(580,90){\makebox(0,0){ 4}}
\put(580.0,839.0){\rule[-0.200pt]{0.400pt}{4.818pt}}
\put(795.0,131.0){\rule[-0.200pt]{0.400pt}{4.818pt}}
\put(795,90){\makebox(0,0){ 6}}
\put(795.0,839.0){\rule[-0.200pt]{0.400pt}{4.818pt}}
\put(1010.0,131.0){\rule[-0.200pt]{0.400pt}{4.818pt}}
\put(1010,90){\makebox(0,0){ 8}}
\put(1010.0,839.0){\rule[-0.200pt]{0.400pt}{4.818pt}}
\put(1224.0,131.0){\rule[-0.200pt]{0.400pt}{4.818pt}}
\put(1224,90){\makebox(0,0){ 10}}
\put(1224.0,839.0){\rule[-0.200pt]{0.400pt}{4.818pt}}
\put(1439.0,131.0){\rule[-0.200pt]{0.400pt}{4.818pt}}
\put(1439,90){\makebox(0,0){ 12}}
\put(1439.0,839.0){\rule[-0.200pt]{0.400pt}{4.818pt}}
\put(151.0,131.0){\rule[-0.200pt]{0.400pt}{175.375pt}}
\put(151.0,131.0){\rule[-0.200pt]{310.279pt}{0.400pt}}
\put(1439.0,131.0){\rule[-0.200pt]{0.400pt}{175.375pt}}
\put(151.0,859.0){\rule[-0.200pt]{310.279pt}{0.400pt}}
\put(30,495){\makebox(0,0){\popi{U_l}{V}}}
\put(795,29){\makebox(0,0){\popi{t}{ms}}}
\put(509,0){\line(0,1){899}}
\put(1186,0){\line(0,1){899}}
\put(1279,819){\makebox(0,0)[r]{Loop voltage $U_l$}}
\put(1299.0,819.0){\rule[-0.200pt]{24.090pt}{0.400pt}}
\put(152,309){\usebox{\plotpoint}}
\put(151.67,309){\rule{0.400pt}{1.927pt}}
\multiput(151.17,309.00)(1.000,4.000){2}{\rule{0.400pt}{0.964pt}}
\put(152.67,309){\rule{0.400pt}{1.927pt}}
\multiput(152.17,313.00)(1.000,-4.000){2}{\rule{0.400pt}{0.964pt}}
\put(153.67,309){\rule{0.400pt}{1.927pt}}
\multiput(153.17,309.00)(1.000,4.000){2}{\rule{0.400pt}{0.964pt}}
\put(154.67,309){\rule{0.400pt}{1.927pt}}
\multiput(154.17,313.00)(1.000,-4.000){2}{\rule{0.400pt}{0.964pt}}
\put(155.67,309){\rule{0.400pt}{1.927pt}}
\multiput(155.17,309.00)(1.000,4.000){2}{\rule{0.400pt}{0.964pt}}
\put(158.67,309){\rule{0.400pt}{1.927pt}}
\multiput(158.17,313.00)(1.000,-4.000){2}{\rule{0.400pt}{0.964pt}}
\put(159.67,309){\rule{0.400pt}{1.927pt}}
\multiput(159.17,309.00)(1.000,4.000){2}{\rule{0.400pt}{0.964pt}}
\put(160.67,309){\rule{0.400pt}{1.927pt}}
\multiput(160.17,313.00)(1.000,-4.000){2}{\rule{0.400pt}{0.964pt}}
\put(157.0,317.0){\rule[-0.200pt]{0.482pt}{0.400pt}}
\put(162.67,309){\rule{0.400pt}{1.927pt}}
\multiput(162.17,309.00)(1.000,4.000){2}{\rule{0.400pt}{0.964pt}}
\put(163.67,309){\rule{0.400pt}{1.927pt}}
\multiput(163.17,313.00)(1.000,-4.000){2}{\rule{0.400pt}{0.964pt}}
\put(164.67,309){\rule{0.400pt}{1.927pt}}
\multiput(164.17,309.00)(1.000,4.000){2}{\rule{0.400pt}{0.964pt}}
\put(162.0,309.0){\usebox{\plotpoint}}
\put(166.67,309){\rule{0.400pt}{1.927pt}}
\multiput(166.17,313.00)(1.000,-4.000){2}{\rule{0.400pt}{0.964pt}}
\put(166.0,317.0){\usebox{\plotpoint}}
\put(168.67,309){\rule{0.400pt}{1.927pt}}
\multiput(168.17,309.00)(1.000,4.000){2}{\rule{0.400pt}{0.964pt}}
\put(168.0,309.0){\usebox{\plotpoint}}
\put(170.67,309){\rule{0.400pt}{1.927pt}}
\multiput(170.17,313.00)(1.000,-4.000){2}{\rule{0.400pt}{0.964pt}}
\put(170.0,317.0){\usebox{\plotpoint}}
\put(173.67,309){\rule{0.400pt}{1.927pt}}
\multiput(173.17,309.00)(1.000,4.000){2}{\rule{0.400pt}{0.964pt}}
\put(174.67,309){\rule{0.400pt}{1.927pt}}
\multiput(174.17,313.00)(1.000,-4.000){2}{\rule{0.400pt}{0.964pt}}
\put(175.67,309){\rule{0.400pt}{1.927pt}}
\multiput(175.17,309.00)(1.000,4.000){2}{\rule{0.400pt}{0.964pt}}
\put(176.67,309){\rule{0.400pt}{1.927pt}}
\multiput(176.17,313.00)(1.000,-4.000){2}{\rule{0.400pt}{0.964pt}}
\put(177.67,309){\rule{0.400pt}{1.927pt}}
\multiput(177.17,309.00)(1.000,4.000){2}{\rule{0.400pt}{0.964pt}}
\put(172.0,309.0){\rule[-0.200pt]{0.482pt}{0.400pt}}
\put(179.67,309){\rule{0.400pt}{1.927pt}}
\multiput(179.17,313.00)(1.000,-4.000){2}{\rule{0.400pt}{0.964pt}}
\put(179.0,317.0){\usebox{\plotpoint}}
\put(181.67,309){\rule{0.400pt}{1.927pt}}
\multiput(181.17,309.00)(1.000,4.000){2}{\rule{0.400pt}{0.964pt}}
\put(181.0,309.0){\usebox{\plotpoint}}
\put(183.67,309){\rule{0.400pt}{1.927pt}}
\multiput(183.17,313.00)(1.000,-4.000){2}{\rule{0.400pt}{0.964pt}}
\put(184.67,309){\rule{0.400pt}{1.927pt}}
\multiput(184.17,309.00)(1.000,4.000){2}{\rule{0.400pt}{0.964pt}}
\put(185.67,309){\rule{0.400pt}{1.927pt}}
\multiput(185.17,313.00)(1.000,-4.000){2}{\rule{0.400pt}{0.964pt}}
\put(187.17,309){\rule{0.400pt}{1.700pt}}
\multiput(186.17,309.00)(2.000,4.472){2}{\rule{0.400pt}{0.850pt}}
\put(188.67,309){\rule{0.400pt}{1.927pt}}
\multiput(188.17,313.00)(1.000,-4.000){2}{\rule{0.400pt}{0.964pt}}
\put(189.67,309){\rule{0.400pt}{1.927pt}}
\multiput(189.17,309.00)(1.000,4.000){2}{\rule{0.400pt}{0.964pt}}
\put(190.67,309){\rule{0.400pt}{1.927pt}}
\multiput(190.17,313.00)(1.000,-4.000){2}{\rule{0.400pt}{0.964pt}}
\put(191.67,309){\rule{0.400pt}{1.927pt}}
\multiput(191.17,309.00)(1.000,4.000){2}{\rule{0.400pt}{0.964pt}}
\put(192.67,309){\rule{0.400pt}{1.927pt}}
\multiput(192.17,313.00)(1.000,-4.000){2}{\rule{0.400pt}{0.964pt}}
\put(183.0,317.0){\usebox{\plotpoint}}
\put(194.67,309){\rule{0.400pt}{1.927pt}}
\multiput(194.17,309.00)(1.000,4.000){2}{\rule{0.400pt}{0.964pt}}
\put(195.67,309){\rule{0.400pt}{1.927pt}}
\multiput(195.17,313.00)(1.000,-4.000){2}{\rule{0.400pt}{0.964pt}}
\put(196.67,309){\rule{0.400pt}{1.927pt}}
\multiput(196.17,309.00)(1.000,4.000){2}{\rule{0.400pt}{0.964pt}}
\put(197.67,309){\rule{0.400pt}{1.927pt}}
\multiput(197.17,313.00)(1.000,-4.000){2}{\rule{0.400pt}{0.964pt}}
\put(198.67,309){\rule{0.400pt}{1.927pt}}
\multiput(198.17,309.00)(1.000,4.000){2}{\rule{0.400pt}{0.964pt}}
\put(199.67,309){\rule{0.400pt}{1.927pt}}
\multiput(199.17,313.00)(1.000,-4.000){2}{\rule{0.400pt}{0.964pt}}
\put(201.17,309){\rule{0.400pt}{1.700pt}}
\multiput(200.17,309.00)(2.000,4.472){2}{\rule{0.400pt}{0.850pt}}
\put(202.67,309){\rule{0.400pt}{1.927pt}}
\multiput(202.17,313.00)(1.000,-4.000){2}{\rule{0.400pt}{0.964pt}}
\put(203.67,309){\rule{0.400pt}{1.927pt}}
\multiput(203.17,309.00)(1.000,4.000){2}{\rule{0.400pt}{0.964pt}}
\put(204.67,309){\rule{0.400pt}{1.927pt}}
\multiput(204.17,313.00)(1.000,-4.000){2}{\rule{0.400pt}{0.964pt}}
\put(205.67,309){\rule{0.400pt}{1.927pt}}
\multiput(205.17,309.00)(1.000,4.000){2}{\rule{0.400pt}{0.964pt}}
\put(206.67,309){\rule{0.400pt}{1.927pt}}
\multiput(206.17,313.00)(1.000,-4.000){2}{\rule{0.400pt}{0.964pt}}
\put(207.67,309){\rule{0.400pt}{1.927pt}}
\multiput(207.17,309.00)(1.000,4.000){2}{\rule{0.400pt}{0.964pt}}
\put(194.0,309.0){\usebox{\plotpoint}}
\put(209.67,309){\rule{0.400pt}{1.927pt}}
\multiput(209.17,313.00)(1.000,-4.000){2}{\rule{0.400pt}{0.964pt}}
\put(210.67,309){\rule{0.400pt}{1.927pt}}
\multiput(210.17,309.00)(1.000,4.000){2}{\rule{0.400pt}{0.964pt}}
\put(211.67,309){\rule{0.400pt}{1.927pt}}
\multiput(211.17,313.00)(1.000,-4.000){2}{\rule{0.400pt}{0.964pt}}
\put(209.0,317.0){\usebox{\plotpoint}}
\put(213.67,309){\rule{0.400pt}{1.927pt}}
\multiput(213.17,309.00)(1.000,4.000){2}{\rule{0.400pt}{0.964pt}}
\put(214.67,309){\rule{0.400pt}{1.927pt}}
\multiput(214.17,313.00)(1.000,-4.000){2}{\rule{0.400pt}{0.964pt}}
\put(216.17,309){\rule{0.400pt}{1.700pt}}
\multiput(215.17,309.00)(2.000,4.472){2}{\rule{0.400pt}{0.850pt}}
\put(217.67,309){\rule{0.400pt}{1.927pt}}
\multiput(217.17,313.00)(1.000,-4.000){2}{\rule{0.400pt}{0.964pt}}
\put(218.67,309){\rule{0.400pt}{1.927pt}}
\multiput(218.17,309.00)(1.000,4.000){2}{\rule{0.400pt}{0.964pt}}
\put(213.0,309.0){\usebox{\plotpoint}}
\put(220.67,309){\rule{0.400pt}{1.927pt}}
\multiput(220.17,313.00)(1.000,-4.000){2}{\rule{0.400pt}{0.964pt}}
\put(220.0,317.0){\usebox{\plotpoint}}
\put(222.67,309){\rule{0.400pt}{1.927pt}}
\multiput(222.17,309.00)(1.000,4.000){2}{\rule{0.400pt}{0.964pt}}
\put(222.0,309.0){\usebox{\plotpoint}}
\put(224.67,309){\rule{0.400pt}{1.927pt}}
\multiput(224.17,313.00)(1.000,-4.000){2}{\rule{0.400pt}{0.964pt}}
\put(225.67,309){\rule{0.400pt}{1.927pt}}
\multiput(225.17,309.00)(1.000,4.000){2}{\rule{0.400pt}{0.964pt}}
\put(226.67,309){\rule{0.400pt}{1.927pt}}
\multiput(226.17,313.00)(1.000,-4.000){2}{\rule{0.400pt}{0.964pt}}
\put(227.67,309){\rule{0.400pt}{1.927pt}}
\multiput(227.17,309.00)(1.000,4.000){2}{\rule{0.400pt}{0.964pt}}
\put(228.67,309){\rule{0.400pt}{1.927pt}}
\multiput(228.17,313.00)(1.000,-4.000){2}{\rule{0.400pt}{0.964pt}}
\put(224.0,317.0){\usebox{\plotpoint}}
\put(231.67,309){\rule{0.400pt}{1.927pt}}
\multiput(231.17,309.00)(1.000,4.000){2}{\rule{0.400pt}{0.964pt}}
\put(232.67,309){\rule{0.400pt}{1.927pt}}
\multiput(232.17,313.00)(1.000,-4.000){2}{\rule{0.400pt}{0.964pt}}
\put(233.67,309){\rule{0.400pt}{1.927pt}}
\multiput(233.17,309.00)(1.000,4.000){2}{\rule{0.400pt}{0.964pt}}
\put(230.0,309.0){\rule[-0.200pt]{0.482pt}{0.400pt}}
\put(235.67,309){\rule{0.400pt}{1.927pt}}
\multiput(235.17,313.00)(1.000,-4.000){2}{\rule{0.400pt}{0.964pt}}
\put(236.67,309){\rule{0.400pt}{1.927pt}}
\multiput(236.17,309.00)(1.000,4.000){2}{\rule{0.400pt}{0.964pt}}
\put(237.67,309){\rule{0.400pt}{1.927pt}}
\multiput(237.17,313.00)(1.000,-4.000){2}{\rule{0.400pt}{0.964pt}}
\put(235.0,317.0){\usebox{\plotpoint}}
\put(239.67,309){\rule{0.400pt}{1.927pt}}
\multiput(239.17,309.00)(1.000,4.000){2}{\rule{0.400pt}{0.964pt}}
\put(240.67,309){\rule{0.400pt}{1.927pt}}
\multiput(240.17,313.00)(1.000,-4.000){2}{\rule{0.400pt}{0.964pt}}
\put(241.67,309){\rule{0.400pt}{1.927pt}}
\multiput(241.17,309.00)(1.000,4.000){2}{\rule{0.400pt}{0.964pt}}
\put(239.0,309.0){\usebox{\plotpoint}}
\put(243.67,309){\rule{0.400pt}{1.927pt}}
\multiput(243.17,313.00)(1.000,-4.000){2}{\rule{0.400pt}{0.964pt}}
\put(245.17,309){\rule{0.400pt}{1.700pt}}
\multiput(244.17,309.00)(2.000,4.472){2}{\rule{0.400pt}{0.850pt}}
\put(246.67,309){\rule{0.400pt}{1.927pt}}
\multiput(246.17,313.00)(1.000,-4.000){2}{\rule{0.400pt}{0.964pt}}
\put(247.67,309){\rule{0.400pt}{1.927pt}}
\multiput(247.17,309.00)(1.000,4.000){2}{\rule{0.400pt}{0.964pt}}
\put(248.67,309){\rule{0.400pt}{1.927pt}}
\multiput(248.17,313.00)(1.000,-4.000){2}{\rule{0.400pt}{0.964pt}}
\put(249.67,309){\rule{0.400pt}{1.927pt}}
\multiput(249.17,309.00)(1.000,4.000){2}{\rule{0.400pt}{0.964pt}}
\put(250.67,309){\rule{0.400pt}{1.927pt}}
\multiput(250.17,313.00)(1.000,-4.000){2}{\rule{0.400pt}{0.964pt}}
\put(251.67,309){\rule{0.400pt}{1.927pt}}
\multiput(251.17,309.00)(1.000,4.000){2}{\rule{0.400pt}{0.964pt}}
\put(252.67,309){\rule{0.400pt}{1.927pt}}
\multiput(252.17,313.00)(1.000,-4.000){2}{\rule{0.400pt}{0.964pt}}
\put(243.0,317.0){\usebox{\plotpoint}}
\put(254.67,309){\rule{0.400pt}{1.927pt}}
\multiput(254.17,309.00)(1.000,4.000){2}{\rule{0.400pt}{0.964pt}}
\put(255.67,309){\rule{0.400pt}{1.927pt}}
\multiput(255.17,313.00)(1.000,-4.000){2}{\rule{0.400pt}{0.964pt}}
\put(256.67,309){\rule{0.400pt}{1.927pt}}
\multiput(256.17,309.00)(1.000,4.000){2}{\rule{0.400pt}{0.964pt}}
\put(257.67,309){\rule{0.400pt}{1.927pt}}
\multiput(257.17,313.00)(1.000,-4.000){2}{\rule{0.400pt}{0.964pt}}
\put(258.67,309){\rule{0.400pt}{1.927pt}}
\multiput(258.17,309.00)(1.000,4.000){2}{\rule{0.400pt}{0.964pt}}
\put(260.17,309){\rule{0.400pt}{1.700pt}}
\multiput(259.17,313.47)(2.000,-4.472){2}{\rule{0.400pt}{0.850pt}}
\put(261.67,309){\rule{0.400pt}{1.927pt}}
\multiput(261.17,309.00)(1.000,4.000){2}{\rule{0.400pt}{0.964pt}}
\put(254.0,309.0){\usebox{\plotpoint}}
\put(263.67,309){\rule{0.400pt}{1.927pt}}
\multiput(263.17,313.00)(1.000,-4.000){2}{\rule{0.400pt}{0.964pt}}
\put(264.67,309){\rule{0.400pt}{1.927pt}}
\multiput(264.17,309.00)(1.000,4.000){2}{\rule{0.400pt}{0.964pt}}
\put(265.67,309){\rule{0.400pt}{1.927pt}}
\multiput(265.17,313.00)(1.000,-4.000){2}{\rule{0.400pt}{0.964pt}}
\put(266.67,309){\rule{0.400pt}{1.927pt}}
\multiput(266.17,309.00)(1.000,4.000){2}{\rule{0.400pt}{0.964pt}}
\put(267.67,309){\rule{0.400pt}{1.927pt}}
\multiput(267.17,313.00)(1.000,-4.000){2}{\rule{0.400pt}{0.964pt}}
\put(263.0,317.0){\usebox{\plotpoint}}
\put(269.67,309){\rule{0.400pt}{1.927pt}}
\multiput(269.17,309.00)(1.000,4.000){2}{\rule{0.400pt}{0.964pt}}
\put(270.67,309){\rule{0.400pt}{1.927pt}}
\multiput(270.17,313.00)(1.000,-4.000){2}{\rule{0.400pt}{0.964pt}}
\put(271.67,309){\rule{0.400pt}{1.927pt}}
\multiput(271.17,309.00)(1.000,4.000){2}{\rule{0.400pt}{0.964pt}}
\put(272.67,309){\rule{0.400pt}{1.927pt}}
\multiput(272.17,313.00)(1.000,-4.000){2}{\rule{0.400pt}{0.964pt}}
\put(274.17,309){\rule{0.400pt}{1.700pt}}
\multiput(273.17,309.00)(2.000,4.472){2}{\rule{0.400pt}{0.850pt}}
\put(269.0,309.0){\usebox{\plotpoint}}
\put(276.67,309){\rule{0.400pt}{1.927pt}}
\multiput(276.17,313.00)(1.000,-4.000){2}{\rule{0.400pt}{0.964pt}}
\put(277.67,309){\rule{0.400pt}{1.927pt}}
\multiput(277.17,309.00)(1.000,4.000){2}{\rule{0.400pt}{0.964pt}}
\put(278.67,309){\rule{0.400pt}{1.927pt}}
\multiput(278.17,313.00)(1.000,-4.000){2}{\rule{0.400pt}{0.964pt}}
\put(276.0,317.0){\usebox{\plotpoint}}
\put(280.67,309){\rule{0.400pt}{1.927pt}}
\multiput(280.17,309.00)(1.000,4.000){2}{\rule{0.400pt}{0.964pt}}
\put(280.0,309.0){\usebox{\plotpoint}}
\put(282.67,309){\rule{0.400pt}{1.927pt}}
\multiput(282.17,313.00)(1.000,-4.000){2}{\rule{0.400pt}{0.964pt}}
\put(282.0,317.0){\usebox{\plotpoint}}
\put(284.67,309){\rule{0.400pt}{1.927pt}}
\multiput(284.17,309.00)(1.000,4.000){2}{\rule{0.400pt}{0.964pt}}
\put(284.0,309.0){\usebox{\plotpoint}}
\put(286.67,309){\rule{0.400pt}{1.927pt}}
\multiput(286.17,313.00)(1.000,-4.000){2}{\rule{0.400pt}{0.964pt}}
\put(286.0,317.0){\usebox{\plotpoint}}
\put(289.17,309){\rule{0.400pt}{1.700pt}}
\multiput(288.17,309.00)(2.000,4.472){2}{\rule{0.400pt}{0.850pt}}
\put(290.67,309){\rule{0.400pt}{1.927pt}}
\multiput(290.17,313.00)(1.000,-4.000){2}{\rule{0.400pt}{0.964pt}}
\put(291.67,309){\rule{0.400pt}{1.927pt}}
\multiput(291.17,309.00)(1.000,4.000){2}{\rule{0.400pt}{0.964pt}}
\put(292.67,309){\rule{0.400pt}{1.927pt}}
\multiput(292.17,313.00)(1.000,-4.000){2}{\rule{0.400pt}{0.964pt}}
\put(293.67,309){\rule{0.400pt}{1.927pt}}
\multiput(293.17,309.00)(1.000,4.000){2}{\rule{0.400pt}{0.964pt}}
\put(294.67,309){\rule{0.400pt}{1.927pt}}
\multiput(294.17,313.00)(1.000,-4.000){2}{\rule{0.400pt}{0.964pt}}
\put(295.67,309){\rule{0.400pt}{1.927pt}}
\multiput(295.17,309.00)(1.000,4.000){2}{\rule{0.400pt}{0.964pt}}
\put(296.67,309){\rule{0.400pt}{1.927pt}}
\multiput(296.17,313.00)(1.000,-4.000){2}{\rule{0.400pt}{0.964pt}}
\put(297.67,309){\rule{0.400pt}{1.927pt}}
\multiput(297.17,309.00)(1.000,4.000){2}{\rule{0.400pt}{0.964pt}}
\put(288.0,309.0){\usebox{\plotpoint}}
\put(299.67,309){\rule{0.400pt}{1.927pt}}
\multiput(299.17,313.00)(1.000,-4.000){2}{\rule{0.400pt}{0.964pt}}
\put(300.67,309){\rule{0.400pt}{1.927pt}}
\multiput(300.17,309.00)(1.000,4.000){2}{\rule{0.400pt}{0.964pt}}
\put(301.67,309){\rule{0.400pt}{1.927pt}}
\multiput(301.17,313.00)(1.000,-4.000){2}{\rule{0.400pt}{0.964pt}}
\put(302.67,309){\rule{0.400pt}{1.927pt}}
\multiput(302.17,309.00)(1.000,4.000){2}{\rule{0.400pt}{0.964pt}}
\put(304.17,309){\rule{0.400pt}{1.700pt}}
\multiput(303.17,313.47)(2.000,-4.472){2}{\rule{0.400pt}{0.850pt}}
\put(299.0,317.0){\usebox{\plotpoint}}
\put(306.67,309){\rule{0.400pt}{1.927pt}}
\multiput(306.17,309.00)(1.000,4.000){2}{\rule{0.400pt}{0.964pt}}
\put(307.67,309){\rule{0.400pt}{1.927pt}}
\multiput(307.17,313.00)(1.000,-4.000){2}{\rule{0.400pt}{0.964pt}}
\put(308.67,309){\rule{0.400pt}{1.927pt}}
\multiput(308.17,309.00)(1.000,4.000){2}{\rule{0.400pt}{0.964pt}}
\put(309.67,309){\rule{0.400pt}{1.927pt}}
\multiput(309.17,313.00)(1.000,-4.000){2}{\rule{0.400pt}{0.964pt}}
\put(310.67,309){\rule{0.400pt}{1.927pt}}
\multiput(310.17,309.00)(1.000,4.000){2}{\rule{0.400pt}{0.964pt}}
\put(311.67,309){\rule{0.400pt}{1.927pt}}
\multiput(311.17,313.00)(1.000,-4.000){2}{\rule{0.400pt}{0.964pt}}
\put(312.67,309){\rule{0.400pt}{1.927pt}}
\multiput(312.17,309.00)(1.000,4.000){2}{\rule{0.400pt}{0.964pt}}
\put(313.67,309){\rule{0.400pt}{1.927pt}}
\multiput(313.17,313.00)(1.000,-4.000){2}{\rule{0.400pt}{0.964pt}}
\put(314.67,309){\rule{0.400pt}{1.927pt}}
\multiput(314.17,309.00)(1.000,4.000){2}{\rule{0.400pt}{0.964pt}}
\put(315.67,309){\rule{0.400pt}{1.927pt}}
\multiput(315.17,313.00)(1.000,-4.000){2}{\rule{0.400pt}{0.964pt}}
\put(316.67,309){\rule{0.400pt}{1.927pt}}
\multiput(316.17,309.00)(1.000,4.000){2}{\rule{0.400pt}{0.964pt}}
\put(318.17,309){\rule{0.400pt}{1.700pt}}
\multiput(317.17,313.47)(2.000,-4.472){2}{\rule{0.400pt}{0.850pt}}
\put(319.67,309){\rule{0.400pt}{1.927pt}}
\multiput(319.17,309.00)(1.000,4.000){2}{\rule{0.400pt}{0.964pt}}
\put(320.67,309){\rule{0.400pt}{1.927pt}}
\multiput(320.17,313.00)(1.000,-4.000){2}{\rule{0.400pt}{0.964pt}}
\put(321.67,309){\rule{0.400pt}{1.927pt}}
\multiput(321.17,309.00)(1.000,4.000){2}{\rule{0.400pt}{0.964pt}}
\put(322.67,309){\rule{0.400pt}{1.927pt}}
\multiput(322.17,313.00)(1.000,-4.000){2}{\rule{0.400pt}{0.964pt}}
\put(323.67,309){\rule{0.400pt}{1.927pt}}
\multiput(323.17,309.00)(1.000,4.000){2}{\rule{0.400pt}{0.964pt}}
\put(324.67,309){\rule{0.400pt}{1.927pt}}
\multiput(324.17,313.00)(1.000,-4.000){2}{\rule{0.400pt}{0.964pt}}
\put(325.67,309){\rule{0.400pt}{1.927pt}}
\multiput(325.17,309.00)(1.000,4.000){2}{\rule{0.400pt}{0.964pt}}
\put(306.0,309.0){\usebox{\plotpoint}}
\put(327.67,309){\rule{0.400pt}{1.927pt}}
\multiput(327.17,313.00)(1.000,-4.000){2}{\rule{0.400pt}{0.964pt}}
\put(328.67,309){\rule{0.400pt}{1.927pt}}
\multiput(328.17,309.00)(1.000,4.000){2}{\rule{0.400pt}{0.964pt}}
\put(329.67,309){\rule{0.400pt}{1.927pt}}
\multiput(329.17,313.00)(1.000,-4.000){2}{\rule{0.400pt}{0.964pt}}
\put(330.67,309){\rule{0.400pt}{1.927pt}}
\multiput(330.17,309.00)(1.000,4.000){2}{\rule{0.400pt}{0.964pt}}
\put(331.67,309){\rule{0.400pt}{1.927pt}}
\multiput(331.17,313.00)(1.000,-4.000){2}{\rule{0.400pt}{0.964pt}}
\put(333.17,309){\rule{0.400pt}{1.700pt}}
\multiput(332.17,309.00)(2.000,4.472){2}{\rule{0.400pt}{0.850pt}}
\put(334.67,309){\rule{0.400pt}{1.927pt}}
\multiput(334.17,313.00)(1.000,-4.000){2}{\rule{0.400pt}{0.964pt}}
\put(327.0,317.0){\usebox{\plotpoint}}
\put(336.67,309){\rule{0.400pt}{1.927pt}}
\multiput(336.17,309.00)(1.000,4.000){2}{\rule{0.400pt}{0.964pt}}
\put(337.67,309){\rule{0.400pt}{1.927pt}}
\multiput(337.17,313.00)(1.000,-4.000){2}{\rule{0.400pt}{0.964pt}}
\put(338.67,309){\rule{0.400pt}{1.927pt}}
\multiput(338.17,309.00)(1.000,4.000){2}{\rule{0.400pt}{0.964pt}}
\put(339.67,309){\rule{0.400pt}{1.927pt}}
\multiput(339.17,313.00)(1.000,-4.000){2}{\rule{0.400pt}{0.964pt}}
\put(340.67,309){\rule{0.400pt}{1.927pt}}
\multiput(340.17,309.00)(1.000,4.000){2}{\rule{0.400pt}{0.964pt}}
\put(336.0,309.0){\usebox{\plotpoint}}
\put(342.67,309){\rule{0.400pt}{1.927pt}}
\multiput(342.17,313.00)(1.000,-4.000){2}{\rule{0.400pt}{0.964pt}}
\put(342.0,317.0){\usebox{\plotpoint}}
\put(344.67,309){\rule{0.400pt}{1.927pt}}
\multiput(344.17,309.00)(1.000,4.000){2}{\rule{0.400pt}{0.964pt}}
\put(344.0,309.0){\usebox{\plotpoint}}
\put(346.67,309){\rule{0.400pt}{1.927pt}}
\multiput(346.17,313.00)(1.000,-4.000){2}{\rule{0.400pt}{0.964pt}}
\put(346.0,317.0){\usebox{\plotpoint}}
\put(349.67,309){\rule{0.400pt}{1.927pt}}
\multiput(349.17,309.00)(1.000,4.000){2}{\rule{0.400pt}{0.964pt}}
\put(348.0,309.0){\rule[-0.200pt]{0.482pt}{0.400pt}}
\put(351.67,309){\rule{0.400pt}{1.927pt}}
\multiput(351.17,313.00)(1.000,-4.000){2}{\rule{0.400pt}{0.964pt}}
\put(351.0,317.0){\usebox{\plotpoint}}
\put(353.67,309){\rule{0.400pt}{1.927pt}}
\multiput(353.17,309.00)(1.000,4.000){2}{\rule{0.400pt}{0.964pt}}
\put(353.0,309.0){\usebox{\plotpoint}}
\put(355.67,309){\rule{0.400pt}{1.927pt}}
\multiput(355.17,313.00)(1.000,-4.000){2}{\rule{0.400pt}{0.964pt}}
\put(356.67,309){\rule{0.400pt}{1.927pt}}
\multiput(356.17,309.00)(1.000,4.000){2}{\rule{0.400pt}{0.964pt}}
\put(357.67,309){\rule{0.400pt}{1.927pt}}
\multiput(357.17,313.00)(1.000,-4.000){2}{\rule{0.400pt}{0.964pt}}
\put(355.0,317.0){\usebox{\plotpoint}}
\put(359.67,309){\rule{0.400pt}{1.927pt}}
\multiput(359.17,309.00)(1.000,4.000){2}{\rule{0.400pt}{0.964pt}}
\put(359.0,309.0){\usebox{\plotpoint}}
\put(362.17,309){\rule{0.400pt}{1.700pt}}
\multiput(361.17,313.47)(2.000,-4.472){2}{\rule{0.400pt}{0.850pt}}
\put(363.67,309){\rule{0.400pt}{1.927pt}}
\multiput(363.17,309.00)(1.000,4.000){2}{\rule{0.400pt}{0.964pt}}
\put(364.67,309){\rule{0.400pt}{1.927pt}}
\multiput(364.17,313.00)(1.000,-4.000){2}{\rule{0.400pt}{0.964pt}}
\put(365.67,309){\rule{0.400pt}{1.927pt}}
\multiput(365.17,309.00)(1.000,4.000){2}{\rule{0.400pt}{0.964pt}}
\put(366.67,309){\rule{0.400pt}{1.927pt}}
\multiput(366.17,313.00)(1.000,-4.000){2}{\rule{0.400pt}{0.964pt}}
\put(361.0,317.0){\usebox{\plotpoint}}
\put(368.67,309){\rule{0.400pt}{1.927pt}}
\multiput(368.17,309.00)(1.000,4.000){2}{\rule{0.400pt}{0.964pt}}
\put(368.0,309.0){\usebox{\plotpoint}}
\put(370.67,309){\rule{0.400pt}{1.927pt}}
\multiput(370.17,313.00)(1.000,-4.000){2}{\rule{0.400pt}{0.964pt}}
\put(370.0,317.0){\usebox{\plotpoint}}
\put(372.67,309){\rule{0.400pt}{1.927pt}}
\multiput(372.17,309.00)(1.000,4.000){2}{\rule{0.400pt}{0.964pt}}
\put(372.0,309.0){\usebox{\plotpoint}}
\put(374.67,309){\rule{0.400pt}{1.927pt}}
\multiput(374.17,313.00)(1.000,-4.000){2}{\rule{0.400pt}{0.964pt}}
\put(375.67,233){\rule{0.400pt}{18.308pt}}
\multiput(375.17,271.00)(1.000,-38.000){2}{\rule{0.400pt}{9.154pt}}
\put(377.17,233){\rule{0.400pt}{27.100pt}}
\multiput(376.17,233.00)(2.000,78.753){2}{\rule{0.400pt}{13.550pt}}
\put(378.67,197){\rule{0.400pt}{41.194pt}}
\multiput(378.17,282.50)(1.000,-85.500){2}{\rule{0.400pt}{20.597pt}}
\put(379.67,197){\rule{0.400pt}{55.889pt}}
\multiput(379.17,197.00)(1.000,116.000){2}{\rule{0.400pt}{27.944pt}}
\put(380.67,317){\rule{0.400pt}{26.981pt}}
\multiput(380.17,373.00)(1.000,-56.000){2}{\rule{0.400pt}{13.490pt}}
\put(381.67,309){\rule{0.400pt}{1.927pt}}
\multiput(381.17,313.00)(1.000,-4.000){2}{\rule{0.400pt}{0.964pt}}
\put(382.67,309){\rule{0.400pt}{1.927pt}}
\multiput(382.17,309.00)(1.000,4.000){2}{\rule{0.400pt}{0.964pt}}
\put(383.67,309){\rule{0.400pt}{1.927pt}}
\multiput(383.17,313.00)(1.000,-4.000){2}{\rule{0.400pt}{0.964pt}}
\put(374.0,317.0){\usebox{\plotpoint}}
\put(385.67,309){\rule{0.400pt}{1.927pt}}
\multiput(385.17,309.00)(1.000,4.000){2}{\rule{0.400pt}{0.964pt}}
\put(386.67,309){\rule{0.400pt}{1.927pt}}
\multiput(386.17,313.00)(1.000,-4.000){2}{\rule{0.400pt}{0.964pt}}
\put(387.67,309){\rule{0.400pt}{1.927pt}}
\multiput(387.17,309.00)(1.000,4.000){2}{\rule{0.400pt}{0.964pt}}
\put(388.67,309){\rule{0.400pt}{1.927pt}}
\multiput(388.17,313.00)(1.000,-4.000){2}{\rule{0.400pt}{0.964pt}}
\put(389.67,309){\rule{0.400pt}{1.927pt}}
\multiput(389.17,309.00)(1.000,4.000){2}{\rule{0.400pt}{0.964pt}}
\put(391.17,309){\rule{0.400pt}{1.700pt}}
\multiput(390.17,313.47)(2.000,-4.472){2}{\rule{0.400pt}{0.850pt}}
\put(392.67,309){\rule{0.400pt}{1.927pt}}
\multiput(392.17,309.00)(1.000,4.000){2}{\rule{0.400pt}{0.964pt}}
\put(393.67,309){\rule{0.400pt}{1.927pt}}
\multiput(393.17,313.00)(1.000,-4.000){2}{\rule{0.400pt}{0.964pt}}
\put(394.67,309){\rule{0.400pt}{1.927pt}}
\multiput(394.17,309.00)(1.000,4.000){2}{\rule{0.400pt}{0.964pt}}
\put(395.67,309){\rule{0.400pt}{1.927pt}}
\multiput(395.17,313.00)(1.000,-4.000){2}{\rule{0.400pt}{0.964pt}}
\put(396.67,309){\rule{0.400pt}{1.927pt}}
\multiput(396.17,309.00)(1.000,4.000){2}{\rule{0.400pt}{0.964pt}}
\put(397.67,309){\rule{0.400pt}{1.927pt}}
\multiput(397.17,313.00)(1.000,-4.000){2}{\rule{0.400pt}{0.964pt}}
\put(398.67,309){\rule{0.400pt}{1.927pt}}
\multiput(398.17,309.00)(1.000,4.000){2}{\rule{0.400pt}{0.964pt}}
\put(399.67,309){\rule{0.400pt}{1.927pt}}
\multiput(399.17,313.00)(1.000,-4.000){2}{\rule{0.400pt}{0.964pt}}
\put(400.67,309){\rule{0.400pt}{1.927pt}}
\multiput(400.17,309.00)(1.000,4.000){2}{\rule{0.400pt}{0.964pt}}
\put(401.67,309){\rule{0.400pt}{1.927pt}}
\multiput(401.17,313.00)(1.000,-4.000){2}{\rule{0.400pt}{0.964pt}}
\put(402.67,309){\rule{0.400pt}{1.927pt}}
\multiput(402.17,309.00)(1.000,4.000){2}{\rule{0.400pt}{0.964pt}}
\put(385.0,309.0){\usebox{\plotpoint}}
\put(404.67,309){\rule{0.400pt}{1.927pt}}
\multiput(404.17,313.00)(1.000,-4.000){2}{\rule{0.400pt}{0.964pt}}
\put(404.0,317.0){\usebox{\plotpoint}}
\put(407.67,309){\rule{0.400pt}{1.927pt}}
\multiput(407.17,309.00)(1.000,4.000){2}{\rule{0.400pt}{0.964pt}}
\put(408.67,309){\rule{0.400pt}{1.927pt}}
\multiput(408.17,313.00)(1.000,-4.000){2}{\rule{0.400pt}{0.964pt}}
\put(409.67,309){\rule{0.400pt}{1.927pt}}
\multiput(409.17,309.00)(1.000,4.000){2}{\rule{0.400pt}{0.964pt}}
\put(410.67,309){\rule{0.400pt}{1.927pt}}
\multiput(410.17,313.00)(1.000,-4.000){2}{\rule{0.400pt}{0.964pt}}
\put(411.67,309){\rule{0.400pt}{1.927pt}}
\multiput(411.17,309.00)(1.000,4.000){2}{\rule{0.400pt}{0.964pt}}
\put(412.67,309){\rule{0.400pt}{1.927pt}}
\multiput(412.17,313.00)(1.000,-4.000){2}{\rule{0.400pt}{0.964pt}}
\put(413.67,309){\rule{0.400pt}{1.927pt}}
\multiput(413.17,309.00)(1.000,4.000){2}{\rule{0.400pt}{0.964pt}}
\put(414.67,309){\rule{0.400pt}{1.927pt}}
\multiput(414.17,313.00)(1.000,-4.000){2}{\rule{0.400pt}{0.964pt}}
\put(415.67,309){\rule{0.400pt}{1.927pt}}
\multiput(415.17,309.00)(1.000,4.000){2}{\rule{0.400pt}{0.964pt}}
\put(416.67,309){\rule{0.400pt}{1.927pt}}
\multiput(416.17,313.00)(1.000,-4.000){2}{\rule{0.400pt}{0.964pt}}
\put(417.67,309){\rule{0.400pt}{1.927pt}}
\multiput(417.17,309.00)(1.000,4.000){2}{\rule{0.400pt}{0.964pt}}
\put(418.67,309){\rule{0.400pt}{1.927pt}}
\multiput(418.17,313.00)(1.000,-4.000){2}{\rule{0.400pt}{0.964pt}}
\put(419.67,309){\rule{0.400pt}{1.927pt}}
\multiput(419.17,309.00)(1.000,4.000){2}{\rule{0.400pt}{0.964pt}}
\put(406.0,309.0){\rule[-0.200pt]{0.482pt}{0.400pt}}
\put(422.67,309){\rule{0.400pt}{1.927pt}}
\multiput(422.17,313.00)(1.000,-4.000){2}{\rule{0.400pt}{0.964pt}}
\put(421.0,317.0){\rule[-0.200pt]{0.482pt}{0.400pt}}
\put(424.67,309){\rule{0.400pt}{1.927pt}}
\multiput(424.17,309.00)(1.000,4.000){2}{\rule{0.400pt}{0.964pt}}
\put(425.67,309){\rule{0.400pt}{1.927pt}}
\multiput(425.17,313.00)(1.000,-4.000){2}{\rule{0.400pt}{0.964pt}}
\put(426.67,309){\rule{0.400pt}{1.927pt}}
\multiput(426.17,309.00)(1.000,4.000){2}{\rule{0.400pt}{0.964pt}}
\put(427.67,309){\rule{0.400pt}{1.927pt}}
\multiput(427.17,313.00)(1.000,-4.000){2}{\rule{0.400pt}{0.964pt}}
\put(428.67,309){\rule{0.400pt}{1.927pt}}
\multiput(428.17,309.00)(1.000,4.000){2}{\rule{0.400pt}{0.964pt}}
\put(429.67,309){\rule{0.400pt}{1.927pt}}
\multiput(429.17,313.00)(1.000,-4.000){2}{\rule{0.400pt}{0.964pt}}
\put(430.67,309){\rule{0.400pt}{1.927pt}}
\multiput(430.17,309.00)(1.000,4.000){2}{\rule{0.400pt}{0.964pt}}
\put(431.67,309){\rule{0.400pt}{1.927pt}}
\multiput(431.17,313.00)(1.000,-4.000){2}{\rule{0.400pt}{0.964pt}}
\put(432.67,309){\rule{0.400pt}{7.950pt}}
\multiput(432.17,309.00)(1.000,16.500){2}{\rule{0.400pt}{3.975pt}}
\put(433.67,335){\rule{0.400pt}{1.686pt}}
\multiput(433.17,338.50)(1.000,-3.500){2}{\rule{0.400pt}{0.843pt}}
\put(435.17,335){\rule{0.400pt}{2.900pt}}
\multiput(434.17,335.00)(2.000,7.981){2}{\rule{0.400pt}{1.450pt}}
\put(436.67,346){\rule{0.400pt}{0.723pt}}
\multiput(436.17,347.50)(1.000,-1.500){2}{\rule{0.400pt}{0.361pt}}
\put(437.67,346){\rule{0.400pt}{3.373pt}}
\multiput(437.17,346.00)(1.000,7.000){2}{\rule{0.400pt}{1.686pt}}
\put(438.67,353){\rule{0.400pt}{1.686pt}}
\multiput(438.17,356.50)(1.000,-3.500){2}{\rule{0.400pt}{0.843pt}}
\put(439.67,353){\rule{0.400pt}{4.336pt}}
\multiput(439.17,353.00)(1.000,9.000){2}{\rule{0.400pt}{2.168pt}}
\put(440.67,364){\rule{0.400pt}{1.686pt}}
\multiput(440.17,367.50)(1.000,-3.500){2}{\rule{0.400pt}{0.843pt}}
\put(441.67,364){\rule{0.400pt}{3.614pt}}
\multiput(441.17,364.00)(1.000,7.500){2}{\rule{0.400pt}{1.807pt}}
\put(442.67,371){\rule{0.400pt}{1.927pt}}
\multiput(442.17,375.00)(1.000,-4.000){2}{\rule{0.400pt}{0.964pt}}
\put(443.67,371){\rule{0.400pt}{3.614pt}}
\multiput(443.17,371.00)(1.000,7.500){2}{\rule{0.400pt}{1.807pt}}
\put(444.67,379){\rule{0.400pt}{1.686pt}}
\multiput(444.17,382.50)(1.000,-3.500){2}{\rule{0.400pt}{0.843pt}}
\put(445.67,379){\rule{0.400pt}{3.373pt}}
\multiput(445.17,379.00)(1.000,7.000){2}{\rule{0.400pt}{1.686pt}}
\put(446.67,389){\rule{0.400pt}{0.964pt}}
\multiput(446.17,391.00)(1.000,-2.000){2}{\rule{0.400pt}{0.482pt}}
\put(447.67,389){\rule{0.400pt}{2.650pt}}
\multiput(447.17,389.00)(1.000,5.500){2}{\rule{0.400pt}{1.325pt}}
\put(448.67,393){\rule{0.400pt}{1.686pt}}
\multiput(448.17,396.50)(1.000,-3.500){2}{\rule{0.400pt}{0.843pt}}
\put(450.17,393){\rule{0.400pt}{3.100pt}}
\multiput(449.17,393.00)(2.000,8.566){2}{\rule{0.400pt}{1.550pt}}
\put(451.67,404){\rule{0.400pt}{0.964pt}}
\multiput(451.17,406.00)(1.000,-2.000){2}{\rule{0.400pt}{0.482pt}}
\put(452.67,404){\rule{0.400pt}{2.650pt}}
\multiput(452.17,404.00)(1.000,5.500){2}{\rule{0.400pt}{1.325pt}}
\put(453.67,411){\rule{0.400pt}{0.964pt}}
\multiput(453.17,413.00)(1.000,-2.000){2}{\rule{0.400pt}{0.482pt}}
\put(454.67,411){\rule{0.400pt}{2.650pt}}
\multiput(454.17,411.00)(1.000,5.500){2}{\rule{0.400pt}{1.325pt}}
\put(455.67,415){\rule{0.400pt}{1.686pt}}
\multiput(455.17,418.50)(1.000,-3.500){2}{\rule{0.400pt}{0.843pt}}
\put(456.67,415){\rule{0.400pt}{3.373pt}}
\multiput(456.17,415.00)(1.000,7.000){2}{\rule{0.400pt}{1.686pt}}
\put(457.67,422){\rule{0.400pt}{1.686pt}}
\multiput(457.17,425.50)(1.000,-3.500){2}{\rule{0.400pt}{0.843pt}}
\put(458.67,422){\rule{0.400pt}{2.650pt}}
\multiput(458.17,422.00)(1.000,5.500){2}{\rule{0.400pt}{1.325pt}}
\put(459.67,429){\rule{0.400pt}{0.964pt}}
\multiput(459.17,431.00)(1.000,-2.000){2}{\rule{0.400pt}{0.482pt}}
\put(460.67,429){\rule{0.400pt}{2.650pt}}
\multiput(460.17,429.00)(1.000,5.500){2}{\rule{0.400pt}{1.325pt}}
\put(461.67,433){\rule{0.400pt}{1.686pt}}
\multiput(461.17,436.50)(1.000,-3.500){2}{\rule{0.400pt}{0.843pt}}
\put(462.67,433){\rule{0.400pt}{3.614pt}}
\multiput(462.17,433.00)(1.000,7.500){2}{\rule{0.400pt}{1.807pt}}
\put(463.67,440){\rule{0.400pt}{1.927pt}}
\multiput(463.17,444.00)(1.000,-4.000){2}{\rule{0.400pt}{0.964pt}}
\put(465.17,440){\rule{0.400pt}{2.300pt}}
\multiput(464.17,440.00)(2.000,6.226){2}{\rule{0.400pt}{1.150pt}}
\put(466.67,448){\rule{0.400pt}{0.723pt}}
\multiput(466.17,449.50)(1.000,-1.500){2}{\rule{0.400pt}{0.361pt}}
\put(467.67,448){\rule{0.400pt}{1.686pt}}
\multiput(467.17,448.00)(1.000,3.500){2}{\rule{0.400pt}{0.843pt}}
\put(468.67,451){\rule{0.400pt}{0.964pt}}
\multiput(468.17,453.00)(1.000,-2.000){2}{\rule{0.400pt}{0.482pt}}
\put(469.67,451){\rule{0.400pt}{1.927pt}}
\multiput(469.17,451.00)(1.000,4.000){2}{\rule{0.400pt}{0.964pt}}
\put(470.67,455){\rule{0.400pt}{0.964pt}}
\multiput(470.17,457.00)(1.000,-2.000){2}{\rule{0.400pt}{0.482pt}}
\put(471.67,455){\rule{0.400pt}{1.686pt}}
\multiput(471.17,455.00)(1.000,3.500){2}{\rule{0.400pt}{0.843pt}}
\put(472.67,459){\rule{0.400pt}{0.723pt}}
\multiput(472.17,460.50)(1.000,-1.500){2}{\rule{0.400pt}{0.361pt}}
\put(473.67,459){\rule{0.400pt}{2.650pt}}
\multiput(473.17,459.00)(1.000,5.500){2}{\rule{0.400pt}{1.325pt}}
\put(474.67,470){\rule{0.400pt}{0.723pt}}
\multiput(474.17,470.00)(1.000,1.500){2}{\rule{0.400pt}{0.361pt}}
\put(475.67,466){\rule{0.400pt}{1.686pt}}
\multiput(475.17,469.50)(1.000,-3.500){2}{\rule{0.400pt}{0.843pt}}
\put(476.67,466){\rule{0.400pt}{0.964pt}}
\multiput(476.17,466.00)(1.000,2.000){2}{\rule{0.400pt}{0.482pt}}
\put(477.67,470){\rule{0.400pt}{1.686pt}}
\multiput(477.17,470.00)(1.000,3.500){2}{\rule{0.400pt}{0.843pt}}
\put(479.17,477){\rule{0.400pt}{0.700pt}}
\multiput(478.17,477.00)(2.000,1.547){2}{\rule{0.400pt}{0.350pt}}
\put(480.67,473){\rule{0.400pt}{1.686pt}}
\multiput(480.17,476.50)(1.000,-3.500){2}{\rule{0.400pt}{0.843pt}}
\put(481.67,473){\rule{0.400pt}{0.964pt}}
\multiput(481.17,473.00)(1.000,2.000){2}{\rule{0.400pt}{0.482pt}}
\put(482.67,477){\rule{0.400pt}{2.650pt}}
\multiput(482.17,477.00)(1.000,5.500){2}{\rule{0.400pt}{1.325pt}}
\put(483.67,484){\rule{0.400pt}{0.964pt}}
\multiput(483.17,486.00)(1.000,-2.000){2}{\rule{0.400pt}{0.482pt}}
\put(484.67,484){\rule{0.400pt}{1.686pt}}
\multiput(484.17,484.00)(1.000,3.500){2}{\rule{0.400pt}{0.843pt}}
\put(485.67,488){\rule{0.400pt}{0.723pt}}
\multiput(485.17,489.50)(1.000,-1.500){2}{\rule{0.400pt}{0.361pt}}
\put(486.67,488){\rule{0.400pt}{1.686pt}}
\multiput(486.17,488.00)(1.000,3.500){2}{\rule{0.400pt}{0.843pt}}
\put(487.67,488){\rule{0.400pt}{1.686pt}}
\multiput(487.17,491.50)(1.000,-3.500){2}{\rule{0.400pt}{0.843pt}}
\put(488.67,488){\rule{0.400pt}{2.650pt}}
\multiput(488.17,488.00)(1.000,5.500){2}{\rule{0.400pt}{1.325pt}}
\put(489.67,491){\rule{0.400pt}{1.927pt}}
\multiput(489.17,495.00)(1.000,-4.000){2}{\rule{0.400pt}{0.964pt}}
\put(490.67,491){\rule{0.400pt}{1.927pt}}
\multiput(490.17,491.00)(1.000,4.000){2}{\rule{0.400pt}{0.964pt}}
\put(491.67,495){\rule{0.400pt}{0.964pt}}
\multiput(491.17,497.00)(1.000,-2.000){2}{\rule{0.400pt}{0.482pt}}
\put(492.67,495){\rule{0.400pt}{1.686pt}}
\multiput(492.17,495.00)(1.000,3.500){2}{\rule{0.400pt}{0.843pt}}
\put(494.17,499){\rule{0.400pt}{0.700pt}}
\multiput(493.17,500.55)(2.000,-1.547){2}{\rule{0.400pt}{0.350pt}}
\put(495.67,499){\rule{0.400pt}{2.650pt}}
\multiput(495.17,499.00)(1.000,5.500){2}{\rule{0.400pt}{1.325pt}}
\put(496.67,502){\rule{0.400pt}{1.927pt}}
\multiput(496.17,506.00)(1.000,-4.000){2}{\rule{0.400pt}{0.964pt}}
\put(497.67,502){\rule{0.400pt}{2.650pt}}
\multiput(497.17,502.00)(1.000,5.500){2}{\rule{0.400pt}{1.325pt}}
\put(498.67,506){\rule{0.400pt}{1.686pt}}
\multiput(498.17,509.50)(1.000,-3.500){2}{\rule{0.400pt}{0.843pt}}
\put(499.67,506){\rule{0.400pt}{1.686pt}}
\multiput(499.17,506.00)(1.000,3.500){2}{\rule{0.400pt}{0.843pt}}
\put(500.67,510){\rule{0.400pt}{0.723pt}}
\multiput(500.17,511.50)(1.000,-1.500){2}{\rule{0.400pt}{0.361pt}}
\put(501.67,510){\rule{0.400pt}{1.686pt}}
\multiput(501.17,510.00)(1.000,3.500){2}{\rule{0.400pt}{0.843pt}}
\put(424.0,309.0){\usebox{\plotpoint}}
\put(503.67,510){\rule{0.400pt}{1.686pt}}
\multiput(503.17,513.50)(1.000,-3.500){2}{\rule{0.400pt}{0.843pt}}
\put(504.67,510){\rule{0.400pt}{0.723pt}}
\multiput(504.17,510.00)(1.000,1.500){2}{\rule{0.400pt}{0.361pt}}
\put(505.67,513){\rule{0.400pt}{1.686pt}}
\multiput(505.17,513.00)(1.000,3.500){2}{\rule{0.400pt}{0.843pt}}
\put(506.67,513){\rule{0.400pt}{1.686pt}}
\multiput(506.17,516.50)(1.000,-3.500){2}{\rule{0.400pt}{0.843pt}}
\put(507.67,513){\rule{0.400pt}{2.650pt}}
\multiput(507.17,513.00)(1.000,5.500){2}{\rule{0.400pt}{1.325pt}}
\put(509.17,517){\rule{0.400pt}{1.500pt}}
\multiput(508.17,520.89)(2.000,-3.887){2}{\rule{0.400pt}{0.750pt}}
\put(510.67,517){\rule{0.400pt}{2.650pt}}
\multiput(510.17,517.00)(1.000,5.500){2}{\rule{0.400pt}{1.325pt}}
\put(511.67,517){\rule{0.400pt}{2.650pt}}
\multiput(511.17,522.50)(1.000,-5.500){2}{\rule{0.400pt}{1.325pt}}
\put(512.67,517){\rule{0.400pt}{1.686pt}}
\multiput(512.17,517.00)(1.000,3.500){2}{\rule{0.400pt}{0.843pt}}
\put(513.67,524){\rule{0.400pt}{0.964pt}}
\multiput(513.17,524.00)(1.000,2.000){2}{\rule{0.400pt}{0.482pt}}
\put(514.67,517){\rule{0.400pt}{2.650pt}}
\multiput(514.17,522.50)(1.000,-5.500){2}{\rule{0.400pt}{1.325pt}}
\put(515.67,517){\rule{0.400pt}{0.723pt}}
\multiput(515.17,517.00)(1.000,1.500){2}{\rule{0.400pt}{0.361pt}}
\put(516.67,506){\rule{0.400pt}{3.373pt}}
\multiput(516.17,513.00)(1.000,-7.000){2}{\rule{0.400pt}{1.686pt}}
\put(503.0,517.0){\usebox{\plotpoint}}
\put(518.67,488){\rule{0.400pt}{4.336pt}}
\multiput(518.17,497.00)(1.000,-9.000){2}{\rule{0.400pt}{2.168pt}}
\put(519.67,488){\rule{0.400pt}{0.723pt}}
\multiput(519.17,488.00)(1.000,1.500){2}{\rule{0.400pt}{0.361pt}}
\put(520.67,480){\rule{0.400pt}{2.650pt}}
\multiput(520.17,485.50)(1.000,-5.500){2}{\rule{0.400pt}{1.325pt}}
\put(521.67,480){\rule{0.400pt}{0.964pt}}
\multiput(521.17,480.00)(1.000,2.000){2}{\rule{0.400pt}{0.482pt}}
\put(523.17,484){\rule{0.400pt}{3.700pt}}
\multiput(522.17,484.00)(2.000,10.320){2}{\rule{0.400pt}{1.850pt}}
\put(524.67,491){\rule{0.400pt}{2.650pt}}
\multiput(524.17,496.50)(1.000,-5.500){2}{\rule{0.400pt}{1.325pt}}
\put(525.67,473){\rule{0.400pt}{4.336pt}}
\multiput(525.17,482.00)(1.000,-9.000){2}{\rule{0.400pt}{2.168pt}}
\put(518.0,506.0){\usebox{\plotpoint}}
\put(527.67,473){\rule{0.400pt}{7.950pt}}
\multiput(527.17,473.00)(1.000,16.500){2}{\rule{0.400pt}{3.975pt}}
\put(528.67,506){\rule{0.400pt}{1.686pt}}
\multiput(528.17,506.00)(1.000,3.500){2}{\rule{0.400pt}{0.843pt}}
\put(529.67,488){\rule{0.400pt}{6.023pt}}
\multiput(529.17,500.50)(1.000,-12.500){2}{\rule{0.400pt}{3.011pt}}
\put(530.67,488){\rule{0.400pt}{0.723pt}}
\multiput(530.17,488.00)(1.000,1.500){2}{\rule{0.400pt}{0.361pt}}
\put(531.67,470){\rule{0.400pt}{5.059pt}}
\multiput(531.17,480.50)(1.000,-10.500){2}{\rule{0.400pt}{2.529pt}}
\put(532.67,466){\rule{0.400pt}{0.964pt}}
\multiput(532.17,468.00)(1.000,-2.000){2}{\rule{0.400pt}{0.482pt}}
\put(533.67,466){\rule{0.400pt}{1.686pt}}
\multiput(533.17,466.00)(1.000,3.500){2}{\rule{0.400pt}{0.843pt}}
\put(534.67,462){\rule{0.400pt}{2.650pt}}
\multiput(534.17,467.50)(1.000,-5.500){2}{\rule{0.400pt}{1.325pt}}
\put(535.67,462){\rule{0.400pt}{5.300pt}}
\multiput(535.17,462.00)(1.000,11.000){2}{\rule{0.400pt}{2.650pt}}
\put(536.67,480){\rule{0.400pt}{0.964pt}}
\multiput(536.17,482.00)(1.000,-2.000){2}{\rule{0.400pt}{0.482pt}}
\put(538.17,480){\rule{0.400pt}{3.100pt}}
\multiput(537.17,480.00)(2.000,8.566){2}{\rule{0.400pt}{1.550pt}}
\put(539.67,488){\rule{0.400pt}{1.686pt}}
\multiput(539.17,491.50)(1.000,-3.500){2}{\rule{0.400pt}{0.843pt}}
\put(540.67,488){\rule{0.400pt}{6.986pt}}
\multiput(540.17,488.00)(1.000,14.500){2}{\rule{0.400pt}{3.493pt}}
\put(541.67,513){\rule{0.400pt}{0.964pt}}
\multiput(541.17,515.00)(1.000,-2.000){2}{\rule{0.400pt}{0.482pt}}
\put(542.67,513){\rule{0.400pt}{4.336pt}}
\multiput(542.17,513.00)(1.000,9.000){2}{\rule{0.400pt}{2.168pt}}
\put(543.67,517){\rule{0.400pt}{3.373pt}}
\multiput(543.17,524.00)(1.000,-7.000){2}{\rule{0.400pt}{1.686pt}}
\put(544.67,491){\rule{0.400pt}{6.263pt}}
\multiput(544.17,504.00)(1.000,-13.000){2}{\rule{0.400pt}{3.132pt}}
\put(545.67,488){\rule{0.400pt}{0.723pt}}
\multiput(545.17,489.50)(1.000,-1.500){2}{\rule{0.400pt}{0.361pt}}
\put(546.67,488){\rule{0.400pt}{2.650pt}}
\multiput(546.17,488.00)(1.000,5.500){2}{\rule{0.400pt}{1.325pt}}
\put(547.67,488){\rule{0.400pt}{2.650pt}}
\multiput(547.17,493.50)(1.000,-5.500){2}{\rule{0.400pt}{1.325pt}}
\put(548.67,488){\rule{0.400pt}{5.300pt}}
\multiput(548.17,488.00)(1.000,11.000){2}{\rule{0.400pt}{2.650pt}}
\put(549.67,499){\rule{0.400pt}{2.650pt}}
\multiput(549.17,504.50)(1.000,-5.500){2}{\rule{0.400pt}{1.325pt}}
\put(550.67,488){\rule{0.400pt}{2.650pt}}
\multiput(550.17,493.50)(1.000,-5.500){2}{\rule{0.400pt}{1.325pt}}
\put(552.17,488){\rule{0.400pt}{0.700pt}}
\multiput(551.17,488.00)(2.000,1.547){2}{\rule{0.400pt}{0.350pt}}
\put(553.67,470){\rule{0.400pt}{5.059pt}}
\multiput(553.17,480.50)(1.000,-10.500){2}{\rule{0.400pt}{2.529pt}}
\put(527.0,473.0){\usebox{\plotpoint}}
\put(555.67,470){\rule{0.400pt}{2.409pt}}
\multiput(555.17,470.00)(1.000,5.000){2}{\rule{0.400pt}{1.204pt}}
\put(556.67,473){\rule{0.400pt}{1.686pt}}
\multiput(556.17,476.50)(1.000,-3.500){2}{\rule{0.400pt}{0.843pt}}
\put(557.67,473){\rule{0.400pt}{2.650pt}}
\multiput(557.17,473.00)(1.000,5.500){2}{\rule{0.400pt}{1.325pt}}
\put(558.67,470){\rule{0.400pt}{3.373pt}}
\multiput(558.17,477.00)(1.000,-7.000){2}{\rule{0.400pt}{1.686pt}}
\put(559.67,470){\rule{0.400pt}{2.409pt}}
\multiput(559.17,470.00)(1.000,5.000){2}{\rule{0.400pt}{1.204pt}}
\put(560.67,477){\rule{0.400pt}{0.723pt}}
\multiput(560.17,478.50)(1.000,-1.500){2}{\rule{0.400pt}{0.361pt}}
\put(561.67,466){\rule{0.400pt}{2.650pt}}
\multiput(561.17,471.50)(1.000,-5.500){2}{\rule{0.400pt}{1.325pt}}
\put(555.0,470.0){\usebox{\plotpoint}}
\put(563.67,466){\rule{0.400pt}{2.650pt}}
\multiput(563.17,466.00)(1.000,5.500){2}{\rule{0.400pt}{1.325pt}}
\put(564.67,473){\rule{0.400pt}{0.964pt}}
\multiput(564.17,475.00)(1.000,-2.000){2}{\rule{0.400pt}{0.482pt}}
\put(565.67,455){\rule{0.400pt}{4.336pt}}
\multiput(565.17,464.00)(1.000,-9.000){2}{\rule{0.400pt}{2.168pt}}
\put(567.17,455){\rule{0.400pt}{1.500pt}}
\multiput(566.17,455.00)(2.000,3.887){2}{\rule{0.400pt}{0.750pt}}
\put(568.67,451){\rule{0.400pt}{2.650pt}}
\multiput(568.17,456.50)(1.000,-5.500){2}{\rule{0.400pt}{1.325pt}}
\put(563.0,466.0){\usebox{\plotpoint}}
\put(570.67,451){\rule{0.400pt}{2.650pt}}
\multiput(570.17,451.00)(1.000,5.500){2}{\rule{0.400pt}{1.325pt}}
\put(571.67,459){\rule{0.400pt}{0.723pt}}
\multiput(571.17,460.50)(1.000,-1.500){2}{\rule{0.400pt}{0.361pt}}
\put(572.67,437){\rule{0.400pt}{5.300pt}}
\multiput(572.17,448.00)(1.000,-11.000){2}{\rule{0.400pt}{2.650pt}}
\put(573.67,437){\rule{0.400pt}{1.686pt}}
\multiput(573.17,437.00)(1.000,3.500){2}{\rule{0.400pt}{0.843pt}}
\put(574.67,433){\rule{0.400pt}{2.650pt}}
\multiput(574.17,438.50)(1.000,-5.500){2}{\rule{0.400pt}{1.325pt}}
\put(575.67,433){\rule{0.400pt}{1.686pt}}
\multiput(575.17,433.00)(1.000,3.500){2}{\rule{0.400pt}{0.843pt}}
\put(576.67,426){\rule{0.400pt}{3.373pt}}
\multiput(576.17,433.00)(1.000,-7.000){2}{\rule{0.400pt}{1.686pt}}
\put(570.0,451.0){\usebox{\plotpoint}}
\put(578.67,426){\rule{0.400pt}{1.686pt}}
\multiput(578.17,426.00)(1.000,3.500){2}{\rule{0.400pt}{0.843pt}}
\put(579.67,422){\rule{0.400pt}{2.650pt}}
\multiput(579.17,427.50)(1.000,-5.500){2}{\rule{0.400pt}{1.325pt}}
\put(580.67,422){\rule{0.400pt}{2.650pt}}
\multiput(580.17,422.00)(1.000,5.500){2}{\rule{0.400pt}{1.325pt}}
\put(578.0,426.0){\usebox{\plotpoint}}
\put(583.67,426){\rule{0.400pt}{1.686pt}}
\multiput(583.17,429.50)(1.000,-3.500){2}{\rule{0.400pt}{0.843pt}}
\put(584.67,426){\rule{0.400pt}{0.723pt}}
\multiput(584.17,426.00)(1.000,1.500){2}{\rule{0.400pt}{0.361pt}}
\put(585.67,419){\rule{0.400pt}{2.409pt}}
\multiput(585.17,424.00)(1.000,-5.000){2}{\rule{0.400pt}{1.204pt}}
\put(586.67,419){\rule{0.400pt}{2.409pt}}
\multiput(586.17,419.00)(1.000,5.000){2}{\rule{0.400pt}{1.204pt}}
\put(587.67,415){\rule{0.400pt}{3.373pt}}
\multiput(587.17,422.00)(1.000,-7.000){2}{\rule{0.400pt}{1.686pt}}
\put(588.67,415){\rule{0.400pt}{0.964pt}}
\multiput(588.17,415.00)(1.000,2.000){2}{\rule{0.400pt}{0.482pt}}
\put(589.67,419){\rule{0.400pt}{2.409pt}}
\multiput(589.17,419.00)(1.000,5.000){2}{\rule{0.400pt}{1.204pt}}
\put(590.67,422){\rule{0.400pt}{1.686pt}}
\multiput(590.17,425.50)(1.000,-3.500){2}{\rule{0.400pt}{0.843pt}}
\put(591.67,415){\rule{0.400pt}{1.686pt}}
\multiput(591.17,418.50)(1.000,-3.500){2}{\rule{0.400pt}{0.843pt}}
\put(592.67,415){\rule{0.400pt}{1.686pt}}
\multiput(592.17,415.00)(1.000,3.500){2}{\rule{0.400pt}{0.843pt}}
\put(593.67,415){\rule{0.400pt}{1.686pt}}
\multiput(593.17,418.50)(1.000,-3.500){2}{\rule{0.400pt}{0.843pt}}
\put(594.67,415){\rule{0.400pt}{1.686pt}}
\multiput(594.17,415.00)(1.000,3.500){2}{\rule{0.400pt}{0.843pt}}
\put(596.17,415){\rule{0.400pt}{1.500pt}}
\multiput(595.17,418.89)(2.000,-3.887){2}{\rule{0.400pt}{0.750pt}}
\put(597.67,415){\rule{0.400pt}{1.686pt}}
\multiput(597.17,415.00)(1.000,3.500){2}{\rule{0.400pt}{0.843pt}}
\put(598.67,415){\rule{0.400pt}{1.686pt}}
\multiput(598.17,418.50)(1.000,-3.500){2}{\rule{0.400pt}{0.843pt}}
\put(599.67,415){\rule{0.400pt}{0.964pt}}
\multiput(599.17,415.00)(1.000,2.000){2}{\rule{0.400pt}{0.482pt}}
\put(600.67,419){\rule{0.400pt}{2.409pt}}
\multiput(600.17,419.00)(1.000,5.000){2}{\rule{0.400pt}{1.204pt}}
\put(582.0,433.0){\rule[-0.200pt]{0.482pt}{0.400pt}}
\put(602.67,415){\rule{0.400pt}{3.373pt}}
\multiput(602.17,422.00)(1.000,-7.000){2}{\rule{0.400pt}{1.686pt}}
\put(603.67,415){\rule{0.400pt}{3.373pt}}
\multiput(603.17,415.00)(1.000,7.000){2}{\rule{0.400pt}{1.686pt}}
\put(604.67,419){\rule{0.400pt}{2.409pt}}
\multiput(604.17,424.00)(1.000,-5.000){2}{\rule{0.400pt}{1.204pt}}
\put(605.67,419){\rule{0.400pt}{2.409pt}}
\multiput(605.17,419.00)(1.000,5.000){2}{\rule{0.400pt}{1.204pt}}
\put(606.67,419){\rule{0.400pt}{2.409pt}}
\multiput(606.17,424.00)(1.000,-5.000){2}{\rule{0.400pt}{1.204pt}}
\put(602.0,429.0){\usebox{\plotpoint}}
\put(608.67,419){\rule{0.400pt}{2.409pt}}
\multiput(608.17,419.00)(1.000,5.000){2}{\rule{0.400pt}{1.204pt}}
\put(609.67,429){\rule{0.400pt}{0.964pt}}
\multiput(609.17,429.00)(1.000,2.000){2}{\rule{0.400pt}{0.482pt}}
\put(611.17,419){\rule{0.400pt}{2.900pt}}
\multiput(610.17,426.98)(2.000,-7.981){2}{\rule{0.400pt}{1.450pt}}
\put(612.67,415){\rule{0.400pt}{0.964pt}}
\multiput(612.17,417.00)(1.000,-2.000){2}{\rule{0.400pt}{0.482pt}}
\put(613.67,415){\rule{0.400pt}{2.650pt}}
\multiput(613.17,415.00)(1.000,5.500){2}{\rule{0.400pt}{1.325pt}}
\put(614.67,426){\rule{0.400pt}{0.723pt}}
\multiput(614.17,426.00)(1.000,1.500){2}{\rule{0.400pt}{0.361pt}}
\put(615.67,419){\rule{0.400pt}{2.409pt}}
\multiput(615.17,424.00)(1.000,-5.000){2}{\rule{0.400pt}{1.204pt}}
\put(608.0,419.0){\usebox{\plotpoint}}
\put(617.67,419){\rule{0.400pt}{2.409pt}}
\multiput(617.17,419.00)(1.000,5.000){2}{\rule{0.400pt}{1.204pt}}
\put(618.67,419){\rule{0.400pt}{2.409pt}}
\multiput(618.17,424.00)(1.000,-5.000){2}{\rule{0.400pt}{1.204pt}}
\put(619.67,419){\rule{0.400pt}{2.409pt}}
\multiput(619.17,419.00)(1.000,5.000){2}{\rule{0.400pt}{1.204pt}}
\put(620.67,419){\rule{0.400pt}{2.409pt}}
\multiput(620.17,424.00)(1.000,-5.000){2}{\rule{0.400pt}{1.204pt}}
\put(621.67,419){\rule{0.400pt}{3.373pt}}
\multiput(621.17,419.00)(1.000,7.000){2}{\rule{0.400pt}{1.686pt}}
\put(622.67,426){\rule{0.400pt}{1.686pt}}
\multiput(622.17,429.50)(1.000,-3.500){2}{\rule{0.400pt}{0.843pt}}
\put(623.67,415){\rule{0.400pt}{2.650pt}}
\multiput(623.17,420.50)(1.000,-5.500){2}{\rule{0.400pt}{1.325pt}}
\put(617.0,419.0){\usebox{\plotpoint}}
\put(626.17,415){\rule{0.400pt}{0.900pt}}
\multiput(625.17,415.00)(2.000,2.132){2}{\rule{0.400pt}{0.450pt}}
\put(627.67,415){\rule{0.400pt}{0.964pt}}
\multiput(627.17,417.00)(1.000,-2.000){2}{\rule{0.400pt}{0.482pt}}
\put(628.67,415){\rule{0.400pt}{4.336pt}}
\multiput(628.17,415.00)(1.000,9.000){2}{\rule{0.400pt}{2.168pt}}
\put(629.67,426){\rule{0.400pt}{1.686pt}}
\multiput(629.17,429.50)(1.000,-3.500){2}{\rule{0.400pt}{0.843pt}}
\put(630.67,426){\rule{0.400pt}{2.650pt}}
\multiput(630.17,426.00)(1.000,5.500){2}{\rule{0.400pt}{1.325pt}}
\put(631.67,419){\rule{0.400pt}{4.336pt}}
\multiput(631.17,428.00)(1.000,-9.000){2}{\rule{0.400pt}{2.168pt}}
\put(632.67,419){\rule{0.400pt}{3.373pt}}
\multiput(632.17,419.00)(1.000,7.000){2}{\rule{0.400pt}{1.686pt}}
\put(633.67,415){\rule{0.400pt}{4.336pt}}
\multiput(633.17,424.00)(1.000,-9.000){2}{\rule{0.400pt}{2.168pt}}
\put(634.67,415){\rule{0.400pt}{4.336pt}}
\multiput(634.17,415.00)(1.000,9.000){2}{\rule{0.400pt}{2.168pt}}
\put(635.67,426){\rule{0.400pt}{1.686pt}}
\multiput(635.17,429.50)(1.000,-3.500){2}{\rule{0.400pt}{0.843pt}}
\put(636.67,426){\rule{0.400pt}{2.650pt}}
\multiput(636.17,426.00)(1.000,5.500){2}{\rule{0.400pt}{1.325pt}}
\put(637.67,419){\rule{0.400pt}{4.336pt}}
\multiput(637.17,428.00)(1.000,-9.000){2}{\rule{0.400pt}{2.168pt}}
\put(638.67,419){\rule{0.400pt}{5.059pt}}
\multiput(638.17,419.00)(1.000,10.500){2}{\rule{0.400pt}{2.529pt}}
\put(625.0,415.0){\usebox{\plotpoint}}
\put(641.67,419){\rule{0.400pt}{5.059pt}}
\multiput(641.17,429.50)(1.000,-10.500){2}{\rule{0.400pt}{2.529pt}}
\put(642.67,415){\rule{0.400pt}{0.964pt}}
\multiput(642.17,417.00)(1.000,-2.000){2}{\rule{0.400pt}{0.482pt}}
\put(643.67,415){\rule{0.400pt}{4.336pt}}
\multiput(643.17,415.00)(1.000,9.000){2}{\rule{0.400pt}{2.168pt}}
\put(644.67,429){\rule{0.400pt}{0.964pt}}
\multiput(644.17,431.00)(1.000,-2.000){2}{\rule{0.400pt}{0.482pt}}
\put(645.67,429){\rule{0.400pt}{2.650pt}}
\multiput(645.17,429.00)(1.000,5.500){2}{\rule{0.400pt}{1.325pt}}
\put(646.67,433){\rule{0.400pt}{1.686pt}}
\multiput(646.17,436.50)(1.000,-3.500){2}{\rule{0.400pt}{0.843pt}}
\put(647.67,422){\rule{0.400pt}{2.650pt}}
\multiput(647.17,427.50)(1.000,-5.500){2}{\rule{0.400pt}{1.325pt}}
\put(648.67,422){\rule{0.400pt}{2.650pt}}
\multiput(648.17,422.00)(1.000,5.500){2}{\rule{0.400pt}{1.325pt}}
\put(649.67,419){\rule{0.400pt}{3.373pt}}
\multiput(649.17,426.00)(1.000,-7.000){2}{\rule{0.400pt}{1.686pt}}
\put(650.67,419){\rule{0.400pt}{1.686pt}}
\multiput(650.17,419.00)(1.000,3.500){2}{\rule{0.400pt}{0.843pt}}
\put(651.67,426){\rule{0.400pt}{2.650pt}}
\multiput(651.17,426.00)(1.000,5.500){2}{\rule{0.400pt}{1.325pt}}
\put(652.67,419){\rule{0.400pt}{4.336pt}}
\multiput(652.17,428.00)(1.000,-9.000){2}{\rule{0.400pt}{2.168pt}}
\put(653.67,419){\rule{0.400pt}{4.336pt}}
\multiput(653.17,419.00)(1.000,9.000){2}{\rule{0.400pt}{2.168pt}}
\put(655.17,419){\rule{0.400pt}{3.700pt}}
\multiput(654.17,429.32)(2.000,-10.320){2}{\rule{0.400pt}{1.850pt}}
\put(656.67,419){\rule{0.400pt}{3.373pt}}
\multiput(656.17,419.00)(1.000,7.000){2}{\rule{0.400pt}{1.686pt}}
\put(657.67,426){\rule{0.400pt}{1.686pt}}
\multiput(657.17,429.50)(1.000,-3.500){2}{\rule{0.400pt}{0.843pt}}
\put(658.67,426){\rule{0.400pt}{2.650pt}}
\multiput(658.17,426.00)(1.000,5.500){2}{\rule{0.400pt}{1.325pt}}
\put(659.67,426){\rule{0.400pt}{2.650pt}}
\multiput(659.17,431.50)(1.000,-5.500){2}{\rule{0.400pt}{1.325pt}}
\put(660.67,426){\rule{0.400pt}{2.650pt}}
\multiput(660.17,426.00)(1.000,5.500){2}{\rule{0.400pt}{1.325pt}}
\put(661.67,426){\rule{0.400pt}{2.650pt}}
\multiput(661.17,431.50)(1.000,-5.500){2}{\rule{0.400pt}{1.325pt}}
\put(662.67,426){\rule{0.400pt}{2.650pt}}
\multiput(662.17,426.00)(1.000,5.500){2}{\rule{0.400pt}{1.325pt}}
\put(663.67,426){\rule{0.400pt}{2.650pt}}
\multiput(663.17,431.50)(1.000,-5.500){2}{\rule{0.400pt}{1.325pt}}
\put(664.67,426){\rule{0.400pt}{1.686pt}}
\multiput(664.17,426.00)(1.000,3.500){2}{\rule{0.400pt}{0.843pt}}
\put(665.67,411){\rule{0.400pt}{5.300pt}}
\multiput(665.17,422.00)(1.000,-11.000){2}{\rule{0.400pt}{2.650pt}}
\put(666.67,411){\rule{0.400pt}{6.263pt}}
\multiput(666.17,411.00)(1.000,13.000){2}{\rule{0.400pt}{3.132pt}}
\put(667.67,429){\rule{0.400pt}{1.927pt}}
\multiput(667.17,433.00)(1.000,-4.000){2}{\rule{0.400pt}{0.964pt}}
\put(668.67,429){\rule{0.400pt}{5.300pt}}
\multiput(668.17,429.00)(1.000,11.000){2}{\rule{0.400pt}{2.650pt}}
\put(670.17,437){\rule{0.400pt}{2.900pt}}
\multiput(669.17,444.98)(2.000,-7.981){2}{\rule{0.400pt}{1.450pt}}
\put(671.67,426){\rule{0.400pt}{2.650pt}}
\multiput(671.17,431.50)(1.000,-5.500){2}{\rule{0.400pt}{1.325pt}}
\put(672.67,426){\rule{0.400pt}{1.686pt}}
\multiput(672.17,426.00)(1.000,3.500){2}{\rule{0.400pt}{0.843pt}}
\put(673.67,419){\rule{0.400pt}{3.373pt}}
\multiput(673.17,426.00)(1.000,-7.000){2}{\rule{0.400pt}{1.686pt}}
\put(674.67,419){\rule{0.400pt}{0.723pt}}
\multiput(674.17,419.00)(1.000,1.500){2}{\rule{0.400pt}{0.361pt}}
\put(675.67,422){\rule{0.400pt}{6.986pt}}
\multiput(675.17,422.00)(1.000,14.500){2}{\rule{0.400pt}{3.493pt}}
\put(640.0,440.0){\rule[-0.200pt]{0.482pt}{0.400pt}}
\put(677.67,433){\rule{0.400pt}{4.336pt}}
\multiput(677.17,442.00)(1.000,-9.000){2}{\rule{0.400pt}{2.168pt}}
\put(678.67,433){\rule{0.400pt}{0.964pt}}
\multiput(678.17,433.00)(1.000,2.000){2}{\rule{0.400pt}{0.482pt}}
\put(679.67,429){\rule{0.400pt}{1.927pt}}
\multiput(679.17,433.00)(1.000,-4.000){2}{\rule{0.400pt}{0.964pt}}
\put(680.67,426){\rule{0.400pt}{0.723pt}}
\multiput(680.17,427.50)(1.000,-1.500){2}{\rule{0.400pt}{0.361pt}}
\put(681.67,426){\rule{0.400pt}{2.650pt}}
\multiput(681.17,426.00)(1.000,5.500){2}{\rule{0.400pt}{1.325pt}}
\put(682.67,426){\rule{0.400pt}{2.650pt}}
\multiput(682.17,431.50)(1.000,-5.500){2}{\rule{0.400pt}{1.325pt}}
\put(684.17,426){\rule{0.400pt}{5.100pt}}
\multiput(683.17,426.00)(2.000,14.415){2}{\rule{0.400pt}{2.550pt}}
\put(677.0,451.0){\usebox{\plotpoint}}
\put(686.67,437){\rule{0.400pt}{3.373pt}}
\multiput(686.17,444.00)(1.000,-7.000){2}{\rule{0.400pt}{1.686pt}}
\put(687.67,437){\rule{0.400pt}{2.650pt}}
\multiput(687.17,437.00)(1.000,5.500){2}{\rule{0.400pt}{1.325pt}}
\put(688.67,433){\rule{0.400pt}{3.614pt}}
\multiput(688.17,440.50)(1.000,-7.500){2}{\rule{0.400pt}{1.807pt}}
\put(689.67,433){\rule{0.400pt}{3.614pt}}
\multiput(689.17,433.00)(1.000,7.500){2}{\rule{0.400pt}{1.807pt}}
\put(690.67,433){\rule{0.400pt}{3.614pt}}
\multiput(690.17,440.50)(1.000,-7.500){2}{\rule{0.400pt}{1.807pt}}
\put(686.0,451.0){\usebox{\plotpoint}}
\put(692.67,433){\rule{0.400pt}{2.650pt}}
\multiput(692.17,433.00)(1.000,5.500){2}{\rule{0.400pt}{1.325pt}}
\put(693.67,444){\rule{0.400pt}{0.964pt}}
\multiput(693.17,444.00)(1.000,2.000){2}{\rule{0.400pt}{0.482pt}}
\put(694.67,437){\rule{0.400pt}{2.650pt}}
\multiput(694.17,442.50)(1.000,-5.500){2}{\rule{0.400pt}{1.325pt}}
\put(695.67,437){\rule{0.400pt}{3.373pt}}
\multiput(695.17,437.00)(1.000,7.000){2}{\rule{0.400pt}{1.686pt}}
\put(696.67,429){\rule{0.400pt}{5.300pt}}
\multiput(696.17,440.00)(1.000,-11.000){2}{\rule{0.400pt}{2.650pt}}
\put(697.67,429){\rule{0.400pt}{3.614pt}}
\multiput(697.17,429.00)(1.000,7.500){2}{\rule{0.400pt}{1.807pt}}
\put(699.17,444){\rule{0.400pt}{3.100pt}}
\multiput(698.17,444.00)(2.000,8.566){2}{\rule{0.400pt}{1.550pt}}
\put(700.67,455){\rule{0.400pt}{0.964pt}}
\multiput(700.17,457.00)(1.000,-2.000){2}{\rule{0.400pt}{0.482pt}}
\put(701.67,444){\rule{0.400pt}{2.650pt}}
\multiput(701.17,449.50)(1.000,-5.500){2}{\rule{0.400pt}{1.325pt}}
\put(702.67,444){\rule{0.400pt}{0.964pt}}
\multiput(702.17,444.00)(1.000,2.000){2}{\rule{0.400pt}{0.482pt}}
\put(703.67,437){\rule{0.400pt}{2.650pt}}
\multiput(703.17,442.50)(1.000,-5.500){2}{\rule{0.400pt}{1.325pt}}
\put(692.0,433.0){\usebox{\plotpoint}}
\put(705.67,437){\rule{0.400pt}{4.336pt}}
\multiput(705.17,437.00)(1.000,9.000){2}{\rule{0.400pt}{2.168pt}}
\put(705.0,437.0){\usebox{\plotpoint}}
\put(707.67,448){\rule{0.400pt}{1.686pt}}
\multiput(707.17,451.50)(1.000,-3.500){2}{\rule{0.400pt}{0.843pt}}
\put(708.67,448){\rule{0.400pt}{1.686pt}}
\multiput(708.17,448.00)(1.000,3.500){2}{\rule{0.400pt}{0.843pt}}
\put(709.67,444){\rule{0.400pt}{2.650pt}}
\multiput(709.17,449.50)(1.000,-5.500){2}{\rule{0.400pt}{1.325pt}}
\put(710.67,444){\rule{0.400pt}{0.964pt}}
\multiput(710.17,444.00)(1.000,2.000){2}{\rule{0.400pt}{0.482pt}}
\put(711.67,448){\rule{0.400pt}{1.686pt}}
\multiput(711.17,448.00)(1.000,3.500){2}{\rule{0.400pt}{0.843pt}}
\put(707.0,455.0){\usebox{\plotpoint}}
\put(714.67,448){\rule{0.400pt}{1.686pt}}
\multiput(714.17,451.50)(1.000,-3.500){2}{\rule{0.400pt}{0.843pt}}
\put(715.67,448){\rule{0.400pt}{2.650pt}}
\multiput(715.17,448.00)(1.000,5.500){2}{\rule{0.400pt}{1.325pt}}
\put(716.67,444){\rule{0.400pt}{3.614pt}}
\multiput(716.17,451.50)(1.000,-7.500){2}{\rule{0.400pt}{1.807pt}}
\put(717.67,444){\rule{0.400pt}{0.964pt}}
\multiput(717.17,444.00)(1.000,2.000){2}{\rule{0.400pt}{0.482pt}}
\put(718.67,448){\rule{0.400pt}{2.650pt}}
\multiput(718.17,448.00)(1.000,5.500){2}{\rule{0.400pt}{1.325pt}}
\put(719.67,451){\rule{0.400pt}{1.927pt}}
\multiput(719.17,455.00)(1.000,-4.000){2}{\rule{0.400pt}{0.964pt}}
\put(720.67,451){\rule{0.400pt}{2.650pt}}
\multiput(720.17,451.00)(1.000,5.500){2}{\rule{0.400pt}{1.325pt}}
\put(713.0,455.0){\rule[-0.200pt]{0.482pt}{0.400pt}}
\put(722.67,455){\rule{0.400pt}{1.686pt}}
\multiput(722.17,458.50)(1.000,-3.500){2}{\rule{0.400pt}{0.843pt}}
\put(722.0,462.0){\usebox{\plotpoint}}
\put(724.67,455){\rule{0.400pt}{1.686pt}}
\multiput(724.17,455.00)(1.000,3.500){2}{\rule{0.400pt}{0.843pt}}
\put(724.0,455.0){\usebox{\plotpoint}}
\put(726.67,451){\rule{0.400pt}{2.650pt}}
\multiput(726.17,456.50)(1.000,-5.500){2}{\rule{0.400pt}{1.325pt}}
\put(728.17,451){\rule{0.400pt}{0.900pt}}
\multiput(727.17,451.00)(2.000,2.132){2}{\rule{0.400pt}{0.450pt}}
\put(729.67,455){\rule{0.400pt}{1.686pt}}
\multiput(729.17,455.00)(1.000,3.500){2}{\rule{0.400pt}{0.843pt}}
\put(726.0,462.0){\usebox{\plotpoint}}
\put(731.67,455){\rule{0.400pt}{1.686pt}}
\multiput(731.17,458.50)(1.000,-3.500){2}{\rule{0.400pt}{0.843pt}}
\put(732.67,455){\rule{0.400pt}{1.686pt}}
\multiput(732.17,455.00)(1.000,3.500){2}{\rule{0.400pt}{0.843pt}}
\put(733.67,455){\rule{0.400pt}{1.686pt}}
\multiput(733.17,458.50)(1.000,-3.500){2}{\rule{0.400pt}{0.843pt}}
\put(734.67,455){\rule{0.400pt}{1.686pt}}
\multiput(734.17,455.00)(1.000,3.500){2}{\rule{0.400pt}{0.843pt}}
\put(735.67,455){\rule{0.400pt}{1.686pt}}
\multiput(735.17,458.50)(1.000,-3.500){2}{\rule{0.400pt}{0.843pt}}
\put(731.0,462.0){\usebox{\plotpoint}}
\put(737.67,455){\rule{0.400pt}{3.614pt}}
\multiput(737.17,455.00)(1.000,7.500){2}{\rule{0.400pt}{1.807pt}}
\put(737.0,455.0){\usebox{\plotpoint}}
\put(739.67,459){\rule{0.400pt}{2.650pt}}
\multiput(739.17,464.50)(1.000,-5.500){2}{\rule{0.400pt}{1.325pt}}
\put(739.0,470.0){\usebox{\plotpoint}}
\put(741.67,459){\rule{0.400pt}{2.650pt}}
\multiput(741.17,459.00)(1.000,5.500){2}{\rule{0.400pt}{1.325pt}}
\put(743.17,470){\rule{0.400pt}{1.500pt}}
\multiput(742.17,470.00)(2.000,3.887){2}{\rule{0.400pt}{0.750pt}}
\put(744.67,466){\rule{0.400pt}{2.650pt}}
\multiput(744.17,471.50)(1.000,-5.500){2}{\rule{0.400pt}{1.325pt}}
\put(745.67,466){\rule{0.400pt}{0.964pt}}
\multiput(745.17,466.00)(1.000,2.000){2}{\rule{0.400pt}{0.482pt}}
\put(746.67,459){\rule{0.400pt}{2.650pt}}
\multiput(746.17,464.50)(1.000,-5.500){2}{\rule{0.400pt}{1.325pt}}
\put(741.0,459.0){\usebox{\plotpoint}}
\put(748.67,459){\rule{0.400pt}{2.650pt}}
\multiput(748.17,459.00)(1.000,5.500){2}{\rule{0.400pt}{1.325pt}}
\put(748.0,459.0){\usebox{\plotpoint}}
\put(750.67,459){\rule{0.400pt}{2.650pt}}
\multiput(750.17,464.50)(1.000,-5.500){2}{\rule{0.400pt}{1.325pt}}
\put(750.0,470.0){\usebox{\plotpoint}}
\put(752.67,459){\rule{0.400pt}{3.373pt}}
\multiput(752.17,459.00)(1.000,7.000){2}{\rule{0.400pt}{1.686pt}}
\put(753.67,473){\rule{0.400pt}{0.964pt}}
\multiput(753.17,473.00)(1.000,2.000){2}{\rule{0.400pt}{0.482pt}}
\put(754.67,466){\rule{0.400pt}{2.650pt}}
\multiput(754.17,471.50)(1.000,-5.500){2}{\rule{0.400pt}{1.325pt}}
\put(755.67,462){\rule{0.400pt}{0.964pt}}
\multiput(755.17,464.00)(1.000,-2.000){2}{\rule{0.400pt}{0.482pt}}
\put(757.17,462){\rule{0.400pt}{3.100pt}}
\multiput(756.17,462.00)(2.000,8.566){2}{\rule{0.400pt}{1.550pt}}
\put(758.67,459){\rule{0.400pt}{4.336pt}}
\multiput(758.17,468.00)(1.000,-9.000){2}{\rule{0.400pt}{2.168pt}}
\put(759.67,459){\rule{0.400pt}{3.373pt}}
\multiput(759.17,459.00)(1.000,7.000){2}{\rule{0.400pt}{1.686pt}}
\put(760.67,466){\rule{0.400pt}{1.686pt}}
\multiput(760.17,469.50)(1.000,-3.500){2}{\rule{0.400pt}{0.843pt}}
\put(761.67,466){\rule{0.400pt}{2.650pt}}
\multiput(761.17,466.00)(1.000,5.500){2}{\rule{0.400pt}{1.325pt}}
\put(762.67,462){\rule{0.400pt}{3.614pt}}
\multiput(762.17,469.50)(1.000,-7.500){2}{\rule{0.400pt}{1.807pt}}
\put(763.67,462){\rule{0.400pt}{2.650pt}}
\multiput(763.17,462.00)(1.000,5.500){2}{\rule{0.400pt}{1.325pt}}
\put(764.67,459){\rule{0.400pt}{3.373pt}}
\multiput(764.17,466.00)(1.000,-7.000){2}{\rule{0.400pt}{1.686pt}}
\put(765.67,459){\rule{0.400pt}{3.373pt}}
\multiput(765.17,459.00)(1.000,7.000){2}{\rule{0.400pt}{1.686pt}}
\put(752.0,459.0){\usebox{\plotpoint}}
\put(767.67,455){\rule{0.400pt}{4.336pt}}
\multiput(767.17,464.00)(1.000,-9.000){2}{\rule{0.400pt}{2.168pt}}
\put(768.67,455){\rule{0.400pt}{4.336pt}}
\multiput(768.17,455.00)(1.000,9.000){2}{\rule{0.400pt}{2.168pt}}
\put(769.67,459){\rule{0.400pt}{3.373pt}}
\multiput(769.17,466.00)(1.000,-7.000){2}{\rule{0.400pt}{1.686pt}}
\put(770.67,459){\rule{0.400pt}{0.723pt}}
\multiput(770.17,459.00)(1.000,1.500){2}{\rule{0.400pt}{0.361pt}}
\put(772.17,462){\rule{0.400pt}{2.300pt}}
\multiput(771.17,462.00)(2.000,6.226){2}{\rule{0.400pt}{1.150pt}}
\put(773.67,459){\rule{0.400pt}{3.373pt}}
\multiput(773.17,466.00)(1.000,-7.000){2}{\rule{0.400pt}{1.686pt}}
\put(774.67,459){\rule{0.400pt}{3.373pt}}
\multiput(774.17,459.00)(1.000,7.000){2}{\rule{0.400pt}{1.686pt}}
\put(775.67,466){\rule{0.400pt}{1.686pt}}
\multiput(775.17,469.50)(1.000,-3.500){2}{\rule{0.400pt}{0.843pt}}
\put(776.67,466){\rule{0.400pt}{1.686pt}}
\multiput(776.17,466.00)(1.000,3.500){2}{\rule{0.400pt}{0.843pt}}
\put(777.67,466){\rule{0.400pt}{1.686pt}}
\multiput(777.17,469.50)(1.000,-3.500){2}{\rule{0.400pt}{0.843pt}}
\put(778.67,466){\rule{0.400pt}{1.686pt}}
\multiput(778.17,466.00)(1.000,3.500){2}{\rule{0.400pt}{0.843pt}}
\put(779.67,473){\rule{0.400pt}{0.964pt}}
\multiput(779.17,473.00)(1.000,2.000){2}{\rule{0.400pt}{0.482pt}}
\put(780.67,459){\rule{0.400pt}{4.336pt}}
\multiput(780.17,468.00)(1.000,-9.000){2}{\rule{0.400pt}{2.168pt}}
\put(781.67,451){\rule{0.400pt}{1.927pt}}
\multiput(781.17,455.00)(1.000,-4.000){2}{\rule{0.400pt}{0.964pt}}
\put(782.67,451){\rule{0.400pt}{4.577pt}}
\multiput(782.17,451.00)(1.000,9.500){2}{\rule{0.400pt}{2.289pt}}
\put(783.67,455){\rule{0.400pt}{3.614pt}}
\multiput(783.17,462.50)(1.000,-7.500){2}{\rule{0.400pt}{1.807pt}}
\put(784.67,455){\rule{0.400pt}{4.336pt}}
\multiput(784.17,455.00)(1.000,9.000){2}{\rule{0.400pt}{2.168pt}}
\put(767.0,473.0){\usebox{\plotpoint}}
\put(787.17,455){\rule{0.400pt}{3.700pt}}
\multiput(786.17,465.32)(2.000,-10.320){2}{\rule{0.400pt}{1.850pt}}
\put(786.0,473.0){\usebox{\plotpoint}}
\put(789.67,455){\rule{0.400pt}{1.686pt}}
\multiput(789.17,455.00)(1.000,3.500){2}{\rule{0.400pt}{0.843pt}}
\put(790.67,455){\rule{0.400pt}{1.686pt}}
\multiput(790.17,458.50)(1.000,-3.500){2}{\rule{0.400pt}{0.843pt}}
\put(791.67,455){\rule{0.400pt}{5.300pt}}
\multiput(791.17,455.00)(1.000,11.000){2}{\rule{0.400pt}{2.650pt}}
\put(792.67,473){\rule{0.400pt}{0.964pt}}
\multiput(792.17,475.00)(1.000,-2.000){2}{\rule{0.400pt}{0.482pt}}
\put(793.67,459){\rule{0.400pt}{3.373pt}}
\multiput(793.17,466.00)(1.000,-7.000){2}{\rule{0.400pt}{1.686pt}}
\put(789.0,455.0){\usebox{\plotpoint}}
\put(795.67,459){\rule{0.400pt}{2.650pt}}
\multiput(795.17,459.00)(1.000,5.500){2}{\rule{0.400pt}{1.325pt}}
\put(796.67,462){\rule{0.400pt}{1.927pt}}
\multiput(796.17,466.00)(1.000,-4.000){2}{\rule{0.400pt}{0.964pt}}
\put(797.67,451){\rule{0.400pt}{2.650pt}}
\multiput(797.17,456.50)(1.000,-5.500){2}{\rule{0.400pt}{1.325pt}}
\put(795.0,459.0){\usebox{\plotpoint}}
\put(799.67,451){\rule{0.400pt}{6.263pt}}
\multiput(799.17,451.00)(1.000,13.000){2}{\rule{0.400pt}{3.132pt}}
\put(801.17,473){\rule{0.400pt}{0.900pt}}
\multiput(800.17,475.13)(2.000,-2.132){2}{\rule{0.400pt}{0.450pt}}
\put(802.67,462){\rule{0.400pt}{2.650pt}}
\multiput(802.17,467.50)(1.000,-5.500){2}{\rule{0.400pt}{1.325pt}}
\put(803.67,462){\rule{0.400pt}{1.927pt}}
\multiput(803.17,462.00)(1.000,4.000){2}{\rule{0.400pt}{0.964pt}}
\put(804.67,455){\rule{0.400pt}{3.614pt}}
\multiput(804.17,462.50)(1.000,-7.500){2}{\rule{0.400pt}{1.807pt}}
\put(799.0,451.0){\usebox{\plotpoint}}
\put(806.67,455){\rule{0.400pt}{1.686pt}}
\multiput(806.17,455.00)(1.000,3.500){2}{\rule{0.400pt}{0.843pt}}
\put(806.0,455.0){\usebox{\plotpoint}}
\put(808.67,462){\rule{0.400pt}{2.650pt}}
\multiput(808.17,462.00)(1.000,5.500){2}{\rule{0.400pt}{1.325pt}}
\put(809.67,462){\rule{0.400pt}{2.650pt}}
\multiput(809.17,467.50)(1.000,-5.500){2}{\rule{0.400pt}{1.325pt}}
\put(810.67,462){\rule{0.400pt}{2.650pt}}
\multiput(810.17,462.00)(1.000,5.500){2}{\rule{0.400pt}{1.325pt}}
\put(811.67,466){\rule{0.400pt}{1.686pt}}
\multiput(811.17,469.50)(1.000,-3.500){2}{\rule{0.400pt}{0.843pt}}
\put(812.67,466){\rule{0.400pt}{1.686pt}}
\multiput(812.17,466.00)(1.000,3.500){2}{\rule{0.400pt}{0.843pt}}
\put(813.67,466){\rule{0.400pt}{1.686pt}}
\multiput(813.17,469.50)(1.000,-3.500){2}{\rule{0.400pt}{0.843pt}}
\put(814.67,466){\rule{0.400pt}{2.650pt}}
\multiput(814.17,466.00)(1.000,5.500){2}{\rule{0.400pt}{1.325pt}}
\put(816.17,473){\rule{0.400pt}{0.900pt}}
\multiput(815.17,475.13)(2.000,-2.132){2}{\rule{0.400pt}{0.450pt}}
\put(817.67,459){\rule{0.400pt}{3.373pt}}
\multiput(817.17,466.00)(1.000,-7.000){2}{\rule{0.400pt}{1.686pt}}
\put(818.67,455){\rule{0.400pt}{0.964pt}}
\multiput(818.17,457.00)(1.000,-2.000){2}{\rule{0.400pt}{0.482pt}}
\put(819.67,455){\rule{0.400pt}{3.614pt}}
\multiput(819.17,455.00)(1.000,7.500){2}{\rule{0.400pt}{1.807pt}}
\put(820.67,459){\rule{0.400pt}{2.650pt}}
\multiput(820.17,464.50)(1.000,-5.500){2}{\rule{0.400pt}{1.325pt}}
\put(821.67,459){\rule{0.400pt}{3.373pt}}
\multiput(821.17,459.00)(1.000,7.000){2}{\rule{0.400pt}{1.686pt}}
\put(822.67,470){\rule{0.400pt}{0.723pt}}
\multiput(822.17,471.50)(1.000,-1.500){2}{\rule{0.400pt}{0.361pt}}
\put(823.67,455){\rule{0.400pt}{3.614pt}}
\multiput(823.17,462.50)(1.000,-7.500){2}{\rule{0.400pt}{1.807pt}}
\put(824.67,455){\rule{0.400pt}{1.686pt}}
\multiput(824.17,455.00)(1.000,3.500){2}{\rule{0.400pt}{0.843pt}}
\put(825.67,455){\rule{0.400pt}{1.686pt}}
\multiput(825.17,458.50)(1.000,-3.500){2}{\rule{0.400pt}{0.843pt}}
\put(808.0,462.0){\usebox{\plotpoint}}
\put(827.67,455){\rule{0.400pt}{1.686pt}}
\multiput(827.17,455.00)(1.000,3.500){2}{\rule{0.400pt}{0.843pt}}
\put(828.67,455){\rule{0.400pt}{1.686pt}}
\multiput(828.17,458.50)(1.000,-3.500){2}{\rule{0.400pt}{0.843pt}}
\put(829.67,455){\rule{0.400pt}{4.336pt}}
\multiput(829.17,455.00)(1.000,9.000){2}{\rule{0.400pt}{2.168pt}}
\put(827.0,455.0){\usebox{\plotpoint}}
\put(832.67,459){\rule{0.400pt}{3.373pt}}
\multiput(832.17,466.00)(1.000,-7.000){2}{\rule{0.400pt}{1.686pt}}
\put(833.67,459){\rule{0.400pt}{1.686pt}}
\multiput(833.17,459.00)(1.000,3.500){2}{\rule{0.400pt}{0.843pt}}
\put(834.67,466){\rule{0.400pt}{2.650pt}}
\multiput(834.17,466.00)(1.000,5.500){2}{\rule{0.400pt}{1.325pt}}
\put(831.0,473.0){\rule[-0.200pt]{0.482pt}{0.400pt}}
\put(836.67,470){\rule{0.400pt}{1.686pt}}
\multiput(836.17,473.50)(1.000,-3.500){2}{\rule{0.400pt}{0.843pt}}
\put(837.67,470){\rule{0.400pt}{0.723pt}}
\multiput(837.17,470.00)(1.000,1.500){2}{\rule{0.400pt}{0.361pt}}
\put(838.67,459){\rule{0.400pt}{3.373pt}}
\multiput(838.17,466.00)(1.000,-7.000){2}{\rule{0.400pt}{1.686pt}}
\put(836.0,477.0){\usebox{\plotpoint}}
\put(840.67,459){\rule{0.400pt}{3.373pt}}
\multiput(840.17,459.00)(1.000,7.000){2}{\rule{0.400pt}{1.686pt}}
\put(841.67,473){\rule{0.400pt}{0.964pt}}
\multiput(841.17,473.00)(1.000,2.000){2}{\rule{0.400pt}{0.482pt}}
\put(842.67,466){\rule{0.400pt}{2.650pt}}
\multiput(842.17,471.50)(1.000,-5.500){2}{\rule{0.400pt}{1.325pt}}
\put(840.0,459.0){\usebox{\plotpoint}}
\put(845.17,466){\rule{0.400pt}{1.500pt}}
\multiput(844.17,466.00)(2.000,3.887){2}{\rule{0.400pt}{0.750pt}}
\put(846.67,466){\rule{0.400pt}{1.686pt}}
\multiput(846.17,469.50)(1.000,-3.500){2}{\rule{0.400pt}{0.843pt}}
\put(847.67,466){\rule{0.400pt}{2.650pt}}
\multiput(847.17,466.00)(1.000,5.500){2}{\rule{0.400pt}{1.325pt}}
\put(844.0,466.0){\usebox{\plotpoint}}
\put(849.67,466){\rule{0.400pt}{2.650pt}}
\multiput(849.17,471.50)(1.000,-5.500){2}{\rule{0.400pt}{1.325pt}}
\put(849.0,477.0){\usebox{\plotpoint}}
\put(851.67,466){\rule{0.400pt}{2.650pt}}
\multiput(851.17,466.00)(1.000,5.500){2}{\rule{0.400pt}{1.325pt}}
\put(851.0,466.0){\usebox{\plotpoint}}
\put(853.67,470){\rule{0.400pt}{1.686pt}}
\multiput(853.17,473.50)(1.000,-3.500){2}{\rule{0.400pt}{0.843pt}}
\put(854.67,470){\rule{0.400pt}{1.686pt}}
\multiput(854.17,470.00)(1.000,3.500){2}{\rule{0.400pt}{0.843pt}}
\put(855.67,470){\rule{0.400pt}{1.686pt}}
\multiput(855.17,473.50)(1.000,-3.500){2}{\rule{0.400pt}{0.843pt}}
\put(856.67,470){\rule{0.400pt}{1.686pt}}
\multiput(856.17,470.00)(1.000,3.500){2}{\rule{0.400pt}{0.843pt}}
\put(857.67,466){\rule{0.400pt}{2.650pt}}
\multiput(857.17,471.50)(1.000,-5.500){2}{\rule{0.400pt}{1.325pt}}
\put(853.0,477.0){\usebox{\plotpoint}}
\put(860.17,466){\rule{0.400pt}{2.300pt}}
\multiput(859.17,466.00)(2.000,6.226){2}{\rule{0.400pt}{1.150pt}}
\put(861.67,473){\rule{0.400pt}{0.964pt}}
\multiput(861.17,475.00)(1.000,-2.000){2}{\rule{0.400pt}{0.482pt}}
\put(862.67,466){\rule{0.400pt}{1.686pt}}
\multiput(862.17,469.50)(1.000,-3.500){2}{\rule{0.400pt}{0.843pt}}
\put(863.67,466){\rule{0.400pt}{1.686pt}}
\multiput(863.17,466.00)(1.000,3.500){2}{\rule{0.400pt}{0.843pt}}
\put(864.67,466){\rule{0.400pt}{1.686pt}}
\multiput(864.17,469.50)(1.000,-3.500){2}{\rule{0.400pt}{0.843pt}}
\put(865.67,466){\rule{0.400pt}{2.650pt}}
\multiput(865.17,466.00)(1.000,5.500){2}{\rule{0.400pt}{1.325pt}}
\put(866.67,466){\rule{0.400pt}{2.650pt}}
\multiput(866.17,471.50)(1.000,-5.500){2}{\rule{0.400pt}{1.325pt}}
\put(867.67,466){\rule{0.400pt}{0.964pt}}
\multiput(867.17,466.00)(1.000,2.000){2}{\rule{0.400pt}{0.482pt}}
\put(868.67,470){\rule{0.400pt}{1.686pt}}
\multiput(868.17,470.00)(1.000,3.500){2}{\rule{0.400pt}{0.843pt}}
\put(869.67,470){\rule{0.400pt}{1.686pt}}
\multiput(869.17,473.50)(1.000,-3.500){2}{\rule{0.400pt}{0.843pt}}
\put(870.67,470){\rule{0.400pt}{1.686pt}}
\multiput(870.17,470.00)(1.000,3.500){2}{\rule{0.400pt}{0.843pt}}
\put(871.67,470){\rule{0.400pt}{1.686pt}}
\multiput(871.17,473.50)(1.000,-3.500){2}{\rule{0.400pt}{0.843pt}}
\put(872.67,470){\rule{0.400pt}{2.409pt}}
\multiput(872.17,470.00)(1.000,5.000){2}{\rule{0.400pt}{1.204pt}}
\put(873.67,470){\rule{0.400pt}{2.409pt}}
\multiput(873.17,475.00)(1.000,-5.000){2}{\rule{0.400pt}{1.204pt}}
\put(875.17,470){\rule{0.400pt}{1.500pt}}
\multiput(874.17,470.00)(2.000,3.887){2}{\rule{0.400pt}{0.750pt}}
\put(876.67,470){\rule{0.400pt}{1.686pt}}
\multiput(876.17,473.50)(1.000,-3.500){2}{\rule{0.400pt}{0.843pt}}
\put(877.67,470){\rule{0.400pt}{1.686pt}}
\multiput(877.17,470.00)(1.000,3.500){2}{\rule{0.400pt}{0.843pt}}
\put(878.67,470){\rule{0.400pt}{1.686pt}}
\multiput(878.17,473.50)(1.000,-3.500){2}{\rule{0.400pt}{0.843pt}}
\put(879.67,470){\rule{0.400pt}{2.409pt}}
\multiput(879.17,470.00)(1.000,5.000){2}{\rule{0.400pt}{1.204pt}}
\put(859.0,466.0){\usebox{\plotpoint}}
\put(881.67,470){\rule{0.400pt}{2.409pt}}
\multiput(881.17,475.00)(1.000,-5.000){2}{\rule{0.400pt}{1.204pt}}
\put(882.67,466){\rule{0.400pt}{0.964pt}}
\multiput(882.17,468.00)(1.000,-2.000){2}{\rule{0.400pt}{0.482pt}}
\put(883.67,466){\rule{0.400pt}{2.650pt}}
\multiput(883.17,466.00)(1.000,5.500){2}{\rule{0.400pt}{1.325pt}}
\put(884.67,470){\rule{0.400pt}{1.686pt}}
\multiput(884.17,473.50)(1.000,-3.500){2}{\rule{0.400pt}{0.843pt}}
\put(885.67,470){\rule{0.400pt}{2.409pt}}
\multiput(885.17,470.00)(1.000,5.000){2}{\rule{0.400pt}{1.204pt}}
\put(886.67,470){\rule{0.400pt}{2.409pt}}
\multiput(886.17,475.00)(1.000,-5.000){2}{\rule{0.400pt}{1.204pt}}
\put(887.67,470){\rule{0.400pt}{2.409pt}}
\multiput(887.17,470.00)(1.000,5.000){2}{\rule{0.400pt}{1.204pt}}
\put(889.17,477){\rule{0.400pt}{0.700pt}}
\multiput(888.17,478.55)(2.000,-1.547){2}{\rule{0.400pt}{0.350pt}}
\put(890.67,470){\rule{0.400pt}{1.686pt}}
\multiput(890.17,473.50)(1.000,-3.500){2}{\rule{0.400pt}{0.843pt}}
\put(891.67,470){\rule{0.400pt}{0.723pt}}
\multiput(891.17,470.00)(1.000,1.500){2}{\rule{0.400pt}{0.361pt}}
\put(892.67,473){\rule{0.400pt}{0.964pt}}
\multiput(892.17,473.00)(1.000,2.000){2}{\rule{0.400pt}{0.482pt}}
\put(881.0,480.0){\usebox{\plotpoint}}
\put(894.67,470){\rule{0.400pt}{1.686pt}}
\multiput(894.17,473.50)(1.000,-3.500){2}{\rule{0.400pt}{0.843pt}}
\put(894.0,477.0){\usebox{\plotpoint}}
\put(896.67,470){\rule{0.400pt}{2.409pt}}
\multiput(896.17,470.00)(1.000,5.000){2}{\rule{0.400pt}{1.204pt}}
\put(897.67,473){\rule{0.400pt}{1.686pt}}
\multiput(897.17,476.50)(1.000,-3.500){2}{\rule{0.400pt}{0.843pt}}
\put(898.67,473){\rule{0.400pt}{3.614pt}}
\multiput(898.17,473.00)(1.000,7.500){2}{\rule{0.400pt}{1.807pt}}
\put(899.67,484){\rule{0.400pt}{0.964pt}}
\multiput(899.17,486.00)(1.000,-2.000){2}{\rule{0.400pt}{0.482pt}}
\put(900.67,473){\rule{0.400pt}{2.650pt}}
\multiput(900.17,478.50)(1.000,-5.500){2}{\rule{0.400pt}{1.325pt}}
\put(901.67,473){\rule{0.400pt}{1.686pt}}
\multiput(901.17,473.00)(1.000,3.500){2}{\rule{0.400pt}{0.843pt}}
\put(902.67,473){\rule{0.400pt}{1.686pt}}
\multiput(902.17,476.50)(1.000,-3.500){2}{\rule{0.400pt}{0.843pt}}
\put(896.0,470.0){\usebox{\plotpoint}}
\put(905.67,473){\rule{0.400pt}{2.650pt}}
\multiput(905.17,473.00)(1.000,5.500){2}{\rule{0.400pt}{1.325pt}}
\put(906.67,477){\rule{0.400pt}{1.686pt}}
\multiput(906.17,480.50)(1.000,-3.500){2}{\rule{0.400pt}{0.843pt}}
\put(907.67,477){\rule{0.400pt}{2.650pt}}
\multiput(907.17,477.00)(1.000,5.500){2}{\rule{0.400pt}{1.325pt}}
\put(908.67,484){\rule{0.400pt}{0.964pt}}
\multiput(908.17,486.00)(1.000,-2.000){2}{\rule{0.400pt}{0.482pt}}
\put(909.67,473){\rule{0.400pt}{2.650pt}}
\multiput(909.17,478.50)(1.000,-5.500){2}{\rule{0.400pt}{1.325pt}}
\put(910.67,470){\rule{0.400pt}{0.723pt}}
\multiput(910.17,471.50)(1.000,-1.500){2}{\rule{0.400pt}{0.361pt}}
\put(911.67,470){\rule{0.400pt}{2.409pt}}
\multiput(911.17,470.00)(1.000,5.000){2}{\rule{0.400pt}{1.204pt}}
\put(912.67,480){\rule{0.400pt}{0.964pt}}
\multiput(912.17,480.00)(1.000,2.000){2}{\rule{0.400pt}{0.482pt}}
\put(913.67,473){\rule{0.400pt}{2.650pt}}
\multiput(913.17,478.50)(1.000,-5.500){2}{\rule{0.400pt}{1.325pt}}
\put(914.67,473){\rule{0.400pt}{0.964pt}}
\multiput(914.17,473.00)(1.000,2.000){2}{\rule{0.400pt}{0.482pt}}
\put(915.67,477){\rule{0.400pt}{2.650pt}}
\multiput(915.17,477.00)(1.000,5.500){2}{\rule{0.400pt}{1.325pt}}
\put(904.0,473.0){\rule[-0.200pt]{0.482pt}{0.400pt}}
\put(918.17,473){\rule{0.400pt}{3.100pt}}
\multiput(917.17,481.57)(2.000,-8.566){2}{\rule{0.400pt}{1.550pt}}
\put(917.0,488.0){\usebox{\plotpoint}}
\put(920.67,473){\rule{0.400pt}{1.686pt}}
\multiput(920.17,473.00)(1.000,3.500){2}{\rule{0.400pt}{0.843pt}}
\put(921.67,473){\rule{0.400pt}{1.686pt}}
\multiput(921.17,476.50)(1.000,-3.500){2}{\rule{0.400pt}{0.843pt}}
\put(922.67,473){\rule{0.400pt}{2.650pt}}
\multiput(922.17,473.00)(1.000,5.500){2}{\rule{0.400pt}{1.325pt}}
\put(923.67,470){\rule{0.400pt}{3.373pt}}
\multiput(923.17,477.00)(1.000,-7.000){2}{\rule{0.400pt}{1.686pt}}
\put(924.67,470){\rule{0.400pt}{2.409pt}}
\multiput(924.17,470.00)(1.000,5.000){2}{\rule{0.400pt}{1.204pt}}
\put(925.67,473){\rule{0.400pt}{1.686pt}}
\multiput(925.17,476.50)(1.000,-3.500){2}{\rule{0.400pt}{0.843pt}}
\put(926.67,473){\rule{0.400pt}{1.686pt}}
\multiput(926.17,473.00)(1.000,3.500){2}{\rule{0.400pt}{0.843pt}}
\put(920.0,473.0){\usebox{\plotpoint}}
\put(928.67,473){\rule{0.400pt}{1.686pt}}
\multiput(928.17,476.50)(1.000,-3.500){2}{\rule{0.400pt}{0.843pt}}
\put(928.0,480.0){\usebox{\plotpoint}}
\put(930.67,473){\rule{0.400pt}{3.614pt}}
\multiput(930.17,473.00)(1.000,7.500){2}{\rule{0.400pt}{1.807pt}}
\put(931.67,488){\rule{0.400pt}{0.723pt}}
\multiput(931.17,488.00)(1.000,1.500){2}{\rule{0.400pt}{0.361pt}}
\put(933.17,473){\rule{0.400pt}{3.700pt}}
\multiput(932.17,483.32)(2.000,-10.320){2}{\rule{0.400pt}{1.850pt}}
\put(934.67,473){\rule{0.400pt}{1.686pt}}
\multiput(934.17,473.00)(1.000,3.500){2}{\rule{0.400pt}{0.843pt}}
\put(935.67,462){\rule{0.400pt}{4.336pt}}
\multiput(935.17,471.00)(1.000,-9.000){2}{\rule{0.400pt}{2.168pt}}
\put(936.67,462){\rule{0.400pt}{2.650pt}}
\multiput(936.17,462.00)(1.000,5.500){2}{\rule{0.400pt}{1.325pt}}
\put(937.67,473){\rule{0.400pt}{4.336pt}}
\multiput(937.17,473.00)(1.000,9.000){2}{\rule{0.400pt}{2.168pt}}
\put(938.67,484){\rule{0.400pt}{1.686pt}}
\multiput(938.17,487.50)(1.000,-3.500){2}{\rule{0.400pt}{0.843pt}}
\put(939.67,484){\rule{0.400pt}{1.686pt}}
\multiput(939.17,484.00)(1.000,3.500){2}{\rule{0.400pt}{0.843pt}}
\put(930.0,473.0){\usebox{\plotpoint}}
\put(941.67,477){\rule{0.400pt}{3.373pt}}
\multiput(941.17,484.00)(1.000,-7.000){2}{\rule{0.400pt}{1.686pt}}
\put(941.0,491.0){\usebox{\plotpoint}}
\put(943.67,477){\rule{0.400pt}{2.650pt}}
\multiput(943.17,477.00)(1.000,5.500){2}{\rule{0.400pt}{1.325pt}}
\put(944.67,477){\rule{0.400pt}{2.650pt}}
\multiput(944.17,482.50)(1.000,-5.500){2}{\rule{0.400pt}{1.325pt}}
\put(945.67,477){\rule{0.400pt}{2.650pt}}
\multiput(945.17,477.00)(1.000,5.500){2}{\rule{0.400pt}{1.325pt}}
\put(946.67,484){\rule{0.400pt}{0.964pt}}
\multiput(946.17,486.00)(1.000,-2.000){2}{\rule{0.400pt}{0.482pt}}
\put(948.17,473){\rule{0.400pt}{2.300pt}}
\multiput(947.17,479.23)(2.000,-6.226){2}{\rule{0.400pt}{1.150pt}}
\put(949.67,473){\rule{0.400pt}{0.964pt}}
\multiput(949.17,473.00)(1.000,2.000){2}{\rule{0.400pt}{0.482pt}}
\put(950.67,466){\rule{0.400pt}{2.650pt}}
\multiput(950.17,471.50)(1.000,-5.500){2}{\rule{0.400pt}{1.325pt}}
\put(943.0,477.0){\usebox{\plotpoint}}
\put(952.67,466){\rule{0.400pt}{3.373pt}}
\multiput(952.17,466.00)(1.000,7.000){2}{\rule{0.400pt}{1.686pt}}
\put(953.67,477){\rule{0.400pt}{0.723pt}}
\multiput(953.17,478.50)(1.000,-1.500){2}{\rule{0.400pt}{0.361pt}}
\put(954.67,477){\rule{0.400pt}{3.373pt}}
\multiput(954.17,477.00)(1.000,7.000){2}{\rule{0.400pt}{1.686pt}}
\put(952.0,466.0){\usebox{\plotpoint}}
\put(956.67,470){\rule{0.400pt}{5.059pt}}
\multiput(956.17,480.50)(1.000,-10.500){2}{\rule{0.400pt}{2.529pt}}
\put(956.0,491.0){\usebox{\plotpoint}}
\put(958.67,470){\rule{0.400pt}{6.023pt}}
\multiput(958.17,470.00)(1.000,12.500){2}{\rule{0.400pt}{3.011pt}}
\put(958.0,470.0){\usebox{\plotpoint}}
\put(960.67,484){\rule{0.400pt}{2.650pt}}
\multiput(960.17,489.50)(1.000,-5.500){2}{\rule{0.400pt}{1.325pt}}
\put(962.17,484){\rule{0.400pt}{1.500pt}}
\multiput(961.17,484.00)(2.000,3.887){2}{\rule{0.400pt}{0.750pt}}
\put(963.67,470){\rule{0.400pt}{5.059pt}}
\multiput(963.17,480.50)(1.000,-10.500){2}{\rule{0.400pt}{2.529pt}}
\put(964.67,470){\rule{0.400pt}{0.723pt}}
\multiput(964.17,470.00)(1.000,1.500){2}{\rule{0.400pt}{0.361pt}}
\put(965.67,473){\rule{0.400pt}{3.614pt}}
\multiput(965.17,473.00)(1.000,7.500){2}{\rule{0.400pt}{1.807pt}}
\put(966.67,484){\rule{0.400pt}{0.964pt}}
\multiput(966.17,486.00)(1.000,-2.000){2}{\rule{0.400pt}{0.482pt}}
\put(967.67,484){\rule{0.400pt}{2.650pt}}
\multiput(967.17,484.00)(1.000,5.500){2}{\rule{0.400pt}{1.325pt}}
\put(968.67,477){\rule{0.400pt}{4.336pt}}
\multiput(968.17,486.00)(1.000,-9.000){2}{\rule{0.400pt}{2.168pt}}
\put(969.67,477){\rule{0.400pt}{3.373pt}}
\multiput(969.17,477.00)(1.000,7.000){2}{\rule{0.400pt}{1.686pt}}
\put(960.0,495.0){\usebox{\plotpoint}}
\put(971.67,477){\rule{0.400pt}{3.373pt}}
\multiput(971.17,484.00)(1.000,-7.000){2}{\rule{0.400pt}{1.686pt}}
\put(971.0,491.0){\usebox{\plotpoint}}
\put(973.67,477){\rule{0.400pt}{2.650pt}}
\multiput(973.17,477.00)(1.000,5.500){2}{\rule{0.400pt}{1.325pt}}
\put(973.0,477.0){\usebox{\plotpoint}}
\put(975.67,473){\rule{0.400pt}{3.614pt}}
\multiput(975.17,480.50)(1.000,-7.500){2}{\rule{0.400pt}{1.807pt}}
\put(975.0,488.0){\usebox{\plotpoint}}
\put(978.67,473){\rule{0.400pt}{3.614pt}}
\multiput(978.17,473.00)(1.000,7.500){2}{\rule{0.400pt}{1.807pt}}
\put(979.67,477){\rule{0.400pt}{2.650pt}}
\multiput(979.17,482.50)(1.000,-5.500){2}{\rule{0.400pt}{1.325pt}}
\put(980.67,477){\rule{0.400pt}{2.650pt}}
\multiput(980.17,477.00)(1.000,5.500){2}{\rule{0.400pt}{1.325pt}}
\put(981.67,480){\rule{0.400pt}{1.927pt}}
\multiput(981.17,484.00)(1.000,-4.000){2}{\rule{0.400pt}{0.964pt}}
\put(982.67,480){\rule{0.400pt}{2.650pt}}
\multiput(982.17,480.00)(1.000,5.500){2}{\rule{0.400pt}{1.325pt}}
\put(983.67,484){\rule{0.400pt}{1.686pt}}
\multiput(983.17,487.50)(1.000,-3.500){2}{\rule{0.400pt}{0.843pt}}
\put(984.67,484){\rule{0.400pt}{2.650pt}}
\multiput(984.17,484.00)(1.000,5.500){2}{\rule{0.400pt}{1.325pt}}
\put(977.0,473.0){\rule[-0.200pt]{0.482pt}{0.400pt}}
\put(986.67,484){\rule{0.400pt}{2.650pt}}
\multiput(986.17,489.50)(1.000,-5.500){2}{\rule{0.400pt}{1.325pt}}
\put(987.67,484){\rule{0.400pt}{1.686pt}}
\multiput(987.17,484.00)(1.000,3.500){2}{\rule{0.400pt}{0.843pt}}
\put(988.67,477){\rule{0.400pt}{3.373pt}}
\multiput(988.17,484.00)(1.000,-7.000){2}{\rule{0.400pt}{1.686pt}}
\put(989.67,477){\rule{0.400pt}{2.650pt}}
\multiput(989.17,477.00)(1.000,5.500){2}{\rule{0.400pt}{1.325pt}}
\put(990.67,477){\rule{0.400pt}{2.650pt}}
\multiput(990.17,482.50)(1.000,-5.500){2}{\rule{0.400pt}{1.325pt}}
\put(986.0,495.0){\usebox{\plotpoint}}
\put(993.67,477){\rule{0.400pt}{3.373pt}}
\multiput(993.17,477.00)(1.000,7.000){2}{\rule{0.400pt}{1.686pt}}
\put(994.67,484){\rule{0.400pt}{1.686pt}}
\multiput(994.17,487.50)(1.000,-3.500){2}{\rule{0.400pt}{0.843pt}}
\put(995.67,484){\rule{0.400pt}{3.614pt}}
\multiput(995.17,484.00)(1.000,7.500){2}{\rule{0.400pt}{1.807pt}}
\put(996.67,495){\rule{0.400pt}{0.964pt}}
\multiput(996.17,497.00)(1.000,-2.000){2}{\rule{0.400pt}{0.482pt}}
\put(997.67,484){\rule{0.400pt}{2.650pt}}
\multiput(997.17,489.50)(1.000,-5.500){2}{\rule{0.400pt}{1.325pt}}
\put(998.67,484){\rule{0.400pt}{2.650pt}}
\multiput(998.17,484.00)(1.000,5.500){2}{\rule{0.400pt}{1.325pt}}
\put(999.67,484){\rule{0.400pt}{2.650pt}}
\multiput(999.17,489.50)(1.000,-5.500){2}{\rule{0.400pt}{1.325pt}}
\put(1000.67,484){\rule{0.400pt}{1.686pt}}
\multiput(1000.17,484.00)(1.000,3.500){2}{\rule{0.400pt}{0.843pt}}
\put(1001.67,477){\rule{0.400pt}{3.373pt}}
\multiput(1001.17,484.00)(1.000,-7.000){2}{\rule{0.400pt}{1.686pt}}
\put(1002.67,477){\rule{0.400pt}{1.686pt}}
\multiput(1002.17,477.00)(1.000,3.500){2}{\rule{0.400pt}{0.843pt}}
\put(1003.67,484){\rule{0.400pt}{2.650pt}}
\multiput(1003.17,484.00)(1.000,5.500){2}{\rule{0.400pt}{1.325pt}}
\put(1004.67,484){\rule{0.400pt}{2.650pt}}
\multiput(1004.17,489.50)(1.000,-5.500){2}{\rule{0.400pt}{1.325pt}}
\put(1006.17,484){\rule{0.400pt}{1.500pt}}
\multiput(1005.17,484.00)(2.000,3.887){2}{\rule{0.400pt}{0.750pt}}
\put(1007.67,484){\rule{0.400pt}{1.686pt}}
\multiput(1007.17,487.50)(1.000,-3.500){2}{\rule{0.400pt}{0.843pt}}
\put(1008.67,484){\rule{0.400pt}{1.686pt}}
\multiput(1008.17,484.00)(1.000,3.500){2}{\rule{0.400pt}{0.843pt}}
\put(1009.67,477){\rule{0.400pt}{3.373pt}}
\multiput(1009.17,484.00)(1.000,-7.000){2}{\rule{0.400pt}{1.686pt}}
\put(1010.67,477){\rule{0.400pt}{3.373pt}}
\multiput(1010.17,477.00)(1.000,7.000){2}{\rule{0.400pt}{1.686pt}}
\put(992.0,477.0){\rule[-0.200pt]{0.482pt}{0.400pt}}
\put(1012.67,484){\rule{0.400pt}{1.686pt}}
\multiput(1012.17,487.50)(1.000,-3.500){2}{\rule{0.400pt}{0.843pt}}
\put(1012.0,491.0){\usebox{\plotpoint}}
\put(1014.67,484){\rule{0.400pt}{2.650pt}}
\multiput(1014.17,484.00)(1.000,5.500){2}{\rule{0.400pt}{1.325pt}}
\put(1015.67,491){\rule{0.400pt}{0.964pt}}
\multiput(1015.17,493.00)(1.000,-2.000){2}{\rule{0.400pt}{0.482pt}}
\put(1016.67,484){\rule{0.400pt}{1.686pt}}
\multiput(1016.17,487.50)(1.000,-3.500){2}{\rule{0.400pt}{0.843pt}}
\put(1014.0,484.0){\usebox{\plotpoint}}
\put(1018.67,484){\rule{0.400pt}{2.650pt}}
\multiput(1018.17,484.00)(1.000,5.500){2}{\rule{0.400pt}{1.325pt}}
\put(1019.67,484){\rule{0.400pt}{2.650pt}}
\multiput(1019.17,489.50)(1.000,-5.500){2}{\rule{0.400pt}{1.325pt}}
\put(1021.17,484){\rule{0.400pt}{3.100pt}}
\multiput(1020.17,484.00)(2.000,8.566){2}{\rule{0.400pt}{1.550pt}}
\put(1018.0,484.0){\usebox{\plotpoint}}
\put(1023.67,488){\rule{0.400pt}{2.650pt}}
\multiput(1023.17,493.50)(1.000,-5.500){2}{\rule{0.400pt}{1.325pt}}
\put(1024.67,488){\rule{0.400pt}{0.723pt}}
\multiput(1024.17,488.00)(1.000,1.500){2}{\rule{0.400pt}{0.361pt}}
\put(1025.67,491){\rule{0.400pt}{1.927pt}}
\multiput(1025.17,491.00)(1.000,4.000){2}{\rule{0.400pt}{0.964pt}}
\put(1026.67,491){\rule{0.400pt}{1.927pt}}
\multiput(1026.17,495.00)(1.000,-4.000){2}{\rule{0.400pt}{0.964pt}}
\put(1027.67,491){\rule{0.400pt}{2.650pt}}
\multiput(1027.17,491.00)(1.000,5.500){2}{\rule{0.400pt}{1.325pt}}
\put(1023.0,499.0){\usebox{\plotpoint}}
\put(1029.67,495){\rule{0.400pt}{1.686pt}}
\multiput(1029.17,498.50)(1.000,-3.500){2}{\rule{0.400pt}{0.843pt}}
\put(1030.67,491){\rule{0.400pt}{0.964pt}}
\multiput(1030.17,493.00)(1.000,-2.000){2}{\rule{0.400pt}{0.482pt}}
\put(1031.67,491){\rule{0.400pt}{2.650pt}}
\multiput(1031.17,491.00)(1.000,5.500){2}{\rule{0.400pt}{1.325pt}}
\put(1029.0,502.0){\usebox{\plotpoint}}
\put(1033.67,491){\rule{0.400pt}{2.650pt}}
\multiput(1033.17,496.50)(1.000,-5.500){2}{\rule{0.400pt}{1.325pt}}
\put(1034.67,491){\rule{0.400pt}{1.927pt}}
\multiput(1034.17,491.00)(1.000,4.000){2}{\rule{0.400pt}{0.964pt}}
\put(1036.17,491){\rule{0.400pt}{1.700pt}}
\multiput(1035.17,495.47)(2.000,-4.472){2}{\rule{0.400pt}{0.850pt}}
\put(1037.67,488){\rule{0.400pt}{0.723pt}}
\multiput(1037.17,489.50)(1.000,-1.500){2}{\rule{0.400pt}{0.361pt}}
\put(1038.67,488){\rule{0.400pt}{2.650pt}}
\multiput(1038.17,488.00)(1.000,5.500){2}{\rule{0.400pt}{1.325pt}}
\put(1033.0,502.0){\usebox{\plotpoint}}
\put(1040.67,488){\rule{0.400pt}{2.650pt}}
\multiput(1040.17,493.50)(1.000,-5.500){2}{\rule{0.400pt}{1.325pt}}
\put(1041.67,488){\rule{0.400pt}{1.686pt}}
\multiput(1041.17,488.00)(1.000,3.500){2}{\rule{0.400pt}{0.843pt}}
\put(1042.67,484){\rule{0.400pt}{2.650pt}}
\multiput(1042.17,489.50)(1.000,-5.500){2}{\rule{0.400pt}{1.325pt}}
\put(1043.67,484){\rule{0.400pt}{0.964pt}}
\multiput(1043.17,484.00)(1.000,2.000){2}{\rule{0.400pt}{0.482pt}}
\put(1044.67,488){\rule{0.400pt}{2.650pt}}
\multiput(1044.17,488.00)(1.000,5.500){2}{\rule{0.400pt}{1.325pt}}
\put(1045.67,495){\rule{0.400pt}{0.964pt}}
\multiput(1045.17,497.00)(1.000,-2.000){2}{\rule{0.400pt}{0.482pt}}
\put(1046.67,495){\rule{0.400pt}{3.614pt}}
\multiput(1046.17,495.00)(1.000,7.500){2}{\rule{0.400pt}{1.807pt}}
\put(1047.67,499){\rule{0.400pt}{2.650pt}}
\multiput(1047.17,504.50)(1.000,-5.500){2}{\rule{0.400pt}{1.325pt}}
\put(1048.67,484){\rule{0.400pt}{3.614pt}}
\multiput(1048.17,491.50)(1.000,-7.500){2}{\rule{0.400pt}{1.807pt}}
\put(1050.17,484){\rule{0.400pt}{1.500pt}}
\multiput(1049.17,484.00)(2.000,3.887){2}{\rule{0.400pt}{0.750pt}}
\put(1051.67,491){\rule{0.400pt}{4.577pt}}
\multiput(1051.17,491.00)(1.000,9.500){2}{\rule{0.400pt}{2.289pt}}
\put(1052.67,506){\rule{0.400pt}{0.964pt}}
\multiput(1052.17,508.00)(1.000,-2.000){2}{\rule{0.400pt}{0.482pt}}
\put(1053.67,491){\rule{0.400pt}{3.614pt}}
\multiput(1053.17,498.50)(1.000,-7.500){2}{\rule{0.400pt}{1.807pt}}
\put(1054.67,488){\rule{0.400pt}{0.723pt}}
\multiput(1054.17,489.50)(1.000,-1.500){2}{\rule{0.400pt}{0.361pt}}
\put(1055.67,488){\rule{0.400pt}{2.650pt}}
\multiput(1055.17,488.00)(1.000,5.500){2}{\rule{0.400pt}{1.325pt}}
\put(1056.67,491){\rule{0.400pt}{1.927pt}}
\multiput(1056.17,495.00)(1.000,-4.000){2}{\rule{0.400pt}{0.964pt}}
\put(1057.67,491){\rule{0.400pt}{1.927pt}}
\multiput(1057.17,491.00)(1.000,4.000){2}{\rule{0.400pt}{0.964pt}}
\put(1040.0,499.0){\usebox{\plotpoint}}
\put(1059.67,491){\rule{0.400pt}{1.927pt}}
\multiput(1059.17,495.00)(1.000,-4.000){2}{\rule{0.400pt}{0.964pt}}
\put(1060.67,488){\rule{0.400pt}{0.723pt}}
\multiput(1060.17,489.50)(1.000,-1.500){2}{\rule{0.400pt}{0.361pt}}
\put(1061.67,488){\rule{0.400pt}{3.373pt}}
\multiput(1061.17,488.00)(1.000,7.000){2}{\rule{0.400pt}{1.686pt}}
\put(1059.0,499.0){\usebox{\plotpoint}}
\put(1063.67,484){\rule{0.400pt}{4.336pt}}
\multiput(1063.17,493.00)(1.000,-9.000){2}{\rule{0.400pt}{2.168pt}}
\put(1065.17,484){\rule{0.400pt}{0.900pt}}
\multiput(1064.17,484.00)(2.000,2.132){2}{\rule{0.400pt}{0.450pt}}
\put(1066.67,488){\rule{0.400pt}{2.650pt}}
\multiput(1066.17,488.00)(1.000,5.500){2}{\rule{0.400pt}{1.325pt}}
\put(1067.67,491){\rule{0.400pt}{1.927pt}}
\multiput(1067.17,495.00)(1.000,-4.000){2}{\rule{0.400pt}{0.964pt}}
\put(1068.67,491){\rule{0.400pt}{2.650pt}}
\multiput(1068.17,491.00)(1.000,5.500){2}{\rule{0.400pt}{1.325pt}}
\put(1069.67,491){\rule{0.400pt}{2.650pt}}
\multiput(1069.17,496.50)(1.000,-5.500){2}{\rule{0.400pt}{1.325pt}}
\put(1070.67,491){\rule{0.400pt}{2.650pt}}
\multiput(1070.17,491.00)(1.000,5.500){2}{\rule{0.400pt}{1.325pt}}
\put(1071.67,495){\rule{0.400pt}{1.686pt}}
\multiput(1071.17,498.50)(1.000,-3.500){2}{\rule{0.400pt}{0.843pt}}
\put(1072.67,495){\rule{0.400pt}{1.686pt}}
\multiput(1072.17,495.00)(1.000,3.500){2}{\rule{0.400pt}{0.843pt}}
\put(1063.0,502.0){\usebox{\plotpoint}}
\put(1074.67,495){\rule{0.400pt}{1.686pt}}
\multiput(1074.17,498.50)(1.000,-3.500){2}{\rule{0.400pt}{0.843pt}}
\put(1075.67,495){\rule{0.400pt}{1.686pt}}
\multiput(1075.17,495.00)(1.000,3.500){2}{\rule{0.400pt}{0.843pt}}
\put(1076.67,488){\rule{0.400pt}{3.373pt}}
\multiput(1076.17,495.00)(1.000,-7.000){2}{\rule{0.400pt}{1.686pt}}
\put(1074.0,502.0){\usebox{\plotpoint}}
\put(1079.17,488){\rule{0.400pt}{2.300pt}}
\multiput(1078.17,488.00)(2.000,6.226){2}{\rule{0.400pt}{1.150pt}}
\put(1080.67,491){\rule{0.400pt}{1.927pt}}
\multiput(1080.17,495.00)(1.000,-4.000){2}{\rule{0.400pt}{0.964pt}}
\put(1081.67,491){\rule{0.400pt}{4.577pt}}
\multiput(1081.17,491.00)(1.000,9.500){2}{\rule{0.400pt}{2.289pt}}
\put(1082.67,499){\rule{0.400pt}{2.650pt}}
\multiput(1082.17,504.50)(1.000,-5.500){2}{\rule{0.400pt}{1.325pt}}
\put(1083.67,499){\rule{0.400pt}{2.650pt}}
\multiput(1083.17,499.00)(1.000,5.500){2}{\rule{0.400pt}{1.325pt}}
\put(1084.67,502){\rule{0.400pt}{1.927pt}}
\multiput(1084.17,506.00)(1.000,-4.000){2}{\rule{0.400pt}{0.964pt}}
\put(1085.67,495){\rule{0.400pt}{1.686pt}}
\multiput(1085.17,498.50)(1.000,-3.500){2}{\rule{0.400pt}{0.843pt}}
\put(1086.67,491){\rule{0.400pt}{0.964pt}}
\multiput(1086.17,493.00)(1.000,-2.000){2}{\rule{0.400pt}{0.482pt}}
\put(1087.67,491){\rule{0.400pt}{2.650pt}}
\multiput(1087.17,491.00)(1.000,5.500){2}{\rule{0.400pt}{1.325pt}}
\put(1088.67,477){\rule{0.400pt}{6.023pt}}
\multiput(1088.17,489.50)(1.000,-12.500){2}{\rule{0.400pt}{3.011pt}}
\put(1089.67,477){\rule{0.400pt}{7.950pt}}
\multiput(1089.17,477.00)(1.000,16.500){2}{\rule{0.400pt}{3.975pt}}
\put(1090.67,499){\rule{0.400pt}{2.650pt}}
\multiput(1090.17,504.50)(1.000,-5.500){2}{\rule{0.400pt}{1.325pt}}
\put(1091.67,499){\rule{0.400pt}{3.373pt}}
\multiput(1091.17,499.00)(1.000,7.000){2}{\rule{0.400pt}{1.686pt}}
\put(1078.0,488.0){\usebox{\plotpoint}}
\put(1094.17,466){\rule{0.400pt}{9.500pt}}
\multiput(1093.17,493.28)(2.000,-27.282){2}{\rule{0.400pt}{4.750pt}}
\put(1095.67,466){\rule{0.400pt}{1.686pt}}
\multiput(1095.17,466.00)(1.000,3.500){2}{\rule{0.400pt}{0.843pt}}
\put(1096.67,473){\rule{0.400pt}{10.600pt}}
\multiput(1096.17,473.00)(1.000,22.000){2}{\rule{0.400pt}{5.300pt}}
\put(1093.0,513.0){\usebox{\plotpoint}}
\put(1098.67,491){\rule{0.400pt}{6.263pt}}
\multiput(1098.17,504.00)(1.000,-13.000){2}{\rule{0.400pt}{3.132pt}}
\put(1099.67,491){\rule{0.400pt}{0.964pt}}
\multiput(1099.17,491.00)(1.000,2.000){2}{\rule{0.400pt}{0.482pt}}
\put(1100.67,495){\rule{0.400pt}{6.023pt}}
\multiput(1100.17,495.00)(1.000,12.500){2}{\rule{0.400pt}{3.011pt}}
\put(1101.67,520){\rule{0.400pt}{1.927pt}}
\multiput(1101.17,520.00)(1.000,4.000){2}{\rule{0.400pt}{0.964pt}}
\put(1102.67,506){\rule{0.400pt}{5.300pt}}
\multiput(1102.17,517.00)(1.000,-11.000){2}{\rule{0.400pt}{2.650pt}}
\put(1103.67,491){\rule{0.400pt}{3.614pt}}
\multiput(1103.17,498.50)(1.000,-7.500){2}{\rule{0.400pt}{1.807pt}}
\put(1104.67,491){\rule{0.400pt}{6.986pt}}
\multiput(1104.17,491.00)(1.000,14.500){2}{\rule{0.400pt}{3.493pt}}
\put(1105.67,506){\rule{0.400pt}{3.373pt}}
\multiput(1105.17,513.00)(1.000,-7.000){2}{\rule{0.400pt}{1.686pt}}
\put(1106.67,506){\rule{0.400pt}{7.950pt}}
\multiput(1106.17,506.00)(1.000,16.500){2}{\rule{0.400pt}{3.975pt}}
\put(1107.67,524){\rule{0.400pt}{3.614pt}}
\multiput(1107.17,531.50)(1.000,-7.500){2}{\rule{0.400pt}{1.807pt}}
\put(1109.17,524){\rule{0.400pt}{7.500pt}}
\multiput(1108.17,524.00)(2.000,21.433){2}{\rule{0.400pt}{3.750pt}}
\put(1110.67,546){\rule{0.400pt}{3.614pt}}
\multiput(1110.17,553.50)(1.000,-7.500){2}{\rule{0.400pt}{1.807pt}}
\put(1111.67,546){\rule{0.400pt}{11.322pt}}
\multiput(1111.17,546.00)(1.000,23.500){2}{\rule{0.400pt}{5.661pt}}
\put(1098.0,517.0){\usebox{\plotpoint}}
\put(1113.67,568){\rule{0.400pt}{6.023pt}}
\multiput(1113.17,580.50)(1.000,-12.500){2}{\rule{0.400pt}{3.011pt}}
\put(1114.67,568){\rule{0.400pt}{2.650pt}}
\multiput(1114.17,568.00)(1.000,5.500){2}{\rule{0.400pt}{1.325pt}}
\put(1115.67,553){\rule{0.400pt}{6.263pt}}
\multiput(1115.17,566.00)(1.000,-13.000){2}{\rule{0.400pt}{3.132pt}}
\put(1116.67,553){\rule{0.400pt}{1.927pt}}
\multiput(1116.17,553.00)(1.000,4.000){2}{\rule{0.400pt}{0.964pt}}
\put(1117.67,539){\rule{0.400pt}{5.300pt}}
\multiput(1117.17,550.00)(1.000,-11.000){2}{\rule{0.400pt}{2.650pt}}
\put(1118.67,539){\rule{0.400pt}{0.723pt}}
\multiput(1118.17,539.00)(1.000,1.500){2}{\rule{0.400pt}{0.361pt}}
\put(1119.67,520){\rule{0.400pt}{5.300pt}}
\multiput(1119.17,531.00)(1.000,-11.000){2}{\rule{0.400pt}{2.650pt}}
\put(1120.67,517){\rule{0.400pt}{0.723pt}}
\multiput(1120.17,518.50)(1.000,-1.500){2}{\rule{0.400pt}{0.361pt}}
\put(1121.67,517){\rule{0.400pt}{7.950pt}}
\multiput(1121.17,517.00)(1.000,16.500){2}{\rule{0.400pt}{3.975pt}}
\put(1113.0,593.0){\usebox{\plotpoint}}
\put(1124.67,535){\rule{0.400pt}{3.614pt}}
\multiput(1124.17,542.50)(1.000,-7.500){2}{\rule{0.400pt}{1.807pt}}
\put(1125.67,535){\rule{0.400pt}{1.686pt}}
\multiput(1125.17,535.00)(1.000,3.500){2}{\rule{0.400pt}{0.843pt}}
\put(1126.67,528){\rule{0.400pt}{3.373pt}}
\multiput(1126.17,535.00)(1.000,-7.000){2}{\rule{0.400pt}{1.686pt}}
\put(1127.67,528){\rule{0.400pt}{0.723pt}}
\multiput(1127.17,528.00)(1.000,1.500){2}{\rule{0.400pt}{0.361pt}}
\put(1128.67,517){\rule{0.400pt}{3.373pt}}
\multiput(1128.17,524.00)(1.000,-7.000){2}{\rule{0.400pt}{1.686pt}}
\put(1129.67,510){\rule{0.400pt}{1.686pt}}
\multiput(1129.17,513.50)(1.000,-3.500){2}{\rule{0.400pt}{0.843pt}}
\put(1130.67,510){\rule{0.400pt}{3.373pt}}
\multiput(1130.17,510.00)(1.000,7.000){2}{\rule{0.400pt}{1.686pt}}
\put(1131.67,524){\rule{0.400pt}{0.964pt}}
\multiput(1131.17,524.00)(1.000,2.000){2}{\rule{0.400pt}{0.482pt}}
\put(1132.67,517){\rule{0.400pt}{2.650pt}}
\multiput(1132.17,522.50)(1.000,-5.500){2}{\rule{0.400pt}{1.325pt}}
\put(1123.0,550.0){\rule[-0.200pt]{0.482pt}{0.400pt}}
\put(1134.67,517){\rule{0.400pt}{3.373pt}}
\multiput(1134.17,517.00)(1.000,7.000){2}{\rule{0.400pt}{1.686pt}}
\put(1135.67,524){\rule{0.400pt}{1.686pt}}
\multiput(1135.17,527.50)(1.000,-3.500){2}{\rule{0.400pt}{0.843pt}}
\put(1136.67,524){\rule{0.400pt}{2.650pt}}
\multiput(1136.17,524.00)(1.000,5.500){2}{\rule{0.400pt}{1.325pt}}
\put(1134.0,517.0){\usebox{\plotpoint}}
\put(1139.67,524){\rule{0.400pt}{2.650pt}}
\multiput(1139.17,529.50)(1.000,-5.500){2}{\rule{0.400pt}{1.325pt}}
\put(1138.0,535.0){\rule[-0.200pt]{0.482pt}{0.400pt}}
\put(1141.67,524){\rule{0.400pt}{4.336pt}}
\multiput(1141.17,524.00)(1.000,9.000){2}{\rule{0.400pt}{2.168pt}}
\put(1141.0,524.0){\usebox{\plotpoint}}
\put(1143.67,531){\rule{0.400pt}{2.650pt}}
\multiput(1143.17,536.50)(1.000,-5.500){2}{\rule{0.400pt}{1.325pt}}
\put(1144.67,531){\rule{0.400pt}{1.927pt}}
\multiput(1144.17,531.00)(1.000,4.000){2}{\rule{0.400pt}{0.964pt}}
\put(1145.67,524){\rule{0.400pt}{3.614pt}}
\multiput(1145.17,531.50)(1.000,-7.500){2}{\rule{0.400pt}{1.807pt}}
\put(1146.67,524){\rule{0.400pt}{0.964pt}}
\multiput(1146.17,524.00)(1.000,2.000){2}{\rule{0.400pt}{0.482pt}}
\put(1147.67,528){\rule{0.400pt}{2.650pt}}
\multiput(1147.17,528.00)(1.000,5.500){2}{\rule{0.400pt}{1.325pt}}
\put(1148.67,528){\rule{0.400pt}{2.650pt}}
\multiput(1148.17,533.50)(1.000,-5.500){2}{\rule{0.400pt}{1.325pt}}
\put(1149.67,528){\rule{0.400pt}{3.373pt}}
\multiput(1149.17,528.00)(1.000,7.000){2}{\rule{0.400pt}{1.686pt}}
\put(1150.67,510){\rule{0.400pt}{7.709pt}}
\multiput(1150.17,526.00)(1.000,-16.000){2}{\rule{0.400pt}{3.854pt}}
\put(1151.67,510){\rule{0.400pt}{10.359pt}}
\multiput(1151.17,510.00)(1.000,21.500){2}{\rule{0.400pt}{5.179pt}}
\put(1153.17,550){\rule{0.400pt}{0.700pt}}
\multiput(1152.17,551.55)(2.000,-1.547){2}{\rule{0.400pt}{0.350pt}}
\put(1154.67,550){\rule{0.400pt}{20.958pt}}
\multiput(1154.17,550.00)(1.000,43.500){2}{\rule{0.400pt}{10.479pt}}
\put(1155.67,633){\rule{0.400pt}{0.964pt}}
\multiput(1155.17,635.00)(1.000,-2.000){2}{\rule{0.400pt}{0.482pt}}
\put(1156.67,633){\rule{0.400pt}{36.135pt}}
\multiput(1156.17,633.00)(1.000,75.000){2}{\rule{0.400pt}{18.067pt}}
\put(1143.0,542.0){\usebox{\plotpoint}}
\put(1158.67,753){\rule{0.400pt}{7.227pt}}
\multiput(1158.17,768.00)(1.000,-15.000){2}{\rule{0.400pt}{3.613pt}}
\put(1159.67,753){\rule{0.400pt}{4.577pt}}
\multiput(1159.17,753.00)(1.000,9.500){2}{\rule{0.400pt}{2.289pt}}
\put(1160.67,746){\rule{0.400pt}{6.263pt}}
\multiput(1160.17,759.00)(1.000,-13.000){2}{\rule{0.400pt}{3.132pt}}
\put(1158.0,783.0){\usebox{\plotpoint}}
\put(1162.67,746){\rule{0.400pt}{7.950pt}}
\multiput(1162.17,746.00)(1.000,16.500){2}{\rule{0.400pt}{3.975pt}}
\put(1163.67,764){\rule{0.400pt}{3.614pt}}
\multiput(1163.17,771.50)(1.000,-7.500){2}{\rule{0.400pt}{1.807pt}}
\put(1164.67,764){\rule{0.400pt}{4.577pt}}
\multiput(1164.17,764.00)(1.000,9.500){2}{\rule{0.400pt}{2.289pt}}
\put(1165.67,779){\rule{0.400pt}{0.964pt}}
\multiput(1165.17,781.00)(1.000,-2.000){2}{\rule{0.400pt}{0.482pt}}
\put(1167.17,779){\rule{0.400pt}{2.900pt}}
\multiput(1166.17,779.00)(2.000,7.981){2}{\rule{0.400pt}{1.450pt}}
\put(1168.67,786){\rule{0.400pt}{1.686pt}}
\multiput(1168.17,789.50)(1.000,-3.500){2}{\rule{0.400pt}{0.843pt}}
\put(1169.67,786){\rule{0.400pt}{2.650pt}}
\multiput(1169.17,786.00)(1.000,5.500){2}{\rule{0.400pt}{1.325pt}}
\put(1170.67,793){\rule{0.400pt}{0.964pt}}
\multiput(1170.17,795.00)(1.000,-2.000){2}{\rule{0.400pt}{0.482pt}}
\put(1171.67,786){\rule{0.400pt}{1.686pt}}
\multiput(1171.17,789.50)(1.000,-3.500){2}{\rule{0.400pt}{0.843pt}}
\put(1172.67,786){\rule{0.400pt}{1.686pt}}
\multiput(1172.17,786.00)(1.000,3.500){2}{\rule{0.400pt}{0.843pt}}
\put(1173.67,783){\rule{0.400pt}{2.409pt}}
\multiput(1173.17,788.00)(1.000,-5.000){2}{\rule{0.400pt}{1.204pt}}
\put(1174.67,783){\rule{0.400pt}{1.686pt}}
\multiput(1174.17,783.00)(1.000,3.500){2}{\rule{0.400pt}{0.843pt}}
\put(1175.67,772){\rule{0.400pt}{4.336pt}}
\multiput(1175.17,781.00)(1.000,-9.000){2}{\rule{0.400pt}{2.168pt}}
\put(1176.67,772){\rule{0.400pt}{0.723pt}}
\multiput(1176.17,772.00)(1.000,1.500){2}{\rule{0.400pt}{0.361pt}}
\put(1177.67,757){\rule{0.400pt}{4.336pt}}
\multiput(1177.17,766.00)(1.000,-9.000){2}{\rule{0.400pt}{2.168pt}}
\put(1178.67,757){\rule{0.400pt}{0.964pt}}
\multiput(1178.17,757.00)(1.000,2.000){2}{\rule{0.400pt}{0.482pt}}
\put(1179.67,743){\rule{0.400pt}{4.336pt}}
\multiput(1179.17,752.00)(1.000,-9.000){2}{\rule{0.400pt}{2.168pt}}
\put(1180.67,743){\rule{0.400pt}{0.723pt}}
\multiput(1180.17,743.00)(1.000,1.500){2}{\rule{0.400pt}{0.361pt}}
\put(1182.17,728){\rule{0.400pt}{3.700pt}}
\multiput(1181.17,738.32)(2.000,-10.320){2}{\rule{0.400pt}{1.850pt}}
\put(1183.67,728){\rule{0.400pt}{0.964pt}}
\multiput(1183.17,728.00)(1.000,2.000){2}{\rule{0.400pt}{0.482pt}}
\put(1184.67,710){\rule{0.400pt}{5.300pt}}
\multiput(1184.17,721.00)(1.000,-11.000){2}{\rule{0.400pt}{2.650pt}}
\put(1185.67,710){\rule{0.400pt}{1.686pt}}
\multiput(1185.17,710.00)(1.000,3.500){2}{\rule{0.400pt}{0.843pt}}
\put(1186.67,695){\rule{0.400pt}{5.300pt}}
\multiput(1186.17,706.00)(1.000,-11.000){2}{\rule{0.400pt}{2.650pt}}
\put(1187.67,695){\rule{0.400pt}{0.964pt}}
\multiput(1187.17,695.00)(1.000,2.000){2}{\rule{0.400pt}{0.482pt}}
\put(1188.67,684){\rule{0.400pt}{3.614pt}}
\multiput(1188.17,691.50)(1.000,-7.500){2}{\rule{0.400pt}{1.807pt}}
\put(1189.67,684){\rule{0.400pt}{0.964pt}}
\multiput(1189.17,684.00)(1.000,2.000){2}{\rule{0.400pt}{0.482pt}}
\put(1190.67,666){\rule{0.400pt}{5.300pt}}
\multiput(1190.17,677.00)(1.000,-11.000){2}{\rule{0.400pt}{2.650pt}}
\put(1191.67,666){\rule{0.400pt}{1.686pt}}
\multiput(1191.17,666.00)(1.000,3.500){2}{\rule{0.400pt}{0.843pt}}
\put(1192.67,652){\rule{0.400pt}{5.059pt}}
\multiput(1192.17,662.50)(1.000,-10.500){2}{\rule{0.400pt}{2.529pt}}
\put(1193.67,652){\rule{0.400pt}{0.723pt}}
\multiput(1193.17,652.00)(1.000,1.500){2}{\rule{0.400pt}{0.361pt}}
\put(1194.67,637){\rule{0.400pt}{4.336pt}}
\multiput(1194.17,646.00)(1.000,-9.000){2}{\rule{0.400pt}{2.168pt}}
\put(1195.67,637){\rule{0.400pt}{0.964pt}}
\multiput(1195.17,637.00)(1.000,2.000){2}{\rule{0.400pt}{0.482pt}}
\put(1197.17,626){\rule{0.400pt}{3.100pt}}
\multiput(1196.17,634.57)(2.000,-8.566){2}{\rule{0.400pt}{1.550pt}}
\put(1198.67,626){\rule{0.400pt}{0.964pt}}
\multiput(1198.17,626.00)(1.000,2.000){2}{\rule{0.400pt}{0.482pt}}
\put(1199.67,611){\rule{0.400pt}{4.577pt}}
\multiput(1199.17,620.50)(1.000,-9.500){2}{\rule{0.400pt}{2.289pt}}
\put(1200.67,611){\rule{0.400pt}{0.964pt}}
\multiput(1200.17,611.00)(1.000,2.000){2}{\rule{0.400pt}{0.482pt}}
\put(1201.67,597){\rule{0.400pt}{4.336pt}}
\multiput(1201.17,606.00)(1.000,-9.000){2}{\rule{0.400pt}{2.168pt}}
\put(1202.67,597){\rule{0.400pt}{0.964pt}}
\multiput(1202.17,597.00)(1.000,2.000){2}{\rule{0.400pt}{0.482pt}}
\put(1203.67,586){\rule{0.400pt}{3.614pt}}
\multiput(1203.17,593.50)(1.000,-7.500){2}{\rule{0.400pt}{1.807pt}}
\put(1204.67,586){\rule{0.400pt}{1.686pt}}
\multiput(1204.17,586.00)(1.000,3.500){2}{\rule{0.400pt}{0.843pt}}
\put(1205.67,575){\rule{0.400pt}{4.336pt}}
\multiput(1205.17,584.00)(1.000,-9.000){2}{\rule{0.400pt}{2.168pt}}
\put(1206.67,575){\rule{0.400pt}{0.964pt}}
\multiput(1206.17,575.00)(1.000,2.000){2}{\rule{0.400pt}{0.482pt}}
\put(1207.67,564){\rule{0.400pt}{3.614pt}}
\multiput(1207.17,571.50)(1.000,-7.500){2}{\rule{0.400pt}{1.807pt}}
\put(1208.67,564){\rule{0.400pt}{0.964pt}}
\multiput(1208.17,564.00)(1.000,2.000){2}{\rule{0.400pt}{0.482pt}}
\put(1209.67,553){\rule{0.400pt}{3.614pt}}
\multiput(1209.17,560.50)(1.000,-7.500){2}{\rule{0.400pt}{1.807pt}}
\put(1211.17,553){\rule{0.400pt}{0.900pt}}
\multiput(1210.17,553.00)(2.000,2.132){2}{\rule{0.400pt}{0.450pt}}
\put(1212.67,542){\rule{0.400pt}{3.614pt}}
\multiput(1212.17,549.50)(1.000,-7.500){2}{\rule{0.400pt}{1.807pt}}
\put(1213.67,542){\rule{0.400pt}{1.927pt}}
\multiput(1213.17,542.00)(1.000,4.000){2}{\rule{0.400pt}{0.964pt}}
\put(1214.67,531){\rule{0.400pt}{4.577pt}}
\multiput(1214.17,540.50)(1.000,-9.500){2}{\rule{0.400pt}{2.289pt}}
\put(1215.67,531){\rule{0.400pt}{0.964pt}}
\multiput(1215.17,531.00)(1.000,2.000){2}{\rule{0.400pt}{0.482pt}}
\put(1216.67,524){\rule{0.400pt}{2.650pt}}
\multiput(1216.17,529.50)(1.000,-5.500){2}{\rule{0.400pt}{1.325pt}}
\put(1217.67,524){\rule{0.400pt}{0.964pt}}
\multiput(1217.17,524.00)(1.000,2.000){2}{\rule{0.400pt}{0.482pt}}
\put(1218.67,513){\rule{0.400pt}{3.614pt}}
\multiput(1218.17,520.50)(1.000,-7.500){2}{\rule{0.400pt}{1.807pt}}
\put(1219.67,513){\rule{0.400pt}{0.964pt}}
\multiput(1219.17,513.00)(1.000,2.000){2}{\rule{0.400pt}{0.482pt}}
\put(1220.67,506){\rule{0.400pt}{2.650pt}}
\multiput(1220.17,511.50)(1.000,-5.500){2}{\rule{0.400pt}{1.325pt}}
\put(1221.67,506){\rule{0.400pt}{0.964pt}}
\multiput(1221.17,506.00)(1.000,2.000){2}{\rule{0.400pt}{0.482pt}}
\put(1222.67,495){\rule{0.400pt}{3.614pt}}
\multiput(1222.17,502.50)(1.000,-7.500){2}{\rule{0.400pt}{1.807pt}}
\put(1223.67,495){\rule{0.400pt}{0.964pt}}
\multiput(1223.17,495.00)(1.000,2.000){2}{\rule{0.400pt}{0.482pt}}
\put(1224.67,488){\rule{0.400pt}{2.650pt}}
\multiput(1224.17,493.50)(1.000,-5.500){2}{\rule{0.400pt}{1.325pt}}
\put(1226.17,488){\rule{0.400pt}{1.500pt}}
\multiput(1225.17,488.00)(2.000,3.887){2}{\rule{0.400pt}{0.750pt}}
\put(1227.67,477){\rule{0.400pt}{4.336pt}}
\multiput(1227.17,486.00)(1.000,-9.000){2}{\rule{0.400pt}{2.168pt}}
\put(1228.67,477){\rule{0.400pt}{1.686pt}}
\multiput(1228.17,477.00)(1.000,3.500){2}{\rule{0.400pt}{0.843pt}}
\put(1229.67,473){\rule{0.400pt}{2.650pt}}
\multiput(1229.17,478.50)(1.000,-5.500){2}{\rule{0.400pt}{1.325pt}}
\put(1230.67,473){\rule{0.400pt}{0.964pt}}
\multiput(1230.17,473.00)(1.000,2.000){2}{\rule{0.400pt}{0.482pt}}
\put(1231.67,466){\rule{0.400pt}{2.650pt}}
\multiput(1231.17,471.50)(1.000,-5.500){2}{\rule{0.400pt}{1.325pt}}
\put(1232.67,466){\rule{0.400pt}{0.964pt}}
\multiput(1232.17,466.00)(1.000,2.000){2}{\rule{0.400pt}{0.482pt}}
\put(1233.67,455){\rule{0.400pt}{3.614pt}}
\multiput(1233.17,462.50)(1.000,-7.500){2}{\rule{0.400pt}{1.807pt}}
\put(1234.67,455){\rule{0.400pt}{1.686pt}}
\multiput(1234.17,455.00)(1.000,3.500){2}{\rule{0.400pt}{0.843pt}}
\put(1235.67,451){\rule{0.400pt}{2.650pt}}
\multiput(1235.17,456.50)(1.000,-5.500){2}{\rule{0.400pt}{1.325pt}}
\put(1236.67,451){\rule{0.400pt}{0.964pt}}
\multiput(1236.17,451.00)(1.000,2.000){2}{\rule{0.400pt}{0.482pt}}
\put(1237.67,448){\rule{0.400pt}{1.686pt}}
\multiput(1237.17,451.50)(1.000,-3.500){2}{\rule{0.400pt}{0.843pt}}
\put(1238.67,448){\rule{0.400pt}{0.723pt}}
\multiput(1238.17,448.00)(1.000,1.500){2}{\rule{0.400pt}{0.361pt}}
\put(1240.17,437){\rule{0.400pt}{2.900pt}}
\multiput(1239.17,444.98)(2.000,-7.981){2}{\rule{0.400pt}{1.450pt}}
\put(1241.67,437){\rule{0.400pt}{1.686pt}}
\multiput(1241.17,437.00)(1.000,3.500){2}{\rule{0.400pt}{0.843pt}}
\put(1242.67,433){\rule{0.400pt}{2.650pt}}
\multiput(1242.17,438.50)(1.000,-5.500){2}{\rule{0.400pt}{1.325pt}}
\put(1243.67,433){\rule{0.400pt}{0.964pt}}
\multiput(1243.17,433.00)(1.000,2.000){2}{\rule{0.400pt}{0.482pt}}
\put(1244.67,429){\rule{0.400pt}{1.927pt}}
\multiput(1244.17,433.00)(1.000,-4.000){2}{\rule{0.400pt}{0.964pt}}
\put(1245.67,429){\rule{0.400pt}{0.964pt}}
\multiput(1245.17,429.00)(1.000,2.000){2}{\rule{0.400pt}{0.482pt}}
\put(1246.67,426){\rule{0.400pt}{1.686pt}}
\multiput(1246.17,429.50)(1.000,-3.500){2}{\rule{0.400pt}{0.843pt}}
\put(1247.67,426){\rule{0.400pt}{0.723pt}}
\multiput(1247.17,426.00)(1.000,1.500){2}{\rule{0.400pt}{0.361pt}}
\put(1248.67,419){\rule{0.400pt}{2.409pt}}
\multiput(1248.17,424.00)(1.000,-5.000){2}{\rule{0.400pt}{1.204pt}}
\put(1249.67,419){\rule{0.400pt}{0.723pt}}
\multiput(1249.17,419.00)(1.000,1.500){2}{\rule{0.400pt}{0.361pt}}
\put(1250.67,415){\rule{0.400pt}{1.686pt}}
\multiput(1250.17,418.50)(1.000,-3.500){2}{\rule{0.400pt}{0.843pt}}
\put(1251.67,411){\rule{0.400pt}{0.964pt}}
\multiput(1251.17,413.00)(1.000,-2.000){2}{\rule{0.400pt}{0.482pt}}
\put(1252.67,411){\rule{0.400pt}{1.927pt}}
\multiput(1252.17,411.00)(1.000,4.000){2}{\rule{0.400pt}{0.964pt}}
\put(1253.67,415){\rule{0.400pt}{0.964pt}}
\multiput(1253.17,417.00)(1.000,-2.000){2}{\rule{0.400pt}{0.482pt}}
\put(1255.17,408){\rule{0.400pt}{1.500pt}}
\multiput(1254.17,411.89)(2.000,-3.887){2}{\rule{0.400pt}{0.750pt}}
\put(1256.67,408){\rule{0.400pt}{0.723pt}}
\multiput(1256.17,408.00)(1.000,1.500){2}{\rule{0.400pt}{0.361pt}}
\put(1257.67,404){\rule{0.400pt}{1.686pt}}
\multiput(1257.17,407.50)(1.000,-3.500){2}{\rule{0.400pt}{0.843pt}}
\put(1258.67,404){\rule{0.400pt}{0.964pt}}
\multiput(1258.17,404.00)(1.000,2.000){2}{\rule{0.400pt}{0.482pt}}
\put(1259.67,397){\rule{0.400pt}{2.650pt}}
\multiput(1259.17,402.50)(1.000,-5.500){2}{\rule{0.400pt}{1.325pt}}
\put(1260.67,397){\rule{0.400pt}{1.686pt}}
\multiput(1260.17,397.00)(1.000,3.500){2}{\rule{0.400pt}{0.843pt}}
\put(1261.67,393){\rule{0.400pt}{2.650pt}}
\multiput(1261.17,398.50)(1.000,-5.500){2}{\rule{0.400pt}{1.325pt}}
\put(1262.67,393){\rule{0.400pt}{1.686pt}}
\multiput(1262.17,393.00)(1.000,3.500){2}{\rule{0.400pt}{0.843pt}}
\put(1263.67,389){\rule{0.400pt}{2.650pt}}
\multiput(1263.17,394.50)(1.000,-5.500){2}{\rule{0.400pt}{1.325pt}}
\put(1162.0,746.0){\usebox{\plotpoint}}
\put(1265.67,389){\rule{0.400pt}{1.927pt}}
\multiput(1265.17,389.00)(1.000,4.000){2}{\rule{0.400pt}{0.964pt}}
\put(1266.67,393){\rule{0.400pt}{0.964pt}}
\multiput(1266.17,395.00)(1.000,-2.000){2}{\rule{0.400pt}{0.482pt}}
\put(1267.67,386){\rule{0.400pt}{1.686pt}}
\multiput(1267.17,389.50)(1.000,-3.500){2}{\rule{0.400pt}{0.843pt}}
\put(1268.67,386){\rule{0.400pt}{0.723pt}}
\multiput(1268.17,386.00)(1.000,1.500){2}{\rule{0.400pt}{0.361pt}}
\put(1270.17,382){\rule{0.400pt}{1.500pt}}
\multiput(1269.17,385.89)(2.000,-3.887){2}{\rule{0.400pt}{0.750pt}}
\put(1271.67,379){\rule{0.400pt}{0.723pt}}
\multiput(1271.17,380.50)(1.000,-1.500){2}{\rule{0.400pt}{0.361pt}}
\put(1272.67,379){\rule{0.400pt}{2.409pt}}
\multiput(1272.17,379.00)(1.000,5.000){2}{\rule{0.400pt}{1.204pt}}
\put(1273.67,382){\rule{0.400pt}{1.686pt}}
\multiput(1273.17,385.50)(1.000,-3.500){2}{\rule{0.400pt}{0.843pt}}
\put(1274.67,375){\rule{0.400pt}{1.686pt}}
\multiput(1274.17,378.50)(1.000,-3.500){2}{\rule{0.400pt}{0.843pt}}
\put(1265.0,389.0){\usebox{\plotpoint}}
\put(1276.67,375){\rule{0.400pt}{1.686pt}}
\multiput(1276.17,375.00)(1.000,3.500){2}{\rule{0.400pt}{0.843pt}}
\put(1277.67,379){\rule{0.400pt}{0.723pt}}
\multiput(1277.17,380.50)(1.000,-1.500){2}{\rule{0.400pt}{0.361pt}}
\put(1278.67,371){\rule{0.400pt}{1.927pt}}
\multiput(1278.17,375.00)(1.000,-4.000){2}{\rule{0.400pt}{0.964pt}}
\put(1276.0,375.0){\usebox{\plotpoint}}
\put(1280.67,371){\rule{0.400pt}{1.927pt}}
\multiput(1280.17,371.00)(1.000,4.000){2}{\rule{0.400pt}{0.964pt}}
\put(1281.67,375){\rule{0.400pt}{0.964pt}}
\multiput(1281.17,377.00)(1.000,-2.000){2}{\rule{0.400pt}{0.482pt}}
\put(1282.67,368){\rule{0.400pt}{1.686pt}}
\multiput(1282.17,371.50)(1.000,-3.500){2}{\rule{0.400pt}{0.843pt}}
\put(1280.0,371.0){\usebox{\plotpoint}}
\put(1285.67,368){\rule{0.400pt}{1.686pt}}
\multiput(1285.17,368.00)(1.000,3.500){2}{\rule{0.400pt}{0.843pt}}
\put(1286.67,368){\rule{0.400pt}{1.686pt}}
\multiput(1286.17,371.50)(1.000,-3.500){2}{\rule{0.400pt}{0.843pt}}
\put(1287.67,368){\rule{0.400pt}{0.723pt}}
\multiput(1287.17,368.00)(1.000,1.500){2}{\rule{0.400pt}{0.361pt}}
\put(1284.0,368.0){\rule[-0.200pt]{0.482pt}{0.400pt}}
\put(1289.67,364){\rule{0.400pt}{1.686pt}}
\multiput(1289.17,367.50)(1.000,-3.500){2}{\rule{0.400pt}{0.843pt}}
\put(1289.0,371.0){\usebox{\plotpoint}}
\put(1291.67,364){\rule{0.400pt}{1.686pt}}
\multiput(1291.17,364.00)(1.000,3.500){2}{\rule{0.400pt}{0.843pt}}
\put(1292.67,368){\rule{0.400pt}{0.723pt}}
\multiput(1292.17,369.50)(1.000,-1.500){2}{\rule{0.400pt}{0.361pt}}
\put(1293.67,360){\rule{0.400pt}{1.927pt}}
\multiput(1293.17,364.00)(1.000,-4.000){2}{\rule{0.400pt}{0.964pt}}
\put(1294.67,357){\rule{0.400pt}{0.723pt}}
\multiput(1294.17,358.50)(1.000,-1.500){2}{\rule{0.400pt}{0.361pt}}
\put(1295.67,357){\rule{0.400pt}{2.650pt}}
\multiput(1295.17,357.00)(1.000,5.500){2}{\rule{0.400pt}{1.325pt}}
\put(1296.67,357){\rule{0.400pt}{2.650pt}}
\multiput(1296.17,362.50)(1.000,-5.500){2}{\rule{0.400pt}{1.325pt}}
\put(1297.67,357){\rule{0.400pt}{2.650pt}}
\multiput(1297.17,357.00)(1.000,5.500){2}{\rule{0.400pt}{1.325pt}}
\put(1299.17,360){\rule{0.400pt}{1.700pt}}
\multiput(1298.17,364.47)(2.000,-4.472){2}{\rule{0.400pt}{0.850pt}}
\put(1300.67,357){\rule{0.400pt}{0.723pt}}
\multiput(1300.17,358.50)(1.000,-1.500){2}{\rule{0.400pt}{0.361pt}}
\put(1301.67,357){\rule{0.400pt}{0.723pt}}
\multiput(1301.17,357.00)(1.000,1.500){2}{\rule{0.400pt}{0.361pt}}
\put(1302.67,353){\rule{0.400pt}{1.686pt}}
\multiput(1302.17,356.50)(1.000,-3.500){2}{\rule{0.400pt}{0.843pt}}
\put(1303.67,353){\rule{0.400pt}{1.686pt}}
\multiput(1303.17,353.00)(1.000,3.500){2}{\rule{0.400pt}{0.843pt}}
\put(1304.67,353){\rule{0.400pt}{1.686pt}}
\multiput(1304.17,356.50)(1.000,-3.500){2}{\rule{0.400pt}{0.843pt}}
\put(1291.0,364.0){\usebox{\plotpoint}}
\put(1306.67,353){\rule{0.400pt}{1.686pt}}
\multiput(1306.17,353.00)(1.000,3.500){2}{\rule{0.400pt}{0.843pt}}
\put(1307.67,357){\rule{0.400pt}{0.723pt}}
\multiput(1307.17,358.50)(1.000,-1.500){2}{\rule{0.400pt}{0.361pt}}
\put(1308.67,353){\rule{0.400pt}{0.964pt}}
\multiput(1308.17,355.00)(1.000,-2.000){2}{\rule{0.400pt}{0.482pt}}
\put(1309.67,353){\rule{0.400pt}{0.964pt}}
\multiput(1309.17,353.00)(1.000,2.000){2}{\rule{0.400pt}{0.482pt}}
\put(1310.67,349){\rule{0.400pt}{1.927pt}}
\multiput(1310.17,353.00)(1.000,-4.000){2}{\rule{0.400pt}{0.964pt}}
\put(1306.0,353.0){\usebox{\plotpoint}}
\put(1312.67,349){\rule{0.400pt}{1.927pt}}
\multiput(1312.17,349.00)(1.000,4.000){2}{\rule{0.400pt}{0.964pt}}
\put(1312.0,349.0){\usebox{\plotpoint}}
\put(1315.67,349){\rule{0.400pt}{1.927pt}}
\multiput(1315.17,353.00)(1.000,-4.000){2}{\rule{0.400pt}{0.964pt}}
\put(1314.0,357.0){\rule[-0.200pt]{0.482pt}{0.400pt}}
\put(1317.67,349){\rule{0.400pt}{0.964pt}}
\multiput(1317.17,349.00)(1.000,2.000){2}{\rule{0.400pt}{0.482pt}}
\put(1318.67,349){\rule{0.400pt}{0.964pt}}
\multiput(1318.17,351.00)(1.000,-2.000){2}{\rule{0.400pt}{0.482pt}}
\put(1319.67,349){\rule{0.400pt}{0.964pt}}
\multiput(1319.17,349.00)(1.000,2.000){2}{\rule{0.400pt}{0.482pt}}
\put(1317.0,349.0){\usebox{\plotpoint}}
\put(1321.67,346){\rule{0.400pt}{1.686pt}}
\multiput(1321.17,349.50)(1.000,-3.500){2}{\rule{0.400pt}{0.843pt}}
\put(1321.0,353.0){\usebox{\plotpoint}}
\put(1323.67,346){\rule{0.400pt}{1.686pt}}
\multiput(1323.17,346.00)(1.000,3.500){2}{\rule{0.400pt}{0.843pt}}
\put(1324.67,346){\rule{0.400pt}{1.686pt}}
\multiput(1324.17,349.50)(1.000,-3.500){2}{\rule{0.400pt}{0.843pt}}
\put(1325.67,346){\rule{0.400pt}{1.686pt}}
\multiput(1325.17,346.00)(1.000,3.500){2}{\rule{0.400pt}{0.843pt}}
\put(1323.0,346.0){\usebox{\plotpoint}}
\put(1328.17,346){\rule{0.400pt}{1.500pt}}
\multiput(1327.17,349.89)(2.000,-3.887){2}{\rule{0.400pt}{0.750pt}}
\put(1327.0,353.0){\usebox{\plotpoint}}
\put(1330.67,346){\rule{0.400pt}{0.723pt}}
\multiput(1330.17,346.00)(1.000,1.500){2}{\rule{0.400pt}{0.361pt}}
\put(1331.67,346){\rule{0.400pt}{0.723pt}}
\multiput(1331.17,347.50)(1.000,-1.500){2}{\rule{0.400pt}{0.361pt}}
\put(1332.67,346){\rule{0.400pt}{0.723pt}}
\multiput(1332.17,346.00)(1.000,1.500){2}{\rule{0.400pt}{0.361pt}}
\put(1330.0,346.0){\usebox{\plotpoint}}
\put(1334.67,338){\rule{0.400pt}{2.650pt}}
\multiput(1334.17,343.50)(1.000,-5.500){2}{\rule{0.400pt}{1.325pt}}
\put(1335.67,338){\rule{0.400pt}{2.650pt}}
\multiput(1335.17,338.00)(1.000,5.500){2}{\rule{0.400pt}{1.325pt}}
\put(1336.67,338){\rule{0.400pt}{2.650pt}}
\multiput(1336.17,343.50)(1.000,-5.500){2}{\rule{0.400pt}{1.325pt}}
\put(1337.67,338){\rule{0.400pt}{2.650pt}}
\multiput(1337.17,338.00)(1.000,5.500){2}{\rule{0.400pt}{1.325pt}}
\put(1338.67,338){\rule{0.400pt}{2.650pt}}
\multiput(1338.17,343.50)(1.000,-5.500){2}{\rule{0.400pt}{1.325pt}}
\put(1339.67,338){\rule{0.400pt}{2.650pt}}
\multiput(1339.17,338.00)(1.000,5.500){2}{\rule{0.400pt}{1.325pt}}
\put(1340.67,338){\rule{0.400pt}{2.650pt}}
\multiput(1340.17,343.50)(1.000,-5.500){2}{\rule{0.400pt}{1.325pt}}
\put(1334.0,349.0){\usebox{\plotpoint}}
\put(1343.17,338){\rule{0.400pt}{2.300pt}}
\multiput(1342.17,338.00)(2.000,6.226){2}{\rule{0.400pt}{1.150pt}}
\put(1342.0,338.0){\usebox{\plotpoint}}
\put(1345.67,338){\rule{0.400pt}{2.650pt}}
\multiput(1345.17,343.50)(1.000,-5.500){2}{\rule{0.400pt}{1.325pt}}
\put(1345.0,349.0){\usebox{\plotpoint}}
\put(1347.67,338){\rule{0.400pt}{0.964pt}}
\multiput(1347.17,338.00)(1.000,2.000){2}{\rule{0.400pt}{0.482pt}}
\put(1347.0,338.0){\usebox{\plotpoint}}
\put(1349.67,338){\rule{0.400pt}{0.964pt}}
\multiput(1349.17,340.00)(1.000,-2.000){2}{\rule{0.400pt}{0.482pt}}
\put(1350.67,338){\rule{0.400pt}{0.964pt}}
\multiput(1350.17,338.00)(1.000,2.000){2}{\rule{0.400pt}{0.482pt}}
\put(1351.67,338){\rule{0.400pt}{0.964pt}}
\multiput(1351.17,340.00)(1.000,-2.000){2}{\rule{0.400pt}{0.482pt}}
\put(1349.0,342.0){\usebox{\plotpoint}}
\put(1353.67,338){\rule{0.400pt}{0.964pt}}
\multiput(1353.17,338.00)(1.000,2.000){2}{\rule{0.400pt}{0.482pt}}
\put(1354.67,335){\rule{0.400pt}{1.686pt}}
\multiput(1354.17,338.50)(1.000,-3.500){2}{\rule{0.400pt}{0.843pt}}
\put(1355.67,335){\rule{0.400pt}{1.686pt}}
\multiput(1355.17,335.00)(1.000,3.500){2}{\rule{0.400pt}{0.843pt}}
\put(1353.0,338.0){\usebox{\plotpoint}}
\put(1358.67,335){\rule{0.400pt}{1.686pt}}
\multiput(1358.17,338.50)(1.000,-3.500){2}{\rule{0.400pt}{0.843pt}}
\put(1357.0,342.0){\rule[-0.200pt]{0.482pt}{0.400pt}}
\put(1360.67,335){\rule{0.400pt}{1.686pt}}
\multiput(1360.17,335.00)(1.000,3.500){2}{\rule{0.400pt}{0.843pt}}
\put(1360.0,335.0){\usebox{\plotpoint}}
\put(1362.67,335){\rule{0.400pt}{1.686pt}}
\multiput(1362.17,338.50)(1.000,-3.500){2}{\rule{0.400pt}{0.843pt}}
\put(1363.67,335){\rule{0.400pt}{1.686pt}}
\multiput(1363.17,335.00)(1.000,3.500){2}{\rule{0.400pt}{0.843pt}}
\put(1364.67,335){\rule{0.400pt}{1.686pt}}
\multiput(1364.17,338.50)(1.000,-3.500){2}{\rule{0.400pt}{0.843pt}}
\put(1362.0,342.0){\usebox{\plotpoint}}
\put(1366.67,335){\rule{0.400pt}{1.686pt}}
\multiput(1366.17,335.00)(1.000,3.500){2}{\rule{0.400pt}{0.843pt}}
\put(1367.67,338){\rule{0.400pt}{0.964pt}}
\multiput(1367.17,340.00)(1.000,-2.000){2}{\rule{0.400pt}{0.482pt}}
\put(1368.67,335){\rule{0.400pt}{0.723pt}}
\multiput(1368.17,336.50)(1.000,-1.500){2}{\rule{0.400pt}{0.361pt}}
\put(1369.67,335){\rule{0.400pt}{0.723pt}}
\multiput(1369.17,335.00)(1.000,1.500){2}{\rule{0.400pt}{0.361pt}}
\put(1370.67,335){\rule{0.400pt}{0.723pt}}
\multiput(1370.17,336.50)(1.000,-1.500){2}{\rule{0.400pt}{0.361pt}}
\put(1372.17,335){\rule{0.400pt}{0.700pt}}
\multiput(1371.17,335.00)(2.000,1.547){2}{\rule{0.400pt}{0.350pt}}
\put(1373.67,335){\rule{0.400pt}{0.723pt}}
\multiput(1373.17,336.50)(1.000,-1.500){2}{\rule{0.400pt}{0.361pt}}
\put(1374.67,335){\rule{0.400pt}{0.723pt}}
\multiput(1374.17,335.00)(1.000,1.500){2}{\rule{0.400pt}{0.361pt}}
\put(1375.67,335){\rule{0.400pt}{0.723pt}}
\multiput(1375.17,336.50)(1.000,-1.500){2}{\rule{0.400pt}{0.361pt}}
\put(1376.67,335){\rule{0.400pt}{0.723pt}}
\multiput(1376.17,335.00)(1.000,1.500){2}{\rule{0.400pt}{0.361pt}}
\put(1377.67,331){\rule{0.400pt}{1.686pt}}
\multiput(1377.17,334.50)(1.000,-3.500){2}{\rule{0.400pt}{0.843pt}}
\put(1378.67,331){\rule{0.400pt}{1.686pt}}
\multiput(1378.17,331.00)(1.000,3.500){2}{\rule{0.400pt}{0.843pt}}
\put(1379.67,335){\rule{0.400pt}{0.723pt}}
\multiput(1379.17,336.50)(1.000,-1.500){2}{\rule{0.400pt}{0.361pt}}
\put(1380.67,335){\rule{0.400pt}{0.723pt}}
\multiput(1380.17,335.00)(1.000,1.500){2}{\rule{0.400pt}{0.361pt}}
\put(1381.67,331){\rule{0.400pt}{1.686pt}}
\multiput(1381.17,334.50)(1.000,-3.500){2}{\rule{0.400pt}{0.843pt}}
\put(1382.67,331){\rule{0.400pt}{1.686pt}}
\multiput(1382.17,331.00)(1.000,3.500){2}{\rule{0.400pt}{0.843pt}}
\put(1383.67,331){\rule{0.400pt}{1.686pt}}
\multiput(1383.17,334.50)(1.000,-3.500){2}{\rule{0.400pt}{0.843pt}}
\put(1384.67,331){\rule{0.400pt}{1.686pt}}
\multiput(1384.17,331.00)(1.000,3.500){2}{\rule{0.400pt}{0.843pt}}
\put(1385.67,331){\rule{0.400pt}{1.686pt}}
\multiput(1385.17,334.50)(1.000,-3.500){2}{\rule{0.400pt}{0.843pt}}
\put(1366.0,335.0){\usebox{\plotpoint}}
\put(1388.67,331){\rule{0.400pt}{1.686pt}}
\multiput(1388.17,331.00)(1.000,3.500){2}{\rule{0.400pt}{0.843pt}}
\put(1387.0,331.0){\rule[-0.200pt]{0.482pt}{0.400pt}}
\put(1390.67,331){\rule{0.400pt}{1.686pt}}
\multiput(1390.17,334.50)(1.000,-3.500){2}{\rule{0.400pt}{0.843pt}}
\put(1390.0,338.0){\usebox{\plotpoint}}
\put(1392.67,331){\rule{0.400pt}{1.686pt}}
\multiput(1392.17,331.00)(1.000,3.500){2}{\rule{0.400pt}{0.843pt}}
\put(1392.0,331.0){\usebox{\plotpoint}}
\put(1394.67,331){\rule{0.400pt}{1.686pt}}
\multiput(1394.17,334.50)(1.000,-3.500){2}{\rule{0.400pt}{0.843pt}}
\put(1394.0,338.0){\usebox{\plotpoint}}
\put(1396.67,331){\rule{0.400pt}{1.686pt}}
\multiput(1396.17,331.00)(1.000,3.500){2}{\rule{0.400pt}{0.843pt}}
\put(1396.0,331.0){\usebox{\plotpoint}}
\put(1398.67,331){\rule{0.400pt}{1.686pt}}
\multiput(1398.17,334.50)(1.000,-3.500){2}{\rule{0.400pt}{0.843pt}}
\put(1399.67,331){\rule{0.400pt}{0.964pt}}
\multiput(1399.17,331.00)(1.000,2.000){2}{\rule{0.400pt}{0.482pt}}
\put(1401.17,331){\rule{0.400pt}{0.900pt}}
\multiput(1400.17,333.13)(2.000,-2.132){2}{\rule{0.400pt}{0.450pt}}
\put(1402.67,331){\rule{0.400pt}{0.964pt}}
\multiput(1402.17,331.00)(1.000,2.000){2}{\rule{0.400pt}{0.482pt}}
\put(1403.67,331){\rule{0.400pt}{0.964pt}}
\multiput(1403.17,333.00)(1.000,-2.000){2}{\rule{0.400pt}{0.482pt}}
\put(1404.67,331){\rule{0.400pt}{0.964pt}}
\multiput(1404.17,331.00)(1.000,2.000){2}{\rule{0.400pt}{0.482pt}}
\put(1405.67,331){\rule{0.400pt}{0.964pt}}
\multiput(1405.17,333.00)(1.000,-2.000){2}{\rule{0.400pt}{0.482pt}}
\put(1406.67,331){\rule{0.400pt}{0.964pt}}
\multiput(1406.17,331.00)(1.000,2.000){2}{\rule{0.400pt}{0.482pt}}
\put(1407.67,331){\rule{0.400pt}{0.964pt}}
\multiput(1407.17,333.00)(1.000,-2.000){2}{\rule{0.400pt}{0.482pt}}
\put(1408.67,331){\rule{0.400pt}{0.964pt}}
\multiput(1408.17,331.00)(1.000,2.000){2}{\rule{0.400pt}{0.482pt}}
\put(1409.67,331){\rule{0.400pt}{0.964pt}}
\multiput(1409.17,333.00)(1.000,-2.000){2}{\rule{0.400pt}{0.482pt}}
\put(1410.67,331){\rule{0.400pt}{0.964pt}}
\multiput(1410.17,331.00)(1.000,2.000){2}{\rule{0.400pt}{0.482pt}}
\put(1411.67,328){\rule{0.400pt}{1.686pt}}
\multiput(1411.17,331.50)(1.000,-3.500){2}{\rule{0.400pt}{0.843pt}}
\put(1412.67,328){\rule{0.400pt}{1.686pt}}
\multiput(1412.17,328.00)(1.000,3.500){2}{\rule{0.400pt}{0.843pt}}
\put(1413.67,328){\rule{0.400pt}{1.686pt}}
\multiput(1413.17,331.50)(1.000,-3.500){2}{\rule{0.400pt}{0.843pt}}
\put(1414.67,328){\rule{0.400pt}{1.686pt}}
\multiput(1414.17,328.00)(1.000,3.500){2}{\rule{0.400pt}{0.843pt}}
\put(1416.17,328){\rule{0.400pt}{1.500pt}}
\multiput(1415.17,331.89)(2.000,-3.887){2}{\rule{0.400pt}{0.750pt}}
\put(1417.67,328){\rule{0.400pt}{1.686pt}}
\multiput(1417.17,328.00)(1.000,3.500){2}{\rule{0.400pt}{0.843pt}}
\put(1418.67,328){\rule{0.400pt}{1.686pt}}
\multiput(1418.17,331.50)(1.000,-3.500){2}{\rule{0.400pt}{0.843pt}}
\put(1398.0,338.0){\usebox{\plotpoint}}
\put(1420.67,328){\rule{0.400pt}{1.686pt}}
\multiput(1420.17,328.00)(1.000,3.500){2}{\rule{0.400pt}{0.843pt}}
\put(1421.67,328){\rule{0.400pt}{1.686pt}}
\multiput(1421.17,331.50)(1.000,-3.500){2}{\rule{0.400pt}{0.843pt}}
\put(1422.67,328){\rule{0.400pt}{1.686pt}}
\multiput(1422.17,328.00)(1.000,3.500){2}{\rule{0.400pt}{0.843pt}}
\put(1423.67,328){\rule{0.400pt}{1.686pt}}
\multiput(1423.17,331.50)(1.000,-3.500){2}{\rule{0.400pt}{0.843pt}}
\put(1424.67,328){\rule{0.400pt}{1.686pt}}
\multiput(1424.17,328.00)(1.000,3.500){2}{\rule{0.400pt}{0.843pt}}
\put(1425.67,328){\rule{0.400pt}{1.686pt}}
\multiput(1425.17,331.50)(1.000,-3.500){2}{\rule{0.400pt}{0.843pt}}
\put(1426.67,328){\rule{0.400pt}{1.686pt}}
\multiput(1426.17,328.00)(1.000,3.500){2}{\rule{0.400pt}{0.843pt}}
\put(1427.67,328){\rule{0.400pt}{1.686pt}}
\multiput(1427.17,331.50)(1.000,-3.500){2}{\rule{0.400pt}{0.843pt}}
\put(1428.67,328){\rule{0.400pt}{1.686pt}}
\multiput(1428.17,328.00)(1.000,3.500){2}{\rule{0.400pt}{0.843pt}}
\put(1429.67,328){\rule{0.400pt}{1.686pt}}
\multiput(1429.17,331.50)(1.000,-3.500){2}{\rule{0.400pt}{0.843pt}}
\put(1431.17,328){\rule{0.400pt}{1.500pt}}
\multiput(1430.17,328.00)(2.000,3.887){2}{\rule{0.400pt}{0.750pt}}
\put(1432.67,328){\rule{0.400pt}{1.686pt}}
\multiput(1432.17,331.50)(1.000,-3.500){2}{\rule{0.400pt}{0.843pt}}
\put(1433.67,328){\rule{0.400pt}{1.686pt}}
\multiput(1433.17,328.00)(1.000,3.500){2}{\rule{0.400pt}{0.843pt}}
\put(1434.67,328){\rule{0.400pt}{1.686pt}}
\multiput(1434.17,331.50)(1.000,-3.500){2}{\rule{0.400pt}{0.843pt}}
\put(1435.67,328){\rule{0.400pt}{1.686pt}}
\multiput(1435.17,328.00)(1.000,3.500){2}{\rule{0.400pt}{0.843pt}}
\put(1436.67,331){\rule{0.400pt}{0.964pt}}
\multiput(1436.17,333.00)(1.000,-2.000){2}{\rule{0.400pt}{0.482pt}}
\put(1437.67,328){\rule{0.400pt}{0.723pt}}
\multiput(1437.17,329.50)(1.000,-1.500){2}{\rule{0.400pt}{0.361pt}}
\put(1420.0,328.0){\usebox{\plotpoint}}
\put(151.0,131.0){\rule[-0.200pt]{0.400pt}{175.375pt}}
\put(151.0,131.0){\rule[-0.200pt]{310.279pt}{0.400pt}}
\put(1439.0,131.0){\rule[-0.200pt]{0.400pt}{175.375pt}}
\put(151.0,859.0){\rule[-0.200pt]{310.279pt}{0.400pt}}
\end{picture}

\caption{
Závislosť napätia $U_l\(t\)$ na čase $t$ pre výstrel \#23728. S určenými časmi začiatku a konca života plazmy.
}\label{G_V-1-U}
\end{figure}

\begin{figure}
% GNUPLOT: LaTeX picture
\setlength{\unitlength}{0.240900pt}
\ifx\plotpoint\undefined\newsavebox{\plotpoint}\fi
\begin{picture}(1500,900)(0,0)
\sbox{\plotpoint}{\rule[-0.200pt]{0.400pt}{0.400pt}}%
\put(231.0,131.0){\rule[-0.200pt]{4.818pt}{0.400pt}}
\put(211,131){\makebox(0,0)[r]{-0.0001}}
\put(1419.0,131.0){\rule[-0.200pt]{4.818pt}{0.400pt}}
\put(231.0,197.0){\rule[-0.200pt]{4.818pt}{0.400pt}}
\put(211,197){\makebox(0,0)[r]{ 0}}
\put(1419.0,197.0){\rule[-0.200pt]{4.818pt}{0.400pt}}
\put(231.0,263.0){\rule[-0.200pt]{4.818pt}{0.400pt}}
\put(211,263){\makebox(0,0)[r]{ 0.0001}}
\put(1419.0,263.0){\rule[-0.200pt]{4.818pt}{0.400pt}}
\put(231.0,330.0){\rule[-0.200pt]{4.818pt}{0.400pt}}
\put(211,330){\makebox(0,0)[r]{ 0.0002}}
\put(1419.0,330.0){\rule[-0.200pt]{4.818pt}{0.400pt}}
\put(231.0,396.0){\rule[-0.200pt]{4.818pt}{0.400pt}}
\put(211,396){\makebox(0,0)[r]{ 0.0003}}
\put(1419.0,396.0){\rule[-0.200pt]{4.818pt}{0.400pt}}
\put(231.0,462.0){\rule[-0.200pt]{4.818pt}{0.400pt}}
\put(211,462){\makebox(0,0)[r]{ 0.0004}}
\put(1419.0,462.0){\rule[-0.200pt]{4.818pt}{0.400pt}}
\put(231.0,528.0){\rule[-0.200pt]{4.818pt}{0.400pt}}
\put(211,528){\makebox(0,0)[r]{ 0.0005}}
\put(1419.0,528.0){\rule[-0.200pt]{4.818pt}{0.400pt}}
\put(231.0,594.0){\rule[-0.200pt]{4.818pt}{0.400pt}}
\put(211,594){\makebox(0,0)[r]{ 0.0006}}
\put(1419.0,594.0){\rule[-0.200pt]{4.818pt}{0.400pt}}
\put(231.0,660.0){\rule[-0.200pt]{4.818pt}{0.400pt}}
\put(211,660){\makebox(0,0)[r]{ 0.0007}}
\put(1419.0,660.0){\rule[-0.200pt]{4.818pt}{0.400pt}}
\put(231.0,727.0){\rule[-0.200pt]{4.818pt}{0.400pt}}
\put(211,727){\makebox(0,0)[r]{ 0.0008}}
\put(1419.0,727.0){\rule[-0.200pt]{4.818pt}{0.400pt}}
\put(231.0,793.0){\rule[-0.200pt]{4.818pt}{0.400pt}}
\put(211,793){\makebox(0,0)[r]{ 0.0009}}
\put(1419.0,793.0){\rule[-0.200pt]{4.818pt}{0.400pt}}
\put(231.0,859.0){\rule[-0.200pt]{4.818pt}{0.400pt}}
\put(211,859){\makebox(0,0)[r]{ 0.001}}
\put(1419.0,859.0){\rule[-0.200pt]{4.818pt}{0.400pt}}
\put(231.0,131.0){\rule[-0.200pt]{0.400pt}{4.818pt}}
\put(231,90){\makebox(0,0){ 0}}
\put(231.0,839.0){\rule[-0.200pt]{0.400pt}{4.818pt}}
\put(432.0,131.0){\rule[-0.200pt]{0.400pt}{4.818pt}}
\put(432,90){\makebox(0,0){ 2}}
\put(432.0,839.0){\rule[-0.200pt]{0.400pt}{4.818pt}}
\put(634.0,131.0){\rule[-0.200pt]{0.400pt}{4.818pt}}
\put(634,90){\makebox(0,0){ 4}}
\put(634.0,839.0){\rule[-0.200pt]{0.400pt}{4.818pt}}
\put(835.0,131.0){\rule[-0.200pt]{0.400pt}{4.818pt}}
\put(835,90){\makebox(0,0){ 6}}
\put(835.0,839.0){\rule[-0.200pt]{0.400pt}{4.818pt}}
\put(1036.0,131.0){\rule[-0.200pt]{0.400pt}{4.818pt}}
\put(1036,90){\makebox(0,0){ 8}}
\put(1036.0,839.0){\rule[-0.200pt]{0.400pt}{4.818pt}}
\put(1238.0,131.0){\rule[-0.200pt]{0.400pt}{4.818pt}}
\put(1238,90){\makebox(0,0){ 10}}
\put(1238.0,839.0){\rule[-0.200pt]{0.400pt}{4.818pt}}
\put(1439.0,131.0){\rule[-0.200pt]{0.400pt}{4.818pt}}
\put(1439,90){\makebox(0,0){ 12}}
\put(1439.0,839.0){\rule[-0.200pt]{0.400pt}{4.818pt}}
\put(231.0,131.0){\rule[-0.200pt]{0.400pt}{175.375pt}}
\put(231.0,131.0){\rule[-0.200pt]{291.007pt}{0.400pt}}
\put(1439.0,131.0){\rule[-0.200pt]{0.400pt}{175.375pt}}
\put(231.0,859.0){\rule[-0.200pt]{291.007pt}{0.400pt}}
\put(30,495){\makebox(0,0){\popi{B_t}{T}}}
\put(835,29){\makebox(0,0){\popi{t}{ms}}}
\put(1279,172){\makebox(0,0)[r]{Toroidal mag. field $B_t$}}
\put(1299.0,172.0){\rule[-0.200pt]{24.090pt}{0.400pt}}
\put(232,197){\usebox{\plotpoint}}
\put(443,195.67){\rule{0.241pt}{0.400pt}}
\multiput(443.00,196.17)(0.500,-1.000){2}{\rule{0.120pt}{0.400pt}}
\put(443.67,196){\rule{0.400pt}{0.964pt}}
\multiput(443.17,196.00)(1.000,2.000){2}{\rule{0.400pt}{0.482pt}}
\put(444.67,198){\rule{0.400pt}{0.482pt}}
\multiput(444.17,199.00)(1.000,-1.000){2}{\rule{0.400pt}{0.241pt}}
\put(446,197.67){\rule{0.241pt}{0.400pt}}
\multiput(446.00,197.17)(0.500,1.000){2}{\rule{0.120pt}{0.400pt}}
\put(446.67,199){\rule{0.400pt}{0.482pt}}
\multiput(446.17,199.00)(1.000,1.000){2}{\rule{0.400pt}{0.241pt}}
\put(448,200.67){\rule{0.241pt}{0.400pt}}
\multiput(448.00,200.17)(0.500,1.000){2}{\rule{0.120pt}{0.400pt}}
\put(449,201.67){\rule{0.241pt}{0.400pt}}
\multiput(449.00,201.17)(0.500,1.000){2}{\rule{0.120pt}{0.400pt}}
\put(449.67,203){\rule{0.400pt}{0.482pt}}
\multiput(449.17,203.00)(1.000,1.000){2}{\rule{0.400pt}{0.241pt}}
\put(451,204.67){\rule{0.241pt}{0.400pt}}
\multiput(451.00,204.17)(0.500,1.000){2}{\rule{0.120pt}{0.400pt}}
\put(452,205.67){\rule{0.241pt}{0.400pt}}
\multiput(452.00,205.17)(0.500,1.000){2}{\rule{0.120pt}{0.400pt}}
\put(452.67,207){\rule{0.400pt}{0.482pt}}
\multiput(452.17,207.00)(1.000,1.000){2}{\rule{0.400pt}{0.241pt}}
\put(454,208.67){\rule{0.241pt}{0.400pt}}
\multiput(454.00,208.17)(0.500,1.000){2}{\rule{0.120pt}{0.400pt}}
\put(455,209.67){\rule{0.241pt}{0.400pt}}
\multiput(455.00,209.17)(0.500,1.000){2}{\rule{0.120pt}{0.400pt}}
\put(456,210.67){\rule{0.482pt}{0.400pt}}
\multiput(456.00,210.17)(1.000,1.000){2}{\rule{0.241pt}{0.400pt}}
\put(457.67,212){\rule{0.400pt}{0.482pt}}
\multiput(457.17,212.00)(1.000,1.000){2}{\rule{0.400pt}{0.241pt}}
\put(459,213.67){\rule{0.241pt}{0.400pt}}
\multiput(459.00,213.17)(0.500,1.000){2}{\rule{0.120pt}{0.400pt}}
\put(460,214.67){\rule{0.241pt}{0.400pt}}
\multiput(460.00,214.17)(0.500,1.000){2}{\rule{0.120pt}{0.400pt}}
\put(461,215.67){\rule{0.241pt}{0.400pt}}
\multiput(461.00,215.17)(0.500,1.000){2}{\rule{0.120pt}{0.400pt}}
\put(461.67,217){\rule{0.400pt}{0.482pt}}
\multiput(461.17,217.00)(1.000,1.000){2}{\rule{0.400pt}{0.241pt}}
\put(463,218.67){\rule{0.241pt}{0.400pt}}
\multiput(463.00,218.17)(0.500,1.000){2}{\rule{0.120pt}{0.400pt}}
\put(464,219.67){\rule{0.241pt}{0.400pt}}
\multiput(464.00,219.17)(0.500,1.000){2}{\rule{0.120pt}{0.400pt}}
\put(465,220.67){\rule{0.241pt}{0.400pt}}
\multiput(465.00,220.17)(0.500,1.000){2}{\rule{0.120pt}{0.400pt}}
\put(465.67,222){\rule{0.400pt}{0.482pt}}
\multiput(465.17,222.00)(1.000,1.000){2}{\rule{0.400pt}{0.241pt}}
\put(467,223.67){\rule{0.241pt}{0.400pt}}
\multiput(467.00,223.17)(0.500,1.000){2}{\rule{0.120pt}{0.400pt}}
\put(468,224.67){\rule{0.241pt}{0.400pt}}
\multiput(468.00,224.17)(0.500,1.000){2}{\rule{0.120pt}{0.400pt}}
\put(469,225.67){\rule{0.241pt}{0.400pt}}
\multiput(469.00,225.17)(0.500,1.000){2}{\rule{0.120pt}{0.400pt}}
\put(470,226.67){\rule{0.241pt}{0.400pt}}
\multiput(470.00,226.17)(0.500,1.000){2}{\rule{0.120pt}{0.400pt}}
\put(470.67,228){\rule{0.400pt}{0.482pt}}
\multiput(470.17,228.00)(1.000,1.000){2}{\rule{0.400pt}{0.241pt}}
\put(472,229.67){\rule{0.241pt}{0.400pt}}
\multiput(472.00,229.17)(0.500,1.000){2}{\rule{0.120pt}{0.400pt}}
\put(473,230.67){\rule{0.241pt}{0.400pt}}
\multiput(473.00,230.17)(0.500,1.000){2}{\rule{0.120pt}{0.400pt}}
\put(474,231.67){\rule{0.241pt}{0.400pt}}
\multiput(474.00,231.17)(0.500,1.000){2}{\rule{0.120pt}{0.400pt}}
\put(474.67,233){\rule{0.400pt}{0.482pt}}
\multiput(474.17,233.00)(1.000,1.000){2}{\rule{0.400pt}{0.241pt}}
\put(476,234.67){\rule{0.241pt}{0.400pt}}
\multiput(476.00,234.17)(0.500,1.000){2}{\rule{0.120pt}{0.400pt}}
\put(477,235.67){\rule{0.241pt}{0.400pt}}
\multiput(477.00,235.17)(0.500,1.000){2}{\rule{0.120pt}{0.400pt}}
\put(478,236.67){\rule{0.241pt}{0.400pt}}
\multiput(478.00,236.17)(0.500,1.000){2}{\rule{0.120pt}{0.400pt}}
\put(479,237.67){\rule{0.241pt}{0.400pt}}
\multiput(479.00,237.17)(0.500,1.000){2}{\rule{0.120pt}{0.400pt}}
\put(480,238.67){\rule{0.241pt}{0.400pt}}
\multiput(480.00,238.17)(0.500,1.000){2}{\rule{0.120pt}{0.400pt}}
\put(480.67,240){\rule{0.400pt}{0.482pt}}
\multiput(480.17,240.00)(1.000,1.000){2}{\rule{0.400pt}{0.241pt}}
\put(482,241.67){\rule{0.241pt}{0.400pt}}
\multiput(482.00,241.17)(0.500,1.000){2}{\rule{0.120pt}{0.400pt}}
\put(483,242.67){\rule{0.241pt}{0.400pt}}
\multiput(483.00,242.17)(0.500,1.000){2}{\rule{0.120pt}{0.400pt}}
\put(484,243.67){\rule{0.241pt}{0.400pt}}
\multiput(484.00,243.17)(0.500,1.000){2}{\rule{0.120pt}{0.400pt}}
\put(485,244.67){\rule{0.241pt}{0.400pt}}
\multiput(485.00,244.17)(0.500,1.000){2}{\rule{0.120pt}{0.400pt}}
\put(485.67,246){\rule{0.400pt}{0.482pt}}
\multiput(485.17,246.00)(1.000,1.000){2}{\rule{0.400pt}{0.241pt}}
\put(487,247.67){\rule{0.241pt}{0.400pt}}
\multiput(487.00,247.17)(0.500,1.000){2}{\rule{0.120pt}{0.400pt}}
\put(488,248.67){\rule{0.241pt}{0.400pt}}
\multiput(488.00,248.17)(0.500,1.000){2}{\rule{0.120pt}{0.400pt}}
\put(489,249.67){\rule{0.241pt}{0.400pt}}
\multiput(489.00,249.17)(0.500,1.000){2}{\rule{0.120pt}{0.400pt}}
\put(490,250.67){\rule{0.241pt}{0.400pt}}
\multiput(490.00,250.17)(0.500,1.000){2}{\rule{0.120pt}{0.400pt}}
\put(491,251.67){\rule{0.241pt}{0.400pt}}
\multiput(491.00,251.17)(0.500,1.000){2}{\rule{0.120pt}{0.400pt}}
\put(491.67,253){\rule{0.400pt}{0.482pt}}
\multiput(491.17,253.00)(1.000,1.000){2}{\rule{0.400pt}{0.241pt}}
\put(493,254.67){\rule{0.241pt}{0.400pt}}
\multiput(493.00,254.17)(0.500,1.000){2}{\rule{0.120pt}{0.400pt}}
\put(494,255.67){\rule{0.241pt}{0.400pt}}
\multiput(494.00,255.17)(0.500,1.000){2}{\rule{0.120pt}{0.400pt}}
\put(495,256.67){\rule{0.241pt}{0.400pt}}
\multiput(495.00,256.17)(0.500,1.000){2}{\rule{0.120pt}{0.400pt}}
\put(496,257.67){\rule{0.241pt}{0.400pt}}
\multiput(496.00,257.17)(0.500,1.000){2}{\rule{0.120pt}{0.400pt}}
\put(497,258.67){\rule{0.241pt}{0.400pt}}
\multiput(497.00,258.17)(0.500,1.000){2}{\rule{0.120pt}{0.400pt}}
\put(498,259.67){\rule{0.241pt}{0.400pt}}
\multiput(498.00,259.17)(0.500,1.000){2}{\rule{0.120pt}{0.400pt}}
\put(498.67,261){\rule{0.400pt}{0.482pt}}
\multiput(498.17,261.00)(1.000,1.000){2}{\rule{0.400pt}{0.241pt}}
\put(500,262.67){\rule{0.241pt}{0.400pt}}
\multiput(500.00,262.17)(0.500,1.000){2}{\rule{0.120pt}{0.400pt}}
\put(501,263.67){\rule{0.241pt}{0.400pt}}
\multiput(501.00,263.17)(0.500,1.000){2}{\rule{0.120pt}{0.400pt}}
\put(502,264.67){\rule{0.241pt}{0.400pt}}
\multiput(502.00,264.17)(0.500,1.000){2}{\rule{0.120pt}{0.400pt}}
\put(503,265.67){\rule{0.241pt}{0.400pt}}
\multiput(503.00,265.17)(0.500,1.000){2}{\rule{0.120pt}{0.400pt}}
\put(504,266.67){\rule{0.241pt}{0.400pt}}
\multiput(504.00,266.17)(0.500,1.000){2}{\rule{0.120pt}{0.400pt}}
\put(505,267.67){\rule{0.241pt}{0.400pt}}
\multiput(505.00,267.17)(0.500,1.000){2}{\rule{0.120pt}{0.400pt}}
\put(505.67,269){\rule{0.400pt}{0.482pt}}
\multiput(505.17,269.00)(1.000,1.000){2}{\rule{0.400pt}{0.241pt}}
\put(507,270.67){\rule{0.241pt}{0.400pt}}
\multiput(507.00,270.17)(0.500,1.000){2}{\rule{0.120pt}{0.400pt}}
\put(508,271.67){\rule{0.241pt}{0.400pt}}
\multiput(508.00,271.17)(0.500,1.000){2}{\rule{0.120pt}{0.400pt}}
\put(509,272.67){\rule{0.241pt}{0.400pt}}
\multiput(509.00,272.17)(0.500,1.000){2}{\rule{0.120pt}{0.400pt}}
\put(510,273.67){\rule{0.241pt}{0.400pt}}
\multiput(510.00,273.17)(0.500,1.000){2}{\rule{0.120pt}{0.400pt}}
\put(511,274.67){\rule{0.241pt}{0.400pt}}
\multiput(511.00,274.17)(0.500,1.000){2}{\rule{0.120pt}{0.400pt}}
\put(512,275.67){\rule{0.241pt}{0.400pt}}
\multiput(512.00,275.17)(0.500,1.000){2}{\rule{0.120pt}{0.400pt}}
\put(513,276.67){\rule{0.241pt}{0.400pt}}
\multiput(513.00,276.17)(0.500,1.000){2}{\rule{0.120pt}{0.400pt}}
\put(513.67,278){\rule{0.400pt}{0.482pt}}
\multiput(513.17,278.00)(1.000,1.000){2}{\rule{0.400pt}{0.241pt}}
\put(515,279.67){\rule{0.241pt}{0.400pt}}
\multiput(515.00,279.17)(0.500,1.000){2}{\rule{0.120pt}{0.400pt}}
\put(516,280.67){\rule{0.241pt}{0.400pt}}
\multiput(516.00,280.17)(0.500,1.000){2}{\rule{0.120pt}{0.400pt}}
\put(517,281.67){\rule{0.241pt}{0.400pt}}
\multiput(517.00,281.17)(0.500,1.000){2}{\rule{0.120pt}{0.400pt}}
\put(518,282.67){\rule{0.241pt}{0.400pt}}
\multiput(518.00,282.17)(0.500,1.000){2}{\rule{0.120pt}{0.400pt}}
\put(519,283.67){\rule{0.241pt}{0.400pt}}
\multiput(519.00,283.17)(0.500,1.000){2}{\rule{0.120pt}{0.400pt}}
\put(520,284.67){\rule{0.241pt}{0.400pt}}
\multiput(520.00,284.17)(0.500,1.000){2}{\rule{0.120pt}{0.400pt}}
\put(520.67,286){\rule{0.400pt}{0.482pt}}
\multiput(520.17,286.00)(1.000,1.000){2}{\rule{0.400pt}{0.241pt}}
\put(522,287.67){\rule{0.241pt}{0.400pt}}
\multiput(522.00,287.17)(0.500,1.000){2}{\rule{0.120pt}{0.400pt}}
\put(523,288.67){\rule{0.241pt}{0.400pt}}
\multiput(523.00,288.17)(0.500,1.000){2}{\rule{0.120pt}{0.400pt}}
\put(524,289.67){\rule{0.241pt}{0.400pt}}
\multiput(524.00,289.17)(0.500,1.000){2}{\rule{0.120pt}{0.400pt}}
\put(525,290.67){\rule{0.241pt}{0.400pt}}
\multiput(525.00,290.17)(0.500,1.000){2}{\rule{0.120pt}{0.400pt}}
\put(526,291.67){\rule{0.241pt}{0.400pt}}
\multiput(526.00,291.17)(0.500,1.000){2}{\rule{0.120pt}{0.400pt}}
\put(527,292.67){\rule{0.241pt}{0.400pt}}
\multiput(527.00,292.17)(0.500,1.000){2}{\rule{0.120pt}{0.400pt}}
\put(528,293.67){\rule{0.241pt}{0.400pt}}
\multiput(528.00,293.17)(0.500,1.000){2}{\rule{0.120pt}{0.400pt}}
\put(529,294.67){\rule{0.241pt}{0.400pt}}
\multiput(529.00,294.17)(0.500,1.000){2}{\rule{0.120pt}{0.400pt}}
\put(529.67,296){\rule{0.400pt}{0.482pt}}
\multiput(529.17,296.00)(1.000,1.000){2}{\rule{0.400pt}{0.241pt}}
\put(531,297.67){\rule{0.241pt}{0.400pt}}
\multiput(531.00,297.17)(0.500,1.000){2}{\rule{0.120pt}{0.400pt}}
\put(532,298.67){\rule{0.241pt}{0.400pt}}
\multiput(532.00,298.17)(0.500,1.000){2}{\rule{0.120pt}{0.400pt}}
\put(533,299.67){\rule{0.241pt}{0.400pt}}
\multiput(533.00,299.17)(0.500,1.000){2}{\rule{0.120pt}{0.400pt}}
\put(534,300.67){\rule{0.241pt}{0.400pt}}
\multiput(534.00,300.17)(0.500,1.000){2}{\rule{0.120pt}{0.400pt}}
\put(535,301.67){\rule{0.241pt}{0.400pt}}
\multiput(535.00,301.17)(0.500,1.000){2}{\rule{0.120pt}{0.400pt}}
\put(536,302.67){\rule{0.241pt}{0.400pt}}
\multiput(536.00,302.17)(0.500,1.000){2}{\rule{0.120pt}{0.400pt}}
\put(536.67,304){\rule{0.400pt}{0.482pt}}
\multiput(536.17,304.00)(1.000,1.000){2}{\rule{0.400pt}{0.241pt}}
\put(538,305.67){\rule{0.241pt}{0.400pt}}
\multiput(538.00,305.17)(0.500,1.000){2}{\rule{0.120pt}{0.400pt}}
\put(539,306.67){\rule{0.241pt}{0.400pt}}
\multiput(539.00,306.17)(0.500,1.000){2}{\rule{0.120pt}{0.400pt}}
\put(540,307.67){\rule{0.241pt}{0.400pt}}
\multiput(540.00,307.17)(0.500,1.000){2}{\rule{0.120pt}{0.400pt}}
\put(541,308.67){\rule{0.241pt}{0.400pt}}
\multiput(541.00,308.17)(0.500,1.000){2}{\rule{0.120pt}{0.400pt}}
\put(542,309.67){\rule{0.241pt}{0.400pt}}
\multiput(542.00,309.17)(0.500,1.000){2}{\rule{0.120pt}{0.400pt}}
\put(543,310.67){\rule{0.241pt}{0.400pt}}
\multiput(543.00,310.17)(0.500,1.000){2}{\rule{0.120pt}{0.400pt}}
\put(544,311.67){\rule{0.241pt}{0.400pt}}
\multiput(544.00,311.17)(0.500,1.000){2}{\rule{0.120pt}{0.400pt}}
\put(545,312.67){\rule{0.241pt}{0.400pt}}
\multiput(545.00,312.17)(0.500,1.000){2}{\rule{0.120pt}{0.400pt}}
\put(546,313.67){\rule{0.241pt}{0.400pt}}
\multiput(546.00,313.17)(0.500,1.000){2}{\rule{0.120pt}{0.400pt}}
\put(546.67,315){\rule{0.400pt}{0.482pt}}
\multiput(546.17,315.00)(1.000,1.000){2}{\rule{0.400pt}{0.241pt}}
\put(548,316.67){\rule{0.241pt}{0.400pt}}
\multiput(548.00,316.17)(0.500,1.000){2}{\rule{0.120pt}{0.400pt}}
\put(549,317.67){\rule{0.241pt}{0.400pt}}
\multiput(549.00,317.17)(0.500,1.000){2}{\rule{0.120pt}{0.400pt}}
\put(550,318.67){\rule{0.241pt}{0.400pt}}
\multiput(550.00,318.17)(0.500,1.000){2}{\rule{0.120pt}{0.400pt}}
\put(551,319.67){\rule{0.241pt}{0.400pt}}
\multiput(551.00,319.17)(0.500,1.000){2}{\rule{0.120pt}{0.400pt}}
\put(552,320.67){\rule{0.241pt}{0.400pt}}
\multiput(552.00,320.17)(0.500,1.000){2}{\rule{0.120pt}{0.400pt}}
\put(553,321.67){\rule{0.241pt}{0.400pt}}
\multiput(553.00,321.17)(0.500,1.000){2}{\rule{0.120pt}{0.400pt}}
\put(554,322.67){\rule{0.241pt}{0.400pt}}
\multiput(554.00,322.17)(0.500,1.000){2}{\rule{0.120pt}{0.400pt}}
\put(555,323.67){\rule{0.241pt}{0.400pt}}
\multiput(555.00,323.17)(0.500,1.000){2}{\rule{0.120pt}{0.400pt}}
\put(556,324.67){\rule{0.241pt}{0.400pt}}
\multiput(556.00,324.17)(0.500,1.000){2}{\rule{0.120pt}{0.400pt}}
\put(557,325.67){\rule{0.241pt}{0.400pt}}
\multiput(557.00,325.17)(0.500,1.000){2}{\rule{0.120pt}{0.400pt}}
\put(558,326.67){\rule{0.241pt}{0.400pt}}
\multiput(558.00,326.17)(0.500,1.000){2}{\rule{0.120pt}{0.400pt}}
\put(558.67,328){\rule{0.400pt}{0.482pt}}
\multiput(558.17,328.00)(1.000,1.000){2}{\rule{0.400pt}{0.241pt}}
\put(560,329.67){\rule{0.241pt}{0.400pt}}
\multiput(560.00,329.17)(0.500,1.000){2}{\rule{0.120pt}{0.400pt}}
\put(561,330.67){\rule{0.241pt}{0.400pt}}
\multiput(561.00,330.17)(0.500,1.000){2}{\rule{0.120pt}{0.400pt}}
\put(562,331.67){\rule{0.241pt}{0.400pt}}
\multiput(562.00,331.17)(0.500,1.000){2}{\rule{0.120pt}{0.400pt}}
\put(563,332.67){\rule{0.241pt}{0.400pt}}
\multiput(563.00,332.17)(0.500,1.000){2}{\rule{0.120pt}{0.400pt}}
\put(564,333.67){\rule{0.241pt}{0.400pt}}
\multiput(564.00,333.17)(0.500,1.000){2}{\rule{0.120pt}{0.400pt}}
\put(565,334.67){\rule{0.241pt}{0.400pt}}
\multiput(565.00,334.17)(0.500,1.000){2}{\rule{0.120pt}{0.400pt}}
\put(566,335.67){\rule{0.241pt}{0.400pt}}
\multiput(566.00,335.17)(0.500,1.000){2}{\rule{0.120pt}{0.400pt}}
\put(567,336.67){\rule{0.241pt}{0.400pt}}
\multiput(567.00,336.17)(0.500,1.000){2}{\rule{0.120pt}{0.400pt}}
\put(568,337.67){\rule{0.241pt}{0.400pt}}
\multiput(568.00,337.17)(0.500,1.000){2}{\rule{0.120pt}{0.400pt}}
\put(568.67,339){\rule{0.400pt}{0.482pt}}
\multiput(568.17,339.00)(1.000,1.000){2}{\rule{0.400pt}{0.241pt}}
\put(232.0,197.0){\rule[-0.200pt]{50.830pt}{0.400pt}}
\put(570.67,341){\rule{0.400pt}{0.482pt}}
\multiput(570.17,341.00)(1.000,1.000){2}{\rule{0.400pt}{0.241pt}}
\put(572,342.67){\rule{0.241pt}{0.400pt}}
\multiput(572.00,342.17)(0.500,1.000){2}{\rule{0.120pt}{0.400pt}}
\put(573,343.67){\rule{0.241pt}{0.400pt}}
\multiput(573.00,343.17)(0.500,1.000){2}{\rule{0.120pt}{0.400pt}}
\put(574,344.67){\rule{0.241pt}{0.400pt}}
\multiput(574.00,344.17)(0.500,1.000){2}{\rule{0.120pt}{0.400pt}}
\put(575,345.67){\rule{0.241pt}{0.400pt}}
\multiput(575.00,345.17)(0.500,1.000){2}{\rule{0.120pt}{0.400pt}}
\put(576,346.67){\rule{0.241pt}{0.400pt}}
\multiput(576.00,346.17)(0.500,1.000){2}{\rule{0.120pt}{0.400pt}}
\put(577,347.67){\rule{0.241pt}{0.400pt}}
\multiput(577.00,347.17)(0.500,1.000){2}{\rule{0.120pt}{0.400pt}}
\put(578,348.67){\rule{0.241pt}{0.400pt}}
\multiput(578.00,348.17)(0.500,1.000){2}{\rule{0.120pt}{0.400pt}}
\put(579,349.67){\rule{0.241pt}{0.400pt}}
\multiput(579.00,349.17)(0.500,1.000){2}{\rule{0.120pt}{0.400pt}}
\put(580,350.67){\rule{0.241pt}{0.400pt}}
\multiput(580.00,350.17)(0.500,1.000){2}{\rule{0.120pt}{0.400pt}}
\put(581,351.67){\rule{0.241pt}{0.400pt}}
\multiput(581.00,351.17)(0.500,1.000){2}{\rule{0.120pt}{0.400pt}}
\put(582,352.67){\rule{0.241pt}{0.400pt}}
\multiput(582.00,352.17)(0.500,1.000){2}{\rule{0.120pt}{0.400pt}}
\put(582.67,354){\rule{0.400pt}{0.482pt}}
\multiput(582.17,354.00)(1.000,1.000){2}{\rule{0.400pt}{0.241pt}}
\put(570.0,341.0){\usebox{\plotpoint}}
\put(584.67,356){\rule{0.400pt}{0.482pt}}
\multiput(584.17,356.00)(1.000,1.000){2}{\rule{0.400pt}{0.241pt}}
\put(586,357.67){\rule{0.241pt}{0.400pt}}
\multiput(586.00,357.17)(0.500,1.000){2}{\rule{0.120pt}{0.400pt}}
\put(587,358.67){\rule{0.241pt}{0.400pt}}
\multiput(587.00,358.17)(0.500,1.000){2}{\rule{0.120pt}{0.400pt}}
\put(588,359.67){\rule{0.241pt}{0.400pt}}
\multiput(588.00,359.17)(0.500,1.000){2}{\rule{0.120pt}{0.400pt}}
\put(589,360.67){\rule{0.241pt}{0.400pt}}
\multiput(589.00,360.17)(0.500,1.000){2}{\rule{0.120pt}{0.400pt}}
\put(590,361.67){\rule{0.241pt}{0.400pt}}
\multiput(590.00,361.17)(0.500,1.000){2}{\rule{0.120pt}{0.400pt}}
\put(591,362.67){\rule{0.241pt}{0.400pt}}
\multiput(591.00,362.17)(0.500,1.000){2}{\rule{0.120pt}{0.400pt}}
\put(592,363.67){\rule{0.241pt}{0.400pt}}
\multiput(592.00,363.17)(0.500,1.000){2}{\rule{0.120pt}{0.400pt}}
\put(593,364.67){\rule{0.241pt}{0.400pt}}
\multiput(593.00,364.17)(0.500,1.000){2}{\rule{0.120pt}{0.400pt}}
\put(594,365.67){\rule{0.241pt}{0.400pt}}
\multiput(594.00,365.17)(0.500,1.000){2}{\rule{0.120pt}{0.400pt}}
\put(595,366.67){\rule{0.241pt}{0.400pt}}
\multiput(595.00,366.17)(0.500,1.000){2}{\rule{0.120pt}{0.400pt}}
\put(596,367.67){\rule{0.241pt}{0.400pt}}
\multiput(596.00,367.17)(0.500,1.000){2}{\rule{0.120pt}{0.400pt}}
\put(597,368.67){\rule{0.241pt}{0.400pt}}
\multiput(597.00,368.17)(0.500,1.000){2}{\rule{0.120pt}{0.400pt}}
\put(598,369.67){\rule{0.241pt}{0.400pt}}
\multiput(598.00,369.17)(0.500,1.000){2}{\rule{0.120pt}{0.400pt}}
\put(599,370.67){\rule{0.241pt}{0.400pt}}
\multiput(599.00,370.17)(0.500,1.000){2}{\rule{0.120pt}{0.400pt}}
\put(600,371.67){\rule{0.241pt}{0.400pt}}
\multiput(600.00,371.17)(0.500,1.000){2}{\rule{0.120pt}{0.400pt}}
\put(601,372.67){\rule{0.241pt}{0.400pt}}
\multiput(601.00,372.17)(0.500,1.000){2}{\rule{0.120pt}{0.400pt}}
\put(602,373.67){\rule{0.241pt}{0.400pt}}
\multiput(602.00,373.17)(0.500,1.000){2}{\rule{0.120pt}{0.400pt}}
\put(603,374.67){\rule{0.241pt}{0.400pt}}
\multiput(603.00,374.17)(0.500,1.000){2}{\rule{0.120pt}{0.400pt}}
\put(603.67,376){\rule{0.400pt}{0.482pt}}
\multiput(603.17,376.00)(1.000,1.000){2}{\rule{0.400pt}{0.241pt}}
\put(605,377.67){\rule{0.241pt}{0.400pt}}
\multiput(605.00,377.17)(0.500,1.000){2}{\rule{0.120pt}{0.400pt}}
\put(606,378.67){\rule{0.241pt}{0.400pt}}
\multiput(606.00,378.17)(0.500,1.000){2}{\rule{0.120pt}{0.400pt}}
\put(607,379.67){\rule{0.482pt}{0.400pt}}
\multiput(607.00,379.17)(1.000,1.000){2}{\rule{0.241pt}{0.400pt}}
\put(609,380.67){\rule{0.241pt}{0.400pt}}
\multiput(609.00,380.17)(0.500,1.000){2}{\rule{0.120pt}{0.400pt}}
\put(610,381.67){\rule{0.241pt}{0.400pt}}
\multiput(610.00,381.17)(0.500,1.000){2}{\rule{0.120pt}{0.400pt}}
\put(611,382.67){\rule{0.241pt}{0.400pt}}
\multiput(611.00,382.17)(0.500,1.000){2}{\rule{0.120pt}{0.400pt}}
\put(612,383.67){\rule{0.241pt}{0.400pt}}
\multiput(612.00,383.17)(0.500,1.000){2}{\rule{0.120pt}{0.400pt}}
\put(613,384.67){\rule{0.241pt}{0.400pt}}
\multiput(613.00,384.17)(0.500,1.000){2}{\rule{0.120pt}{0.400pt}}
\put(614,385.67){\rule{0.241pt}{0.400pt}}
\multiput(614.00,385.17)(0.500,1.000){2}{\rule{0.120pt}{0.400pt}}
\put(615,386.67){\rule{0.241pt}{0.400pt}}
\multiput(615.00,386.17)(0.500,1.000){2}{\rule{0.120pt}{0.400pt}}
\put(616,387.67){\rule{0.241pt}{0.400pt}}
\multiput(616.00,387.17)(0.500,1.000){2}{\rule{0.120pt}{0.400pt}}
\put(617,388.67){\rule{0.241pt}{0.400pt}}
\multiput(617.00,388.17)(0.500,1.000){2}{\rule{0.120pt}{0.400pt}}
\put(618,389.67){\rule{0.241pt}{0.400pt}}
\multiput(618.00,389.17)(0.500,1.000){2}{\rule{0.120pt}{0.400pt}}
\put(619,390.67){\rule{0.241pt}{0.400pt}}
\multiput(619.00,390.17)(0.500,1.000){2}{\rule{0.120pt}{0.400pt}}
\put(620,391.67){\rule{0.241pt}{0.400pt}}
\multiput(620.00,391.17)(0.500,1.000){2}{\rule{0.120pt}{0.400pt}}
\put(621,392.67){\rule{0.241pt}{0.400pt}}
\multiput(621.00,392.17)(0.500,1.000){2}{\rule{0.120pt}{0.400pt}}
\put(622,393.67){\rule{0.241pt}{0.400pt}}
\multiput(622.00,393.17)(0.500,1.000){2}{\rule{0.120pt}{0.400pt}}
\put(623,394.67){\rule{0.241pt}{0.400pt}}
\multiput(623.00,394.17)(0.500,1.000){2}{\rule{0.120pt}{0.400pt}}
\put(624,395.67){\rule{0.241pt}{0.400pt}}
\multiput(624.00,395.17)(0.500,1.000){2}{\rule{0.120pt}{0.400pt}}
\put(625,396.67){\rule{0.241pt}{0.400pt}}
\multiput(625.00,396.17)(0.500,1.000){2}{\rule{0.120pt}{0.400pt}}
\put(626,397.67){\rule{0.241pt}{0.400pt}}
\multiput(626.00,397.17)(0.500,1.000){2}{\rule{0.120pt}{0.400pt}}
\put(627,398.67){\rule{0.241pt}{0.400pt}}
\multiput(627.00,398.17)(0.500,1.000){2}{\rule{0.120pt}{0.400pt}}
\put(628,399.67){\rule{0.241pt}{0.400pt}}
\multiput(628.00,399.17)(0.500,1.000){2}{\rule{0.120pt}{0.400pt}}
\put(629,400.67){\rule{0.241pt}{0.400pt}}
\multiput(629.00,400.17)(0.500,1.000){2}{\rule{0.120pt}{0.400pt}}
\put(630,401.67){\rule{0.241pt}{0.400pt}}
\multiput(630.00,401.17)(0.500,1.000){2}{\rule{0.120pt}{0.400pt}}
\put(631,402.67){\rule{0.241pt}{0.400pt}}
\multiput(631.00,402.17)(0.500,1.000){2}{\rule{0.120pt}{0.400pt}}
\put(632,403.67){\rule{0.241pt}{0.400pt}}
\multiput(632.00,403.17)(0.500,1.000){2}{\rule{0.120pt}{0.400pt}}
\put(633,404.67){\rule{0.241pt}{0.400pt}}
\multiput(633.00,404.17)(0.500,1.000){2}{\rule{0.120pt}{0.400pt}}
\put(633.67,406){\rule{0.400pt}{0.482pt}}
\multiput(633.17,406.00)(1.000,1.000){2}{\rule{0.400pt}{0.241pt}}
\put(584.0,356.0){\usebox{\plotpoint}}
\put(635.67,408){\rule{0.400pt}{0.482pt}}
\multiput(635.17,408.00)(1.000,1.000){2}{\rule{0.400pt}{0.241pt}}
\put(635.0,408.0){\usebox{\plotpoint}}
\put(637.67,410){\rule{0.400pt}{0.482pt}}
\multiput(637.17,410.00)(1.000,1.000){2}{\rule{0.400pt}{0.241pt}}
\put(637.0,410.0){\usebox{\plotpoint}}
\put(640,411.67){\rule{0.241pt}{0.400pt}}
\multiput(640.00,411.17)(0.500,1.000){2}{\rule{0.120pt}{0.400pt}}
\put(640.67,413){\rule{0.400pt}{0.482pt}}
\multiput(640.17,413.00)(1.000,1.000){2}{\rule{0.400pt}{0.241pt}}
\put(642,414.67){\rule{0.241pt}{0.400pt}}
\multiput(642.00,414.17)(0.500,1.000){2}{\rule{0.120pt}{0.400pt}}
\put(643,415.67){\rule{0.241pt}{0.400pt}}
\multiput(643.00,415.17)(0.500,1.000){2}{\rule{0.120pt}{0.400pt}}
\put(644,416.67){\rule{0.241pt}{0.400pt}}
\multiput(644.00,416.17)(0.500,1.000){2}{\rule{0.120pt}{0.400pt}}
\put(645,417.67){\rule{0.241pt}{0.400pt}}
\multiput(645.00,417.17)(0.500,1.000){2}{\rule{0.120pt}{0.400pt}}
\put(646,418.67){\rule{0.241pt}{0.400pt}}
\multiput(646.00,418.17)(0.500,1.000){2}{\rule{0.120pt}{0.400pt}}
\put(647,419.67){\rule{0.241pt}{0.400pt}}
\multiput(647.00,419.17)(0.500,1.000){2}{\rule{0.120pt}{0.400pt}}
\put(648,420.67){\rule{0.241pt}{0.400pt}}
\multiput(648.00,420.17)(0.500,1.000){2}{\rule{0.120pt}{0.400pt}}
\put(639.0,412.0){\usebox{\plotpoint}}
\put(650,421.67){\rule{0.241pt}{0.400pt}}
\multiput(650.00,421.17)(0.500,1.000){2}{\rule{0.120pt}{0.400pt}}
\put(651,422.67){\rule{0.241pt}{0.400pt}}
\multiput(651.00,422.17)(0.500,1.000){2}{\rule{0.120pt}{0.400pt}}
\put(652,423.67){\rule{0.241pt}{0.400pt}}
\multiput(652.00,423.17)(0.500,1.000){2}{\rule{0.120pt}{0.400pt}}
\put(653,424.67){\rule{0.241pt}{0.400pt}}
\multiput(653.00,424.17)(0.500,1.000){2}{\rule{0.120pt}{0.400pt}}
\put(654,425.67){\rule{0.241pt}{0.400pt}}
\multiput(654.00,425.17)(0.500,1.000){2}{\rule{0.120pt}{0.400pt}}
\put(655,426.67){\rule{0.241pt}{0.400pt}}
\multiput(655.00,426.17)(0.500,1.000){2}{\rule{0.120pt}{0.400pt}}
\put(656,427.67){\rule{0.241pt}{0.400pt}}
\multiput(656.00,427.17)(0.500,1.000){2}{\rule{0.120pt}{0.400pt}}
\put(657,428.67){\rule{0.241pt}{0.400pt}}
\multiput(657.00,428.17)(0.500,1.000){2}{\rule{0.120pt}{0.400pt}}
\put(658,429.67){\rule{0.241pt}{0.400pt}}
\multiput(658.00,429.17)(0.500,1.000){2}{\rule{0.120pt}{0.400pt}}
\put(659,430.67){\rule{0.241pt}{0.400pt}}
\multiput(659.00,430.17)(0.500,1.000){2}{\rule{0.120pt}{0.400pt}}
\put(660,431.67){\rule{0.241pt}{0.400pt}}
\multiput(660.00,431.17)(0.500,1.000){2}{\rule{0.120pt}{0.400pt}}
\put(661,432.67){\rule{0.241pt}{0.400pt}}
\multiput(661.00,432.17)(0.500,1.000){2}{\rule{0.120pt}{0.400pt}}
\put(662,433.67){\rule{0.241pt}{0.400pt}}
\multiput(662.00,433.17)(0.500,1.000){2}{\rule{0.120pt}{0.400pt}}
\put(663,434.67){\rule{0.241pt}{0.400pt}}
\multiput(663.00,434.17)(0.500,1.000){2}{\rule{0.120pt}{0.400pt}}
\put(664,435.67){\rule{0.241pt}{0.400pt}}
\multiput(664.00,435.17)(0.500,1.000){2}{\rule{0.120pt}{0.400pt}}
\put(665,436.67){\rule{0.241pt}{0.400pt}}
\multiput(665.00,436.17)(0.500,1.000){2}{\rule{0.120pt}{0.400pt}}
\put(666,437.67){\rule{0.241pt}{0.400pt}}
\multiput(666.00,437.17)(0.500,1.000){2}{\rule{0.120pt}{0.400pt}}
\put(667,438.67){\rule{0.241pt}{0.400pt}}
\multiput(667.00,438.17)(0.500,1.000){2}{\rule{0.120pt}{0.400pt}}
\put(668,439.67){\rule{0.241pt}{0.400pt}}
\multiput(668.00,439.17)(0.500,1.000){2}{\rule{0.120pt}{0.400pt}}
\put(669,440.67){\rule{0.241pt}{0.400pt}}
\multiput(669.00,440.17)(0.500,1.000){2}{\rule{0.120pt}{0.400pt}}
\put(670,441.67){\rule{0.241pt}{0.400pt}}
\multiput(670.00,441.17)(0.500,1.000){2}{\rule{0.120pt}{0.400pt}}
\put(671,442.67){\rule{0.241pt}{0.400pt}}
\multiput(671.00,442.17)(0.500,1.000){2}{\rule{0.120pt}{0.400pt}}
\put(672,443.67){\rule{0.241pt}{0.400pt}}
\multiput(672.00,443.17)(0.500,1.000){2}{\rule{0.120pt}{0.400pt}}
\put(673,444.67){\rule{0.241pt}{0.400pt}}
\multiput(673.00,444.17)(0.500,1.000){2}{\rule{0.120pt}{0.400pt}}
\put(674,445.67){\rule{0.241pt}{0.400pt}}
\multiput(674.00,445.17)(0.500,1.000){2}{\rule{0.120pt}{0.400pt}}
\put(675,446.67){\rule{0.241pt}{0.400pt}}
\multiput(675.00,446.17)(0.500,1.000){2}{\rule{0.120pt}{0.400pt}}
\put(676,447.67){\rule{0.241pt}{0.400pt}}
\multiput(676.00,447.17)(0.500,1.000){2}{\rule{0.120pt}{0.400pt}}
\put(677,448.67){\rule{0.241pt}{0.400pt}}
\multiput(677.00,448.17)(0.500,1.000){2}{\rule{0.120pt}{0.400pt}}
\put(678,449.67){\rule{0.241pt}{0.400pt}}
\multiput(678.00,449.17)(0.500,1.000){2}{\rule{0.120pt}{0.400pt}}
\put(679,450.67){\rule{0.241pt}{0.400pt}}
\multiput(679.00,450.17)(0.500,1.000){2}{\rule{0.120pt}{0.400pt}}
\put(680,451.67){\rule{0.241pt}{0.400pt}}
\multiput(680.00,451.17)(0.500,1.000){2}{\rule{0.120pt}{0.400pt}}
\put(681,452.67){\rule{0.241pt}{0.400pt}}
\multiput(681.00,452.17)(0.500,1.000){2}{\rule{0.120pt}{0.400pt}}
\put(682,453.67){\rule{0.241pt}{0.400pt}}
\multiput(682.00,453.17)(0.500,1.000){2}{\rule{0.120pt}{0.400pt}}
\put(683,454.67){\rule{0.241pt}{0.400pt}}
\multiput(683.00,454.17)(0.500,1.000){2}{\rule{0.120pt}{0.400pt}}
\put(684,455.67){\rule{0.241pt}{0.400pt}}
\multiput(684.00,455.17)(0.500,1.000){2}{\rule{0.120pt}{0.400pt}}
\put(685,456.67){\rule{0.241pt}{0.400pt}}
\multiput(685.00,456.17)(0.500,1.000){2}{\rule{0.120pt}{0.400pt}}
\put(686,457.67){\rule{0.241pt}{0.400pt}}
\multiput(686.00,457.17)(0.500,1.000){2}{\rule{0.120pt}{0.400pt}}
\put(687,458.67){\rule{0.241pt}{0.400pt}}
\multiput(687.00,458.17)(0.500,1.000){2}{\rule{0.120pt}{0.400pt}}
\put(688,459.67){\rule{0.241pt}{0.400pt}}
\multiput(688.00,459.17)(0.500,1.000){2}{\rule{0.120pt}{0.400pt}}
\put(689,460.67){\rule{0.241pt}{0.400pt}}
\multiput(689.00,460.17)(0.500,1.000){2}{\rule{0.120pt}{0.400pt}}
\put(690,461.67){\rule{0.241pt}{0.400pt}}
\multiput(690.00,461.17)(0.500,1.000){2}{\rule{0.120pt}{0.400pt}}
\put(691,462.67){\rule{0.241pt}{0.400pt}}
\multiput(691.00,462.17)(0.500,1.000){2}{\rule{0.120pt}{0.400pt}}
\put(649.0,422.0){\usebox{\plotpoint}}
\put(692.67,464){\rule{0.400pt}{0.482pt}}
\multiput(692.17,464.00)(1.000,1.000){2}{\rule{0.400pt}{0.241pt}}
\put(694,465.67){\rule{0.241pt}{0.400pt}}
\multiput(694.00,465.17)(0.500,1.000){2}{\rule{0.120pt}{0.400pt}}
\put(692.0,464.0){\usebox{\plotpoint}}
\put(696,466.67){\rule{0.241pt}{0.400pt}}
\multiput(696.00,466.17)(0.500,1.000){2}{\rule{0.120pt}{0.400pt}}
\put(697,467.67){\rule{0.241pt}{0.400pt}}
\multiput(697.00,467.17)(0.500,1.000){2}{\rule{0.120pt}{0.400pt}}
\put(698,468.67){\rule{0.241pt}{0.400pt}}
\multiput(698.00,468.17)(0.500,1.000){2}{\rule{0.120pt}{0.400pt}}
\put(699,469.67){\rule{0.241pt}{0.400pt}}
\multiput(699.00,469.17)(0.500,1.000){2}{\rule{0.120pt}{0.400pt}}
\put(700,470.67){\rule{0.241pt}{0.400pt}}
\multiput(700.00,470.17)(0.500,1.000){2}{\rule{0.120pt}{0.400pt}}
\put(701,471.67){\rule{0.241pt}{0.400pt}}
\multiput(701.00,471.17)(0.500,1.000){2}{\rule{0.120pt}{0.400pt}}
\put(702,472.67){\rule{0.241pt}{0.400pt}}
\multiput(702.00,472.17)(0.500,1.000){2}{\rule{0.120pt}{0.400pt}}
\put(703,473.67){\rule{0.241pt}{0.400pt}}
\multiput(703.00,473.17)(0.500,1.000){2}{\rule{0.120pt}{0.400pt}}
\put(704,474.67){\rule{0.241pt}{0.400pt}}
\multiput(704.00,474.17)(0.500,1.000){2}{\rule{0.120pt}{0.400pt}}
\put(705,475.67){\rule{0.241pt}{0.400pt}}
\multiput(705.00,475.17)(0.500,1.000){2}{\rule{0.120pt}{0.400pt}}
\put(706,476.67){\rule{0.241pt}{0.400pt}}
\multiput(706.00,476.17)(0.500,1.000){2}{\rule{0.120pt}{0.400pt}}
\put(707,477.67){\rule{0.241pt}{0.400pt}}
\multiput(707.00,477.17)(0.500,1.000){2}{\rule{0.120pt}{0.400pt}}
\put(708,478.67){\rule{0.241pt}{0.400pt}}
\multiput(708.00,478.17)(0.500,1.000){2}{\rule{0.120pt}{0.400pt}}
\put(709,479.67){\rule{0.241pt}{0.400pt}}
\multiput(709.00,479.17)(0.500,1.000){2}{\rule{0.120pt}{0.400pt}}
\put(710,480.67){\rule{0.241pt}{0.400pt}}
\multiput(710.00,480.17)(0.500,1.000){2}{\rule{0.120pt}{0.400pt}}
\put(711,481.67){\rule{0.241pt}{0.400pt}}
\multiput(711.00,481.17)(0.500,1.000){2}{\rule{0.120pt}{0.400pt}}
\put(712,482.67){\rule{0.241pt}{0.400pt}}
\multiput(712.00,482.17)(0.500,1.000){2}{\rule{0.120pt}{0.400pt}}
\put(713,483.67){\rule{0.241pt}{0.400pt}}
\multiput(713.00,483.17)(0.500,1.000){2}{\rule{0.120pt}{0.400pt}}
\put(695.0,467.0){\usebox{\plotpoint}}
\put(715,484.67){\rule{0.241pt}{0.400pt}}
\multiput(715.00,484.17)(0.500,1.000){2}{\rule{0.120pt}{0.400pt}}
\put(716,485.67){\rule{0.241pt}{0.400pt}}
\multiput(716.00,485.17)(0.500,1.000){2}{\rule{0.120pt}{0.400pt}}
\put(717,486.67){\rule{0.241pt}{0.400pt}}
\multiput(717.00,486.17)(0.500,1.000){2}{\rule{0.120pt}{0.400pt}}
\put(718,487.67){\rule{0.241pt}{0.400pt}}
\multiput(718.00,487.17)(0.500,1.000){2}{\rule{0.120pt}{0.400pt}}
\put(719,488.67){\rule{0.241pt}{0.400pt}}
\multiput(719.00,488.17)(0.500,1.000){2}{\rule{0.120pt}{0.400pt}}
\put(720,489.67){\rule{0.241pt}{0.400pt}}
\multiput(720.00,489.17)(0.500,1.000){2}{\rule{0.120pt}{0.400pt}}
\put(721,490.67){\rule{0.241pt}{0.400pt}}
\multiput(721.00,490.17)(0.500,1.000){2}{\rule{0.120pt}{0.400pt}}
\put(722,491.67){\rule{0.241pt}{0.400pt}}
\multiput(722.00,491.17)(0.500,1.000){2}{\rule{0.120pt}{0.400pt}}
\put(723,492.67){\rule{0.241pt}{0.400pt}}
\multiput(723.00,492.17)(0.500,1.000){2}{\rule{0.120pt}{0.400pt}}
\put(724,493.67){\rule{0.241pt}{0.400pt}}
\multiput(724.00,493.17)(0.500,1.000){2}{\rule{0.120pt}{0.400pt}}
\put(725,494.67){\rule{0.241pt}{0.400pt}}
\multiput(725.00,494.17)(0.500,1.000){2}{\rule{0.120pt}{0.400pt}}
\put(726,495.67){\rule{0.241pt}{0.400pt}}
\multiput(726.00,495.17)(0.500,1.000){2}{\rule{0.120pt}{0.400pt}}
\put(727,496.67){\rule{0.241pt}{0.400pt}}
\multiput(727.00,496.17)(0.500,1.000){2}{\rule{0.120pt}{0.400pt}}
\put(714.0,485.0){\usebox{\plotpoint}}
\put(729,497.67){\rule{0.241pt}{0.400pt}}
\multiput(729.00,497.17)(0.500,1.000){2}{\rule{0.120pt}{0.400pt}}
\put(730,498.67){\rule{0.241pt}{0.400pt}}
\multiput(730.00,498.17)(0.500,1.000){2}{\rule{0.120pt}{0.400pt}}
\put(731,499.67){\rule{0.241pt}{0.400pt}}
\multiput(731.00,499.17)(0.500,1.000){2}{\rule{0.120pt}{0.400pt}}
\put(732,500.67){\rule{0.241pt}{0.400pt}}
\multiput(732.00,500.17)(0.500,1.000){2}{\rule{0.120pt}{0.400pt}}
\put(733,501.67){\rule{0.241pt}{0.400pt}}
\multiput(733.00,501.17)(0.500,1.000){2}{\rule{0.120pt}{0.400pt}}
\put(734,502.67){\rule{0.241pt}{0.400pt}}
\multiput(734.00,502.17)(0.500,1.000){2}{\rule{0.120pt}{0.400pt}}
\put(735,503.67){\rule{0.241pt}{0.400pt}}
\multiput(735.00,503.17)(0.500,1.000){2}{\rule{0.120pt}{0.400pt}}
\put(736,504.67){\rule{0.241pt}{0.400pt}}
\multiput(736.00,504.17)(0.500,1.000){2}{\rule{0.120pt}{0.400pt}}
\put(737,505.67){\rule{0.241pt}{0.400pt}}
\multiput(737.00,505.17)(0.500,1.000){2}{\rule{0.120pt}{0.400pt}}
\put(738,506.67){\rule{0.241pt}{0.400pt}}
\multiput(738.00,506.17)(0.500,1.000){2}{\rule{0.120pt}{0.400pt}}
\put(739,507.67){\rule{0.241pt}{0.400pt}}
\multiput(739.00,507.17)(0.500,1.000){2}{\rule{0.120pt}{0.400pt}}
\put(740,508.67){\rule{0.241pt}{0.400pt}}
\multiput(740.00,508.17)(0.500,1.000){2}{\rule{0.120pt}{0.400pt}}
\put(728.0,498.0){\usebox{\plotpoint}}
\put(742,509.67){\rule{0.241pt}{0.400pt}}
\multiput(742.00,509.17)(0.500,1.000){2}{\rule{0.120pt}{0.400pt}}
\put(743,510.67){\rule{0.241pt}{0.400pt}}
\multiput(743.00,510.17)(0.500,1.000){2}{\rule{0.120pt}{0.400pt}}
\put(744,511.67){\rule{0.241pt}{0.400pt}}
\multiput(744.00,511.17)(0.500,1.000){2}{\rule{0.120pt}{0.400pt}}
\put(745,512.67){\rule{0.241pt}{0.400pt}}
\multiput(745.00,512.17)(0.500,1.000){2}{\rule{0.120pt}{0.400pt}}
\put(746,513.67){\rule{0.241pt}{0.400pt}}
\multiput(746.00,513.17)(0.500,1.000){2}{\rule{0.120pt}{0.400pt}}
\put(747,514.67){\rule{0.241pt}{0.400pt}}
\multiput(747.00,514.17)(0.500,1.000){2}{\rule{0.120pt}{0.400pt}}
\put(748,515.67){\rule{0.241pt}{0.400pt}}
\multiput(748.00,515.17)(0.500,1.000){2}{\rule{0.120pt}{0.400pt}}
\put(749,516.67){\rule{0.241pt}{0.400pt}}
\multiput(749.00,516.17)(0.500,1.000){2}{\rule{0.120pt}{0.400pt}}
\put(741.0,510.0){\usebox{\plotpoint}}
\put(751,517.67){\rule{0.241pt}{0.400pt}}
\multiput(751.00,517.17)(0.500,1.000){2}{\rule{0.120pt}{0.400pt}}
\put(752,518.67){\rule{0.241pt}{0.400pt}}
\multiput(752.00,518.17)(0.500,1.000){2}{\rule{0.120pt}{0.400pt}}
\put(753,519.67){\rule{0.241pt}{0.400pt}}
\multiput(753.00,519.17)(0.500,1.000){2}{\rule{0.120pt}{0.400pt}}
\put(754,520.67){\rule{0.241pt}{0.400pt}}
\multiput(754.00,520.17)(0.500,1.000){2}{\rule{0.120pt}{0.400pt}}
\put(755,521.67){\rule{0.241pt}{0.400pt}}
\multiput(755.00,521.17)(0.500,1.000){2}{\rule{0.120pt}{0.400pt}}
\put(756,522.67){\rule{0.241pt}{0.400pt}}
\multiput(756.00,522.17)(0.500,1.000){2}{\rule{0.120pt}{0.400pt}}
\put(757,523.67){\rule{0.241pt}{0.400pt}}
\multiput(757.00,523.17)(0.500,1.000){2}{\rule{0.120pt}{0.400pt}}
\put(758,524.67){\rule{0.482pt}{0.400pt}}
\multiput(758.00,524.17)(1.000,1.000){2}{\rule{0.241pt}{0.400pt}}
\put(760,525.67){\rule{0.241pt}{0.400pt}}
\multiput(760.00,525.17)(0.500,1.000){2}{\rule{0.120pt}{0.400pt}}
\put(750.0,518.0){\usebox{\plotpoint}}
\put(762,526.67){\rule{0.241pt}{0.400pt}}
\multiput(762.00,526.17)(0.500,1.000){2}{\rule{0.120pt}{0.400pt}}
\put(763,527.67){\rule{0.241pt}{0.400pt}}
\multiput(763.00,527.17)(0.500,1.000){2}{\rule{0.120pt}{0.400pt}}
\put(764,528.67){\rule{0.241pt}{0.400pt}}
\multiput(764.00,528.17)(0.500,1.000){2}{\rule{0.120pt}{0.400pt}}
\put(765,529.67){\rule{0.241pt}{0.400pt}}
\multiput(765.00,529.17)(0.500,1.000){2}{\rule{0.120pt}{0.400pt}}
\put(766,530.67){\rule{0.241pt}{0.400pt}}
\multiput(766.00,530.17)(0.500,1.000){2}{\rule{0.120pt}{0.400pt}}
\put(767,531.67){\rule{0.241pt}{0.400pt}}
\multiput(767.00,531.17)(0.500,1.000){2}{\rule{0.120pt}{0.400pt}}
\put(768,532.67){\rule{0.241pt}{0.400pt}}
\multiput(768.00,532.17)(0.500,1.000){2}{\rule{0.120pt}{0.400pt}}
\put(761.0,527.0){\usebox{\plotpoint}}
\put(770,533.67){\rule{0.241pt}{0.400pt}}
\multiput(770.00,533.17)(0.500,1.000){2}{\rule{0.120pt}{0.400pt}}
\put(771,534.67){\rule{0.241pt}{0.400pt}}
\multiput(771.00,534.17)(0.500,1.000){2}{\rule{0.120pt}{0.400pt}}
\put(772,535.67){\rule{0.241pt}{0.400pt}}
\multiput(772.00,535.17)(0.500,1.000){2}{\rule{0.120pt}{0.400pt}}
\put(773,536.67){\rule{0.241pt}{0.400pt}}
\multiput(773.00,536.17)(0.500,1.000){2}{\rule{0.120pt}{0.400pt}}
\put(774,537.67){\rule{0.241pt}{0.400pt}}
\multiput(774.00,537.17)(0.500,1.000){2}{\rule{0.120pt}{0.400pt}}
\put(775,538.67){\rule{0.241pt}{0.400pt}}
\multiput(775.00,538.17)(0.500,1.000){2}{\rule{0.120pt}{0.400pt}}
\put(776,539.67){\rule{0.241pt}{0.400pt}}
\multiput(776.00,539.17)(0.500,1.000){2}{\rule{0.120pt}{0.400pt}}
\put(769.0,534.0){\usebox{\plotpoint}}
\put(778,540.67){\rule{0.241pt}{0.400pt}}
\multiput(778.00,540.17)(0.500,1.000){2}{\rule{0.120pt}{0.400pt}}
\put(779,541.67){\rule{0.241pt}{0.400pt}}
\multiput(779.00,541.17)(0.500,1.000){2}{\rule{0.120pt}{0.400pt}}
\put(780,542.67){\rule{0.241pt}{0.400pt}}
\multiput(780.00,542.17)(0.500,1.000){2}{\rule{0.120pt}{0.400pt}}
\put(781,543.67){\rule{0.241pt}{0.400pt}}
\multiput(781.00,543.17)(0.500,1.000){2}{\rule{0.120pt}{0.400pt}}
\put(782,544.67){\rule{0.241pt}{0.400pt}}
\multiput(782.00,544.17)(0.500,1.000){2}{\rule{0.120pt}{0.400pt}}
\put(783,545.67){\rule{0.241pt}{0.400pt}}
\multiput(783.00,545.17)(0.500,1.000){2}{\rule{0.120pt}{0.400pt}}
\put(777.0,541.0){\usebox{\plotpoint}}
\put(785,546.67){\rule{0.241pt}{0.400pt}}
\multiput(785.00,546.17)(0.500,1.000){2}{\rule{0.120pt}{0.400pt}}
\put(786,547.67){\rule{0.241pt}{0.400pt}}
\multiput(786.00,547.17)(0.500,1.000){2}{\rule{0.120pt}{0.400pt}}
\put(787,548.67){\rule{0.241pt}{0.400pt}}
\multiput(787.00,548.17)(0.500,1.000){2}{\rule{0.120pt}{0.400pt}}
\put(788,549.67){\rule{0.241pt}{0.400pt}}
\multiput(788.00,549.17)(0.500,1.000){2}{\rule{0.120pt}{0.400pt}}
\put(789,550.67){\rule{0.241pt}{0.400pt}}
\multiput(789.00,550.17)(0.500,1.000){2}{\rule{0.120pt}{0.400pt}}
\put(790,551.67){\rule{0.241pt}{0.400pt}}
\multiput(790.00,551.17)(0.500,1.000){2}{\rule{0.120pt}{0.400pt}}
\put(784.0,547.0){\usebox{\plotpoint}}
\put(792,552.67){\rule{0.241pt}{0.400pt}}
\multiput(792.00,552.17)(0.500,1.000){2}{\rule{0.120pt}{0.400pt}}
\put(793,553.67){\rule{0.241pt}{0.400pt}}
\multiput(793.00,553.17)(0.500,1.000){2}{\rule{0.120pt}{0.400pt}}
\put(794,554.67){\rule{0.241pt}{0.400pt}}
\multiput(794.00,554.17)(0.500,1.000){2}{\rule{0.120pt}{0.400pt}}
\put(795,555.67){\rule{0.241pt}{0.400pt}}
\multiput(795.00,555.17)(0.500,1.000){2}{\rule{0.120pt}{0.400pt}}
\put(796,556.67){\rule{0.241pt}{0.400pt}}
\multiput(796.00,556.17)(0.500,1.000){2}{\rule{0.120pt}{0.400pt}}
\put(791.0,553.0){\usebox{\plotpoint}}
\put(798,557.67){\rule{0.241pt}{0.400pt}}
\multiput(798.00,557.17)(0.500,1.000){2}{\rule{0.120pt}{0.400pt}}
\put(799,558.67){\rule{0.241pt}{0.400pt}}
\multiput(799.00,558.17)(0.500,1.000){2}{\rule{0.120pt}{0.400pt}}
\put(800,559.67){\rule{0.241pt}{0.400pt}}
\multiput(800.00,559.17)(0.500,1.000){2}{\rule{0.120pt}{0.400pt}}
\put(801,560.67){\rule{0.241pt}{0.400pt}}
\multiput(801.00,560.17)(0.500,1.000){2}{\rule{0.120pt}{0.400pt}}
\put(802,561.67){\rule{0.241pt}{0.400pt}}
\multiput(802.00,561.17)(0.500,1.000){2}{\rule{0.120pt}{0.400pt}}
\put(803,562.67){\rule{0.241pt}{0.400pt}}
\multiput(803.00,562.17)(0.500,1.000){2}{\rule{0.120pt}{0.400pt}}
\put(797.0,558.0){\usebox{\plotpoint}}
\put(805,563.67){\rule{0.241pt}{0.400pt}}
\multiput(805.00,563.17)(0.500,1.000){2}{\rule{0.120pt}{0.400pt}}
\put(806,564.67){\rule{0.241pt}{0.400pt}}
\multiput(806.00,564.17)(0.500,1.000){2}{\rule{0.120pt}{0.400pt}}
\put(807,565.67){\rule{0.241pt}{0.400pt}}
\multiput(807.00,565.17)(0.500,1.000){2}{\rule{0.120pt}{0.400pt}}
\put(808,566.67){\rule{0.241pt}{0.400pt}}
\multiput(808.00,566.17)(0.500,1.000){2}{\rule{0.120pt}{0.400pt}}
\put(804.0,564.0){\usebox{\plotpoint}}
\put(810,567.67){\rule{0.241pt}{0.400pt}}
\multiput(810.00,567.17)(0.500,1.000){2}{\rule{0.120pt}{0.400pt}}
\put(811,568.67){\rule{0.241pt}{0.400pt}}
\multiput(811.00,568.17)(0.500,1.000){2}{\rule{0.120pt}{0.400pt}}
\put(812,569.67){\rule{0.241pt}{0.400pt}}
\multiput(812.00,569.17)(0.500,1.000){2}{\rule{0.120pt}{0.400pt}}
\put(813,570.67){\rule{0.241pt}{0.400pt}}
\multiput(813.00,570.17)(0.500,1.000){2}{\rule{0.120pt}{0.400pt}}
\put(814,571.67){\rule{0.241pt}{0.400pt}}
\multiput(814.00,571.17)(0.500,1.000){2}{\rule{0.120pt}{0.400pt}}
\put(809.0,568.0){\usebox{\plotpoint}}
\put(816,572.67){\rule{0.241pt}{0.400pt}}
\multiput(816.00,572.17)(0.500,1.000){2}{\rule{0.120pt}{0.400pt}}
\put(817,573.67){\rule{0.241pt}{0.400pt}}
\multiput(817.00,573.17)(0.500,1.000){2}{\rule{0.120pt}{0.400pt}}
\put(818,574.67){\rule{0.241pt}{0.400pt}}
\multiput(818.00,574.17)(0.500,1.000){2}{\rule{0.120pt}{0.400pt}}
\put(819,575.67){\rule{0.241pt}{0.400pt}}
\multiput(819.00,575.17)(0.500,1.000){2}{\rule{0.120pt}{0.400pt}}
\put(820,576.67){\rule{0.241pt}{0.400pt}}
\multiput(820.00,576.17)(0.500,1.000){2}{\rule{0.120pt}{0.400pt}}
\put(815.0,573.0){\usebox{\plotpoint}}
\put(822,577.67){\rule{0.241pt}{0.400pt}}
\multiput(822.00,577.17)(0.500,1.000){2}{\rule{0.120pt}{0.400pt}}
\put(823,578.67){\rule{0.241pt}{0.400pt}}
\multiput(823.00,578.17)(0.500,1.000){2}{\rule{0.120pt}{0.400pt}}
\put(824,579.67){\rule{0.241pt}{0.400pt}}
\multiput(824.00,579.17)(0.500,1.000){2}{\rule{0.120pt}{0.400pt}}
\put(825,580.67){\rule{0.241pt}{0.400pt}}
\multiput(825.00,580.17)(0.500,1.000){2}{\rule{0.120pt}{0.400pt}}
\put(821.0,578.0){\usebox{\plotpoint}}
\put(827,581.67){\rule{0.241pt}{0.400pt}}
\multiput(827.00,581.17)(0.500,1.000){2}{\rule{0.120pt}{0.400pt}}
\put(828,582.67){\rule{0.241pt}{0.400pt}}
\multiput(828.00,582.17)(0.500,1.000){2}{\rule{0.120pt}{0.400pt}}
\put(829,583.67){\rule{0.241pt}{0.400pt}}
\multiput(829.00,583.17)(0.500,1.000){2}{\rule{0.120pt}{0.400pt}}
\put(830,584.67){\rule{0.241pt}{0.400pt}}
\multiput(830.00,584.17)(0.500,1.000){2}{\rule{0.120pt}{0.400pt}}
\put(826.0,582.0){\usebox{\plotpoint}}
\put(832,585.67){\rule{0.241pt}{0.400pt}}
\multiput(832.00,585.17)(0.500,1.000){2}{\rule{0.120pt}{0.400pt}}
\put(833,586.67){\rule{0.241pt}{0.400pt}}
\multiput(833.00,586.17)(0.500,1.000){2}{\rule{0.120pt}{0.400pt}}
\put(834,587.67){\rule{0.241pt}{0.400pt}}
\multiput(834.00,587.17)(0.500,1.000){2}{\rule{0.120pt}{0.400pt}}
\put(835,588.67){\rule{0.241pt}{0.400pt}}
\multiput(835.00,588.17)(0.500,1.000){2}{\rule{0.120pt}{0.400pt}}
\put(831.0,586.0){\usebox{\plotpoint}}
\put(837,589.67){\rule{0.241pt}{0.400pt}}
\multiput(837.00,589.17)(0.500,1.000){2}{\rule{0.120pt}{0.400pt}}
\put(838,590.67){\rule{0.241pt}{0.400pt}}
\multiput(838.00,590.17)(0.500,1.000){2}{\rule{0.120pt}{0.400pt}}
\put(839,591.67){\rule{0.241pt}{0.400pt}}
\multiput(839.00,591.17)(0.500,1.000){2}{\rule{0.120pt}{0.400pt}}
\put(840,592.67){\rule{0.241pt}{0.400pt}}
\multiput(840.00,592.17)(0.500,1.000){2}{\rule{0.120pt}{0.400pt}}
\put(841,593.67){\rule{0.241pt}{0.400pt}}
\multiput(841.00,593.17)(0.500,1.000){2}{\rule{0.120pt}{0.400pt}}
\put(836.0,590.0){\usebox{\plotpoint}}
\put(843,594.67){\rule{0.241pt}{0.400pt}}
\multiput(843.00,594.17)(0.500,1.000){2}{\rule{0.120pt}{0.400pt}}
\put(844,595.67){\rule{0.241pt}{0.400pt}}
\multiput(844.00,595.17)(0.500,1.000){2}{\rule{0.120pt}{0.400pt}}
\put(845,596.67){\rule{0.241pt}{0.400pt}}
\multiput(845.00,596.17)(0.500,1.000){2}{\rule{0.120pt}{0.400pt}}
\put(842.0,595.0){\usebox{\plotpoint}}
\put(847,597.67){\rule{0.241pt}{0.400pt}}
\multiput(847.00,597.17)(0.500,1.000){2}{\rule{0.120pt}{0.400pt}}
\put(848,598.67){\rule{0.241pt}{0.400pt}}
\multiput(848.00,598.17)(0.500,1.000){2}{\rule{0.120pt}{0.400pt}}
\put(849,599.67){\rule{0.241pt}{0.400pt}}
\multiput(849.00,599.17)(0.500,1.000){2}{\rule{0.120pt}{0.400pt}}
\put(850,600.67){\rule{0.241pt}{0.400pt}}
\multiput(850.00,600.17)(0.500,1.000){2}{\rule{0.120pt}{0.400pt}}
\put(846.0,598.0){\usebox{\plotpoint}}
\put(852,601.67){\rule{0.241pt}{0.400pt}}
\multiput(852.00,601.17)(0.500,1.000){2}{\rule{0.120pt}{0.400pt}}
\put(853,602.67){\rule{0.241pt}{0.400pt}}
\multiput(853.00,602.17)(0.500,1.000){2}{\rule{0.120pt}{0.400pt}}
\put(854,603.67){\rule{0.241pt}{0.400pt}}
\multiput(854.00,603.17)(0.500,1.000){2}{\rule{0.120pt}{0.400pt}}
\put(851.0,602.0){\usebox{\plotpoint}}
\put(856,604.67){\rule{0.241pt}{0.400pt}}
\multiput(856.00,604.17)(0.500,1.000){2}{\rule{0.120pt}{0.400pt}}
\put(857,605.67){\rule{0.241pt}{0.400pt}}
\multiput(857.00,605.17)(0.500,1.000){2}{\rule{0.120pt}{0.400pt}}
\put(858,606.67){\rule{0.241pt}{0.400pt}}
\multiput(858.00,606.17)(0.500,1.000){2}{\rule{0.120pt}{0.400pt}}
\put(859,607.67){\rule{0.241pt}{0.400pt}}
\multiput(859.00,607.17)(0.500,1.000){2}{\rule{0.120pt}{0.400pt}}
\put(855.0,605.0){\usebox{\plotpoint}}
\put(861,608.67){\rule{0.241pt}{0.400pt}}
\multiput(861.00,608.17)(0.500,1.000){2}{\rule{0.120pt}{0.400pt}}
\put(862,609.67){\rule{0.241pt}{0.400pt}}
\multiput(862.00,609.17)(0.500,1.000){2}{\rule{0.120pt}{0.400pt}}
\put(860.0,609.0){\usebox{\plotpoint}}
\put(864,610.67){\rule{0.241pt}{0.400pt}}
\multiput(864.00,610.17)(0.500,1.000){2}{\rule{0.120pt}{0.400pt}}
\put(865,611.67){\rule{0.241pt}{0.400pt}}
\multiput(865.00,611.17)(0.500,1.000){2}{\rule{0.120pt}{0.400pt}}
\put(866,612.67){\rule{0.241pt}{0.400pt}}
\multiput(866.00,612.17)(0.500,1.000){2}{\rule{0.120pt}{0.400pt}}
\put(867,613.67){\rule{0.241pt}{0.400pt}}
\multiput(867.00,613.17)(0.500,1.000){2}{\rule{0.120pt}{0.400pt}}
\put(863.0,611.0){\usebox{\plotpoint}}
\put(869,614.67){\rule{0.241pt}{0.400pt}}
\multiput(869.00,614.17)(0.500,1.000){2}{\rule{0.120pt}{0.400pt}}
\put(870,615.67){\rule{0.241pt}{0.400pt}}
\multiput(870.00,615.17)(0.500,1.000){2}{\rule{0.120pt}{0.400pt}}
\put(871,616.67){\rule{0.241pt}{0.400pt}}
\multiput(871.00,616.17)(0.500,1.000){2}{\rule{0.120pt}{0.400pt}}
\put(868.0,615.0){\usebox{\plotpoint}}
\put(873,617.67){\rule{0.241pt}{0.400pt}}
\multiput(873.00,617.17)(0.500,1.000){2}{\rule{0.120pt}{0.400pt}}
\put(874,618.67){\rule{0.241pt}{0.400pt}}
\multiput(874.00,618.17)(0.500,1.000){2}{\rule{0.120pt}{0.400pt}}
\put(875,619.67){\rule{0.241pt}{0.400pt}}
\multiput(875.00,619.17)(0.500,1.000){2}{\rule{0.120pt}{0.400pt}}
\put(872.0,618.0){\usebox{\plotpoint}}
\put(877,620.67){\rule{0.241pt}{0.400pt}}
\multiput(877.00,620.17)(0.500,1.000){2}{\rule{0.120pt}{0.400pt}}
\put(878,621.67){\rule{0.241pt}{0.400pt}}
\multiput(878.00,621.17)(0.500,1.000){2}{\rule{0.120pt}{0.400pt}}
\put(879,622.67){\rule{0.241pt}{0.400pt}}
\multiput(879.00,622.17)(0.500,1.000){2}{\rule{0.120pt}{0.400pt}}
\put(876.0,621.0){\usebox{\plotpoint}}
\put(881,623.67){\rule{0.241pt}{0.400pt}}
\multiput(881.00,623.17)(0.500,1.000){2}{\rule{0.120pt}{0.400pt}}
\put(882,624.67){\rule{0.241pt}{0.400pt}}
\multiput(882.00,624.17)(0.500,1.000){2}{\rule{0.120pt}{0.400pt}}
\put(883,625.67){\rule{0.241pt}{0.400pt}}
\multiput(883.00,625.17)(0.500,1.000){2}{\rule{0.120pt}{0.400pt}}
\put(880.0,624.0){\usebox{\plotpoint}}
\put(885,626.67){\rule{0.241pt}{0.400pt}}
\multiput(885.00,626.17)(0.500,1.000){2}{\rule{0.120pt}{0.400pt}}
\put(886,627.67){\rule{0.241pt}{0.400pt}}
\multiput(886.00,627.17)(0.500,1.000){2}{\rule{0.120pt}{0.400pt}}
\put(884.0,627.0){\usebox{\plotpoint}}
\put(888,628.67){\rule{0.241pt}{0.400pt}}
\multiput(888.00,628.17)(0.500,1.000){2}{\rule{0.120pt}{0.400pt}}
\put(889,629.67){\rule{0.241pt}{0.400pt}}
\multiput(889.00,629.17)(0.500,1.000){2}{\rule{0.120pt}{0.400pt}}
\put(890,630.67){\rule{0.241pt}{0.400pt}}
\multiput(890.00,630.17)(0.500,1.000){2}{\rule{0.120pt}{0.400pt}}
\put(891,631.67){\rule{0.241pt}{0.400pt}}
\multiput(891.00,631.17)(0.500,1.000){2}{\rule{0.120pt}{0.400pt}}
\put(887.0,629.0){\usebox{\plotpoint}}
\put(893,632.67){\rule{0.241pt}{0.400pt}}
\multiput(893.00,632.17)(0.500,1.000){2}{\rule{0.120pt}{0.400pt}}
\put(894,633.67){\rule{0.241pt}{0.400pt}}
\multiput(894.00,633.17)(0.500,1.000){2}{\rule{0.120pt}{0.400pt}}
\put(892.0,633.0){\usebox{\plotpoint}}
\put(896,634.67){\rule{0.241pt}{0.400pt}}
\multiput(896.00,634.17)(0.500,1.000){2}{\rule{0.120pt}{0.400pt}}
\put(897,635.67){\rule{0.241pt}{0.400pt}}
\multiput(897.00,635.17)(0.500,1.000){2}{\rule{0.120pt}{0.400pt}}
\put(898,636.67){\rule{0.241pt}{0.400pt}}
\multiput(898.00,636.17)(0.500,1.000){2}{\rule{0.120pt}{0.400pt}}
\put(895.0,635.0){\usebox{\plotpoint}}
\put(900,637.67){\rule{0.241pt}{0.400pt}}
\multiput(900.00,637.17)(0.500,1.000){2}{\rule{0.120pt}{0.400pt}}
\put(901,638.67){\rule{0.241pt}{0.400pt}}
\multiput(901.00,638.17)(0.500,1.000){2}{\rule{0.120pt}{0.400pt}}
\put(899.0,638.0){\usebox{\plotpoint}}
\put(903,639.67){\rule{0.241pt}{0.400pt}}
\multiput(903.00,639.17)(0.500,1.000){2}{\rule{0.120pt}{0.400pt}}
\put(904,640.67){\rule{0.241pt}{0.400pt}}
\multiput(904.00,640.17)(0.500,1.000){2}{\rule{0.120pt}{0.400pt}}
\put(905,641.67){\rule{0.241pt}{0.400pt}}
\multiput(905.00,641.17)(0.500,1.000){2}{\rule{0.120pt}{0.400pt}}
\put(902.0,640.0){\usebox{\plotpoint}}
\put(907,642.67){\rule{0.241pt}{0.400pt}}
\multiput(907.00,642.17)(0.500,1.000){2}{\rule{0.120pt}{0.400pt}}
\put(908,643.67){\rule{0.241pt}{0.400pt}}
\multiput(908.00,643.17)(0.500,1.000){2}{\rule{0.120pt}{0.400pt}}
\put(906.0,643.0){\usebox{\plotpoint}}
\put(910,644.67){\rule{0.482pt}{0.400pt}}
\multiput(910.00,644.17)(1.000,1.000){2}{\rule{0.241pt}{0.400pt}}
\put(912,645.67){\rule{0.241pt}{0.400pt}}
\multiput(912.00,645.17)(0.500,1.000){2}{\rule{0.120pt}{0.400pt}}
\put(913,646.67){\rule{0.241pt}{0.400pt}}
\multiput(913.00,646.17)(0.500,1.000){2}{\rule{0.120pt}{0.400pt}}
\put(909.0,645.0){\usebox{\plotpoint}}
\put(915,647.67){\rule{0.241pt}{0.400pt}}
\multiput(915.00,647.17)(0.500,1.000){2}{\rule{0.120pt}{0.400pt}}
\put(916,648.67){\rule{0.241pt}{0.400pt}}
\multiput(916.00,648.17)(0.500,1.000){2}{\rule{0.120pt}{0.400pt}}
\put(914.0,648.0){\usebox{\plotpoint}}
\put(918,649.67){\rule{0.241pt}{0.400pt}}
\multiput(918.00,649.17)(0.500,1.000){2}{\rule{0.120pt}{0.400pt}}
\put(919,650.67){\rule{0.241pt}{0.400pt}}
\multiput(919.00,650.17)(0.500,1.000){2}{\rule{0.120pt}{0.400pt}}
\put(920,651.67){\rule{0.241pt}{0.400pt}}
\multiput(920.00,651.17)(0.500,1.000){2}{\rule{0.120pt}{0.400pt}}
\put(917.0,650.0){\usebox{\plotpoint}}
\put(922,652.67){\rule{0.241pt}{0.400pt}}
\multiput(922.00,652.17)(0.500,1.000){2}{\rule{0.120pt}{0.400pt}}
\put(921.0,653.0){\usebox{\plotpoint}}
\put(924,653.67){\rule{0.241pt}{0.400pt}}
\multiput(924.00,653.17)(0.500,1.000){2}{\rule{0.120pt}{0.400pt}}
\put(925,654.67){\rule{0.241pt}{0.400pt}}
\multiput(925.00,654.17)(0.500,1.000){2}{\rule{0.120pt}{0.400pt}}
\put(926,655.67){\rule{0.241pt}{0.400pt}}
\multiput(926.00,655.17)(0.500,1.000){2}{\rule{0.120pt}{0.400pt}}
\put(923.0,654.0){\usebox{\plotpoint}}
\put(928,656.67){\rule{0.241pt}{0.400pt}}
\multiput(928.00,656.17)(0.500,1.000){2}{\rule{0.120pt}{0.400pt}}
\put(929,657.67){\rule{0.241pt}{0.400pt}}
\multiput(929.00,657.17)(0.500,1.000){2}{\rule{0.120pt}{0.400pt}}
\put(927.0,657.0){\usebox{\plotpoint}}
\put(931,658.67){\rule{0.241pt}{0.400pt}}
\multiput(931.00,658.17)(0.500,1.000){2}{\rule{0.120pt}{0.400pt}}
\put(932,659.67){\rule{0.241pt}{0.400pt}}
\multiput(932.00,659.17)(0.500,1.000){2}{\rule{0.120pt}{0.400pt}}
\put(930.0,659.0){\usebox{\plotpoint}}
\put(934,660.67){\rule{0.241pt}{0.400pt}}
\multiput(934.00,660.17)(0.500,1.000){2}{\rule{0.120pt}{0.400pt}}
\put(935,661.67){\rule{0.241pt}{0.400pt}}
\multiput(935.00,661.17)(0.500,1.000){2}{\rule{0.120pt}{0.400pt}}
\put(933.0,661.0){\usebox{\plotpoint}}
\put(937,662.67){\rule{0.241pt}{0.400pt}}
\multiput(937.00,662.17)(0.500,1.000){2}{\rule{0.120pt}{0.400pt}}
\put(938,663.67){\rule{0.241pt}{0.400pt}}
\multiput(938.00,663.17)(0.500,1.000){2}{\rule{0.120pt}{0.400pt}}
\put(936.0,663.0){\usebox{\plotpoint}}
\put(940,664.67){\rule{0.241pt}{0.400pt}}
\multiput(940.00,664.17)(0.500,1.000){2}{\rule{0.120pt}{0.400pt}}
\put(941,665.67){\rule{0.241pt}{0.400pt}}
\multiput(941.00,665.17)(0.500,1.000){2}{\rule{0.120pt}{0.400pt}}
\put(939.0,665.0){\usebox{\plotpoint}}
\put(943,666.67){\rule{0.241pt}{0.400pt}}
\multiput(943.00,666.17)(0.500,1.000){2}{\rule{0.120pt}{0.400pt}}
\put(944,667.67){\rule{0.241pt}{0.400pt}}
\multiput(944.00,667.17)(0.500,1.000){2}{\rule{0.120pt}{0.400pt}}
\put(942.0,667.0){\usebox{\plotpoint}}
\put(946,668.67){\rule{0.241pt}{0.400pt}}
\multiput(946.00,668.17)(0.500,1.000){2}{\rule{0.120pt}{0.400pt}}
\put(947,669.67){\rule{0.241pt}{0.400pt}}
\multiput(947.00,669.17)(0.500,1.000){2}{\rule{0.120pt}{0.400pt}}
\put(945.0,669.0){\usebox{\plotpoint}}
\put(949,670.67){\rule{0.241pt}{0.400pt}}
\multiput(949.00,670.17)(0.500,1.000){2}{\rule{0.120pt}{0.400pt}}
\put(950,671.67){\rule{0.241pt}{0.400pt}}
\multiput(950.00,671.17)(0.500,1.000){2}{\rule{0.120pt}{0.400pt}}
\put(948.0,671.0){\usebox{\plotpoint}}
\put(952,672.67){\rule{0.241pt}{0.400pt}}
\multiput(952.00,672.17)(0.500,1.000){2}{\rule{0.120pt}{0.400pt}}
\put(951.0,673.0){\usebox{\plotpoint}}
\put(954,673.67){\rule{0.241pt}{0.400pt}}
\multiput(954.00,673.17)(0.500,1.000){2}{\rule{0.120pt}{0.400pt}}
\put(955,674.67){\rule{0.241pt}{0.400pt}}
\multiput(955.00,674.17)(0.500,1.000){2}{\rule{0.120pt}{0.400pt}}
\put(953.0,674.0){\usebox{\plotpoint}}
\put(957,675.67){\rule{0.241pt}{0.400pt}}
\multiput(957.00,675.17)(0.500,1.000){2}{\rule{0.120pt}{0.400pt}}
\put(958,676.67){\rule{0.241pt}{0.400pt}}
\multiput(958.00,676.17)(0.500,1.000){2}{\rule{0.120pt}{0.400pt}}
\put(956.0,676.0){\usebox{\plotpoint}}
\put(960,677.67){\rule{0.241pt}{0.400pt}}
\multiput(960.00,677.17)(0.500,1.000){2}{\rule{0.120pt}{0.400pt}}
\put(961,678.67){\rule{0.241pt}{0.400pt}}
\multiput(961.00,678.17)(0.500,1.000){2}{\rule{0.120pt}{0.400pt}}
\put(959.0,678.0){\usebox{\plotpoint}}
\put(963,679.67){\rule{0.241pt}{0.400pt}}
\multiput(963.00,679.17)(0.500,1.000){2}{\rule{0.120pt}{0.400pt}}
\put(964,680.67){\rule{0.241pt}{0.400pt}}
\multiput(964.00,680.17)(0.500,1.000){2}{\rule{0.120pt}{0.400pt}}
\put(962.0,680.0){\usebox{\plotpoint}}
\put(966,681.67){\rule{0.241pt}{0.400pt}}
\multiput(966.00,681.17)(0.500,1.000){2}{\rule{0.120pt}{0.400pt}}
\put(965.0,682.0){\usebox{\plotpoint}}
\put(968,682.67){\rule{0.241pt}{0.400pt}}
\multiput(968.00,682.17)(0.500,1.000){2}{\rule{0.120pt}{0.400pt}}
\put(969,683.67){\rule{0.241pt}{0.400pt}}
\multiput(969.00,683.17)(0.500,1.000){2}{\rule{0.120pt}{0.400pt}}
\put(967.0,683.0){\usebox{\plotpoint}}
\put(971,684.67){\rule{0.241pt}{0.400pt}}
\multiput(971.00,684.17)(0.500,1.000){2}{\rule{0.120pt}{0.400pt}}
\put(972,685.67){\rule{0.241pt}{0.400pt}}
\multiput(972.00,685.17)(0.500,1.000){2}{\rule{0.120pt}{0.400pt}}
\put(970.0,685.0){\usebox{\plotpoint}}
\put(974,686.67){\rule{0.241pt}{0.400pt}}
\multiput(974.00,686.17)(0.500,1.000){2}{\rule{0.120pt}{0.400pt}}
\put(973.0,687.0){\usebox{\plotpoint}}
\put(976,687.67){\rule{0.241pt}{0.400pt}}
\multiput(976.00,687.17)(0.500,1.000){2}{\rule{0.120pt}{0.400pt}}
\put(977,688.67){\rule{0.241pt}{0.400pt}}
\multiput(977.00,688.17)(0.500,1.000){2}{\rule{0.120pt}{0.400pt}}
\put(975.0,688.0){\usebox{\plotpoint}}
\put(979,689.67){\rule{0.241pt}{0.400pt}}
\multiput(979.00,689.17)(0.500,1.000){2}{\rule{0.120pt}{0.400pt}}
\put(980,690.67){\rule{0.241pt}{0.400pt}}
\multiput(980.00,690.17)(0.500,1.000){2}{\rule{0.120pt}{0.400pt}}
\put(978.0,690.0){\usebox{\plotpoint}}
\put(982,691.67){\rule{0.241pt}{0.400pt}}
\multiput(982.00,691.17)(0.500,1.000){2}{\rule{0.120pt}{0.400pt}}
\put(981.0,692.0){\usebox{\plotpoint}}
\put(984,692.67){\rule{0.241pt}{0.400pt}}
\multiput(984.00,692.17)(0.500,1.000){2}{\rule{0.120pt}{0.400pt}}
\put(985,693.67){\rule{0.241pt}{0.400pt}}
\multiput(985.00,693.17)(0.500,1.000){2}{\rule{0.120pt}{0.400pt}}
\put(983.0,693.0){\usebox{\plotpoint}}
\put(987,694.67){\rule{0.241pt}{0.400pt}}
\multiput(987.00,694.17)(0.500,1.000){2}{\rule{0.120pt}{0.400pt}}
\put(986.0,695.0){\usebox{\plotpoint}}
\put(989,695.67){\rule{0.241pt}{0.400pt}}
\multiput(989.00,695.17)(0.500,1.000){2}{\rule{0.120pt}{0.400pt}}
\put(990,696.67){\rule{0.241pt}{0.400pt}}
\multiput(990.00,696.17)(0.500,1.000){2}{\rule{0.120pt}{0.400pt}}
\put(988.0,696.0){\usebox{\plotpoint}}
\put(992,697.67){\rule{0.241pt}{0.400pt}}
\multiput(992.00,697.17)(0.500,1.000){2}{\rule{0.120pt}{0.400pt}}
\put(991.0,698.0){\usebox{\plotpoint}}
\put(994,698.67){\rule{0.241pt}{0.400pt}}
\multiput(994.00,698.17)(0.500,1.000){2}{\rule{0.120pt}{0.400pt}}
\put(995,699.67){\rule{0.241pt}{0.400pt}}
\multiput(995.00,699.17)(0.500,1.000){2}{\rule{0.120pt}{0.400pt}}
\put(993.0,699.0){\usebox{\plotpoint}}
\put(997,700.67){\rule{0.241pt}{0.400pt}}
\multiput(997.00,700.17)(0.500,1.000){2}{\rule{0.120pt}{0.400pt}}
\put(996.0,701.0){\usebox{\plotpoint}}
\put(999,701.67){\rule{0.241pt}{0.400pt}}
\multiput(999.00,701.17)(0.500,1.000){2}{\rule{0.120pt}{0.400pt}}
\put(1000,702.67){\rule{0.241pt}{0.400pt}}
\multiput(1000.00,702.17)(0.500,1.000){2}{\rule{0.120pt}{0.400pt}}
\put(998.0,702.0){\usebox{\plotpoint}}
\put(1002,703.67){\rule{0.241pt}{0.400pt}}
\multiput(1002.00,703.17)(0.500,1.000){2}{\rule{0.120pt}{0.400pt}}
\put(1001.0,704.0){\usebox{\plotpoint}}
\put(1004,704.67){\rule{0.241pt}{0.400pt}}
\multiput(1004.00,704.17)(0.500,1.000){2}{\rule{0.120pt}{0.400pt}}
\put(1003.0,705.0){\usebox{\plotpoint}}
\put(1006,705.67){\rule{0.241pt}{0.400pt}}
\multiput(1006.00,705.17)(0.500,1.000){2}{\rule{0.120pt}{0.400pt}}
\put(1007,706.67){\rule{0.241pt}{0.400pt}}
\multiput(1007.00,706.17)(0.500,1.000){2}{\rule{0.120pt}{0.400pt}}
\put(1005.0,706.0){\usebox{\plotpoint}}
\put(1009,707.67){\rule{0.241pt}{0.400pt}}
\multiput(1009.00,707.17)(0.500,1.000){2}{\rule{0.120pt}{0.400pt}}
\put(1008.0,708.0){\usebox{\plotpoint}}
\put(1011,708.67){\rule{0.241pt}{0.400pt}}
\multiput(1011.00,708.17)(0.500,1.000){2}{\rule{0.120pt}{0.400pt}}
\put(1010.0,709.0){\usebox{\plotpoint}}
\put(1013,709.67){\rule{0.241pt}{0.400pt}}
\multiput(1013.00,709.17)(0.500,1.000){2}{\rule{0.120pt}{0.400pt}}
\put(1012.0,710.0){\usebox{\plotpoint}}
\put(1015,710.67){\rule{0.241pt}{0.400pt}}
\multiput(1015.00,710.17)(0.500,1.000){2}{\rule{0.120pt}{0.400pt}}
\put(1016,711.67){\rule{0.241pt}{0.400pt}}
\multiput(1016.00,711.17)(0.500,1.000){2}{\rule{0.120pt}{0.400pt}}
\put(1014.0,711.0){\usebox{\plotpoint}}
\put(1018,712.67){\rule{0.241pt}{0.400pt}}
\multiput(1018.00,712.17)(0.500,1.000){2}{\rule{0.120pt}{0.400pt}}
\put(1017.0,713.0){\usebox{\plotpoint}}
\put(1020,713.67){\rule{0.241pt}{0.400pt}}
\multiput(1020.00,713.17)(0.500,1.000){2}{\rule{0.120pt}{0.400pt}}
\put(1021,714.67){\rule{0.241pt}{0.400pt}}
\multiput(1021.00,714.17)(0.500,1.000){2}{\rule{0.120pt}{0.400pt}}
\put(1019.0,714.0){\usebox{\plotpoint}}
\put(1023,715.67){\rule{0.241pt}{0.400pt}}
\multiput(1023.00,715.17)(0.500,1.000){2}{\rule{0.120pt}{0.400pt}}
\put(1022.0,716.0){\usebox{\plotpoint}}
\put(1025,716.67){\rule{0.241pt}{0.400pt}}
\multiput(1025.00,716.17)(0.500,1.000){2}{\rule{0.120pt}{0.400pt}}
\put(1024.0,717.0){\usebox{\plotpoint}}
\put(1027,717.67){\rule{0.241pt}{0.400pt}}
\multiput(1027.00,717.17)(0.500,1.000){2}{\rule{0.120pt}{0.400pt}}
\put(1026.0,718.0){\usebox{\plotpoint}}
\put(1029,718.67){\rule{0.241pt}{0.400pt}}
\multiput(1029.00,718.17)(0.500,1.000){2}{\rule{0.120pt}{0.400pt}}
\put(1028.0,719.0){\usebox{\plotpoint}}
\put(1031,719.67){\rule{0.241pt}{0.400pt}}
\multiput(1031.00,719.17)(0.500,1.000){2}{\rule{0.120pt}{0.400pt}}
\put(1032,720.67){\rule{0.241pt}{0.400pt}}
\multiput(1032.00,720.17)(0.500,1.000){2}{\rule{0.120pt}{0.400pt}}
\put(1030.0,720.0){\usebox{\plotpoint}}
\put(1034,721.67){\rule{0.241pt}{0.400pt}}
\multiput(1034.00,721.17)(0.500,1.000){2}{\rule{0.120pt}{0.400pt}}
\put(1033.0,722.0){\usebox{\plotpoint}}
\put(1036,722.67){\rule{0.241pt}{0.400pt}}
\multiput(1036.00,722.17)(0.500,1.000){2}{\rule{0.120pt}{0.400pt}}
\put(1035.0,723.0){\usebox{\plotpoint}}
\put(1038,723.67){\rule{0.241pt}{0.400pt}}
\multiput(1038.00,723.17)(0.500,1.000){2}{\rule{0.120pt}{0.400pt}}
\put(1037.0,724.0){\usebox{\plotpoint}}
\put(1040,724.67){\rule{0.241pt}{0.400pt}}
\multiput(1040.00,724.17)(0.500,1.000){2}{\rule{0.120pt}{0.400pt}}
\put(1039.0,725.0){\usebox{\plotpoint}}
\put(1042,725.67){\rule{0.241pt}{0.400pt}}
\multiput(1042.00,725.17)(0.500,1.000){2}{\rule{0.120pt}{0.400pt}}
\put(1041.0,726.0){\usebox{\plotpoint}}
\put(1044,726.67){\rule{0.241pt}{0.400pt}}
\multiput(1044.00,726.17)(0.500,1.000){2}{\rule{0.120pt}{0.400pt}}
\put(1043.0,727.0){\usebox{\plotpoint}}
\put(1046,727.67){\rule{0.241pt}{0.400pt}}
\multiput(1046.00,727.17)(0.500,1.000){2}{\rule{0.120pt}{0.400pt}}
\put(1045.0,728.0){\usebox{\plotpoint}}
\put(1048,728.67){\rule{0.241pt}{0.400pt}}
\multiput(1048.00,728.17)(0.500,1.000){2}{\rule{0.120pt}{0.400pt}}
\put(1047.0,729.0){\usebox{\plotpoint}}
\put(1050,729.67){\rule{0.241pt}{0.400pt}}
\multiput(1050.00,729.17)(0.500,1.000){2}{\rule{0.120pt}{0.400pt}}
\put(1049.0,730.0){\usebox{\plotpoint}}
\put(1052,730.67){\rule{0.241pt}{0.400pt}}
\multiput(1052.00,730.17)(0.500,1.000){2}{\rule{0.120pt}{0.400pt}}
\put(1053,731.67){\rule{0.241pt}{0.400pt}}
\multiput(1053.00,731.17)(0.500,1.000){2}{\rule{0.120pt}{0.400pt}}
\put(1051.0,731.0){\usebox{\plotpoint}}
\put(1055,732.67){\rule{0.241pt}{0.400pt}}
\multiput(1055.00,732.17)(0.500,1.000){2}{\rule{0.120pt}{0.400pt}}
\put(1054.0,733.0){\usebox{\plotpoint}}
\put(1057,733.67){\rule{0.241pt}{0.400pt}}
\multiput(1057.00,733.17)(0.500,1.000){2}{\rule{0.120pt}{0.400pt}}
\put(1056.0,734.0){\usebox{\plotpoint}}
\put(1059,734.67){\rule{0.241pt}{0.400pt}}
\multiput(1059.00,734.17)(0.500,1.000){2}{\rule{0.120pt}{0.400pt}}
\put(1058.0,735.0){\usebox{\plotpoint}}
\put(1061,735.67){\rule{0.482pt}{0.400pt}}
\multiput(1061.00,735.17)(1.000,1.000){2}{\rule{0.241pt}{0.400pt}}
\put(1060.0,736.0){\usebox{\plotpoint}}
\put(1064,736.67){\rule{0.241pt}{0.400pt}}
\multiput(1064.00,736.17)(0.500,1.000){2}{\rule{0.120pt}{0.400pt}}
\put(1063.0,737.0){\usebox{\plotpoint}}
\put(1066,737.67){\rule{0.241pt}{0.400pt}}
\multiput(1066.00,737.17)(0.500,1.000){2}{\rule{0.120pt}{0.400pt}}
\put(1065.0,738.0){\usebox{\plotpoint}}
\put(1068,738.67){\rule{0.241pt}{0.400pt}}
\multiput(1068.00,738.17)(0.500,1.000){2}{\rule{0.120pt}{0.400pt}}
\put(1067.0,739.0){\usebox{\plotpoint}}
\put(1070,739.67){\rule{0.241pt}{0.400pt}}
\multiput(1070.00,739.17)(0.500,1.000){2}{\rule{0.120pt}{0.400pt}}
\put(1069.0,740.0){\usebox{\plotpoint}}
\put(1073,740.67){\rule{0.241pt}{0.400pt}}
\multiput(1073.00,740.17)(0.500,1.000){2}{\rule{0.120pt}{0.400pt}}
\put(1071.0,741.0){\rule[-0.200pt]{0.482pt}{0.400pt}}
\put(1075,741.67){\rule{0.241pt}{0.400pt}}
\multiput(1075.00,741.17)(0.500,1.000){2}{\rule{0.120pt}{0.400pt}}
\put(1074.0,742.0){\usebox{\plotpoint}}
\put(1077,742.67){\rule{0.241pt}{0.400pt}}
\multiput(1077.00,742.17)(0.500,1.000){2}{\rule{0.120pt}{0.400pt}}
\put(1076.0,743.0){\usebox{\plotpoint}}
\put(1079,743.67){\rule{0.241pt}{0.400pt}}
\multiput(1079.00,743.17)(0.500,1.000){2}{\rule{0.120pt}{0.400pt}}
\put(1078.0,744.0){\usebox{\plotpoint}}
\put(1081,744.67){\rule{0.241pt}{0.400pt}}
\multiput(1081.00,744.17)(0.500,1.000){2}{\rule{0.120pt}{0.400pt}}
\put(1080.0,745.0){\usebox{\plotpoint}}
\put(1083,745.67){\rule{0.241pt}{0.400pt}}
\multiput(1083.00,745.17)(0.500,1.000){2}{\rule{0.120pt}{0.400pt}}
\put(1082.0,746.0){\usebox{\plotpoint}}
\put(1085,746.67){\rule{0.241pt}{0.400pt}}
\multiput(1085.00,746.17)(0.500,1.000){2}{\rule{0.120pt}{0.400pt}}
\put(1084.0,747.0){\usebox{\plotpoint}}
\put(1087,747.67){\rule{0.241pt}{0.400pt}}
\multiput(1087.00,747.17)(0.500,1.000){2}{\rule{0.120pt}{0.400pt}}
\put(1086.0,748.0){\usebox{\plotpoint}}
\put(1090,748.67){\rule{0.241pt}{0.400pt}}
\multiput(1090.00,748.17)(0.500,1.000){2}{\rule{0.120pt}{0.400pt}}
\put(1088.0,749.0){\rule[-0.200pt]{0.482pt}{0.400pt}}
\put(1092,749.67){\rule{0.241pt}{0.400pt}}
\multiput(1092.00,749.17)(0.500,1.000){2}{\rule{0.120pt}{0.400pt}}
\put(1091.0,750.0){\usebox{\plotpoint}}
\put(1094,750.67){\rule{0.241pt}{0.400pt}}
\multiput(1094.00,750.17)(0.500,1.000){2}{\rule{0.120pt}{0.400pt}}
\put(1093.0,751.0){\usebox{\plotpoint}}
\put(1096,751.67){\rule{0.241pt}{0.400pt}}
\multiput(1096.00,751.17)(0.500,1.000){2}{\rule{0.120pt}{0.400pt}}
\put(1095.0,752.0){\usebox{\plotpoint}}
\put(1098,752.67){\rule{0.241pt}{0.400pt}}
\multiput(1098.00,752.17)(0.500,1.000){2}{\rule{0.120pt}{0.400pt}}
\put(1097.0,753.0){\usebox{\plotpoint}}
\put(1100,753.67){\rule{0.241pt}{0.400pt}}
\multiput(1100.00,753.17)(0.500,1.000){2}{\rule{0.120pt}{0.400pt}}
\put(1099.0,754.0){\usebox{\plotpoint}}
\put(1103,754.67){\rule{0.241pt}{0.400pt}}
\multiput(1103.00,754.17)(0.500,1.000){2}{\rule{0.120pt}{0.400pt}}
\put(1101.0,755.0){\rule[-0.200pt]{0.482pt}{0.400pt}}
\put(1105,755.67){\rule{0.241pt}{0.400pt}}
\multiput(1105.00,755.17)(0.500,1.000){2}{\rule{0.120pt}{0.400pt}}
\put(1104.0,756.0){\usebox{\plotpoint}}
\put(1107,756.67){\rule{0.241pt}{0.400pt}}
\multiput(1107.00,756.17)(0.500,1.000){2}{\rule{0.120pt}{0.400pt}}
\put(1106.0,757.0){\usebox{\plotpoint}}
\put(1110,757.67){\rule{0.241pt}{0.400pt}}
\multiput(1110.00,757.17)(0.500,1.000){2}{\rule{0.120pt}{0.400pt}}
\put(1108.0,758.0){\rule[-0.200pt]{0.482pt}{0.400pt}}
\put(1112,758.67){\rule{0.241pt}{0.400pt}}
\multiput(1112.00,758.17)(0.500,1.000){2}{\rule{0.120pt}{0.400pt}}
\put(1111.0,759.0){\usebox{\plotpoint}}
\put(1114,759.67){\rule{0.241pt}{0.400pt}}
\multiput(1114.00,759.17)(0.500,1.000){2}{\rule{0.120pt}{0.400pt}}
\put(1113.0,760.0){\usebox{\plotpoint}}
\put(1116,760.67){\rule{0.241pt}{0.400pt}}
\multiput(1116.00,760.17)(0.500,1.000){2}{\rule{0.120pt}{0.400pt}}
\put(1115.0,761.0){\usebox{\plotpoint}}
\put(1119,761.67){\rule{0.241pt}{0.400pt}}
\multiput(1119.00,761.17)(0.500,1.000){2}{\rule{0.120pt}{0.400pt}}
\put(1117.0,762.0){\rule[-0.200pt]{0.482pt}{0.400pt}}
\put(1122,762.67){\rule{0.241pt}{0.400pt}}
\multiput(1122.00,762.17)(0.500,1.000){2}{\rule{0.120pt}{0.400pt}}
\put(1120.0,763.0){\rule[-0.200pt]{0.482pt}{0.400pt}}
\put(1124,763.67){\rule{0.241pt}{0.400pt}}
\multiput(1124.00,763.17)(0.500,1.000){2}{\rule{0.120pt}{0.400pt}}
\put(1123.0,764.0){\usebox{\plotpoint}}
\put(1126,764.67){\rule{0.241pt}{0.400pt}}
\multiput(1126.00,764.17)(0.500,1.000){2}{\rule{0.120pt}{0.400pt}}
\put(1125.0,765.0){\usebox{\plotpoint}}
\put(1129,765.67){\rule{0.241pt}{0.400pt}}
\multiput(1129.00,765.17)(0.500,1.000){2}{\rule{0.120pt}{0.400pt}}
\put(1127.0,766.0){\rule[-0.200pt]{0.482pt}{0.400pt}}
\put(1131,766.67){\rule{0.241pt}{0.400pt}}
\multiput(1131.00,766.17)(0.500,1.000){2}{\rule{0.120pt}{0.400pt}}
\put(1130.0,767.0){\usebox{\plotpoint}}
\put(1134,767.67){\rule{0.241pt}{0.400pt}}
\multiput(1134.00,767.17)(0.500,1.000){2}{\rule{0.120pt}{0.400pt}}
\put(1132.0,768.0){\rule[-0.200pt]{0.482pt}{0.400pt}}
\put(1136,768.67){\rule{0.241pt}{0.400pt}}
\multiput(1136.00,768.17)(0.500,1.000){2}{\rule{0.120pt}{0.400pt}}
\put(1135.0,769.0){\usebox{\plotpoint}}
\put(1138,769.67){\rule{0.241pt}{0.400pt}}
\multiput(1138.00,769.17)(0.500,1.000){2}{\rule{0.120pt}{0.400pt}}
\put(1137.0,770.0){\usebox{\plotpoint}}
\put(1141,770.67){\rule{0.241pt}{0.400pt}}
\multiput(1141.00,770.17)(0.500,1.000){2}{\rule{0.120pt}{0.400pt}}
\put(1139.0,771.0){\rule[-0.200pt]{0.482pt}{0.400pt}}
\put(1143,771.67){\rule{0.241pt}{0.400pt}}
\multiput(1143.00,771.17)(0.500,1.000){2}{\rule{0.120pt}{0.400pt}}
\put(1142.0,772.0){\usebox{\plotpoint}}
\put(1146,772.67){\rule{0.241pt}{0.400pt}}
\multiput(1146.00,772.17)(0.500,1.000){2}{\rule{0.120pt}{0.400pt}}
\put(1144.0,773.0){\rule[-0.200pt]{0.482pt}{0.400pt}}
\put(1149,773.67){\rule{0.241pt}{0.400pt}}
\multiput(1149.00,773.17)(0.500,1.000){2}{\rule{0.120pt}{0.400pt}}
\put(1147.0,774.0){\rule[-0.200pt]{0.482pt}{0.400pt}}
\put(1151,774.67){\rule{0.241pt}{0.400pt}}
\multiput(1151.00,774.17)(0.500,1.000){2}{\rule{0.120pt}{0.400pt}}
\put(1150.0,775.0){\usebox{\plotpoint}}
\put(1154,775.67){\rule{0.241pt}{0.400pt}}
\multiput(1154.00,775.17)(0.500,1.000){2}{\rule{0.120pt}{0.400pt}}
\put(1152.0,776.0){\rule[-0.200pt]{0.482pt}{0.400pt}}
\put(1157,776.67){\rule{0.241pt}{0.400pt}}
\multiput(1157.00,776.17)(0.500,1.000){2}{\rule{0.120pt}{0.400pt}}
\put(1155.0,777.0){\rule[-0.200pt]{0.482pt}{0.400pt}}
\put(1160,777.67){\rule{0.241pt}{0.400pt}}
\multiput(1160.00,777.17)(0.500,1.000){2}{\rule{0.120pt}{0.400pt}}
\put(1158.0,778.0){\rule[-0.200pt]{0.482pt}{0.400pt}}
\put(1162,778.67){\rule{0.241pt}{0.400pt}}
\multiput(1162.00,778.17)(0.500,1.000){2}{\rule{0.120pt}{0.400pt}}
\put(1161.0,779.0){\usebox{\plotpoint}}
\put(1165,779.67){\rule{0.241pt}{0.400pt}}
\multiput(1165.00,779.17)(0.500,1.000){2}{\rule{0.120pt}{0.400pt}}
\put(1163.0,780.0){\rule[-0.200pt]{0.482pt}{0.400pt}}
\put(1168,780.67){\rule{0.241pt}{0.400pt}}
\multiput(1168.00,780.17)(0.500,1.000){2}{\rule{0.120pt}{0.400pt}}
\put(1166.0,781.0){\rule[-0.200pt]{0.482pt}{0.400pt}}
\put(1171,781.67){\rule{0.241pt}{0.400pt}}
\multiput(1171.00,781.17)(0.500,1.000){2}{\rule{0.120pt}{0.400pt}}
\put(1169.0,782.0){\rule[-0.200pt]{0.482pt}{0.400pt}}
\put(1174,782.67){\rule{0.241pt}{0.400pt}}
\multiput(1174.00,782.17)(0.500,1.000){2}{\rule{0.120pt}{0.400pt}}
\put(1172.0,783.0){\rule[-0.200pt]{0.482pt}{0.400pt}}
\put(1177,783.67){\rule{0.241pt}{0.400pt}}
\multiput(1177.00,783.17)(0.500,1.000){2}{\rule{0.120pt}{0.400pt}}
\put(1175.0,784.0){\rule[-0.200pt]{0.482pt}{0.400pt}}
\put(1180,784.67){\rule{0.241pt}{0.400pt}}
\multiput(1180.00,784.17)(0.500,1.000){2}{\rule{0.120pt}{0.400pt}}
\put(1178.0,785.0){\rule[-0.200pt]{0.482pt}{0.400pt}}
\put(1183,785.67){\rule{0.241pt}{0.400pt}}
\multiput(1183.00,785.17)(0.500,1.000){2}{\rule{0.120pt}{0.400pt}}
\put(1181.0,786.0){\rule[-0.200pt]{0.482pt}{0.400pt}}
\put(1186,786.67){\rule{0.241pt}{0.400pt}}
\multiput(1186.00,786.17)(0.500,1.000){2}{\rule{0.120pt}{0.400pt}}
\put(1184.0,787.0){\rule[-0.200pt]{0.482pt}{0.400pt}}
\put(1189,787.67){\rule{0.241pt}{0.400pt}}
\multiput(1189.00,787.17)(0.500,1.000){2}{\rule{0.120pt}{0.400pt}}
\put(1187.0,788.0){\rule[-0.200pt]{0.482pt}{0.400pt}}
\put(1193,788.67){\rule{0.241pt}{0.400pt}}
\multiput(1193.00,788.17)(0.500,1.000){2}{\rule{0.120pt}{0.400pt}}
\put(1190.0,789.0){\rule[-0.200pt]{0.723pt}{0.400pt}}
\put(1196,789.67){\rule{0.241pt}{0.400pt}}
\multiput(1196.00,789.17)(0.500,1.000){2}{\rule{0.120pt}{0.400pt}}
\put(1194.0,790.0){\rule[-0.200pt]{0.482pt}{0.400pt}}
\put(1199,790.67){\rule{0.241pt}{0.400pt}}
\multiput(1199.00,790.17)(0.500,1.000){2}{\rule{0.120pt}{0.400pt}}
\put(1197.0,791.0){\rule[-0.200pt]{0.482pt}{0.400pt}}
\put(1202,791.67){\rule{0.241pt}{0.400pt}}
\multiput(1202.00,791.17)(0.500,1.000){2}{\rule{0.120pt}{0.400pt}}
\put(1200.0,792.0){\rule[-0.200pt]{0.482pt}{0.400pt}}
\put(1206,792.67){\rule{0.241pt}{0.400pt}}
\multiput(1206.00,792.17)(0.500,1.000){2}{\rule{0.120pt}{0.400pt}}
\put(1203.0,793.0){\rule[-0.200pt]{0.723pt}{0.400pt}}
\put(1209,793.67){\rule{0.241pt}{0.400pt}}
\multiput(1209.00,793.17)(0.500,1.000){2}{\rule{0.120pt}{0.400pt}}
\put(1207.0,794.0){\rule[-0.200pt]{0.482pt}{0.400pt}}
\put(1212,794.67){\rule{0.482pt}{0.400pt}}
\multiput(1212.00,794.17)(1.000,1.000){2}{\rule{0.241pt}{0.400pt}}
\put(1210.0,795.0){\rule[-0.200pt]{0.482pt}{0.400pt}}
\put(1216,795.67){\rule{0.241pt}{0.400pt}}
\multiput(1216.00,795.17)(0.500,1.000){2}{\rule{0.120pt}{0.400pt}}
\put(1214.0,796.0){\rule[-0.200pt]{0.482pt}{0.400pt}}
\put(1220,796.67){\rule{0.241pt}{0.400pt}}
\multiput(1220.00,796.17)(0.500,1.000){2}{\rule{0.120pt}{0.400pt}}
\put(1217.0,797.0){\rule[-0.200pt]{0.723pt}{0.400pt}}
\put(1224,797.67){\rule{0.241pt}{0.400pt}}
\multiput(1224.00,797.17)(0.500,1.000){2}{\rule{0.120pt}{0.400pt}}
\put(1221.0,798.0){\rule[-0.200pt]{0.723pt}{0.400pt}}
\put(1228,798.67){\rule{0.241pt}{0.400pt}}
\multiput(1228.00,798.17)(0.500,1.000){2}{\rule{0.120pt}{0.400pt}}
\put(1225.0,799.0){\rule[-0.200pt]{0.723pt}{0.400pt}}
\put(1232,799.67){\rule{0.241pt}{0.400pt}}
\multiput(1232.00,799.17)(0.500,1.000){2}{\rule{0.120pt}{0.400pt}}
\put(1229.0,800.0){\rule[-0.200pt]{0.723pt}{0.400pt}}
\put(1235,800.67){\rule{0.241pt}{0.400pt}}
\multiput(1235.00,800.17)(0.500,1.000){2}{\rule{0.120pt}{0.400pt}}
\put(1233.0,801.0){\rule[-0.200pt]{0.482pt}{0.400pt}}
\put(1239,801.67){\rule{0.241pt}{0.400pt}}
\multiput(1239.00,801.17)(0.500,1.000){2}{\rule{0.120pt}{0.400pt}}
\put(1236.0,802.0){\rule[-0.200pt]{0.723pt}{0.400pt}}
\put(1244,802.67){\rule{0.241pt}{0.400pt}}
\multiput(1244.00,802.17)(0.500,1.000){2}{\rule{0.120pt}{0.400pt}}
\put(1240.0,803.0){\rule[-0.200pt]{0.964pt}{0.400pt}}
\put(1248,803.67){\rule{0.241pt}{0.400pt}}
\multiput(1248.00,803.17)(0.500,1.000){2}{\rule{0.120pt}{0.400pt}}
\put(1245.0,804.0){\rule[-0.200pt]{0.723pt}{0.400pt}}
\put(1252,804.67){\rule{0.241pt}{0.400pt}}
\multiput(1252.00,804.17)(0.500,1.000){2}{\rule{0.120pt}{0.400pt}}
\put(1249.0,805.0){\rule[-0.200pt]{0.723pt}{0.400pt}}
\put(1257,805.67){\rule{0.241pt}{0.400pt}}
\multiput(1257.00,805.17)(0.500,1.000){2}{\rule{0.120pt}{0.400pt}}
\put(1253.0,806.0){\rule[-0.200pt]{0.964pt}{0.400pt}}
\put(1262,806.67){\rule{0.241pt}{0.400pt}}
\multiput(1262.00,806.17)(0.500,1.000){2}{\rule{0.120pt}{0.400pt}}
\put(1258.0,807.0){\rule[-0.200pt]{0.964pt}{0.400pt}}
\put(1266,807.67){\rule{0.241pt}{0.400pt}}
\multiput(1266.00,807.17)(0.500,1.000){2}{\rule{0.120pt}{0.400pt}}
\put(1263.0,808.0){\rule[-0.200pt]{0.723pt}{0.400pt}}
\put(1272,808.67){\rule{0.241pt}{0.400pt}}
\multiput(1272.00,808.17)(0.500,1.000){2}{\rule{0.120pt}{0.400pt}}
\put(1267.0,809.0){\rule[-0.200pt]{1.204pt}{0.400pt}}
\put(1277,809.67){\rule{0.241pt}{0.400pt}}
\multiput(1277.00,809.17)(0.500,1.000){2}{\rule{0.120pt}{0.400pt}}
\put(1273.0,810.0){\rule[-0.200pt]{0.964pt}{0.400pt}}
\put(1282,810.67){\rule{0.241pt}{0.400pt}}
\multiput(1282.00,810.17)(0.500,1.000){2}{\rule{0.120pt}{0.400pt}}
\put(1278.0,811.0){\rule[-0.200pt]{0.964pt}{0.400pt}}
\put(1288,811.67){\rule{0.241pt}{0.400pt}}
\multiput(1288.00,811.17)(0.500,1.000){2}{\rule{0.120pt}{0.400pt}}
\put(1283.0,812.0){\rule[-0.200pt]{1.204pt}{0.400pt}}
\put(1295,812.67){\rule{0.241pt}{0.400pt}}
\multiput(1295.00,812.17)(0.500,1.000){2}{\rule{0.120pt}{0.400pt}}
\put(1289.0,813.0){\rule[-0.200pt]{1.445pt}{0.400pt}}
\put(1302,813.67){\rule{0.241pt}{0.400pt}}
\multiput(1302.00,813.17)(0.500,1.000){2}{\rule{0.120pt}{0.400pt}}
\put(1296.0,814.0){\rule[-0.200pt]{1.445pt}{0.400pt}}
\put(1309,814.67){\rule{0.241pt}{0.400pt}}
\multiput(1309.00,814.17)(0.500,1.000){2}{\rule{0.120pt}{0.400pt}}
\put(1303.0,815.0){\rule[-0.200pt]{1.445pt}{0.400pt}}
\put(1317,815.67){\rule{0.241pt}{0.400pt}}
\multiput(1317.00,815.17)(0.500,1.000){2}{\rule{0.120pt}{0.400pt}}
\put(1310.0,816.0){\rule[-0.200pt]{1.686pt}{0.400pt}}
\put(1326,816.67){\rule{0.241pt}{0.400pt}}
\multiput(1326.00,816.17)(0.500,1.000){2}{\rule{0.120pt}{0.400pt}}
\put(1318.0,817.0){\rule[-0.200pt]{1.927pt}{0.400pt}}
\put(1337,817.67){\rule{0.241pt}{0.400pt}}
\multiput(1337.00,817.17)(0.500,1.000){2}{\rule{0.120pt}{0.400pt}}
\put(1327.0,818.0){\rule[-0.200pt]{2.409pt}{0.400pt}}
\put(1348,818.67){\rule{0.241pt}{0.400pt}}
\multiput(1348.00,818.17)(0.500,1.000){2}{\rule{0.120pt}{0.400pt}}
\put(1338.0,819.0){\rule[-0.200pt]{2.409pt}{0.400pt}}
\put(1368,819.67){\rule{0.241pt}{0.400pt}}
\multiput(1368.00,819.17)(0.500,1.000){2}{\rule{0.120pt}{0.400pt}}
\put(1349.0,820.0){\rule[-0.200pt]{4.577pt}{0.400pt}}
\put(1433,819.67){\rule{0.241pt}{0.400pt}}
\multiput(1433.00,820.17)(0.500,-1.000){2}{\rule{0.120pt}{0.400pt}}
\put(1434,819.67){\rule{0.241pt}{0.400pt}}
\multiput(1434.00,819.17)(0.500,1.000){2}{\rule{0.120pt}{0.400pt}}
\put(1369.0,821.0){\rule[-0.200pt]{15.418pt}{0.400pt}}
\put(1436,819.67){\rule{0.241pt}{0.400pt}}
\multiput(1436.00,820.17)(0.500,-1.000){2}{\rule{0.120pt}{0.400pt}}
\put(1435.0,821.0){\usebox{\plotpoint}}
\put(1437.0,820.0){\rule[-0.200pt]{0.482pt}{0.400pt}}
\put(231.0,131.0){\rule[-0.200pt]{0.400pt}{175.375pt}}
\put(231.0,131.0){\rule[-0.200pt]{291.007pt}{0.400pt}}
\put(1439.0,131.0){\rule[-0.200pt]{0.400pt}{175.375pt}}
\put(231.0,859.0){\rule[-0.200pt]{291.007pt}{0.400pt}}
\end{picture}

\caption{
Nekalibrovaná závislosť toroidálneho magnetického poľa $B_t\(t\)$ na čase $t$ pre výstrel \#23728.
}\label{G_V-1-B}
\end{figure}

\begin{figure}
% GNUPLOT: LaTeX picture
\setlength{\unitlength}{0.240900pt}
\ifx\plotpoint\undefined\newsavebox{\plotpoint}\fi
\begin{picture}(1500,900)(0,0)
\sbox{\plotpoint}{\rule[-0.200pt]{0.400pt}{0.400pt}}%
\put(191.0,131.0){\rule[-0.200pt]{4.818pt}{0.400pt}}
\put(171,131){\makebox(0,0)[r]{-500}}
\put(1419.0,131.0){\rule[-0.200pt]{4.818pt}{0.400pt}}
\put(191.0,235.0){\rule[-0.200pt]{4.818pt}{0.400pt}}
\put(171,235){\makebox(0,0)[r]{ 0}}
\put(1419.0,235.0){\rule[-0.200pt]{4.818pt}{0.400pt}}
\put(191.0,339.0){\rule[-0.200pt]{4.818pt}{0.400pt}}
\put(171,339){\makebox(0,0)[r]{ 500}}
\put(1419.0,339.0){\rule[-0.200pt]{4.818pt}{0.400pt}}
\put(191.0,443.0){\rule[-0.200pt]{4.818pt}{0.400pt}}
\put(171,443){\makebox(0,0)[r]{ 1000}}
\put(1419.0,443.0){\rule[-0.200pt]{4.818pt}{0.400pt}}
\put(191.0,547.0){\rule[-0.200pt]{4.818pt}{0.400pt}}
\put(171,547){\makebox(0,0)[r]{ 1500}}
\put(1419.0,547.0){\rule[-0.200pt]{4.818pt}{0.400pt}}
\put(191.0,651.0){\rule[-0.200pt]{4.818pt}{0.400pt}}
\put(171,651){\makebox(0,0)[r]{ 2000}}
\put(1419.0,651.0){\rule[-0.200pt]{4.818pt}{0.400pt}}
\put(191.0,755.0){\rule[-0.200pt]{4.818pt}{0.400pt}}
\put(171,755){\makebox(0,0)[r]{ 2500}}
\put(1419.0,755.0){\rule[-0.200pt]{4.818pt}{0.400pt}}
\put(191.0,859.0){\rule[-0.200pt]{4.818pt}{0.400pt}}
\put(171,859){\makebox(0,0)[r]{ 3000}}
\put(1419.0,859.0){\rule[-0.200pt]{4.818pt}{0.400pt}}
\put(191.0,131.0){\rule[-0.200pt]{0.400pt}{4.818pt}}
\put(191,90){\makebox(0,0){ 0}}
\put(191.0,839.0){\rule[-0.200pt]{0.400pt}{4.818pt}}
\put(399.0,131.0){\rule[-0.200pt]{0.400pt}{4.818pt}}
\put(399,90){\makebox(0,0){ 2}}
\put(399.0,839.0){\rule[-0.200pt]{0.400pt}{4.818pt}}
\put(607.0,131.0){\rule[-0.200pt]{0.400pt}{4.818pt}}
\put(607,90){\makebox(0,0){ 4}}
\put(607.0,839.0){\rule[-0.200pt]{0.400pt}{4.818pt}}
\put(815.0,131.0){\rule[-0.200pt]{0.400pt}{4.818pt}}
\put(815,90){\makebox(0,0){ 6}}
\put(815.0,839.0){\rule[-0.200pt]{0.400pt}{4.818pt}}
\put(1023.0,131.0){\rule[-0.200pt]{0.400pt}{4.818pt}}
\put(1023,90){\makebox(0,0){ 8}}
\put(1023.0,839.0){\rule[-0.200pt]{0.400pt}{4.818pt}}
\put(1231.0,131.0){\rule[-0.200pt]{0.400pt}{4.818pt}}
\put(1231,90){\makebox(0,0){ 10}}
\put(1231.0,839.0){\rule[-0.200pt]{0.400pt}{4.818pt}}
\put(1439.0,131.0){\rule[-0.200pt]{0.400pt}{4.818pt}}
\put(1439,90){\makebox(0,0){ 12}}
\put(1439.0,839.0){\rule[-0.200pt]{0.400pt}{4.818pt}}
\put(191.0,131.0){\rule[-0.200pt]{0.400pt}{175.375pt}}
\put(191.0,131.0){\rule[-0.200pt]{300.643pt}{0.400pt}}
\put(1439.0,131.0){\rule[-0.200pt]{0.400pt}{175.375pt}}
\put(191.0,859.0){\rule[-0.200pt]{300.643pt}{0.400pt}}
\put(30,495){\makebox(0,0){\popi{T_p}{A}}}
\put(815,29){\makebox(0,0){\popi{t}{ms}}}
\put(538,0){\line(0,1){899}}
\put(1194,0){\line(0,1){899}}
\put(551,172){\makebox(0,0)[r]{Plasma current $I_p$}}
\put(571.0,172.0){\rule[-0.200pt]{24.090pt}{0.400pt}}
\put(192,238){\usebox{\plotpoint}}
\put(191.67,232){\rule{0.400pt}{1.445pt}}
\multiput(191.17,235.00)(1.000,-3.000){2}{\rule{0.400pt}{0.723pt}}
\put(192.67,232){\rule{0.400pt}{1.445pt}}
\multiput(192.17,232.00)(1.000,3.000){2}{\rule{0.400pt}{0.723pt}}
\put(193.67,232){\rule{0.400pt}{1.445pt}}
\multiput(193.17,235.00)(1.000,-3.000){2}{\rule{0.400pt}{0.723pt}}
\put(194.67,232){\rule{0.400pt}{1.445pt}}
\multiput(194.17,232.00)(1.000,3.000){2}{\rule{0.400pt}{0.723pt}}
\put(195.67,232){\rule{0.400pt}{1.445pt}}
\multiput(195.17,235.00)(1.000,-3.000){2}{\rule{0.400pt}{0.723pt}}
\put(197.67,232){\rule{0.400pt}{1.445pt}}
\multiput(197.17,232.00)(1.000,3.000){2}{\rule{0.400pt}{0.723pt}}
\put(198.67,232){\rule{0.400pt}{1.445pt}}
\multiput(198.17,235.00)(1.000,-3.000){2}{\rule{0.400pt}{0.723pt}}
\put(199.67,232){\rule{0.400pt}{1.445pt}}
\multiput(199.17,232.00)(1.000,3.000){2}{\rule{0.400pt}{0.723pt}}
\put(197.0,232.0){\usebox{\plotpoint}}
\put(201.67,232){\rule{0.400pt}{1.445pt}}
\multiput(201.17,235.00)(1.000,-3.000){2}{\rule{0.400pt}{0.723pt}}
\put(203.17,232){\rule{0.400pt}{1.300pt}}
\multiput(202.17,232.00)(2.000,3.302){2}{\rule{0.400pt}{0.650pt}}
\put(204.67,232){\rule{0.400pt}{1.445pt}}
\multiput(204.17,235.00)(1.000,-3.000){2}{\rule{0.400pt}{0.723pt}}
\put(201.0,238.0){\usebox{\plotpoint}}
\put(206.67,232){\rule{0.400pt}{1.445pt}}
\multiput(206.17,232.00)(1.000,3.000){2}{\rule{0.400pt}{0.723pt}}
\put(206.0,232.0){\usebox{\plotpoint}}
\put(208.67,232){\rule{0.400pt}{1.445pt}}
\multiput(208.17,235.00)(1.000,-3.000){2}{\rule{0.400pt}{0.723pt}}
\put(208.0,238.0){\usebox{\plotpoint}}
\put(210.67,232){\rule{0.400pt}{1.445pt}}
\multiput(210.17,232.00)(1.000,3.000){2}{\rule{0.400pt}{0.723pt}}
\put(210.0,232.0){\usebox{\plotpoint}}
\put(212.67,232){\rule{0.400pt}{1.445pt}}
\multiput(212.17,235.00)(1.000,-3.000){2}{\rule{0.400pt}{0.723pt}}
\put(213.67,232){\rule{0.400pt}{1.445pt}}
\multiput(213.17,232.00)(1.000,3.000){2}{\rule{0.400pt}{0.723pt}}
\put(214.67,232){\rule{0.400pt}{1.445pt}}
\multiput(214.17,235.00)(1.000,-3.000){2}{\rule{0.400pt}{0.723pt}}
\put(215.67,232){\rule{0.400pt}{1.445pt}}
\multiput(215.17,232.00)(1.000,3.000){2}{\rule{0.400pt}{0.723pt}}
\put(216.67,232){\rule{0.400pt}{1.445pt}}
\multiput(216.17,235.00)(1.000,-3.000){2}{\rule{0.400pt}{0.723pt}}
\put(212.0,238.0){\usebox{\plotpoint}}
\put(218.67,232){\rule{0.400pt}{1.445pt}}
\multiput(218.17,232.00)(1.000,3.000){2}{\rule{0.400pt}{0.723pt}}
\put(218.0,232.0){\usebox{\plotpoint}}
\put(220.67,232){\rule{0.400pt}{1.445pt}}
\multiput(220.17,235.00)(1.000,-3.000){2}{\rule{0.400pt}{0.723pt}}
\put(220.0,238.0){\usebox{\plotpoint}}
\put(222.67,232){\rule{0.400pt}{1.445pt}}
\multiput(222.17,232.00)(1.000,3.000){2}{\rule{0.400pt}{0.723pt}}
\put(223.67,232){\rule{0.400pt}{1.445pt}}
\multiput(223.17,235.00)(1.000,-3.000){2}{\rule{0.400pt}{0.723pt}}
\put(224.67,232){\rule{0.400pt}{1.445pt}}
\multiput(224.17,232.00)(1.000,3.000){2}{\rule{0.400pt}{0.723pt}}
\put(225.67,232){\rule{0.400pt}{1.445pt}}
\multiput(225.17,235.00)(1.000,-3.000){2}{\rule{0.400pt}{0.723pt}}
\put(226.67,232){\rule{0.400pt}{1.445pt}}
\multiput(226.17,232.00)(1.000,3.000){2}{\rule{0.400pt}{0.723pt}}
\put(227.67,232){\rule{0.400pt}{1.445pt}}
\multiput(227.17,235.00)(1.000,-3.000){2}{\rule{0.400pt}{0.723pt}}
\put(229.17,232){\rule{0.400pt}{1.300pt}}
\multiput(228.17,232.00)(2.000,3.302){2}{\rule{0.400pt}{0.650pt}}
\put(230.67,232){\rule{0.400pt}{1.445pt}}
\multiput(230.17,235.00)(1.000,-3.000){2}{\rule{0.400pt}{0.723pt}}
\put(231.67,232){\rule{0.400pt}{1.445pt}}
\multiput(231.17,232.00)(1.000,3.000){2}{\rule{0.400pt}{0.723pt}}
\put(222.0,232.0){\usebox{\plotpoint}}
\put(233.67,232){\rule{0.400pt}{1.445pt}}
\multiput(233.17,235.00)(1.000,-3.000){2}{\rule{0.400pt}{0.723pt}}
\put(234.67,232){\rule{0.400pt}{1.445pt}}
\multiput(234.17,232.00)(1.000,3.000){2}{\rule{0.400pt}{0.723pt}}
\put(235.67,232){\rule{0.400pt}{1.445pt}}
\multiput(235.17,235.00)(1.000,-3.000){2}{\rule{0.400pt}{0.723pt}}
\put(236.67,232){\rule{0.400pt}{1.445pt}}
\multiput(236.17,232.00)(1.000,3.000){2}{\rule{0.400pt}{0.723pt}}
\put(237.67,232){\rule{0.400pt}{1.445pt}}
\multiput(237.17,235.00)(1.000,-3.000){2}{\rule{0.400pt}{0.723pt}}
\put(238.67,232){\rule{0.400pt}{1.445pt}}
\multiput(238.17,232.00)(1.000,3.000){2}{\rule{0.400pt}{0.723pt}}
\put(239.67,232){\rule{0.400pt}{1.445pt}}
\multiput(239.17,235.00)(1.000,-3.000){2}{\rule{0.400pt}{0.723pt}}
\put(240.67,232){\rule{0.400pt}{1.445pt}}
\multiput(240.17,232.00)(1.000,3.000){2}{\rule{0.400pt}{0.723pt}}
\put(241.67,232){\rule{0.400pt}{1.445pt}}
\multiput(241.17,235.00)(1.000,-3.000){2}{\rule{0.400pt}{0.723pt}}
\put(242.67,232){\rule{0.400pt}{1.445pt}}
\multiput(242.17,232.00)(1.000,3.000){2}{\rule{0.400pt}{0.723pt}}
\put(243.67,232){\rule{0.400pt}{1.445pt}}
\multiput(243.17,235.00)(1.000,-3.000){2}{\rule{0.400pt}{0.723pt}}
\put(244.67,232){\rule{0.400pt}{1.445pt}}
\multiput(244.17,232.00)(1.000,3.000){2}{\rule{0.400pt}{0.723pt}}
\put(245.67,232){\rule{0.400pt}{1.445pt}}
\multiput(245.17,235.00)(1.000,-3.000){2}{\rule{0.400pt}{0.723pt}}
\put(247,230.67){\rule{0.241pt}{0.400pt}}
\multiput(247.00,231.17)(0.500,-1.000){2}{\rule{0.120pt}{0.400pt}}
\put(247.67,231){\rule{0.400pt}{1.686pt}}
\multiput(247.17,231.00)(1.000,3.500){2}{\rule{0.400pt}{0.843pt}}
\put(248.67,232){\rule{0.400pt}{1.445pt}}
\multiput(248.17,235.00)(1.000,-3.000){2}{\rule{0.400pt}{0.723pt}}
\put(249.67,232){\rule{0.400pt}{1.445pt}}
\multiput(249.17,232.00)(1.000,3.000){2}{\rule{0.400pt}{0.723pt}}
\put(233.0,238.0){\usebox{\plotpoint}}
\put(251.67,232){\rule{0.400pt}{1.445pt}}
\multiput(251.17,235.00)(1.000,-3.000){2}{\rule{0.400pt}{0.723pt}}
\put(252.67,232){\rule{0.400pt}{1.445pt}}
\multiput(252.17,232.00)(1.000,3.000){2}{\rule{0.400pt}{0.723pt}}
\put(253.67,232){\rule{0.400pt}{1.445pt}}
\multiput(253.17,235.00)(1.000,-3.000){2}{\rule{0.400pt}{0.723pt}}
\put(255.17,232){\rule{0.400pt}{1.300pt}}
\multiput(254.17,232.00)(2.000,3.302){2}{\rule{0.400pt}{0.650pt}}
\put(256.67,232){\rule{0.400pt}{1.445pt}}
\multiput(256.17,235.00)(1.000,-3.000){2}{\rule{0.400pt}{0.723pt}}
\put(251.0,238.0){\usebox{\plotpoint}}
\put(258.67,232){\rule{0.400pt}{1.204pt}}
\multiput(258.17,232.00)(1.000,2.500){2}{\rule{0.400pt}{0.602pt}}
\put(260,236.67){\rule{0.241pt}{0.400pt}}
\multiput(260.00,236.17)(0.500,1.000){2}{\rule{0.120pt}{0.400pt}}
\put(260.67,232){\rule{0.400pt}{1.445pt}}
\multiput(260.17,235.00)(1.000,-3.000){2}{\rule{0.400pt}{0.723pt}}
\put(258.0,232.0){\usebox{\plotpoint}}
\put(262.67,232){\rule{0.400pt}{1.204pt}}
\multiput(262.17,232.00)(1.000,2.500){2}{\rule{0.400pt}{0.602pt}}
\put(263.67,232){\rule{0.400pt}{1.204pt}}
\multiput(263.17,234.50)(1.000,-2.500){2}{\rule{0.400pt}{0.602pt}}
\put(264.67,232){\rule{0.400pt}{1.204pt}}
\multiput(264.17,232.00)(1.000,2.500){2}{\rule{0.400pt}{0.602pt}}
\put(265.67,232){\rule{0.400pt}{1.204pt}}
\multiput(265.17,234.50)(1.000,-2.500){2}{\rule{0.400pt}{0.602pt}}
\put(266.67,232){\rule{0.400pt}{1.204pt}}
\multiput(266.17,232.00)(1.000,2.500){2}{\rule{0.400pt}{0.602pt}}
\put(268,236.67){\rule{0.241pt}{0.400pt}}
\multiput(268.00,236.17)(0.500,1.000){2}{\rule{0.120pt}{0.400pt}}
\put(268.67,232){\rule{0.400pt}{1.445pt}}
\multiput(268.17,235.00)(1.000,-3.000){2}{\rule{0.400pt}{0.723pt}}
\put(269.67,232){\rule{0.400pt}{1.204pt}}
\multiput(269.17,232.00)(1.000,2.500){2}{\rule{0.400pt}{0.602pt}}
\put(270.67,232){\rule{0.400pt}{1.204pt}}
\multiput(270.17,234.50)(1.000,-2.500){2}{\rule{0.400pt}{0.602pt}}
\put(262.0,232.0){\usebox{\plotpoint}}
\put(272.67,232){\rule{0.400pt}{1.204pt}}
\multiput(272.17,232.00)(1.000,2.500){2}{\rule{0.400pt}{0.602pt}}
\put(273.67,231){\rule{0.400pt}{1.445pt}}
\multiput(273.17,234.00)(1.000,-3.000){2}{\rule{0.400pt}{0.723pt}}
\put(274.67,231){\rule{0.400pt}{1.445pt}}
\multiput(274.17,231.00)(1.000,3.000){2}{\rule{0.400pt}{0.723pt}}
\put(276,236.67){\rule{0.241pt}{0.400pt}}
\multiput(276.00,236.17)(0.500,1.000){2}{\rule{0.120pt}{0.400pt}}
\put(276.67,232){\rule{0.400pt}{1.445pt}}
\multiput(276.17,235.00)(1.000,-3.000){2}{\rule{0.400pt}{0.723pt}}
\put(277.67,232){\rule{0.400pt}{1.204pt}}
\multiput(277.17,232.00)(1.000,2.500){2}{\rule{0.400pt}{0.602pt}}
\put(278.67,232){\rule{0.400pt}{1.204pt}}
\multiput(278.17,234.50)(1.000,-2.500){2}{\rule{0.400pt}{0.602pt}}
\put(280,230.67){\rule{0.241pt}{0.400pt}}
\multiput(280.00,231.17)(0.500,-1.000){2}{\rule{0.120pt}{0.400pt}}
\put(281.17,231){\rule{0.400pt}{1.500pt}}
\multiput(280.17,231.00)(2.000,3.887){2}{\rule{0.400pt}{0.750pt}}
\put(282.67,231){\rule{0.400pt}{1.686pt}}
\multiput(282.17,234.50)(1.000,-3.500){2}{\rule{0.400pt}{0.843pt}}
\put(283.67,231){\rule{0.400pt}{1.686pt}}
\multiput(283.17,231.00)(1.000,3.500){2}{\rule{0.400pt}{0.843pt}}
\put(284.67,231){\rule{0.400pt}{1.686pt}}
\multiput(284.17,234.50)(1.000,-3.500){2}{\rule{0.400pt}{0.843pt}}
\put(285.67,231){\rule{0.400pt}{1.686pt}}
\multiput(285.17,231.00)(1.000,3.500){2}{\rule{0.400pt}{0.843pt}}
\put(286.67,231){\rule{0.400pt}{1.686pt}}
\multiput(286.17,234.50)(1.000,-3.500){2}{\rule{0.400pt}{0.843pt}}
\put(287.67,231){\rule{0.400pt}{1.686pt}}
\multiput(287.17,231.00)(1.000,3.500){2}{\rule{0.400pt}{0.843pt}}
\put(288.67,232){\rule{0.400pt}{1.445pt}}
\multiput(288.17,235.00)(1.000,-3.000){2}{\rule{0.400pt}{0.723pt}}
\put(289.67,232){\rule{0.400pt}{1.445pt}}
\multiput(289.17,232.00)(1.000,3.000){2}{\rule{0.400pt}{0.723pt}}
\put(291,236.67){\rule{0.241pt}{0.400pt}}
\multiput(291.00,237.17)(0.500,-1.000){2}{\rule{0.120pt}{0.400pt}}
\put(291.67,232){\rule{0.400pt}{1.204pt}}
\multiput(291.17,234.50)(1.000,-2.500){2}{\rule{0.400pt}{0.602pt}}
\put(292.67,232){\rule{0.400pt}{1.204pt}}
\multiput(292.17,232.00)(1.000,2.500){2}{\rule{0.400pt}{0.602pt}}
\put(293.67,232){\rule{0.400pt}{1.204pt}}
\multiput(293.17,234.50)(1.000,-2.500){2}{\rule{0.400pt}{0.602pt}}
\put(294.67,232){\rule{0.400pt}{1.204pt}}
\multiput(294.17,232.00)(1.000,2.500){2}{\rule{0.400pt}{0.602pt}}
\put(295.67,232){\rule{0.400pt}{1.204pt}}
\multiput(295.17,234.50)(1.000,-2.500){2}{\rule{0.400pt}{0.602pt}}
\put(296.67,232){\rule{0.400pt}{1.204pt}}
\multiput(296.17,232.00)(1.000,2.500){2}{\rule{0.400pt}{0.602pt}}
\put(297.67,232){\rule{0.400pt}{1.204pt}}
\multiput(297.17,234.50)(1.000,-2.500){2}{\rule{0.400pt}{0.602pt}}
\put(272.0,232.0){\usebox{\plotpoint}}
\put(299.67,232){\rule{0.400pt}{1.445pt}}
\multiput(299.17,232.00)(1.000,3.000){2}{\rule{0.400pt}{0.723pt}}
\put(300.67,232){\rule{0.400pt}{1.445pt}}
\multiput(300.17,235.00)(1.000,-3.000){2}{\rule{0.400pt}{0.723pt}}
\put(301.67,232){\rule{0.400pt}{1.445pt}}
\multiput(301.17,232.00)(1.000,3.000){2}{\rule{0.400pt}{0.723pt}}
\put(302.67,232){\rule{0.400pt}{1.445pt}}
\multiput(302.17,235.00)(1.000,-3.000){2}{\rule{0.400pt}{0.723pt}}
\put(303.67,232){\rule{0.400pt}{1.445pt}}
\multiput(303.17,232.00)(1.000,3.000){2}{\rule{0.400pt}{0.723pt}}
\put(305,236.67){\rule{0.241pt}{0.400pt}}
\multiput(305.00,237.17)(0.500,-1.000){2}{\rule{0.120pt}{0.400pt}}
\put(305.67,232){\rule{0.400pt}{1.204pt}}
\multiput(305.17,234.50)(1.000,-2.500){2}{\rule{0.400pt}{0.602pt}}
\put(307.17,232){\rule{0.400pt}{1.300pt}}
\multiput(306.17,232.00)(2.000,3.302){2}{\rule{0.400pt}{0.650pt}}
\put(308.67,232){\rule{0.400pt}{1.445pt}}
\multiput(308.17,235.00)(1.000,-3.000){2}{\rule{0.400pt}{0.723pt}}
\put(309.67,232){\rule{0.400pt}{1.204pt}}
\multiput(309.17,232.00)(1.000,2.500){2}{\rule{0.400pt}{0.602pt}}
\put(310.67,232){\rule{0.400pt}{1.204pt}}
\multiput(310.17,234.50)(1.000,-2.500){2}{\rule{0.400pt}{0.602pt}}
\put(299.0,232.0){\usebox{\plotpoint}}
\put(312.67,232){\rule{0.400pt}{1.445pt}}
\multiput(312.17,232.00)(1.000,3.000){2}{\rule{0.400pt}{0.723pt}}
\put(313.67,232){\rule{0.400pt}{1.445pt}}
\multiput(313.17,235.00)(1.000,-3.000){2}{\rule{0.400pt}{0.723pt}}
\put(314.67,232){\rule{0.400pt}{1.445pt}}
\multiput(314.17,232.00)(1.000,3.000){2}{\rule{0.400pt}{0.723pt}}
\put(316,236.67){\rule{0.241pt}{0.400pt}}
\multiput(316.00,237.17)(0.500,-1.000){2}{\rule{0.120pt}{0.400pt}}
\put(316.67,232){\rule{0.400pt}{1.204pt}}
\multiput(316.17,234.50)(1.000,-2.500){2}{\rule{0.400pt}{0.602pt}}
\put(312.0,232.0){\usebox{\plotpoint}}
\put(318.67,232){\rule{0.400pt}{1.445pt}}
\multiput(318.17,232.00)(1.000,3.000){2}{\rule{0.400pt}{0.723pt}}
\put(318.0,232.0){\usebox{\plotpoint}}
\put(320.67,232){\rule{0.400pt}{1.445pt}}
\multiput(320.17,235.00)(1.000,-3.000){2}{\rule{0.400pt}{0.723pt}}
\put(320.0,238.0){\usebox{\plotpoint}}
\put(322.67,232){\rule{0.400pt}{1.445pt}}
\multiput(322.17,232.00)(1.000,3.000){2}{\rule{0.400pt}{0.723pt}}
\put(322.0,232.0){\usebox{\plotpoint}}
\put(324.67,232){\rule{0.400pt}{1.445pt}}
\multiput(324.17,235.00)(1.000,-3.000){2}{\rule{0.400pt}{0.723pt}}
\put(325.67,232){\rule{0.400pt}{1.204pt}}
\multiput(325.17,232.00)(1.000,2.500){2}{\rule{0.400pt}{0.602pt}}
\put(326.67,232){\rule{0.400pt}{1.204pt}}
\multiput(326.17,234.50)(1.000,-2.500){2}{\rule{0.400pt}{0.602pt}}
\put(327.67,232){\rule{0.400pt}{1.445pt}}
\multiput(327.17,232.00)(1.000,3.000){2}{\rule{0.400pt}{0.723pt}}
\put(328.67,232){\rule{0.400pt}{1.445pt}}
\multiput(328.17,235.00)(1.000,-3.000){2}{\rule{0.400pt}{0.723pt}}
\put(329.67,232){\rule{0.400pt}{1.204pt}}
\multiput(329.17,232.00)(1.000,2.500){2}{\rule{0.400pt}{0.602pt}}
\put(330.67,232){\rule{0.400pt}{1.204pt}}
\multiput(330.17,234.50)(1.000,-2.500){2}{\rule{0.400pt}{0.602pt}}
\put(331.67,232){\rule{0.400pt}{1.204pt}}
\multiput(331.17,232.00)(1.000,2.500){2}{\rule{0.400pt}{0.602pt}}
\put(333.17,232){\rule{0.400pt}{1.100pt}}
\multiput(332.17,234.72)(2.000,-2.717){2}{\rule{0.400pt}{0.550pt}}
\put(335,230.67){\rule{0.241pt}{0.400pt}}
\multiput(335.00,231.17)(0.500,-1.000){2}{\rule{0.120pt}{0.400pt}}
\put(335.67,231){\rule{0.400pt}{1.686pt}}
\multiput(335.17,231.00)(1.000,3.500){2}{\rule{0.400pt}{0.843pt}}
\put(336.67,232){\rule{0.400pt}{1.445pt}}
\multiput(336.17,235.00)(1.000,-3.000){2}{\rule{0.400pt}{0.723pt}}
\put(337.67,232){\rule{0.400pt}{1.445pt}}
\multiput(337.17,232.00)(1.000,3.000){2}{\rule{0.400pt}{0.723pt}}
\put(338.67,232){\rule{0.400pt}{1.445pt}}
\multiput(338.17,235.00)(1.000,-3.000){2}{\rule{0.400pt}{0.723pt}}
\put(339.67,232){\rule{0.400pt}{1.445pt}}
\multiput(339.17,232.00)(1.000,3.000){2}{\rule{0.400pt}{0.723pt}}
\put(324.0,238.0){\usebox{\plotpoint}}
\put(341.67,232){\rule{0.400pt}{1.445pt}}
\multiput(341.17,235.00)(1.000,-3.000){2}{\rule{0.400pt}{0.723pt}}
\put(342.67,232){\rule{0.400pt}{1.445pt}}
\multiput(342.17,232.00)(1.000,3.000){2}{\rule{0.400pt}{0.723pt}}
\put(343.67,232){\rule{0.400pt}{1.445pt}}
\multiput(343.17,235.00)(1.000,-3.000){2}{\rule{0.400pt}{0.723pt}}
\put(344.67,232){\rule{0.400pt}{1.445pt}}
\multiput(344.17,232.00)(1.000,3.000){2}{\rule{0.400pt}{0.723pt}}
\put(345.67,232){\rule{0.400pt}{1.445pt}}
\multiput(345.17,235.00)(1.000,-3.000){2}{\rule{0.400pt}{0.723pt}}
\put(346.67,232){\rule{0.400pt}{1.204pt}}
\multiput(346.17,232.00)(1.000,2.500){2}{\rule{0.400pt}{0.602pt}}
\put(347.67,232){\rule{0.400pt}{1.204pt}}
\multiput(347.17,234.50)(1.000,-2.500){2}{\rule{0.400pt}{0.602pt}}
\put(348.67,232){\rule{0.400pt}{1.204pt}}
\multiput(348.17,232.00)(1.000,2.500){2}{\rule{0.400pt}{0.602pt}}
\put(349.67,232){\rule{0.400pt}{1.204pt}}
\multiput(349.17,234.50)(1.000,-2.500){2}{\rule{0.400pt}{0.602pt}}
\put(350.67,232){\rule{0.400pt}{1.204pt}}
\multiput(350.17,232.00)(1.000,2.500){2}{\rule{0.400pt}{0.602pt}}
\put(351.67,232){\rule{0.400pt}{1.204pt}}
\multiput(351.17,234.50)(1.000,-2.500){2}{\rule{0.400pt}{0.602pt}}
\put(352.67,232){\rule{0.400pt}{1.445pt}}
\multiput(352.17,232.00)(1.000,3.000){2}{\rule{0.400pt}{0.723pt}}
\put(353.67,232){\rule{0.400pt}{1.445pt}}
\multiput(353.17,235.00)(1.000,-3.000){2}{\rule{0.400pt}{0.723pt}}
\put(354.67,232){\rule{0.400pt}{1.445pt}}
\multiput(354.17,232.00)(1.000,3.000){2}{\rule{0.400pt}{0.723pt}}
\put(355.67,232){\rule{0.400pt}{1.445pt}}
\multiput(355.17,235.00)(1.000,-3.000){2}{\rule{0.400pt}{0.723pt}}
\put(356.67,232){\rule{0.400pt}{1.204pt}}
\multiput(356.17,232.00)(1.000,2.500){2}{\rule{0.400pt}{0.602pt}}
\put(357.67,232){\rule{0.400pt}{1.204pt}}
\multiput(357.17,234.50)(1.000,-2.500){2}{\rule{0.400pt}{0.602pt}}
\put(359.17,232){\rule{0.400pt}{1.100pt}}
\multiput(358.17,232.00)(2.000,2.717){2}{\rule{0.400pt}{0.550pt}}
\put(360.67,232){\rule{0.400pt}{1.204pt}}
\multiput(360.17,234.50)(1.000,-2.500){2}{\rule{0.400pt}{0.602pt}}
\put(362,230.67){\rule{0.241pt}{0.400pt}}
\multiput(362.00,231.17)(0.500,-1.000){2}{\rule{0.120pt}{0.400pt}}
\put(362.67,231){\rule{0.400pt}{1.445pt}}
\multiput(362.17,231.00)(1.000,3.000){2}{\rule{0.400pt}{0.723pt}}
\put(363.67,232){\rule{0.400pt}{1.204pt}}
\multiput(363.17,234.50)(1.000,-2.500){2}{\rule{0.400pt}{0.602pt}}
\put(364.67,232){\rule{0.400pt}{1.445pt}}
\multiput(364.17,232.00)(1.000,3.000){2}{\rule{0.400pt}{0.723pt}}
\put(365.67,231){\rule{0.400pt}{1.686pt}}
\multiput(365.17,234.50)(1.000,-3.500){2}{\rule{0.400pt}{0.843pt}}
\put(366.67,231){\rule{0.400pt}{1.686pt}}
\multiput(366.17,231.00)(1.000,3.500){2}{\rule{0.400pt}{0.843pt}}
\put(367.67,232){\rule{0.400pt}{1.445pt}}
\multiput(367.17,235.00)(1.000,-3.000){2}{\rule{0.400pt}{0.723pt}}
\put(368.67,232){\rule{0.400pt}{1.204pt}}
\multiput(368.17,232.00)(1.000,2.500){2}{\rule{0.400pt}{0.602pt}}
\put(370,236.67){\rule{0.241pt}{0.400pt}}
\multiput(370.00,236.17)(0.500,1.000){2}{\rule{0.120pt}{0.400pt}}
\put(370.67,232){\rule{0.400pt}{1.445pt}}
\multiput(370.17,235.00)(1.000,-3.000){2}{\rule{0.400pt}{0.723pt}}
\put(371.67,232){\rule{0.400pt}{1.445pt}}
\multiput(371.17,232.00)(1.000,3.000){2}{\rule{0.400pt}{0.723pt}}
\put(372.67,232){\rule{0.400pt}{1.445pt}}
\multiput(372.17,235.00)(1.000,-3.000){2}{\rule{0.400pt}{0.723pt}}
\put(373.67,232){\rule{0.400pt}{1.204pt}}
\multiput(373.17,232.00)(1.000,2.500){2}{\rule{0.400pt}{0.602pt}}
\put(374.67,232){\rule{0.400pt}{1.204pt}}
\multiput(374.17,234.50)(1.000,-2.500){2}{\rule{0.400pt}{0.602pt}}
\put(376,230.67){\rule{0.241pt}{0.400pt}}
\multiput(376.00,231.17)(0.500,-1.000){2}{\rule{0.120pt}{0.400pt}}
\put(376.67,231){\rule{0.400pt}{1.445pt}}
\multiput(376.17,231.00)(1.000,3.000){2}{\rule{0.400pt}{0.723pt}}
\put(341.0,238.0){\usebox{\plotpoint}}
\put(378.67,232){\rule{0.400pt}{1.204pt}}
\multiput(378.17,234.50)(1.000,-2.500){2}{\rule{0.400pt}{0.602pt}}
\put(380,230.67){\rule{0.241pt}{0.400pt}}
\multiput(380.00,231.17)(0.500,-1.000){2}{\rule{0.120pt}{0.400pt}}
\put(380.67,231){\rule{0.400pt}{1.445pt}}
\multiput(380.17,231.00)(1.000,3.000){2}{\rule{0.400pt}{0.723pt}}
\put(382,236.67){\rule{0.241pt}{0.400pt}}
\multiput(382.00,236.17)(0.500,1.000){2}{\rule{0.120pt}{0.400pt}}
\put(382.67,231){\rule{0.400pt}{1.686pt}}
\multiput(382.17,234.50)(1.000,-3.500){2}{\rule{0.400pt}{0.843pt}}
\put(378.0,237.0){\usebox{\plotpoint}}
\put(385.17,231){\rule{0.400pt}{1.300pt}}
\multiput(384.17,231.00)(2.000,3.302){2}{\rule{0.400pt}{0.650pt}}
\put(387,236.67){\rule{0.241pt}{0.400pt}}
\multiput(387.00,236.17)(0.500,1.000){2}{\rule{0.120pt}{0.400pt}}
\put(387.67,232){\rule{0.400pt}{1.445pt}}
\multiput(387.17,235.00)(1.000,-3.000){2}{\rule{0.400pt}{0.723pt}}
\put(389,230.67){\rule{0.241pt}{0.400pt}}
\multiput(389.00,231.17)(0.500,-1.000){2}{\rule{0.120pt}{0.400pt}}
\put(389.67,231){\rule{0.400pt}{1.445pt}}
\multiput(389.17,231.00)(1.000,3.000){2}{\rule{0.400pt}{0.723pt}}
\put(390.67,232){\rule{0.400pt}{1.204pt}}
\multiput(390.17,234.50)(1.000,-2.500){2}{\rule{0.400pt}{0.602pt}}
\put(391.67,232){\rule{0.400pt}{1.204pt}}
\multiput(391.17,232.00)(1.000,2.500){2}{\rule{0.400pt}{0.602pt}}
\put(384.0,231.0){\usebox{\plotpoint}}
\put(393.67,232){\rule{0.400pt}{1.204pt}}
\multiput(393.17,234.50)(1.000,-2.500){2}{\rule{0.400pt}{0.602pt}}
\put(395,230.67){\rule{0.241pt}{0.400pt}}
\multiput(395.00,231.17)(0.500,-1.000){2}{\rule{0.120pt}{0.400pt}}
\put(395.67,231){\rule{0.400pt}{1.445pt}}
\multiput(395.17,231.00)(1.000,3.000){2}{\rule{0.400pt}{0.723pt}}
\put(396.67,231){\rule{0.400pt}{1.445pt}}
\multiput(396.17,234.00)(1.000,-3.000){2}{\rule{0.400pt}{0.723pt}}
\put(397.67,231){\rule{0.400pt}{1.445pt}}
\multiput(397.17,231.00)(1.000,3.000){2}{\rule{0.400pt}{0.723pt}}
\put(398.67,231){\rule{0.400pt}{1.445pt}}
\multiput(398.17,234.00)(1.000,-3.000){2}{\rule{0.400pt}{0.723pt}}
\put(399.67,231){\rule{0.400pt}{1.445pt}}
\multiput(399.17,231.00)(1.000,3.000){2}{\rule{0.400pt}{0.723pt}}
\put(393.0,237.0){\usebox{\plotpoint}}
\put(401.67,232){\rule{0.400pt}{1.204pt}}
\multiput(401.17,234.50)(1.000,-2.500){2}{\rule{0.400pt}{0.602pt}}
\put(403,230.67){\rule{0.241pt}{0.400pt}}
\multiput(403.00,231.17)(0.500,-1.000){2}{\rule{0.120pt}{0.400pt}}
\put(403.67,231){\rule{0.400pt}{1.445pt}}
\multiput(403.17,231.00)(1.000,3.000){2}{\rule{0.400pt}{0.723pt}}
\put(405,236.67){\rule{0.241pt}{0.400pt}}
\multiput(405.00,236.17)(0.500,1.000){2}{\rule{0.120pt}{0.400pt}}
\put(405.67,231){\rule{0.400pt}{1.686pt}}
\multiput(405.17,234.50)(1.000,-3.500){2}{\rule{0.400pt}{0.843pt}}
\put(401.0,237.0){\usebox{\plotpoint}}
\put(407.67,231){\rule{0.400pt}{1.445pt}}
\multiput(407.17,231.00)(1.000,3.000){2}{\rule{0.400pt}{0.723pt}}
\put(408.67,237){\rule{0.400pt}{16.863pt}}
\multiput(408.17,237.00)(1.000,35.000){2}{\rule{0.400pt}{8.431pt}}
\put(409.67,199){\rule{0.400pt}{26.017pt}}
\multiput(409.17,253.00)(1.000,-54.000){2}{\rule{0.400pt}{13.009pt}}
\put(411.17,199){\rule{0.400pt}{25.100pt}}
\multiput(410.17,199.00)(2.000,72.904){2}{\rule{0.400pt}{12.550pt}}
\put(412.67,149){\rule{0.400pt}{42.158pt}}
\multiput(412.17,236.50)(1.000,-87.500){2}{\rule{0.400pt}{21.079pt}}
\put(413.67,149){\rule{0.400pt}{21.681pt}}
\multiput(413.17,149.00)(1.000,45.000){2}{\rule{0.400pt}{10.840pt}}
\put(414.67,239){\rule{0.400pt}{3.854pt}}
\multiput(414.17,239.00)(1.000,8.000){2}{\rule{0.400pt}{1.927pt}}
\put(416,254.67){\rule{0.241pt}{0.400pt}}
\multiput(416.00,254.17)(0.500,1.000){2}{\rule{0.120pt}{0.400pt}}
\put(416.67,256){\rule{0.400pt}{2.168pt}}
\multiput(416.17,256.00)(1.000,4.500){2}{\rule{0.400pt}{1.084pt}}
\put(417.67,265){\rule{0.400pt}{0.964pt}}
\multiput(417.17,265.00)(1.000,2.000){2}{\rule{0.400pt}{0.482pt}}
\put(407.0,231.0){\usebox{\plotpoint}}
\put(419.67,269){\rule{0.400pt}{2.650pt}}
\multiput(419.17,269.00)(1.000,5.500){2}{\rule{0.400pt}{1.325pt}}
\put(421,278.67){\rule{0.241pt}{0.400pt}}
\multiput(421.00,279.17)(0.500,-1.000){2}{\rule{0.120pt}{0.400pt}}
\put(421.67,279){\rule{0.400pt}{2.409pt}}
\multiput(421.17,279.00)(1.000,5.000){2}{\rule{0.400pt}{1.204pt}}
\put(423,287.67){\rule{0.241pt}{0.400pt}}
\multiput(423.00,288.17)(0.500,-1.000){2}{\rule{0.120pt}{0.400pt}}
\put(423.67,288){\rule{0.400pt}{2.650pt}}
\multiput(423.17,288.00)(1.000,5.500){2}{\rule{0.400pt}{1.325pt}}
\put(424.67,297){\rule{0.400pt}{0.482pt}}
\multiput(424.17,298.00)(1.000,-1.000){2}{\rule{0.400pt}{0.241pt}}
\put(425.67,297){\rule{0.400pt}{2.409pt}}
\multiput(425.17,297.00)(1.000,5.000){2}{\rule{0.400pt}{1.204pt}}
\put(426.67,305){\rule{0.400pt}{0.482pt}}
\multiput(426.17,306.00)(1.000,-1.000){2}{\rule{0.400pt}{0.241pt}}
\put(427.67,305){\rule{0.400pt}{2.650pt}}
\multiput(427.17,305.00)(1.000,5.500){2}{\rule{0.400pt}{1.325pt}}
\put(428.67,314){\rule{0.400pt}{0.482pt}}
\multiput(428.17,315.00)(1.000,-1.000){2}{\rule{0.400pt}{0.241pt}}
\put(429.67,314){\rule{0.400pt}{2.409pt}}
\multiput(429.17,314.00)(1.000,5.000){2}{\rule{0.400pt}{1.204pt}}
\put(430.67,322){\rule{0.400pt}{0.482pt}}
\multiput(430.17,323.00)(1.000,-1.000){2}{\rule{0.400pt}{0.241pt}}
\put(431.67,322){\rule{0.400pt}{2.409pt}}
\multiput(431.17,322.00)(1.000,5.000){2}{\rule{0.400pt}{1.204pt}}
\put(432.67,330){\rule{0.400pt}{0.482pt}}
\multiput(432.17,331.00)(1.000,-1.000){2}{\rule{0.400pt}{0.241pt}}
\put(433.67,330){\rule{0.400pt}{2.409pt}}
\multiput(433.17,330.00)(1.000,5.000){2}{\rule{0.400pt}{1.204pt}}
\put(434.67,338){\rule{0.400pt}{0.482pt}}
\multiput(434.17,339.00)(1.000,-1.000){2}{\rule{0.400pt}{0.241pt}}
\put(435.67,338){\rule{0.400pt}{0.964pt}}
\multiput(435.17,338.00)(1.000,2.000){2}{\rule{0.400pt}{0.482pt}}
\put(437.17,342){\rule{0.400pt}{1.900pt}}
\multiput(436.17,342.00)(2.000,5.056){2}{\rule{0.400pt}{0.950pt}}
\put(438.67,351){\rule{0.400pt}{0.723pt}}
\multiput(438.17,351.00)(1.000,1.500){2}{\rule{0.400pt}{0.361pt}}
\put(439.67,352){\rule{0.400pt}{0.482pt}}
\multiput(439.17,353.00)(1.000,-1.000){2}{\rule{0.400pt}{0.241pt}}
\put(440.67,352){\rule{0.400pt}{2.409pt}}
\multiput(440.17,352.00)(1.000,5.000){2}{\rule{0.400pt}{1.204pt}}
\put(441.67,359){\rule{0.400pt}{0.723pt}}
\multiput(441.17,360.50)(1.000,-1.500){2}{\rule{0.400pt}{0.361pt}}
\put(442.67,359){\rule{0.400pt}{2.168pt}}
\multiput(442.17,359.00)(1.000,4.500){2}{\rule{0.400pt}{1.084pt}}
\put(443.67,365){\rule{0.400pt}{0.723pt}}
\multiput(443.17,366.50)(1.000,-1.500){2}{\rule{0.400pt}{0.361pt}}
\put(444.67,365){\rule{0.400pt}{2.168pt}}
\multiput(444.17,365.00)(1.000,4.500){2}{\rule{0.400pt}{1.084pt}}
\put(445.67,372){\rule{0.400pt}{0.482pt}}
\multiput(445.17,373.00)(1.000,-1.000){2}{\rule{0.400pt}{0.241pt}}
\put(446.67,372){\rule{0.400pt}{2.168pt}}
\multiput(446.17,372.00)(1.000,4.500){2}{\rule{0.400pt}{1.084pt}}
\put(447.67,378){\rule{0.400pt}{0.723pt}}
\multiput(447.17,379.50)(1.000,-1.500){2}{\rule{0.400pt}{0.361pt}}
\put(448.67,378){\rule{0.400pt}{2.168pt}}
\multiput(448.17,378.00)(1.000,4.500){2}{\rule{0.400pt}{1.084pt}}
\put(449.67,383){\rule{0.400pt}{0.964pt}}
\multiput(449.17,385.00)(1.000,-2.000){2}{\rule{0.400pt}{0.482pt}}
\put(450.67,383){\rule{0.400pt}{2.168pt}}
\multiput(450.17,383.00)(1.000,4.500){2}{\rule{0.400pt}{1.084pt}}
\put(451.67,389){\rule{0.400pt}{0.723pt}}
\multiput(451.17,390.50)(1.000,-1.500){2}{\rule{0.400pt}{0.361pt}}
\put(452.67,389){\rule{0.400pt}{0.723pt}}
\multiput(452.17,389.00)(1.000,1.500){2}{\rule{0.400pt}{0.361pt}}
\put(453.67,392){\rule{0.400pt}{1.927pt}}
\multiput(453.17,392.00)(1.000,4.000){2}{\rule{0.400pt}{0.964pt}}
\put(454.67,400){\rule{0.400pt}{0.723pt}}
\multiput(454.17,400.00)(1.000,1.500){2}{\rule{0.400pt}{0.361pt}}
\put(455.67,399){\rule{0.400pt}{0.964pt}}
\multiput(455.17,401.00)(1.000,-2.000){2}{\rule{0.400pt}{0.482pt}}
\put(456.67,399){\rule{0.400pt}{1.927pt}}
\multiput(456.17,399.00)(1.000,4.000){2}{\rule{0.400pt}{0.964pt}}
\put(457.67,404){\rule{0.400pt}{0.723pt}}
\multiput(457.17,405.50)(1.000,-1.500){2}{\rule{0.400pt}{0.361pt}}
\put(458.67,404){\rule{0.400pt}{1.686pt}}
\multiput(458.17,404.00)(1.000,3.500){2}{\rule{0.400pt}{0.843pt}}
\put(459.67,408){\rule{0.400pt}{0.723pt}}
\multiput(459.17,409.50)(1.000,-1.500){2}{\rule{0.400pt}{0.361pt}}
\put(460.67,408){\rule{0.400pt}{1.927pt}}
\multiput(460.17,408.00)(1.000,4.000){2}{\rule{0.400pt}{0.964pt}}
\put(461.67,413){\rule{0.400pt}{0.723pt}}
\multiput(461.17,414.50)(1.000,-1.500){2}{\rule{0.400pt}{0.361pt}}
\put(463.17,413){\rule{0.400pt}{1.700pt}}
\multiput(462.17,413.00)(2.000,4.472){2}{\rule{0.400pt}{0.850pt}}
\put(464.67,403){\rule{0.400pt}{4.336pt}}
\multiput(464.17,412.00)(1.000,-9.000){2}{\rule{0.400pt}{2.168pt}}
\put(465.67,403){\rule{0.400pt}{2.891pt}}
\multiput(465.17,403.00)(1.000,6.000){2}{\rule{0.400pt}{1.445pt}}
\put(466.67,409){\rule{0.400pt}{1.445pt}}
\multiput(466.17,412.00)(1.000,-3.000){2}{\rule{0.400pt}{0.723pt}}
\put(467.67,409){\rule{0.400pt}{2.168pt}}
\multiput(467.17,409.00)(1.000,4.500){2}{\rule{0.400pt}{1.084pt}}
\put(468.67,411){\rule{0.400pt}{1.686pt}}
\multiput(468.17,414.50)(1.000,-3.500){2}{\rule{0.400pt}{0.843pt}}
\put(469.67,411){\rule{0.400pt}{2.891pt}}
\multiput(469.17,411.00)(1.000,6.000){2}{\rule{0.400pt}{1.445pt}}
\put(470.67,413){\rule{0.400pt}{2.409pt}}
\multiput(470.17,418.00)(1.000,-5.000){2}{\rule{0.400pt}{1.204pt}}
\put(471.67,413){\rule{0.400pt}{2.891pt}}
\multiput(471.17,413.00)(1.000,6.000){2}{\rule{0.400pt}{1.445pt}}
\put(472.67,417){\rule{0.400pt}{1.927pt}}
\multiput(472.17,421.00)(1.000,-4.000){2}{\rule{0.400pt}{0.964pt}}
\put(473.67,417){\rule{0.400pt}{2.650pt}}
\multiput(473.17,417.00)(1.000,5.500){2}{\rule{0.400pt}{1.325pt}}
\put(474.67,421){\rule{0.400pt}{1.686pt}}
\multiput(474.17,424.50)(1.000,-3.500){2}{\rule{0.400pt}{0.843pt}}
\put(475.67,421){\rule{0.400pt}{2.650pt}}
\multiput(475.17,421.00)(1.000,5.500){2}{\rule{0.400pt}{1.325pt}}
\put(476.67,424){\rule{0.400pt}{1.927pt}}
\multiput(476.17,428.00)(1.000,-4.000){2}{\rule{0.400pt}{0.964pt}}
\put(477.67,424){\rule{0.400pt}{1.927pt}}
\multiput(477.17,424.00)(1.000,4.000){2}{\rule{0.400pt}{0.964pt}}
\put(478.67,427){\rule{0.400pt}{1.204pt}}
\multiput(478.17,429.50)(1.000,-2.500){2}{\rule{0.400pt}{0.602pt}}
\put(479.67,427){\rule{0.400pt}{2.409pt}}
\multiput(479.17,427.00)(1.000,5.000){2}{\rule{0.400pt}{1.204pt}}
\put(480.67,429){\rule{0.400pt}{1.927pt}}
\multiput(480.17,433.00)(1.000,-4.000){2}{\rule{0.400pt}{0.964pt}}
\put(481.67,429){\rule{0.400pt}{1.686pt}}
\multiput(481.17,429.00)(1.000,3.500){2}{\rule{0.400pt}{0.843pt}}
\put(482.67,431){\rule{0.400pt}{1.204pt}}
\multiput(482.17,433.50)(1.000,-2.500){2}{\rule{0.400pt}{0.602pt}}
\put(483.67,431){\rule{0.400pt}{1.686pt}}
\multiput(483.17,431.00)(1.000,3.500){2}{\rule{0.400pt}{0.843pt}}
\put(484.67,432){\rule{0.400pt}{1.445pt}}
\multiput(484.17,435.00)(1.000,-3.000){2}{\rule{0.400pt}{0.723pt}}
\put(485.67,432){\rule{0.400pt}{2.409pt}}
\multiput(485.17,432.00)(1.000,5.000){2}{\rule{0.400pt}{1.204pt}}
\put(486.67,434){\rule{0.400pt}{1.927pt}}
\multiput(486.17,438.00)(1.000,-4.000){2}{\rule{0.400pt}{0.964pt}}
\put(487.67,434){\rule{0.400pt}{2.168pt}}
\multiput(487.17,434.00)(1.000,4.500){2}{\rule{0.400pt}{1.084pt}}
\put(489.17,437){\rule{0.400pt}{1.300pt}}
\multiput(488.17,440.30)(2.000,-3.302){2}{\rule{0.400pt}{0.650pt}}
\put(490.67,437){\rule{0.400pt}{1.445pt}}
\multiput(490.17,437.00)(1.000,3.000){2}{\rule{0.400pt}{0.723pt}}
\put(491.67,437){\rule{0.400pt}{1.445pt}}
\multiput(491.17,440.00)(1.000,-3.000){2}{\rule{0.400pt}{0.723pt}}
\put(492.67,437){\rule{0.400pt}{2.168pt}}
\multiput(492.17,437.00)(1.000,4.500){2}{\rule{0.400pt}{1.084pt}}
\put(493.67,437){\rule{0.400pt}{2.168pt}}
\multiput(493.17,441.50)(1.000,-4.500){2}{\rule{0.400pt}{1.084pt}}
\put(494.67,437){\rule{0.400pt}{2.168pt}}
\multiput(494.17,437.00)(1.000,4.500){2}{\rule{0.400pt}{1.084pt}}
\put(495.67,440){\rule{0.400pt}{1.445pt}}
\multiput(495.17,443.00)(1.000,-3.000){2}{\rule{0.400pt}{0.723pt}}
\put(496.67,440){\rule{0.400pt}{1.445pt}}
\multiput(496.17,440.00)(1.000,3.000){2}{\rule{0.400pt}{0.723pt}}
\put(497.67,442){\rule{0.400pt}{0.964pt}}
\multiput(497.17,444.00)(1.000,-2.000){2}{\rule{0.400pt}{0.482pt}}
\put(498.67,442){\rule{0.400pt}{1.445pt}}
\multiput(498.17,442.00)(1.000,3.000){2}{\rule{0.400pt}{0.723pt}}
\put(499.67,444){\rule{0.400pt}{0.964pt}}
\multiput(499.17,446.00)(1.000,-2.000){2}{\rule{0.400pt}{0.482pt}}
\put(500.67,444){\rule{0.400pt}{1.204pt}}
\multiput(500.17,444.00)(1.000,2.500){2}{\rule{0.400pt}{0.602pt}}
\put(501.67,445){\rule{0.400pt}{0.964pt}}
\multiput(501.17,447.00)(1.000,-2.000){2}{\rule{0.400pt}{0.482pt}}
\put(502.67,445){\rule{0.400pt}{1.445pt}}
\multiput(502.17,445.00)(1.000,3.000){2}{\rule{0.400pt}{0.723pt}}
\put(503.67,444){\rule{0.400pt}{1.686pt}}
\multiput(503.17,447.50)(1.000,-3.500){2}{\rule{0.400pt}{0.843pt}}
\put(505,442.67){\rule{0.241pt}{0.400pt}}
\multiput(505.00,443.17)(0.500,-1.000){2}{\rule{0.120pt}{0.400pt}}
\put(505.67,443){\rule{0.400pt}{1.686pt}}
\multiput(505.17,443.00)(1.000,3.500){2}{\rule{0.400pt}{0.843pt}}
\put(419.0,269.0){\usebox{\plotpoint}}
\put(507.67,445){\rule{0.400pt}{1.204pt}}
\multiput(507.17,447.50)(1.000,-2.500){2}{\rule{0.400pt}{0.602pt}}
\put(509,443.67){\rule{0.241pt}{0.400pt}}
\multiput(509.00,444.17)(0.500,-1.000){2}{\rule{0.120pt}{0.400pt}}
\put(509.67,444){\rule{0.400pt}{1.686pt}}
\multiput(509.17,444.00)(1.000,3.500){2}{\rule{0.400pt}{0.843pt}}
\put(510.67,449){\rule{0.400pt}{0.482pt}}
\multiput(510.17,450.00)(1.000,-1.000){2}{\rule{0.400pt}{0.241pt}}
\put(511.67,443){\rule{0.400pt}{1.445pt}}
\multiput(511.17,446.00)(1.000,-3.000){2}{\rule{0.400pt}{0.723pt}}
\put(512.67,443){\rule{0.400pt}{0.964pt}}
\multiput(512.17,443.00)(1.000,2.000){2}{\rule{0.400pt}{0.482pt}}
\put(513.67,442){\rule{0.400pt}{1.204pt}}
\multiput(513.17,444.50)(1.000,-2.500){2}{\rule{0.400pt}{0.602pt}}
\put(515.17,442){\rule{0.400pt}{1.100pt}}
\multiput(514.17,442.00)(2.000,2.717){2}{\rule{0.400pt}{0.550pt}}
\put(516.67,442){\rule{0.400pt}{1.204pt}}
\multiput(516.17,444.50)(1.000,-2.500){2}{\rule{0.400pt}{0.602pt}}
\put(517.67,442){\rule{0.400pt}{1.445pt}}
\multiput(517.17,442.00)(1.000,3.000){2}{\rule{0.400pt}{0.723pt}}
\put(518.67,441){\rule{0.400pt}{1.686pt}}
\multiput(518.17,444.50)(1.000,-3.500){2}{\rule{0.400pt}{0.843pt}}
\put(519.67,441){\rule{0.400pt}{1.686pt}}
\multiput(519.17,441.00)(1.000,3.500){2}{\rule{0.400pt}{0.843pt}}
\put(520.67,443){\rule{0.400pt}{1.204pt}}
\multiput(520.17,445.50)(1.000,-2.500){2}{\rule{0.400pt}{0.602pt}}
\put(521.67,443){\rule{0.400pt}{0.964pt}}
\multiput(521.17,443.00)(1.000,2.000){2}{\rule{0.400pt}{0.482pt}}
\put(522.67,442){\rule{0.400pt}{1.204pt}}
\multiput(522.17,444.50)(1.000,-2.500){2}{\rule{0.400pt}{0.602pt}}
\put(523.67,442){\rule{0.400pt}{0.723pt}}
\multiput(523.17,442.00)(1.000,1.500){2}{\rule{0.400pt}{0.361pt}}
\put(524.67,437){\rule{0.400pt}{1.927pt}}
\multiput(524.17,441.00)(1.000,-4.000){2}{\rule{0.400pt}{0.964pt}}
\put(525.67,437){\rule{0.400pt}{1.686pt}}
\multiput(525.17,437.00)(1.000,3.500){2}{\rule{0.400pt}{0.843pt}}
\put(526.67,435){\rule{0.400pt}{2.168pt}}
\multiput(526.17,439.50)(1.000,-4.500){2}{\rule{0.400pt}{1.084pt}}
\put(527.67,435){\rule{0.400pt}{1.445pt}}
\multiput(527.17,435.00)(1.000,3.000){2}{\rule{0.400pt}{0.723pt}}
\put(528.67,437){\rule{0.400pt}{0.964pt}}
\multiput(528.17,439.00)(1.000,-2.000){2}{\rule{0.400pt}{0.482pt}}
\put(529.67,437){\rule{0.400pt}{0.723pt}}
\multiput(529.17,437.00)(1.000,1.500){2}{\rule{0.400pt}{0.361pt}}
\put(530.67,435){\rule{0.400pt}{1.204pt}}
\multiput(530.17,437.50)(1.000,-2.500){2}{\rule{0.400pt}{0.602pt}}
\put(507.0,450.0){\usebox{\plotpoint}}
\put(532.67,435){\rule{0.400pt}{1.445pt}}
\multiput(532.17,435.00)(1.000,3.000){2}{\rule{0.400pt}{0.723pt}}
\put(533.67,439){\rule{0.400pt}{0.482pt}}
\multiput(533.17,440.00)(1.000,-1.000){2}{\rule{0.400pt}{0.241pt}}
\put(534.67,433){\rule{0.400pt}{1.445pt}}
\multiput(534.17,436.00)(1.000,-3.000){2}{\rule{0.400pt}{0.723pt}}
\put(535.67,433){\rule{0.400pt}{1.445pt}}
\multiput(535.17,433.00)(1.000,3.000){2}{\rule{0.400pt}{0.723pt}}
\put(536.67,430){\rule{0.400pt}{2.168pt}}
\multiput(536.17,434.50)(1.000,-4.500){2}{\rule{0.400pt}{1.084pt}}
\put(537.67,430){\rule{0.400pt}{1.445pt}}
\multiput(537.17,430.00)(1.000,3.000){2}{\rule{0.400pt}{0.723pt}}
\put(538.67,427){\rule{0.400pt}{2.168pt}}
\multiput(538.17,431.50)(1.000,-4.500){2}{\rule{0.400pt}{1.084pt}}
\put(539.67,427){\rule{0.400pt}{2.168pt}}
\multiput(539.17,427.00)(1.000,4.500){2}{\rule{0.400pt}{1.084pt}}
\put(541.17,430){\rule{0.400pt}{1.300pt}}
\multiput(540.17,433.30)(2.000,-3.302){2}{\rule{0.400pt}{0.650pt}}
\put(542.67,426){\rule{0.400pt}{0.964pt}}
\multiput(542.17,428.00)(1.000,-2.000){2}{\rule{0.400pt}{0.482pt}}
\put(543.67,426){\rule{0.400pt}{2.409pt}}
\multiput(543.17,426.00)(1.000,5.000){2}{\rule{0.400pt}{1.204pt}}
\put(544.67,432){\rule{0.400pt}{0.964pt}}
\multiput(544.17,434.00)(1.000,-2.000){2}{\rule{0.400pt}{0.482pt}}
\put(545.67,432){\rule{0.400pt}{2.891pt}}
\multiput(545.17,432.00)(1.000,6.000){2}{\rule{0.400pt}{1.445pt}}
\put(532.0,435.0){\usebox{\plotpoint}}
\put(547.67,444){\rule{0.400pt}{3.854pt}}
\multiput(547.17,444.00)(1.000,8.000){2}{\rule{0.400pt}{1.927pt}}
\put(548.67,458){\rule{0.400pt}{0.482pt}}
\multiput(548.17,459.00)(1.000,-1.000){2}{\rule{0.400pt}{0.241pt}}
\put(549.67,458){\rule{0.400pt}{2.168pt}}
\multiput(549.17,458.00)(1.000,4.500){2}{\rule{0.400pt}{1.084pt}}
\put(550.67,463){\rule{0.400pt}{0.964pt}}
\multiput(550.17,465.00)(1.000,-2.000){2}{\rule{0.400pt}{0.482pt}}
\put(551.67,450){\rule{0.400pt}{3.132pt}}
\multiput(551.17,456.50)(1.000,-6.500){2}{\rule{0.400pt}{1.566pt}}
\put(552.67,450){\rule{0.400pt}{1.927pt}}
\multiput(552.17,450.00)(1.000,4.000){2}{\rule{0.400pt}{0.964pt}}
\put(553.67,458){\rule{0.400pt}{3.854pt}}
\multiput(553.17,458.00)(1.000,8.000){2}{\rule{0.400pt}{1.927pt}}
\put(555,473.67){\rule{0.241pt}{0.400pt}}
\multiput(555.00,473.17)(0.500,1.000){2}{\rule{0.120pt}{0.400pt}}
\put(555.67,448){\rule{0.400pt}{6.504pt}}
\multiput(555.17,461.50)(1.000,-13.500){2}{\rule{0.400pt}{3.252pt}}
\put(556.67,441){\rule{0.400pt}{1.686pt}}
\multiput(556.17,444.50)(1.000,-3.500){2}{\rule{0.400pt}{0.843pt}}
\put(557.67,441){\rule{0.400pt}{5.300pt}}
\multiput(557.17,441.00)(1.000,11.000){2}{\rule{0.400pt}{2.650pt}}
\put(558.67,460){\rule{0.400pt}{0.723pt}}
\multiput(558.17,461.50)(1.000,-1.500){2}{\rule{0.400pt}{0.361pt}}
\put(559.67,460){\rule{0.400pt}{4.336pt}}
\multiput(559.17,460.00)(1.000,9.000){2}{\rule{0.400pt}{2.168pt}}
\put(560.67,478){\rule{0.400pt}{0.964pt}}
\multiput(560.17,478.00)(1.000,2.000){2}{\rule{0.400pt}{0.482pt}}
\put(561.67,477){\rule{0.400pt}{1.204pt}}
\multiput(561.17,479.50)(1.000,-2.500){2}{\rule{0.400pt}{0.602pt}}
\put(562.67,477){\rule{0.400pt}{1.686pt}}
\multiput(562.17,477.00)(1.000,3.500){2}{\rule{0.400pt}{0.843pt}}
\put(563.67,468){\rule{0.400pt}{3.854pt}}
\multiput(563.17,476.00)(1.000,-8.000){2}{\rule{0.400pt}{1.927pt}}
\put(564.67,468){\rule{0.400pt}{0.964pt}}
\multiput(564.17,468.00)(1.000,2.000){2}{\rule{0.400pt}{0.482pt}}
\put(565.67,460){\rule{0.400pt}{2.891pt}}
\multiput(565.17,466.00)(1.000,-6.000){2}{\rule{0.400pt}{1.445pt}}
\put(567.17,460){\rule{0.400pt}{1.300pt}}
\multiput(566.17,460.00)(2.000,3.302){2}{\rule{0.400pt}{0.650pt}}
\put(568.67,442){\rule{0.400pt}{5.782pt}}
\multiput(568.17,454.00)(1.000,-12.000){2}{\rule{0.400pt}{2.891pt}}
\put(570,441.67){\rule{0.241pt}{0.400pt}}
\multiput(570.00,441.17)(0.500,1.000){2}{\rule{0.120pt}{0.400pt}}
\put(570.67,429){\rule{0.400pt}{3.373pt}}
\multiput(570.17,436.00)(1.000,-7.000){2}{\rule{0.400pt}{1.686pt}}
\put(571.67,429){\rule{0.400pt}{2.650pt}}
\multiput(571.17,429.00)(1.000,5.500){2}{\rule{0.400pt}{1.325pt}}
\put(572.67,440){\rule{0.400pt}{5.059pt}}
\multiput(572.17,440.00)(1.000,10.500){2}{\rule{0.400pt}{2.529pt}}
\put(573.67,461){\rule{0.400pt}{0.723pt}}
\multiput(573.17,461.00)(1.000,1.500){2}{\rule{0.400pt}{0.361pt}}
\put(574.67,455){\rule{0.400pt}{2.168pt}}
\multiput(574.17,459.50)(1.000,-4.500){2}{\rule{0.400pt}{1.084pt}}
\put(575.67,455){\rule{0.400pt}{1.927pt}}
\multiput(575.17,455.00)(1.000,4.000){2}{\rule{0.400pt}{0.964pt}}
\put(576.67,446){\rule{0.400pt}{4.095pt}}
\multiput(576.17,454.50)(1.000,-8.500){2}{\rule{0.400pt}{2.048pt}}
\put(577.67,446){\rule{0.400pt}{2.168pt}}
\multiput(577.17,446.00)(1.000,4.500){2}{\rule{0.400pt}{1.084pt}}
\put(578.67,455){\rule{0.400pt}{1.927pt}}
\multiput(578.17,455.00)(1.000,4.000){2}{\rule{0.400pt}{0.964pt}}
\put(579.67,460){\rule{0.400pt}{0.723pt}}
\multiput(579.17,461.50)(1.000,-1.500){2}{\rule{0.400pt}{0.361pt}}
\put(580.67,460){\rule{0.400pt}{4.336pt}}
\multiput(580.17,460.00)(1.000,9.000){2}{\rule{0.400pt}{2.168pt}}
\put(582,477.67){\rule{0.241pt}{0.400pt}}
\multiput(582.00,477.17)(0.500,1.000){2}{\rule{0.120pt}{0.400pt}}
\put(582.67,469){\rule{0.400pt}{2.409pt}}
\multiput(582.17,474.00)(1.000,-5.000){2}{\rule{0.400pt}{1.204pt}}
\put(583.67,469){\rule{0.400pt}{1.204pt}}
\multiput(583.17,469.00)(1.000,2.500){2}{\rule{0.400pt}{0.602pt}}
\put(584.67,466){\rule{0.400pt}{1.927pt}}
\multiput(584.17,470.00)(1.000,-4.000){2}{\rule{0.400pt}{0.964pt}}
\put(585.67,466){\rule{0.400pt}{3.132pt}}
\multiput(585.17,466.00)(1.000,6.500){2}{\rule{0.400pt}{1.566pt}}
\put(586.67,469){\rule{0.400pt}{2.409pt}}
\multiput(586.17,474.00)(1.000,-5.000){2}{\rule{0.400pt}{1.204pt}}
\put(587.67,469){\rule{0.400pt}{0.964pt}}
\multiput(587.17,469.00)(1.000,2.000){2}{\rule{0.400pt}{0.482pt}}
\put(588.67,473){\rule{0.400pt}{2.168pt}}
\multiput(588.17,473.00)(1.000,4.500){2}{\rule{0.400pt}{1.084pt}}
\put(590,480.67){\rule{0.241pt}{0.400pt}}
\multiput(590.00,481.17)(0.500,-1.000){2}{\rule{0.120pt}{0.400pt}}
\put(590.67,473){\rule{0.400pt}{1.927pt}}
\multiput(590.17,477.00)(1.000,-4.000){2}{\rule{0.400pt}{0.964pt}}
\put(591.67,473){\rule{0.400pt}{0.964pt}}
\multiput(591.17,473.00)(1.000,2.000){2}{\rule{0.400pt}{0.482pt}}
\put(593.17,477){\rule{0.400pt}{2.900pt}}
\multiput(592.17,477.00)(2.000,7.981){2}{\rule{0.400pt}{1.450pt}}
\put(594.67,485){\rule{0.400pt}{1.445pt}}
\multiput(594.17,488.00)(1.000,-3.000){2}{\rule{0.400pt}{0.723pt}}
\put(595.67,485){\rule{0.400pt}{2.409pt}}
\multiput(595.17,485.00)(1.000,5.000){2}{\rule{0.400pt}{1.204pt}}
\put(597,494.67){\rule{0.241pt}{0.400pt}}
\multiput(597.00,494.17)(0.500,1.000){2}{\rule{0.120pt}{0.400pt}}
\put(597.67,487){\rule{0.400pt}{2.168pt}}
\multiput(597.17,491.50)(1.000,-4.500){2}{\rule{0.400pt}{1.084pt}}
\put(598.67,487){\rule{0.400pt}{0.723pt}}
\multiput(598.17,487.00)(1.000,1.500){2}{\rule{0.400pt}{0.361pt}}
\put(599.67,490){\rule{0.400pt}{4.577pt}}
\multiput(599.17,490.00)(1.000,9.500){2}{\rule{0.400pt}{2.289pt}}
\put(600.67,503){\rule{0.400pt}{1.445pt}}
\multiput(600.17,506.00)(1.000,-3.000){2}{\rule{0.400pt}{0.723pt}}
\put(601.67,503){\rule{0.400pt}{2.409pt}}
\multiput(601.17,503.00)(1.000,5.000){2}{\rule{0.400pt}{1.204pt}}
\put(602.67,508){\rule{0.400pt}{1.204pt}}
\multiput(602.17,510.50)(1.000,-2.500){2}{\rule{0.400pt}{0.602pt}}
\put(603.67,508){\rule{0.400pt}{2.891pt}}
\multiput(603.17,508.00)(1.000,6.000){2}{\rule{0.400pt}{1.445pt}}
\put(605,519.67){\rule{0.241pt}{0.400pt}}
\multiput(605.00,519.17)(0.500,1.000){2}{\rule{0.120pt}{0.400pt}}
\put(605.67,516){\rule{0.400pt}{1.204pt}}
\multiput(605.17,518.50)(1.000,-2.500){2}{\rule{0.400pt}{0.602pt}}
\put(606.67,516){\rule{0.400pt}{2.409pt}}
\multiput(606.17,516.00)(1.000,5.000){2}{\rule{0.400pt}{1.204pt}}
\put(607.67,518){\rule{0.400pt}{1.927pt}}
\multiput(607.17,522.00)(1.000,-4.000){2}{\rule{0.400pt}{0.964pt}}
\put(609,517.67){\rule{0.241pt}{0.400pt}}
\multiput(609.00,517.17)(0.500,1.000){2}{\rule{0.120pt}{0.400pt}}
\put(609.67,519){\rule{0.400pt}{1.686pt}}
\multiput(609.17,519.00)(1.000,3.500){2}{\rule{0.400pt}{0.843pt}}
\put(610.67,524){\rule{0.400pt}{0.482pt}}
\multiput(610.17,525.00)(1.000,-1.000){2}{\rule{0.400pt}{0.241pt}}
\put(611.67,524){\rule{0.400pt}{2.168pt}}
\multiput(611.17,524.00)(1.000,4.500){2}{\rule{0.400pt}{1.084pt}}
\put(612.67,525){\rule{0.400pt}{1.927pt}}
\multiput(612.17,529.00)(1.000,-4.000){2}{\rule{0.400pt}{0.964pt}}
\put(613.67,525){\rule{0.400pt}{3.132pt}}
\multiput(613.17,525.00)(1.000,6.500){2}{\rule{0.400pt}{1.566pt}}
\put(614.67,536){\rule{0.400pt}{0.482pt}}
\multiput(614.17,537.00)(1.000,-1.000){2}{\rule{0.400pt}{0.241pt}}
\put(615.67,528){\rule{0.400pt}{1.927pt}}
\multiput(615.17,532.00)(1.000,-4.000){2}{\rule{0.400pt}{0.964pt}}
\put(616.67,528){\rule{0.400pt}{1.686pt}}
\multiput(616.17,528.00)(1.000,3.500){2}{\rule{0.400pt}{0.843pt}}
\put(617.67,535){\rule{0.400pt}{1.686pt}}
\multiput(617.17,535.00)(1.000,3.500){2}{\rule{0.400pt}{0.843pt}}
\put(619.17,537){\rule{0.400pt}{1.100pt}}
\multiput(618.17,539.72)(2.000,-2.717){2}{\rule{0.400pt}{0.550pt}}
\put(620.67,537){\rule{0.400pt}{1.686pt}}
\multiput(620.17,537.00)(1.000,3.500){2}{\rule{0.400pt}{0.843pt}}
\put(621.67,539){\rule{0.400pt}{1.204pt}}
\multiput(621.17,541.50)(1.000,-2.500){2}{\rule{0.400pt}{0.602pt}}
\put(622.67,539){\rule{0.400pt}{1.445pt}}
\multiput(622.17,539.00)(1.000,3.000){2}{\rule{0.400pt}{0.723pt}}
\put(623.67,540){\rule{0.400pt}{1.204pt}}
\multiput(623.17,542.50)(1.000,-2.500){2}{\rule{0.400pt}{0.602pt}}
\put(624.67,540){\rule{0.400pt}{1.686pt}}
\multiput(624.17,540.00)(1.000,3.500){2}{\rule{0.400pt}{0.843pt}}
\put(625.67,545){\rule{0.400pt}{0.482pt}}
\multiput(625.17,546.00)(1.000,-1.000){2}{\rule{0.400pt}{0.241pt}}
\put(626.67,537){\rule{0.400pt}{1.927pt}}
\multiput(626.17,541.00)(1.000,-4.000){2}{\rule{0.400pt}{0.964pt}}
\put(627.67,537){\rule{0.400pt}{0.482pt}}
\multiput(627.17,537.00)(1.000,1.000){2}{\rule{0.400pt}{0.241pt}}
\put(628.67,539){\rule{0.400pt}{2.891pt}}
\multiput(628.17,539.00)(1.000,6.000){2}{\rule{0.400pt}{1.445pt}}
\put(629.67,540){\rule{0.400pt}{2.650pt}}
\multiput(629.17,545.50)(1.000,-5.500){2}{\rule{0.400pt}{1.325pt}}
\put(630.67,540){\rule{0.400pt}{2.409pt}}
\multiput(630.17,540.00)(1.000,5.000){2}{\rule{0.400pt}{1.204pt}}
\put(631.67,541){\rule{0.400pt}{2.168pt}}
\multiput(631.17,545.50)(1.000,-4.500){2}{\rule{0.400pt}{1.084pt}}
\put(632.67,541){\rule{0.400pt}{2.409pt}}
\multiput(632.17,541.00)(1.000,5.000){2}{\rule{0.400pt}{1.204pt}}
\put(634,550.67){\rule{0.241pt}{0.400pt}}
\multiput(634.00,550.17)(0.500,1.000){2}{\rule{0.120pt}{0.400pt}}
\put(634.67,544){\rule{0.400pt}{1.927pt}}
\multiput(634.17,548.00)(1.000,-4.000){2}{\rule{0.400pt}{0.964pt}}
\put(635.67,542){\rule{0.400pt}{0.482pt}}
\multiput(635.17,543.00)(1.000,-1.000){2}{\rule{0.400pt}{0.241pt}}
\put(636.67,542){\rule{0.400pt}{3.132pt}}
\multiput(636.17,542.00)(1.000,6.500){2}{\rule{0.400pt}{1.566pt}}
\put(637.67,555){\rule{0.400pt}{0.964pt}}
\multiput(637.17,555.00)(1.000,2.000){2}{\rule{0.400pt}{0.482pt}}
\put(638.67,551){\rule{0.400pt}{1.927pt}}
\multiput(638.17,555.00)(1.000,-4.000){2}{\rule{0.400pt}{0.964pt}}
\put(639.67,549){\rule{0.400pt}{0.482pt}}
\multiput(639.17,550.00)(1.000,-1.000){2}{\rule{0.400pt}{0.241pt}}
\put(640.67,549){\rule{0.400pt}{2.409pt}}
\multiput(640.17,549.00)(1.000,5.000){2}{\rule{0.400pt}{1.204pt}}
\put(642,558.67){\rule{0.241pt}{0.400pt}}
\multiput(642.00,558.17)(0.500,1.000){2}{\rule{0.120pt}{0.400pt}}
\put(642.67,552){\rule{0.400pt}{1.927pt}}
\multiput(642.17,556.00)(1.000,-4.000){2}{\rule{0.400pt}{0.964pt}}
\put(643.67,552){\rule{0.400pt}{2.409pt}}
\multiput(643.17,552.00)(1.000,5.000){2}{\rule{0.400pt}{1.204pt}}
\put(645.17,554){\rule{0.400pt}{1.700pt}}
\multiput(644.17,558.47)(2.000,-4.472){2}{\rule{0.400pt}{0.850pt}}
\put(646.67,554){\rule{0.400pt}{2.409pt}}
\multiput(646.17,554.00)(1.000,5.000){2}{\rule{0.400pt}{1.204pt}}
\put(647.67,553){\rule{0.400pt}{2.650pt}}
\multiput(647.17,558.50)(1.000,-5.500){2}{\rule{0.400pt}{1.325pt}}
\put(648.67,553){\rule{0.400pt}{1.686pt}}
\multiput(648.17,553.00)(1.000,3.500){2}{\rule{0.400pt}{0.843pt}}
\put(649.67,560){\rule{0.400pt}{2.409pt}}
\multiput(649.17,560.00)(1.000,5.000){2}{\rule{0.400pt}{1.204pt}}
\put(651,569.67){\rule{0.241pt}{0.400pt}}
\multiput(651.00,569.17)(0.500,1.000){2}{\rule{0.120pt}{0.400pt}}
\put(651.67,569){\rule{0.400pt}{0.482pt}}
\multiput(651.17,570.00)(1.000,-1.000){2}{\rule{0.400pt}{0.241pt}}
\put(652.67,569){\rule{0.400pt}{0.964pt}}
\multiput(652.17,569.00)(1.000,2.000){2}{\rule{0.400pt}{0.482pt}}
\put(653.67,559){\rule{0.400pt}{3.373pt}}
\multiput(653.17,566.00)(1.000,-7.000){2}{\rule{0.400pt}{1.686pt}}
\put(654.67,559){\rule{0.400pt}{1.927pt}}
\multiput(654.17,559.00)(1.000,4.000){2}{\rule{0.400pt}{0.964pt}}
\put(655.67,558){\rule{0.400pt}{2.168pt}}
\multiput(655.17,562.50)(1.000,-4.500){2}{\rule{0.400pt}{1.084pt}}
\put(656.67,558){\rule{0.400pt}{3.854pt}}
\multiput(656.17,558.00)(1.000,8.000){2}{\rule{0.400pt}{1.927pt}}
\put(657.67,563){\rule{0.400pt}{2.650pt}}
\multiput(657.17,568.50)(1.000,-5.500){2}{\rule{0.400pt}{1.325pt}}
\put(658.67,563){\rule{0.400pt}{3.614pt}}
\multiput(658.17,563.00)(1.000,7.500){2}{\rule{0.400pt}{1.807pt}}
\put(659.67,565){\rule{0.400pt}{3.132pt}}
\multiput(659.17,571.50)(1.000,-6.500){2}{\rule{0.400pt}{1.566pt}}
\put(660.67,565){\rule{0.400pt}{1.927pt}}
\multiput(660.17,565.00)(1.000,4.000){2}{\rule{0.400pt}{0.964pt}}
\put(661.67,564){\rule{0.400pt}{2.168pt}}
\multiput(661.17,568.50)(1.000,-4.500){2}{\rule{0.400pt}{1.084pt}}
\put(662.67,564){\rule{0.400pt}{4.095pt}}
\multiput(662.17,564.00)(1.000,8.500){2}{\rule{0.400pt}{2.048pt}}
\put(663.67,563){\rule{0.400pt}{4.336pt}}
\multiput(663.17,572.00)(1.000,-9.000){2}{\rule{0.400pt}{2.168pt}}
\put(664.67,563){\rule{0.400pt}{0.723pt}}
\multiput(664.17,563.00)(1.000,1.500){2}{\rule{0.400pt}{0.361pt}}
\put(665.67,566){\rule{0.400pt}{4.095pt}}
\multiput(665.17,566.00)(1.000,8.500){2}{\rule{0.400pt}{2.048pt}}
\put(666.67,583){\rule{0.400pt}{1.204pt}}
\multiput(666.17,583.00)(1.000,2.500){2}{\rule{0.400pt}{0.602pt}}
\put(667.67,574){\rule{0.400pt}{3.373pt}}
\multiput(667.17,581.00)(1.000,-7.000){2}{\rule{0.400pt}{1.686pt}}
\put(668.67,574){\rule{0.400pt}{0.964pt}}
\multiput(668.17,574.00)(1.000,2.000){2}{\rule{0.400pt}{0.482pt}}
\put(669.67,570){\rule{0.400pt}{1.927pt}}
\multiput(669.17,574.00)(1.000,-4.000){2}{\rule{0.400pt}{0.964pt}}
\put(671.17,570){\rule{0.400pt}{1.500pt}}
\multiput(670.17,570.00)(2.000,3.887){2}{\rule{0.400pt}{0.750pt}}
\put(672.67,577){\rule{0.400pt}{2.168pt}}
\multiput(672.17,577.00)(1.000,4.500){2}{\rule{0.400pt}{1.084pt}}
\put(673.67,579){\rule{0.400pt}{1.686pt}}
\multiput(673.17,582.50)(1.000,-3.500){2}{\rule{0.400pt}{0.843pt}}
\put(674.67,579){\rule{0.400pt}{2.891pt}}
\multiput(674.17,579.00)(1.000,6.000){2}{\rule{0.400pt}{1.445pt}}
\put(675.67,586){\rule{0.400pt}{1.204pt}}
\multiput(675.17,588.50)(1.000,-2.500){2}{\rule{0.400pt}{0.602pt}}
\put(676.67,579){\rule{0.400pt}{1.686pt}}
\multiput(676.17,582.50)(1.000,-3.500){2}{\rule{0.400pt}{0.843pt}}
\put(677.67,579){\rule{0.400pt}{4.095pt}}
\multiput(677.17,579.00)(1.000,8.500){2}{\rule{0.400pt}{2.048pt}}
\put(678.67,582){\rule{0.400pt}{3.373pt}}
\multiput(678.17,589.00)(1.000,-7.000){2}{\rule{0.400pt}{1.686pt}}
\put(679.67,582){\rule{0.400pt}{3.854pt}}
\multiput(679.17,582.00)(1.000,8.000){2}{\rule{0.400pt}{1.927pt}}
\put(680.67,587){\rule{0.400pt}{2.650pt}}
\multiput(680.17,592.50)(1.000,-5.500){2}{\rule{0.400pt}{1.325pt}}
\put(681.67,587){\rule{0.400pt}{1.686pt}}
\multiput(681.17,587.00)(1.000,3.500){2}{\rule{0.400pt}{0.843pt}}
\put(682.67,586){\rule{0.400pt}{1.927pt}}
\multiput(682.17,590.00)(1.000,-4.000){2}{\rule{0.400pt}{0.964pt}}
\put(683.67,586){\rule{0.400pt}{2.409pt}}
\multiput(683.17,586.00)(1.000,5.000){2}{\rule{0.400pt}{1.204pt}}
\put(684.67,588){\rule{0.400pt}{1.927pt}}
\multiput(684.17,592.00)(1.000,-4.000){2}{\rule{0.400pt}{0.964pt}}
\put(685.67,588){\rule{0.400pt}{2.168pt}}
\multiput(685.17,588.00)(1.000,4.500){2}{\rule{0.400pt}{1.084pt}}
\put(686.67,590){\rule{0.400pt}{1.686pt}}
\multiput(686.17,593.50)(1.000,-3.500){2}{\rule{0.400pt}{0.843pt}}
\put(687.67,590){\rule{0.400pt}{2.650pt}}
\multiput(687.17,590.00)(1.000,5.500){2}{\rule{0.400pt}{1.325pt}}
\put(688.67,596){\rule{0.400pt}{1.204pt}}
\multiput(688.17,598.50)(1.000,-2.500){2}{\rule{0.400pt}{0.602pt}}
\put(689.67,596){\rule{0.400pt}{4.818pt}}
\multiput(689.17,596.00)(1.000,10.000){2}{\rule{0.400pt}{2.409pt}}
\put(690.67,595){\rule{0.400pt}{5.059pt}}
\multiput(690.17,605.50)(1.000,-10.500){2}{\rule{0.400pt}{2.529pt}}
\put(691.67,595){\rule{0.400pt}{1.927pt}}
\multiput(691.17,595.00)(1.000,4.000){2}{\rule{0.400pt}{0.964pt}}
\put(692.67,585){\rule{0.400pt}{4.336pt}}
\multiput(692.17,594.00)(1.000,-9.000){2}{\rule{0.400pt}{2.168pt}}
\put(693.67,585){\rule{0.400pt}{2.891pt}}
\multiput(693.17,585.00)(1.000,6.000){2}{\rule{0.400pt}{1.445pt}}
\put(694.67,597){\rule{0.400pt}{2.650pt}}
\multiput(694.17,597.00)(1.000,5.500){2}{\rule{0.400pt}{1.325pt}}
\put(695.67,603){\rule{0.400pt}{1.204pt}}
\multiput(695.17,605.50)(1.000,-2.500){2}{\rule{0.400pt}{0.602pt}}
\put(697.17,603){\rule{0.400pt}{2.900pt}}
\multiput(696.17,603.00)(2.000,7.981){2}{\rule{0.400pt}{1.450pt}}
\put(699,615.67){\rule{0.241pt}{0.400pt}}
\multiput(699.00,616.17)(0.500,-1.000){2}{\rule{0.120pt}{0.400pt}}
\put(699.67,593){\rule{0.400pt}{5.541pt}}
\multiput(699.17,604.50)(1.000,-11.500){2}{\rule{0.400pt}{2.770pt}}
\put(547.0,444.0){\usebox{\plotpoint}}
\put(701.67,593){\rule{0.400pt}{4.095pt}}
\multiput(701.17,593.00)(1.000,8.500){2}{\rule{0.400pt}{2.048pt}}
\put(702.67,607){\rule{0.400pt}{0.723pt}}
\multiput(702.17,608.50)(1.000,-1.500){2}{\rule{0.400pt}{0.361pt}}
\put(703.67,607){\rule{0.400pt}{1.927pt}}
\multiput(703.17,607.00)(1.000,4.000){2}{\rule{0.400pt}{0.964pt}}
\put(704.67,615){\rule{0.400pt}{0.723pt}}
\multiput(704.17,615.00)(1.000,1.500){2}{\rule{0.400pt}{0.361pt}}
\put(705.67,612){\rule{0.400pt}{1.445pt}}
\multiput(705.17,615.00)(1.000,-3.000){2}{\rule{0.400pt}{0.723pt}}
\put(706.67,612){\rule{0.400pt}{2.409pt}}
\multiput(706.17,612.00)(1.000,5.000){2}{\rule{0.400pt}{1.204pt}}
\put(707.67,602){\rule{0.400pt}{4.818pt}}
\multiput(707.17,612.00)(1.000,-10.000){2}{\rule{0.400pt}{2.409pt}}
\put(709,601.67){\rule{0.241pt}{0.400pt}}
\multiput(709.00,601.17)(0.500,1.000){2}{\rule{0.120pt}{0.400pt}}
\put(709.67,603){\rule{0.400pt}{3.132pt}}
\multiput(709.17,603.00)(1.000,6.500){2}{\rule{0.400pt}{1.566pt}}
\put(710.67,609){\rule{0.400pt}{1.686pt}}
\multiput(710.17,612.50)(1.000,-3.500){2}{\rule{0.400pt}{0.843pt}}
\put(711.67,609){\rule{0.400pt}{3.132pt}}
\multiput(711.17,609.00)(1.000,6.500){2}{\rule{0.400pt}{1.566pt}}
\put(712.67,611){\rule{0.400pt}{2.650pt}}
\multiput(712.17,616.50)(1.000,-5.500){2}{\rule{0.400pt}{1.325pt}}
\put(713.67,611){\rule{0.400pt}{3.132pt}}
\multiput(713.17,611.00)(1.000,6.500){2}{\rule{0.400pt}{1.566pt}}
\put(714.67,624){\rule{0.400pt}{0.482pt}}
\multiput(714.17,624.00)(1.000,1.000){2}{\rule{0.400pt}{0.241pt}}
\put(715.67,618){\rule{0.400pt}{1.927pt}}
\multiput(715.17,622.00)(1.000,-4.000){2}{\rule{0.400pt}{0.964pt}}
\put(716.67,616){\rule{0.400pt}{0.482pt}}
\multiput(716.17,617.00)(1.000,-1.000){2}{\rule{0.400pt}{0.241pt}}
\put(717.67,616){\rule{0.400pt}{2.409pt}}
\multiput(717.17,616.00)(1.000,5.000){2}{\rule{0.400pt}{1.204pt}}
\put(718.67,616){\rule{0.400pt}{2.409pt}}
\multiput(718.17,621.00)(1.000,-5.000){2}{\rule{0.400pt}{1.204pt}}
\put(719.67,616){\rule{0.400pt}{4.577pt}}
\multiput(719.17,616.00)(1.000,9.500){2}{\rule{0.400pt}{2.289pt}}
\put(720.67,625){\rule{0.400pt}{2.409pt}}
\multiput(720.17,630.00)(1.000,-5.000){2}{\rule{0.400pt}{1.204pt}}
\put(721.67,613){\rule{0.400pt}{2.891pt}}
\multiput(721.17,619.00)(1.000,-6.000){2}{\rule{0.400pt}{1.445pt}}
\put(723.17,613){\rule{0.400pt}{0.900pt}}
\multiput(722.17,613.00)(2.000,2.132){2}{\rule{0.400pt}{0.450pt}}
\put(724.67,617){\rule{0.400pt}{2.409pt}}
\multiput(724.17,617.00)(1.000,5.000){2}{\rule{0.400pt}{1.204pt}}
\put(726,625.67){\rule{0.241pt}{0.400pt}}
\multiput(726.00,626.17)(0.500,-1.000){2}{\rule{0.120pt}{0.400pt}}
\put(726.67,626){\rule{0.400pt}{2.409pt}}
\multiput(726.17,626.00)(1.000,5.000){2}{\rule{0.400pt}{1.204pt}}
\put(728,635.67){\rule{0.241pt}{0.400pt}}
\multiput(728.00,635.17)(0.500,1.000){2}{\rule{0.120pt}{0.400pt}}
\put(728.67,624){\rule{0.400pt}{3.132pt}}
\multiput(728.17,630.50)(1.000,-6.500){2}{\rule{0.400pt}{1.566pt}}
\put(730,623.67){\rule{0.241pt}{0.400pt}}
\multiput(730.00,623.17)(0.500,1.000){2}{\rule{0.120pt}{0.400pt}}
\put(730.67,625){\rule{0.400pt}{1.686pt}}
\multiput(730.17,625.00)(1.000,3.500){2}{\rule{0.400pt}{0.843pt}}
\put(731.67,627){\rule{0.400pt}{1.204pt}}
\multiput(731.17,629.50)(1.000,-2.500){2}{\rule{0.400pt}{0.602pt}}
\put(732.67,627){\rule{0.400pt}{2.409pt}}
\multiput(732.17,627.00)(1.000,5.000){2}{\rule{0.400pt}{1.204pt}}
\put(734,635.67){\rule{0.241pt}{0.400pt}}
\multiput(734.00,636.17)(0.500,-1.000){2}{\rule{0.120pt}{0.400pt}}
\put(734.67,631){\rule{0.400pt}{1.204pt}}
\multiput(734.17,633.50)(1.000,-2.500){2}{\rule{0.400pt}{0.602pt}}
\put(736,630.67){\rule{0.241pt}{0.400pt}}
\multiput(736.00,630.17)(0.500,1.000){2}{\rule{0.120pt}{0.400pt}}
\put(736.67,632){\rule{0.400pt}{1.686pt}}
\multiput(736.17,632.00)(1.000,3.500){2}{\rule{0.400pt}{0.843pt}}
\put(737.67,631){\rule{0.400pt}{1.927pt}}
\multiput(737.17,635.00)(1.000,-4.000){2}{\rule{0.400pt}{0.964pt}}
\put(738.67,631){\rule{0.400pt}{3.373pt}}
\multiput(738.17,631.00)(1.000,7.000){2}{\rule{0.400pt}{1.686pt}}
\put(739.67,643){\rule{0.400pt}{0.482pt}}
\multiput(739.17,644.00)(1.000,-1.000){2}{\rule{0.400pt}{0.241pt}}
\put(740.67,636){\rule{0.400pt}{1.686pt}}
\multiput(740.17,639.50)(1.000,-3.500){2}{\rule{0.400pt}{0.843pt}}
\put(741.67,636){\rule{0.400pt}{1.445pt}}
\multiput(741.17,636.00)(1.000,3.000){2}{\rule{0.400pt}{0.723pt}}
\put(742.67,635){\rule{0.400pt}{1.686pt}}
\multiput(742.17,638.50)(1.000,-3.500){2}{\rule{0.400pt}{0.843pt}}
\put(744,634.67){\rule{0.241pt}{0.400pt}}
\multiput(744.00,634.17)(0.500,1.000){2}{\rule{0.120pt}{0.400pt}}
\put(744.67,636){\rule{0.400pt}{1.686pt}}
\multiput(744.17,636.00)(1.000,3.500){2}{\rule{0.400pt}{0.843pt}}
\put(745.67,643){\rule{0.400pt}{0.482pt}}
\multiput(745.17,643.00)(1.000,1.000){2}{\rule{0.400pt}{0.241pt}}
\put(746.67,640){\rule{0.400pt}{1.204pt}}
\multiput(746.17,642.50)(1.000,-2.500){2}{\rule{0.400pt}{0.602pt}}
\put(701.0,593.0){\usebox{\plotpoint}}
\put(749.17,640){\rule{0.400pt}{2.300pt}}
\multiput(748.17,640.00)(2.000,6.226){2}{\rule{0.400pt}{1.150pt}}
\put(750.67,649){\rule{0.400pt}{0.482pt}}
\multiput(750.17,650.00)(1.000,-1.000){2}{\rule{0.400pt}{0.241pt}}
\put(751.67,644){\rule{0.400pt}{1.204pt}}
\multiput(751.17,646.50)(1.000,-2.500){2}{\rule{0.400pt}{0.602pt}}
\put(752.67,644){\rule{0.400pt}{0.482pt}}
\multiput(752.17,644.00)(1.000,1.000){2}{\rule{0.400pt}{0.241pt}}
\put(753.67,646){\rule{0.400pt}{1.445pt}}
\multiput(753.17,646.00)(1.000,3.000){2}{\rule{0.400pt}{0.723pt}}
\put(754.67,648){\rule{0.400pt}{0.964pt}}
\multiput(754.17,650.00)(1.000,-2.000){2}{\rule{0.400pt}{0.482pt}}
\put(755.67,648){\rule{0.400pt}{1.686pt}}
\multiput(755.17,648.00)(1.000,3.500){2}{\rule{0.400pt}{0.843pt}}
\put(756.67,650){\rule{0.400pt}{1.204pt}}
\multiput(756.17,652.50)(1.000,-2.500){2}{\rule{0.400pt}{0.602pt}}
\put(757.67,650){\rule{0.400pt}{1.686pt}}
\multiput(757.17,650.00)(1.000,3.500){2}{\rule{0.400pt}{0.843pt}}
\put(758.67,657){\rule{0.400pt}{0.482pt}}
\multiput(758.17,657.00)(1.000,1.000){2}{\rule{0.400pt}{0.241pt}}
\put(759.67,648){\rule{0.400pt}{2.650pt}}
\multiput(759.17,653.50)(1.000,-5.500){2}{\rule{0.400pt}{1.325pt}}
\put(748.0,640.0){\usebox{\plotpoint}}
\put(761.67,648){\rule{0.400pt}{2.650pt}}
\multiput(761.17,648.00)(1.000,5.500){2}{\rule{0.400pt}{1.325pt}}
\put(763,658.67){\rule{0.241pt}{0.400pt}}
\multiput(763.00,658.17)(0.500,1.000){2}{\rule{0.120pt}{0.400pt}}
\put(763.67,652){\rule{0.400pt}{1.927pt}}
\multiput(763.17,656.00)(1.000,-4.000){2}{\rule{0.400pt}{0.964pt}}
\put(764.67,647){\rule{0.400pt}{1.204pt}}
\multiput(764.17,649.50)(1.000,-2.500){2}{\rule{0.400pt}{0.602pt}}
\put(765.67,647){\rule{0.400pt}{2.409pt}}
\multiput(765.17,647.00)(1.000,5.000){2}{\rule{0.400pt}{1.204pt}}
\put(766.67,655){\rule{0.400pt}{0.482pt}}
\multiput(766.17,656.00)(1.000,-1.000){2}{\rule{0.400pt}{0.241pt}}
\put(767.67,655){\rule{0.400pt}{2.409pt}}
\multiput(767.17,655.00)(1.000,5.000){2}{\rule{0.400pt}{1.204pt}}
\put(769,664.67){\rule{0.241pt}{0.400pt}}
\multiput(769.00,664.17)(0.500,1.000){2}{\rule{0.120pt}{0.400pt}}
\put(769.67,659){\rule{0.400pt}{1.686pt}}
\multiput(769.17,662.50)(1.000,-3.500){2}{\rule{0.400pt}{0.843pt}}
\put(771,658.67){\rule{0.241pt}{0.400pt}}
\multiput(771.00,658.17)(0.500,1.000){2}{\rule{0.120pt}{0.400pt}}
\put(771.67,660){\rule{0.400pt}{2.409pt}}
\multiput(771.17,660.00)(1.000,5.000){2}{\rule{0.400pt}{1.204pt}}
\put(772.67,670){\rule{0.400pt}{0.482pt}}
\multiput(772.17,670.00)(1.000,1.000){2}{\rule{0.400pt}{0.241pt}}
\put(773.67,661){\rule{0.400pt}{2.650pt}}
\multiput(773.17,666.50)(1.000,-5.500){2}{\rule{0.400pt}{1.325pt}}
\put(775,659.67){\rule{0.482pt}{0.400pt}}
\multiput(775.00,660.17)(1.000,-1.000){2}{\rule{0.241pt}{0.400pt}}
\put(776.67,660){\rule{0.400pt}{2.168pt}}
\multiput(776.17,660.00)(1.000,4.500){2}{\rule{0.400pt}{1.084pt}}
\put(777.67,669){\rule{0.400pt}{0.723pt}}
\multiput(777.17,669.00)(1.000,1.500){2}{\rule{0.400pt}{0.361pt}}
\put(778.67,663){\rule{0.400pt}{2.168pt}}
\multiput(778.17,667.50)(1.000,-4.500){2}{\rule{0.400pt}{1.084pt}}
\put(779.67,663){\rule{0.400pt}{3.854pt}}
\multiput(779.17,663.00)(1.000,8.000){2}{\rule{0.400pt}{1.927pt}}
\put(780.67,668){\rule{0.400pt}{2.650pt}}
\multiput(780.17,673.50)(1.000,-5.500){2}{\rule{0.400pt}{1.325pt}}
\put(781.67,668){\rule{0.400pt}{1.927pt}}
\multiput(781.17,668.00)(1.000,4.000){2}{\rule{0.400pt}{0.964pt}}
\put(782.67,667){\rule{0.400pt}{2.168pt}}
\multiput(782.17,671.50)(1.000,-4.500){2}{\rule{0.400pt}{1.084pt}}
\put(783.67,667){\rule{0.400pt}{3.373pt}}
\multiput(783.17,667.00)(1.000,7.000){2}{\rule{0.400pt}{1.686pt}}
\put(784.67,673){\rule{0.400pt}{1.927pt}}
\multiput(784.17,677.00)(1.000,-4.000){2}{\rule{0.400pt}{0.964pt}}
\put(785.67,673){\rule{0.400pt}{3.132pt}}
\multiput(785.17,673.00)(1.000,6.500){2}{\rule{0.400pt}{1.566pt}}
\put(786.67,675){\rule{0.400pt}{2.650pt}}
\multiput(786.17,680.50)(1.000,-5.500){2}{\rule{0.400pt}{1.325pt}}
\put(787.67,675){\rule{0.400pt}{0.482pt}}
\multiput(787.17,675.00)(1.000,1.000){2}{\rule{0.400pt}{0.241pt}}
\put(788.67,677){\rule{0.400pt}{3.614pt}}
\multiput(788.17,677.00)(1.000,7.500){2}{\rule{0.400pt}{1.807pt}}
\put(789.67,678){\rule{0.400pt}{3.373pt}}
\multiput(789.17,685.00)(1.000,-7.000){2}{\rule{0.400pt}{1.686pt}}
\put(790.67,678){\rule{0.400pt}{3.373pt}}
\multiput(790.17,678.00)(1.000,7.000){2}{\rule{0.400pt}{1.686pt}}
\put(791.67,690){\rule{0.400pt}{0.482pt}}
\multiput(791.17,691.00)(1.000,-1.000){2}{\rule{0.400pt}{0.241pt}}
\put(792.67,682){\rule{0.400pt}{1.927pt}}
\multiput(792.17,686.00)(1.000,-4.000){2}{\rule{0.400pt}{0.964pt}}
\put(793.67,682){\rule{0.400pt}{3.132pt}}
\multiput(793.17,682.00)(1.000,6.500){2}{\rule{0.400pt}{1.566pt}}
\put(794.67,685){\rule{0.400pt}{2.409pt}}
\multiput(794.17,690.00)(1.000,-5.000){2}{\rule{0.400pt}{1.204pt}}
\put(795.67,685){\rule{0.400pt}{1.445pt}}
\multiput(795.17,685.00)(1.000,3.000){2}{\rule{0.400pt}{0.723pt}}
\put(796.67,687){\rule{0.400pt}{0.964pt}}
\multiput(796.17,689.00)(1.000,-2.000){2}{\rule{0.400pt}{0.482pt}}
\put(797.67,687){\rule{0.400pt}{1.927pt}}
\multiput(797.17,687.00)(1.000,4.000){2}{\rule{0.400pt}{0.964pt}}
\put(798.67,690){\rule{0.400pt}{1.204pt}}
\multiput(798.17,692.50)(1.000,-2.500){2}{\rule{0.400pt}{0.602pt}}
\put(799.67,688){\rule{0.400pt}{0.482pt}}
\multiput(799.17,689.00)(1.000,-1.000){2}{\rule{0.400pt}{0.241pt}}
\put(801.17,688){\rule{0.400pt}{3.300pt}}
\multiput(800.17,688.00)(2.000,9.151){2}{\rule{0.400pt}{1.650pt}}
\put(802.67,704){\rule{0.400pt}{1.927pt}}
\multiput(802.17,704.00)(1.000,4.000){2}{\rule{0.400pt}{0.964pt}}
\put(803.67,698){\rule{0.400pt}{3.373pt}}
\multiput(803.17,705.00)(1.000,-7.000){2}{\rule{0.400pt}{1.686pt}}
\put(804.67,698){\rule{0.400pt}{3.373pt}}
\multiput(804.17,698.00)(1.000,7.000){2}{\rule{0.400pt}{1.686pt}}
\put(805.67,698){\rule{0.400pt}{3.373pt}}
\multiput(805.17,705.00)(1.000,-7.000){2}{\rule{0.400pt}{1.686pt}}
\put(807,697.67){\rule{0.241pt}{0.400pt}}
\multiput(807.00,697.17)(0.500,1.000){2}{\rule{0.120pt}{0.400pt}}
\put(807.67,699){\rule{0.400pt}{3.854pt}}
\multiput(807.17,699.00)(1.000,8.000){2}{\rule{0.400pt}{1.927pt}}
\put(808.67,715){\rule{0.400pt}{0.482pt}}
\multiput(808.17,715.00)(1.000,1.000){2}{\rule{0.400pt}{0.241pt}}
\put(809.67,712){\rule{0.400pt}{1.204pt}}
\multiput(809.17,714.50)(1.000,-2.500){2}{\rule{0.400pt}{0.602pt}}
\put(810.67,712){\rule{0.400pt}{1.927pt}}
\multiput(810.17,712.00)(1.000,4.000){2}{\rule{0.400pt}{0.964pt}}
\put(811.67,703){\rule{0.400pt}{4.095pt}}
\multiput(811.17,711.50)(1.000,-8.500){2}{\rule{0.400pt}{2.048pt}}
\put(812.67,703){\rule{0.400pt}{0.723pt}}
\multiput(812.17,703.00)(1.000,1.500){2}{\rule{0.400pt}{0.361pt}}
\put(813.67,706){\rule{0.400pt}{3.373pt}}
\multiput(813.17,706.00)(1.000,7.000){2}{\rule{0.400pt}{1.686pt}}
\put(815,719.67){\rule{0.241pt}{0.400pt}}
\multiput(815.00,719.17)(0.500,1.000){2}{\rule{0.120pt}{0.400pt}}
\put(815.67,713){\rule{0.400pt}{1.927pt}}
\multiput(815.17,717.00)(1.000,-4.000){2}{\rule{0.400pt}{0.964pt}}
\put(816.67,713){\rule{0.400pt}{1.927pt}}
\multiput(816.17,713.00)(1.000,4.000){2}{\rule{0.400pt}{0.964pt}}
\put(817.67,721){\rule{0.400pt}{2.409pt}}
\multiput(817.17,721.00)(1.000,5.000){2}{\rule{0.400pt}{1.204pt}}
\put(818.67,731){\rule{0.400pt}{0.482pt}}
\multiput(818.17,731.00)(1.000,1.000){2}{\rule{0.400pt}{0.241pt}}
\put(819.67,713){\rule{0.400pt}{4.818pt}}
\multiput(819.17,723.00)(1.000,-10.000){2}{\rule{0.400pt}{2.409pt}}
\put(820.67,713){\rule{0.400pt}{0.964pt}}
\multiput(820.17,713.00)(1.000,2.000){2}{\rule{0.400pt}{0.482pt}}
\put(821.67,717){\rule{0.400pt}{2.409pt}}
\multiput(821.17,717.00)(1.000,5.000){2}{\rule{0.400pt}{1.204pt}}
\put(822.67,722){\rule{0.400pt}{1.204pt}}
\multiput(822.17,724.50)(1.000,-2.500){2}{\rule{0.400pt}{0.602pt}}
\put(823.67,722){\rule{0.400pt}{3.132pt}}
\multiput(823.17,722.00)(1.000,6.500){2}{\rule{0.400pt}{1.566pt}}
\put(825,734.67){\rule{0.241pt}{0.400pt}}
\multiput(825.00,734.17)(0.500,1.000){2}{\rule{0.120pt}{0.400pt}}
\put(825.67,732){\rule{0.400pt}{0.964pt}}
\multiput(825.17,734.00)(1.000,-2.000){2}{\rule{0.400pt}{0.482pt}}
\put(827,732.17){\rule{0.482pt}{0.400pt}}
\multiput(827.00,731.17)(1.000,2.000){2}{\rule{0.241pt}{0.400pt}}
\put(828.67,725){\rule{0.400pt}{2.168pt}}
\multiput(828.17,729.50)(1.000,-4.500){2}{\rule{0.400pt}{1.084pt}}
\put(829.67,725){\rule{0.400pt}{2.409pt}}
\multiput(829.17,725.00)(1.000,5.000){2}{\rule{0.400pt}{1.204pt}}
\put(830.67,728){\rule{0.400pt}{1.686pt}}
\multiput(830.17,731.50)(1.000,-3.500){2}{\rule{0.400pt}{0.843pt}}
\put(831.67,728){\rule{0.400pt}{1.686pt}}
\multiput(831.17,728.00)(1.000,3.500){2}{\rule{0.400pt}{0.843pt}}
\put(832.67,730){\rule{0.400pt}{1.204pt}}
\multiput(832.17,732.50)(1.000,-2.500){2}{\rule{0.400pt}{0.602pt}}
\put(833.67,730){\rule{0.400pt}{1.927pt}}
\multiput(833.17,730.00)(1.000,4.000){2}{\rule{0.400pt}{0.964pt}}
\put(834.67,729){\rule{0.400pt}{2.168pt}}
\multiput(834.17,733.50)(1.000,-4.500){2}{\rule{0.400pt}{1.084pt}}
\put(835.67,729){\rule{0.400pt}{0.964pt}}
\multiput(835.17,729.00)(1.000,2.000){2}{\rule{0.400pt}{0.482pt}}
\put(836.67,733){\rule{0.400pt}{3.373pt}}
\multiput(836.17,733.00)(1.000,7.000){2}{\rule{0.400pt}{1.686pt}}
\put(837.67,747){\rule{0.400pt}{0.964pt}}
\multiput(837.17,747.00)(1.000,2.000){2}{\rule{0.400pt}{0.482pt}}
\put(838.67,740){\rule{0.400pt}{2.650pt}}
\multiput(838.17,745.50)(1.000,-5.500){2}{\rule{0.400pt}{1.325pt}}
\put(839.67,740){\rule{0.400pt}{2.650pt}}
\multiput(839.17,740.00)(1.000,5.500){2}{\rule{0.400pt}{1.325pt}}
\put(840.67,740){\rule{0.400pt}{2.650pt}}
\multiput(840.17,745.50)(1.000,-5.500){2}{\rule{0.400pt}{1.325pt}}
\put(841.67,740){\rule{0.400pt}{0.964pt}}
\multiput(841.17,740.00)(1.000,2.000){2}{\rule{0.400pt}{0.482pt}}
\put(842.67,744){\rule{0.400pt}{3.132pt}}
\multiput(842.17,744.00)(1.000,6.500){2}{\rule{0.400pt}{1.566pt}}
\put(843.67,752){\rule{0.400pt}{1.204pt}}
\multiput(843.17,754.50)(1.000,-2.500){2}{\rule{0.400pt}{0.602pt}}
\put(844.67,752){\rule{0.400pt}{1.927pt}}
\multiput(844.17,752.00)(1.000,4.000){2}{\rule{0.400pt}{0.964pt}}
\put(761.0,648.0){\usebox{\plotpoint}}
\put(846.67,756){\rule{0.400pt}{0.964pt}}
\multiput(846.17,758.00)(1.000,-2.000){2}{\rule{0.400pt}{0.482pt}}
\put(847.67,756){\rule{0.400pt}{1.927pt}}
\multiput(847.17,756.00)(1.000,4.000){2}{\rule{0.400pt}{0.964pt}}
\put(848.67,750){\rule{0.400pt}{3.373pt}}
\multiput(848.17,757.00)(1.000,-7.000){2}{\rule{0.400pt}{1.686pt}}
\put(849.67,750){\rule{0.400pt}{0.482pt}}
\multiput(849.17,750.00)(1.000,1.000){2}{\rule{0.400pt}{0.241pt}}
\put(850.67,752){\rule{0.400pt}{2.891pt}}
\multiput(850.17,752.00)(1.000,6.000){2}{\rule{0.400pt}{1.445pt}}
\put(851.67,757){\rule{0.400pt}{1.686pt}}
\multiput(851.17,760.50)(1.000,-3.500){2}{\rule{0.400pt}{0.843pt}}
\put(853.17,751){\rule{0.400pt}{1.300pt}}
\multiput(852.17,754.30)(2.000,-3.302){2}{\rule{0.400pt}{0.650pt}}
\put(846.0,760.0){\usebox{\plotpoint}}
\put(855.67,751){\rule{0.400pt}{1.927pt}}
\multiput(855.17,751.00)(1.000,4.000){2}{\rule{0.400pt}{0.964pt}}
\put(857,757.67){\rule{0.241pt}{0.400pt}}
\multiput(857.00,758.17)(0.500,-1.000){2}{\rule{0.120pt}{0.400pt}}
\put(857.67,758){\rule{0.400pt}{2.891pt}}
\multiput(857.17,758.00)(1.000,6.000){2}{\rule{0.400pt}{1.445pt}}
\put(859,769.67){\rule{0.241pt}{0.400pt}}
\multiput(859.00,769.17)(0.500,1.000){2}{\rule{0.120pt}{0.400pt}}
\put(859.67,761){\rule{0.400pt}{2.409pt}}
\multiput(859.17,766.00)(1.000,-5.000){2}{\rule{0.400pt}{1.204pt}}
\put(860.67,759){\rule{0.400pt}{0.482pt}}
\multiput(860.17,760.00)(1.000,-1.000){2}{\rule{0.400pt}{0.241pt}}
\put(861.67,759){\rule{0.400pt}{2.409pt}}
\multiput(861.17,759.00)(1.000,5.000){2}{\rule{0.400pt}{1.204pt}}
\put(863,768.67){\rule{0.241pt}{0.400pt}}
\multiput(863.00,768.17)(0.500,1.000){2}{\rule{0.120pt}{0.400pt}}
\put(863.67,766){\rule{0.400pt}{0.964pt}}
\multiput(863.17,768.00)(1.000,-2.000){2}{\rule{0.400pt}{0.482pt}}
\put(864.67,766){\rule{0.400pt}{1.445pt}}
\multiput(864.17,766.00)(1.000,3.000){2}{\rule{0.400pt}{0.723pt}}
\put(865.67,765){\rule{0.400pt}{1.686pt}}
\multiput(865.17,768.50)(1.000,-3.500){2}{\rule{0.400pt}{0.843pt}}
\put(855.0,751.0){\usebox{\plotpoint}}
\put(867.67,765){\rule{0.400pt}{2.409pt}}
\multiput(867.17,765.00)(1.000,5.000){2}{\rule{0.400pt}{1.204pt}}
\put(868.67,775){\rule{0.400pt}{0.482pt}}
\multiput(868.17,775.00)(1.000,1.000){2}{\rule{0.400pt}{0.241pt}}
\put(869.67,769){\rule{0.400pt}{1.927pt}}
\multiput(869.17,773.00)(1.000,-4.000){2}{\rule{0.400pt}{0.964pt}}
\put(871,768.67){\rule{0.241pt}{0.400pt}}
\multiput(871.00,768.17)(0.500,1.000){2}{\rule{0.120pt}{0.400pt}}
\put(871.67,770){\rule{0.400pt}{1.445pt}}
\multiput(871.17,770.00)(1.000,3.000){2}{\rule{0.400pt}{0.723pt}}
\put(872.67,772){\rule{0.400pt}{0.964pt}}
\multiput(872.17,774.00)(1.000,-2.000){2}{\rule{0.400pt}{0.482pt}}
\put(873.67,772){\rule{0.400pt}{1.445pt}}
\multiput(873.17,772.00)(1.000,3.000){2}{\rule{0.400pt}{0.723pt}}
\put(874.67,773){\rule{0.400pt}{1.204pt}}
\multiput(874.17,775.50)(1.000,-2.500){2}{\rule{0.400pt}{0.602pt}}
\put(875.67,773){\rule{0.400pt}{2.409pt}}
\multiput(875.17,773.00)(1.000,5.000){2}{\rule{0.400pt}{1.204pt}}
\put(867.0,765.0){\usebox{\plotpoint}}
\put(877.67,776){\rule{0.400pt}{1.686pt}}
\multiput(877.17,779.50)(1.000,-3.500){2}{\rule{0.400pt}{0.843pt}}
\put(879.17,776){\rule{0.400pt}{0.900pt}}
\multiput(878.17,776.00)(2.000,2.132){2}{\rule{0.400pt}{0.450pt}}
\put(880.67,780){\rule{0.400pt}{1.686pt}}
\multiput(880.17,780.00)(1.000,3.500){2}{\rule{0.400pt}{0.843pt}}
\put(881.67,783){\rule{0.400pt}{0.964pt}}
\multiput(881.17,785.00)(1.000,-2.000){2}{\rule{0.400pt}{0.482pt}}
\put(882.67,783){\rule{0.400pt}{1.445pt}}
\multiput(882.17,783.00)(1.000,3.000){2}{\rule{0.400pt}{0.723pt}}
\put(883.67,781){\rule{0.400pt}{1.927pt}}
\multiput(883.17,785.00)(1.000,-4.000){2}{\rule{0.400pt}{0.964pt}}
\put(884.67,781){\rule{0.400pt}{2.409pt}}
\multiput(884.17,781.00)(1.000,5.000){2}{\rule{0.400pt}{1.204pt}}
\put(885.67,789){\rule{0.400pt}{0.482pt}}
\multiput(885.17,790.00)(1.000,-1.000){2}{\rule{0.400pt}{0.241pt}}
\put(886.67,784){\rule{0.400pt}{1.204pt}}
\multiput(886.17,786.50)(1.000,-2.500){2}{\rule{0.400pt}{0.602pt}}
\put(887.67,784){\rule{0.400pt}{1.686pt}}
\multiput(887.17,784.00)(1.000,3.500){2}{\rule{0.400pt}{0.843pt}}
\put(888.67,786){\rule{0.400pt}{1.204pt}}
\multiput(888.17,788.50)(1.000,-2.500){2}{\rule{0.400pt}{0.602pt}}
\put(889.67,786){\rule{0.400pt}{1.445pt}}
\multiput(889.17,786.00)(1.000,3.000){2}{\rule{0.400pt}{0.723pt}}
\put(890.67,785){\rule{0.400pt}{1.686pt}}
\multiput(890.17,788.50)(1.000,-3.500){2}{\rule{0.400pt}{0.843pt}}
\put(891.67,785){\rule{0.400pt}{2.168pt}}
\multiput(891.17,785.00)(1.000,4.500){2}{\rule{0.400pt}{1.084pt}}
\put(892.67,790){\rule{0.400pt}{0.964pt}}
\multiput(892.17,792.00)(1.000,-2.000){2}{\rule{0.400pt}{0.482pt}}
\put(893.67,790){\rule{0.400pt}{1.204pt}}
\multiput(893.17,790.00)(1.000,2.500){2}{\rule{0.400pt}{0.602pt}}
\put(894.67,791){\rule{0.400pt}{0.964pt}}
\multiput(894.17,793.00)(1.000,-2.000){2}{\rule{0.400pt}{0.482pt}}
\put(895.67,791){\rule{0.400pt}{1.686pt}}
\multiput(895.17,791.00)(1.000,3.500){2}{\rule{0.400pt}{0.843pt}}
\put(896.67,790){\rule{0.400pt}{1.927pt}}
\multiput(896.17,794.00)(1.000,-4.000){2}{\rule{0.400pt}{0.964pt}}
\put(877.0,783.0){\usebox{\plotpoint}}
\put(898.67,790){\rule{0.400pt}{2.409pt}}
\multiput(898.17,790.00)(1.000,5.000){2}{\rule{0.400pt}{1.204pt}}
\put(899.67,800){\rule{0.400pt}{0.964pt}}
\multiput(899.17,800.00)(1.000,2.000){2}{\rule{0.400pt}{0.482pt}}
\put(900.67,796){\rule{0.400pt}{1.927pt}}
\multiput(900.17,800.00)(1.000,-4.000){2}{\rule{0.400pt}{0.964pt}}
\put(901.67,796){\rule{0.400pt}{1.445pt}}
\multiput(901.17,796.00)(1.000,3.000){2}{\rule{0.400pt}{0.723pt}}
\put(902.67,795){\rule{0.400pt}{1.686pt}}
\multiput(902.17,798.50)(1.000,-3.500){2}{\rule{0.400pt}{0.843pt}}
\put(903.67,795){\rule{0.400pt}{2.409pt}}
\multiput(903.17,795.00)(1.000,5.000){2}{\rule{0.400pt}{1.204pt}}
\put(905.17,796){\rule{0.400pt}{1.900pt}}
\multiput(904.17,801.06)(2.000,-5.056){2}{\rule{0.400pt}{0.950pt}}
\put(906.67,796){\rule{0.400pt}{0.723pt}}
\multiput(906.17,796.00)(1.000,1.500){2}{\rule{0.400pt}{0.361pt}}
\put(907.67,799){\rule{0.400pt}{1.927pt}}
\multiput(907.17,799.00)(1.000,4.000){2}{\rule{0.400pt}{0.964pt}}
\put(908.67,805){\rule{0.400pt}{0.482pt}}
\multiput(908.17,806.00)(1.000,-1.000){2}{\rule{0.400pt}{0.241pt}}
\put(909.67,802){\rule{0.400pt}{0.723pt}}
\multiput(909.17,803.50)(1.000,-1.500){2}{\rule{0.400pt}{0.361pt}}
\put(911,801.67){\rule{0.241pt}{0.400pt}}
\multiput(911.00,801.17)(0.500,1.000){2}{\rule{0.120pt}{0.400pt}}
\put(911.67,803){\rule{0.400pt}{1.686pt}}
\multiput(911.17,803.00)(1.000,3.500){2}{\rule{0.400pt}{0.843pt}}
\put(913,809.67){\rule{0.241pt}{0.400pt}}
\multiput(913.00,809.17)(0.500,1.000){2}{\rule{0.120pt}{0.400pt}}
\put(913.67,802){\rule{0.400pt}{2.168pt}}
\multiput(913.17,806.50)(1.000,-4.500){2}{\rule{0.400pt}{1.084pt}}
\put(914.67,802){\rule{0.400pt}{1.686pt}}
\multiput(914.17,802.00)(1.000,3.500){2}{\rule{0.400pt}{0.843pt}}
\put(915.67,797){\rule{0.400pt}{2.891pt}}
\multiput(915.17,803.00)(1.000,-6.000){2}{\rule{0.400pt}{1.445pt}}
\put(916.67,797){\rule{0.400pt}{0.723pt}}
\multiput(916.17,797.00)(1.000,1.500){2}{\rule{0.400pt}{0.361pt}}
\put(917.67,800){\rule{0.400pt}{2.650pt}}
\multiput(917.17,800.00)(1.000,5.500){2}{\rule{0.400pt}{1.325pt}}
\put(918.67,806){\rule{0.400pt}{1.204pt}}
\multiput(918.17,808.50)(1.000,-2.500){2}{\rule{0.400pt}{0.602pt}}
\put(919.67,806){\rule{0.400pt}{1.445pt}}
\multiput(919.17,806.00)(1.000,3.000){2}{\rule{0.400pt}{0.723pt}}
\put(921,811.67){\rule{0.241pt}{0.400pt}}
\multiput(921.00,811.17)(0.500,1.000){2}{\rule{0.120pt}{0.400pt}}
\put(921.67,804){\rule{0.400pt}{2.168pt}}
\multiput(921.17,808.50)(1.000,-4.500){2}{\rule{0.400pt}{1.084pt}}
\put(922.67,804){\rule{0.400pt}{1.686pt}}
\multiput(922.17,804.00)(1.000,3.500){2}{\rule{0.400pt}{0.843pt}}
\put(923.67,803){\rule{0.400pt}{1.927pt}}
\multiput(923.17,807.00)(1.000,-4.000){2}{\rule{0.400pt}{0.964pt}}
\put(924.67,803){\rule{0.400pt}{0.723pt}}
\multiput(924.17,803.00)(1.000,1.500){2}{\rule{0.400pt}{0.361pt}}
\put(925.67,806){\rule{0.400pt}{2.409pt}}
\multiput(925.17,806.00)(1.000,5.000){2}{\rule{0.400pt}{1.204pt}}
\put(926.67,816){\rule{0.400pt}{0.964pt}}
\multiput(926.17,816.00)(1.000,2.000){2}{\rule{0.400pt}{0.482pt}}
\put(927.67,811){\rule{0.400pt}{2.168pt}}
\multiput(927.17,815.50)(1.000,-4.500){2}{\rule{0.400pt}{1.084pt}}
\put(928.67,808){\rule{0.400pt}{0.723pt}}
\multiput(928.17,809.50)(1.000,-1.500){2}{\rule{0.400pt}{0.361pt}}
\put(929.67,808){\rule{0.400pt}{2.409pt}}
\multiput(929.17,808.00)(1.000,5.000){2}{\rule{0.400pt}{1.204pt}}
\put(931,816.17){\rule{0.482pt}{0.400pt}}
\multiput(931.00,817.17)(1.000,-2.000){2}{\rule{0.241pt}{0.400pt}}
\put(932.67,807){\rule{0.400pt}{2.168pt}}
\multiput(932.17,811.50)(1.000,-4.500){2}{\rule{0.400pt}{1.084pt}}
\put(934,806.67){\rule{0.241pt}{0.400pt}}
\multiput(934.00,806.17)(0.500,1.000){2}{\rule{0.120pt}{0.400pt}}
\put(934.67,808){\rule{0.400pt}{2.891pt}}
\multiput(934.17,808.00)(1.000,6.000){2}{\rule{0.400pt}{1.445pt}}
\put(936,819.67){\rule{0.241pt}{0.400pt}}
\multiput(936.00,819.17)(0.500,1.000){2}{\rule{0.120pt}{0.400pt}}
\put(936.67,815){\rule{0.400pt}{1.445pt}}
\multiput(936.17,818.00)(1.000,-3.000){2}{\rule{0.400pt}{0.723pt}}
\put(937.67,815){\rule{0.400pt}{1.445pt}}
\multiput(937.17,815.00)(1.000,3.000){2}{\rule{0.400pt}{0.723pt}}
\put(938.67,813){\rule{0.400pt}{1.927pt}}
\multiput(938.17,817.00)(1.000,-4.000){2}{\rule{0.400pt}{0.964pt}}
\put(939.67,813){\rule{0.400pt}{3.132pt}}
\multiput(939.17,813.00)(1.000,6.500){2}{\rule{0.400pt}{1.566pt}}
\put(940.67,817){\rule{0.400pt}{2.168pt}}
\multiput(940.17,821.50)(1.000,-4.500){2}{\rule{0.400pt}{1.084pt}}
\put(941.67,817){\rule{0.400pt}{1.204pt}}
\multiput(941.17,817.00)(1.000,2.500){2}{\rule{0.400pt}{0.602pt}}
\put(942.67,817){\rule{0.400pt}{1.204pt}}
\multiput(942.17,819.50)(1.000,-2.500){2}{\rule{0.400pt}{0.602pt}}
\put(943.67,817){\rule{0.400pt}{0.482pt}}
\multiput(943.17,817.00)(1.000,1.000){2}{\rule{0.400pt}{0.241pt}}
\put(944.67,819){\rule{0.400pt}{1.204pt}}
\multiput(944.17,819.00)(1.000,2.500){2}{\rule{0.400pt}{0.602pt}}
\put(898.0,790.0){\usebox{\plotpoint}}
\put(946.67,814){\rule{0.400pt}{2.409pt}}
\multiput(946.17,819.00)(1.000,-5.000){2}{\rule{0.400pt}{1.204pt}}
\put(947.67,810){\rule{0.400pt}{0.964pt}}
\multiput(947.17,812.00)(1.000,-2.000){2}{\rule{0.400pt}{0.482pt}}
\put(948.67,810){\rule{0.400pt}{3.854pt}}
\multiput(948.17,810.00)(1.000,8.000){2}{\rule{0.400pt}{1.927pt}}
\put(949.67,822){\rule{0.400pt}{0.964pt}}
\multiput(949.17,824.00)(1.000,-2.000){2}{\rule{0.400pt}{0.482pt}}
\put(950.67,822){\rule{0.400pt}{3.373pt}}
\multiput(950.17,822.00)(1.000,7.000){2}{\rule{0.400pt}{1.686pt}}
\put(951.67,828){\rule{0.400pt}{1.927pt}}
\multiput(951.17,832.00)(1.000,-4.000){2}{\rule{0.400pt}{0.964pt}}
\put(952.67,813){\rule{0.400pt}{3.614pt}}
\multiput(952.17,820.50)(1.000,-7.500){2}{\rule{0.400pt}{1.807pt}}
\put(953.67,813){\rule{0.400pt}{1.686pt}}
\multiput(953.17,813.00)(1.000,3.500){2}{\rule{0.400pt}{0.843pt}}
\put(954.67,814){\rule{0.400pt}{1.445pt}}
\multiput(954.17,817.00)(1.000,-3.000){2}{\rule{0.400pt}{0.723pt}}
\put(956,812.67){\rule{0.241pt}{0.400pt}}
\multiput(956.00,813.17)(0.500,-1.000){2}{\rule{0.120pt}{0.400pt}}
\put(957.17,813){\rule{0.400pt}{2.700pt}}
\multiput(956.17,813.00)(2.000,7.396){2}{\rule{0.400pt}{1.350pt}}
\put(946.0,824.0){\usebox{\plotpoint}}
\put(959.67,818){\rule{0.400pt}{1.927pt}}
\multiput(959.17,822.00)(1.000,-4.000){2}{\rule{0.400pt}{0.964pt}}
\put(960.67,818){\rule{0.400pt}{1.927pt}}
\multiput(960.17,818.00)(1.000,4.000){2}{\rule{0.400pt}{0.964pt}}
\put(961.67,818){\rule{0.400pt}{1.927pt}}
\multiput(961.17,822.00)(1.000,-4.000){2}{\rule{0.400pt}{0.964pt}}
\put(962.67,818){\rule{0.400pt}{0.723pt}}
\multiput(962.17,818.00)(1.000,1.500){2}{\rule{0.400pt}{0.361pt}}
\put(963.67,821){\rule{0.400pt}{2.409pt}}
\multiput(963.17,821.00)(1.000,5.000){2}{\rule{0.400pt}{1.204pt}}
\put(964.67,828){\rule{0.400pt}{0.723pt}}
\multiput(964.17,829.50)(1.000,-1.500){2}{\rule{0.400pt}{0.361pt}}
\put(965.67,828){\rule{0.400pt}{2.168pt}}
\multiput(965.17,828.00)(1.000,4.500){2}{\rule{0.400pt}{1.084pt}}
\put(966.67,837){\rule{0.400pt}{0.482pt}}
\multiput(966.17,837.00)(1.000,1.000){2}{\rule{0.400pt}{0.241pt}}
\put(967.67,826){\rule{0.400pt}{3.132pt}}
\multiput(967.17,832.50)(1.000,-6.500){2}{\rule{0.400pt}{1.566pt}}
\put(968.67,826){\rule{0.400pt}{0.964pt}}
\multiput(968.17,826.00)(1.000,2.000){2}{\rule{0.400pt}{0.482pt}}
\put(969.67,818){\rule{0.400pt}{2.891pt}}
\multiput(969.17,824.00)(1.000,-6.000){2}{\rule{0.400pt}{1.445pt}}
\put(971,816.67){\rule{0.241pt}{0.400pt}}
\multiput(971.00,817.17)(0.500,-1.000){2}{\rule{0.120pt}{0.400pt}}
\put(971.67,817){\rule{0.400pt}{4.577pt}}
\multiput(971.17,817.00)(1.000,9.500){2}{\rule{0.400pt}{2.289pt}}
\put(973,835.67){\rule{0.241pt}{0.400pt}}
\multiput(973.00,835.17)(0.500,1.000){2}{\rule{0.120pt}{0.400pt}}
\put(973.67,815){\rule{0.400pt}{5.300pt}}
\multiput(973.17,826.00)(1.000,-11.000){2}{\rule{0.400pt}{2.650pt}}
\put(959.0,826.0){\usebox{\plotpoint}}
\put(975.67,815){\rule{0.400pt}{2.168pt}}
\multiput(975.17,815.00)(1.000,4.500){2}{\rule{0.400pt}{1.084pt}}
\put(976.67,818){\rule{0.400pt}{1.445pt}}
\multiput(976.17,821.00)(1.000,-3.000){2}{\rule{0.400pt}{0.723pt}}
\put(977.67,818){\rule{0.400pt}{4.336pt}}
\multiput(977.17,818.00)(1.000,9.000){2}{\rule{0.400pt}{2.168pt}}
\put(978.67,834){\rule{0.400pt}{0.482pt}}
\multiput(978.17,835.00)(1.000,-1.000){2}{\rule{0.400pt}{0.241pt}}
\put(979.67,822){\rule{0.400pt}{2.891pt}}
\multiput(979.17,828.00)(1.000,-6.000){2}{\rule{0.400pt}{1.445pt}}
\put(980.67,822){\rule{0.400pt}{0.723pt}}
\multiput(980.17,822.00)(1.000,1.500){2}{\rule{0.400pt}{0.361pt}}
\put(981.67,816){\rule{0.400pt}{2.168pt}}
\multiput(981.17,820.50)(1.000,-4.500){2}{\rule{0.400pt}{1.084pt}}
\put(983.17,816){\rule{0.400pt}{3.100pt}}
\multiput(982.17,816.00)(2.000,8.566){2}{\rule{0.400pt}{1.550pt}}
\put(984.67,819){\rule{0.400pt}{2.891pt}}
\multiput(984.17,825.00)(1.000,-6.000){2}{\rule{0.400pt}{1.445pt}}
\put(986,817.67){\rule{0.241pt}{0.400pt}}
\multiput(986.00,818.17)(0.500,-1.000){2}{\rule{0.120pt}{0.400pt}}
\put(986.67,818){\rule{0.400pt}{3.132pt}}
\multiput(986.17,818.00)(1.000,6.500){2}{\rule{0.400pt}{1.566pt}}
\put(975.0,815.0){\usebox{\plotpoint}}
\put(988.67,822){\rule{0.400pt}{2.168pt}}
\multiput(988.17,826.50)(1.000,-4.500){2}{\rule{0.400pt}{1.084pt}}
\put(990,821.67){\rule{0.241pt}{0.400pt}}
\multiput(990.00,821.17)(0.500,1.000){2}{\rule{0.120pt}{0.400pt}}
\put(990.67,823){\rule{0.400pt}{2.650pt}}
\multiput(990.17,823.00)(1.000,5.500){2}{\rule{0.400pt}{1.325pt}}
\put(992,832.67){\rule{0.241pt}{0.400pt}}
\multiput(992.00,833.17)(0.500,-1.000){2}{\rule{0.120pt}{0.400pt}}
\put(992.67,822){\rule{0.400pt}{2.650pt}}
\multiput(992.17,827.50)(1.000,-5.500){2}{\rule{0.400pt}{1.325pt}}
\put(993.67,822){\rule{0.400pt}{2.409pt}}
\multiput(993.17,822.00)(1.000,5.000){2}{\rule{0.400pt}{1.204pt}}
\put(994.67,822){\rule{0.400pt}{2.409pt}}
\multiput(994.17,827.00)(1.000,-5.000){2}{\rule{0.400pt}{1.204pt}}
\put(995.67,822){\rule{0.400pt}{1.445pt}}
\multiput(995.17,822.00)(1.000,3.000){2}{\rule{0.400pt}{0.723pt}}
\put(996.67,819){\rule{0.400pt}{2.168pt}}
\multiput(996.17,823.50)(1.000,-4.500){2}{\rule{0.400pt}{1.084pt}}
\put(997.67,819){\rule{0.400pt}{1.445pt}}
\multiput(997.17,819.00)(1.000,3.000){2}{\rule{0.400pt}{0.723pt}}
\put(998.67,816){\rule{0.400pt}{2.168pt}}
\multiput(998.17,820.50)(1.000,-4.500){2}{\rule{0.400pt}{1.084pt}}
\put(1000,815.67){\rule{0.241pt}{0.400pt}}
\multiput(1000.00,815.17)(0.500,1.000){2}{\rule{0.120pt}{0.400pt}}
\put(1000.67,817){\rule{0.400pt}{1.927pt}}
\multiput(1000.17,817.00)(1.000,4.000){2}{\rule{0.400pt}{0.964pt}}
\put(1001.67,820){\rule{0.400pt}{1.204pt}}
\multiput(1001.17,822.50)(1.000,-2.500){2}{\rule{0.400pt}{0.602pt}}
\put(1002.67,820){\rule{0.400pt}{2.650pt}}
\multiput(1002.17,820.00)(1.000,5.500){2}{\rule{0.400pt}{1.325pt}}
\put(1003.67,821){\rule{0.400pt}{2.409pt}}
\multiput(1003.17,826.00)(1.000,-5.000){2}{\rule{0.400pt}{1.204pt}}
\put(1004.67,821){\rule{0.400pt}{2.409pt}}
\multiput(1004.17,821.00)(1.000,5.000){2}{\rule{0.400pt}{1.204pt}}
\put(988.0,831.0){\usebox{\plotpoint}}
\put(1006.67,819){\rule{0.400pt}{2.891pt}}
\multiput(1006.17,825.00)(1.000,-6.000){2}{\rule{0.400pt}{1.445pt}}
\put(1007.67,819){\rule{0.400pt}{1.445pt}}
\multiput(1007.17,819.00)(1.000,3.000){2}{\rule{0.400pt}{0.723pt}}
\put(1009.17,812){\rule{0.400pt}{2.700pt}}
\multiput(1008.17,819.40)(2.000,-7.396){2}{\rule{0.400pt}{1.350pt}}
\put(1010.67,812){\rule{0.400pt}{0.723pt}}
\multiput(1010.17,812.00)(1.000,1.500){2}{\rule{0.400pt}{0.361pt}}
\put(1011.67,815){\rule{0.400pt}{1.927pt}}
\multiput(1011.17,815.00)(1.000,4.000){2}{\rule{0.400pt}{0.964pt}}
\put(1012.67,814){\rule{0.400pt}{2.168pt}}
\multiput(1012.17,818.50)(1.000,-4.500){2}{\rule{0.400pt}{1.084pt}}
\put(1013.67,814){\rule{0.400pt}{2.168pt}}
\multiput(1013.17,814.00)(1.000,4.500){2}{\rule{0.400pt}{1.084pt}}
\put(1014.67,817){\rule{0.400pt}{1.445pt}}
\multiput(1014.17,820.00)(1.000,-3.000){2}{\rule{0.400pt}{0.723pt}}
\put(1015.67,817){\rule{0.400pt}{2.650pt}}
\multiput(1015.17,817.00)(1.000,5.500){2}{\rule{0.400pt}{1.325pt}}
\put(1016.67,822){\rule{0.400pt}{1.445pt}}
\multiput(1016.17,825.00)(1.000,-3.000){2}{\rule{0.400pt}{0.723pt}}
\put(1017.67,813){\rule{0.400pt}{2.168pt}}
\multiput(1017.17,817.50)(1.000,-4.500){2}{\rule{0.400pt}{1.084pt}}
\put(1018.67,813){\rule{0.400pt}{1.927pt}}
\multiput(1018.17,813.00)(1.000,4.000){2}{\rule{0.400pt}{0.964pt}}
\put(1019.67,815){\rule{0.400pt}{1.445pt}}
\multiput(1019.17,818.00)(1.000,-3.000){2}{\rule{0.400pt}{0.723pt}}
\put(1020.67,815){\rule{0.400pt}{1.204pt}}
\multiput(1020.17,815.00)(1.000,2.500){2}{\rule{0.400pt}{0.602pt}}
\put(1021.67,815){\rule{0.400pt}{1.204pt}}
\multiput(1021.17,817.50)(1.000,-2.500){2}{\rule{0.400pt}{0.602pt}}
\put(1022.67,815){\rule{0.400pt}{2.891pt}}
\multiput(1022.17,815.00)(1.000,6.000){2}{\rule{0.400pt}{1.445pt}}
\put(1023.67,814){\rule{0.400pt}{3.132pt}}
\multiput(1023.17,820.50)(1.000,-6.500){2}{\rule{0.400pt}{1.566pt}}
\put(1025,812.67){\rule{0.241pt}{0.400pt}}
\multiput(1025.00,813.17)(0.500,-1.000){2}{\rule{0.120pt}{0.400pt}}
\put(1025.67,813){\rule{0.400pt}{1.445pt}}
\multiput(1025.17,813.00)(1.000,3.000){2}{\rule{0.400pt}{0.723pt}}
\put(1027,817.67){\rule{0.241pt}{0.400pt}}
\multiput(1027.00,818.17)(0.500,-1.000){2}{\rule{0.120pt}{0.400pt}}
\put(1027.67,810){\rule{0.400pt}{1.927pt}}
\multiput(1027.17,814.00)(1.000,-4.000){2}{\rule{0.400pt}{0.964pt}}
\put(1028.67,810){\rule{0.400pt}{0.723pt}}
\multiput(1028.17,810.00)(1.000,1.500){2}{\rule{0.400pt}{0.361pt}}
\put(1029.67,813){\rule{0.400pt}{1.204pt}}
\multiput(1029.17,813.00)(1.000,2.500){2}{\rule{0.400pt}{0.602pt}}
\put(1006.0,831.0){\usebox{\plotpoint}}
\put(1031.67,808){\rule{0.400pt}{2.409pt}}
\multiput(1031.17,813.00)(1.000,-5.000){2}{\rule{0.400pt}{1.204pt}}
\put(1032.67,808){\rule{0.400pt}{2.168pt}}
\multiput(1032.17,808.00)(1.000,4.500){2}{\rule{0.400pt}{1.084pt}}
\put(1033.67,804){\rule{0.400pt}{3.132pt}}
\multiput(1033.17,810.50)(1.000,-6.500){2}{\rule{0.400pt}{1.566pt}}
\put(1035,802.67){\rule{0.482pt}{0.400pt}}
\multiput(1035.00,803.17)(1.000,-1.000){2}{\rule{0.241pt}{0.400pt}}
\put(1036.67,803){\rule{0.400pt}{1.927pt}}
\multiput(1036.17,803.00)(1.000,4.000){2}{\rule{0.400pt}{0.964pt}}
\put(1037.67,808){\rule{0.400pt}{0.723pt}}
\multiput(1037.17,809.50)(1.000,-1.500){2}{\rule{0.400pt}{0.361pt}}
\put(1038.67,801){\rule{0.400pt}{1.686pt}}
\multiput(1038.17,804.50)(1.000,-3.500){2}{\rule{0.400pt}{0.843pt}}
\put(1039.67,801){\rule{0.400pt}{1.445pt}}
\multiput(1039.17,801.00)(1.000,3.000){2}{\rule{0.400pt}{0.723pt}}
\put(1040.67,797){\rule{0.400pt}{2.409pt}}
\multiput(1040.17,802.00)(1.000,-5.000){2}{\rule{0.400pt}{1.204pt}}
\put(1041.67,795){\rule{0.400pt}{0.482pt}}
\multiput(1041.17,796.00)(1.000,-1.000){2}{\rule{0.400pt}{0.241pt}}
\put(1042.67,795){\rule{0.400pt}{1.445pt}}
\multiput(1042.17,795.00)(1.000,3.000){2}{\rule{0.400pt}{0.723pt}}
\put(1043.67,801){\rule{0.400pt}{0.723pt}}
\multiput(1043.17,801.00)(1.000,1.500){2}{\rule{0.400pt}{0.361pt}}
\put(1044.67,794){\rule{0.400pt}{2.409pt}}
\multiput(1044.17,799.00)(1.000,-5.000){2}{\rule{0.400pt}{1.204pt}}
\put(1031.0,818.0){\usebox{\plotpoint}}
\put(1046.67,794){\rule{0.400pt}{1.927pt}}
\multiput(1046.17,794.00)(1.000,4.000){2}{\rule{0.400pt}{0.964pt}}
\put(1047.67,795){\rule{0.400pt}{1.686pt}}
\multiput(1047.17,798.50)(1.000,-3.500){2}{\rule{0.400pt}{0.843pt}}
\put(1048.67,795){\rule{0.400pt}{1.445pt}}
\multiput(1048.17,795.00)(1.000,3.000){2}{\rule{0.400pt}{0.723pt}}
\put(1050,800.67){\rule{0.241pt}{0.400pt}}
\multiput(1050.00,800.17)(0.500,1.000){2}{\rule{0.120pt}{0.400pt}}
\put(1050.67,794){\rule{0.400pt}{1.927pt}}
\multiput(1050.17,798.00)(1.000,-4.000){2}{\rule{0.400pt}{0.964pt}}
\put(1046.0,794.0){\usebox{\plotpoint}}
\put(1052.67,794){\rule{0.400pt}{1.686pt}}
\multiput(1052.17,794.00)(1.000,3.500){2}{\rule{0.400pt}{0.843pt}}
\put(1053.67,794){\rule{0.400pt}{1.686pt}}
\multiput(1053.17,797.50)(1.000,-3.500){2}{\rule{0.400pt}{0.843pt}}
\put(1054.67,794){\rule{0.400pt}{2.168pt}}
\multiput(1054.17,794.00)(1.000,4.500){2}{\rule{0.400pt}{1.084pt}}
\put(1055.67,800){\rule{0.400pt}{0.723pt}}
\multiput(1055.17,801.50)(1.000,-1.500){2}{\rule{0.400pt}{0.361pt}}
\put(1056.67,790){\rule{0.400pt}{2.409pt}}
\multiput(1056.17,795.00)(1.000,-5.000){2}{\rule{0.400pt}{1.204pt}}
\put(1058,789.67){\rule{0.241pt}{0.400pt}}
\multiput(1058.00,789.17)(0.500,1.000){2}{\rule{0.120pt}{0.400pt}}
\put(1058.67,779){\rule{0.400pt}{2.891pt}}
\multiput(1058.17,785.00)(1.000,-6.000){2}{\rule{0.400pt}{1.445pt}}
\put(1059.67,779){\rule{0.400pt}{1.927pt}}
\multiput(1059.17,779.00)(1.000,4.000){2}{\rule{0.400pt}{0.964pt}}
\put(1061.17,787){\rule{0.400pt}{2.300pt}}
\multiput(1060.17,787.00)(2.000,6.226){2}{\rule{0.400pt}{1.150pt}}
\put(1062.67,792){\rule{0.400pt}{1.445pt}}
\multiput(1062.17,795.00)(1.000,-3.000){2}{\rule{0.400pt}{0.723pt}}
\put(1063.67,776){\rule{0.400pt}{3.854pt}}
\multiput(1063.17,784.00)(1.000,-8.000){2}{\rule{0.400pt}{1.927pt}}
\put(1064.67,776){\rule{0.400pt}{0.482pt}}
\multiput(1064.17,776.00)(1.000,1.000){2}{\rule{0.400pt}{0.241pt}}
\put(1065.67,778){\rule{0.400pt}{2.650pt}}
\multiput(1065.17,778.00)(1.000,5.500){2}{\rule{0.400pt}{1.325pt}}
\put(1066.67,789){\rule{0.400pt}{0.723pt}}
\multiput(1066.17,789.00)(1.000,1.500){2}{\rule{0.400pt}{0.361pt}}
\put(1067.67,782){\rule{0.400pt}{2.409pt}}
\multiput(1067.17,787.00)(1.000,-5.000){2}{\rule{0.400pt}{1.204pt}}
\put(1068.67,782){\rule{0.400pt}{1.204pt}}
\multiput(1068.17,782.00)(1.000,2.500){2}{\rule{0.400pt}{0.602pt}}
\put(1069.67,780){\rule{0.400pt}{1.686pt}}
\multiput(1069.17,783.50)(1.000,-3.500){2}{\rule{0.400pt}{0.843pt}}
\put(1070.67,778){\rule{0.400pt}{0.482pt}}
\multiput(1070.17,779.00)(1.000,-1.000){2}{\rule{0.400pt}{0.241pt}}
\put(1071.67,778){\rule{0.400pt}{1.445pt}}
\multiput(1071.17,778.00)(1.000,3.000){2}{\rule{0.400pt}{0.723pt}}
\put(1072.67,784){\rule{0.400pt}{0.723pt}}
\multiput(1072.17,784.00)(1.000,1.500){2}{\rule{0.400pt}{0.361pt}}
\put(1073.67,774){\rule{0.400pt}{3.132pt}}
\multiput(1073.17,780.50)(1.000,-6.500){2}{\rule{0.400pt}{1.566pt}}
\put(1052.0,794.0){\usebox{\plotpoint}}
\put(1075.67,774){\rule{0.400pt}{3.132pt}}
\multiput(1075.17,774.00)(1.000,6.500){2}{\rule{0.400pt}{1.566pt}}
\put(1076.67,785){\rule{0.400pt}{0.482pt}}
\multiput(1076.17,786.00)(1.000,-1.000){2}{\rule{0.400pt}{0.241pt}}
\put(1077.67,774){\rule{0.400pt}{2.650pt}}
\multiput(1077.17,779.50)(1.000,-5.500){2}{\rule{0.400pt}{1.325pt}}
\put(1078.67,774){\rule{0.400pt}{0.964pt}}
\multiput(1078.17,774.00)(1.000,2.000){2}{\rule{0.400pt}{0.482pt}}
\put(1079.67,769){\rule{0.400pt}{2.168pt}}
\multiput(1079.17,773.50)(1.000,-4.500){2}{\rule{0.400pt}{1.084pt}}
\put(1080.67,769){\rule{0.400pt}{2.409pt}}
\multiput(1080.17,769.00)(1.000,5.000){2}{\rule{0.400pt}{1.204pt}}
\put(1081.67,768){\rule{0.400pt}{2.650pt}}
\multiput(1081.17,773.50)(1.000,-5.500){2}{\rule{0.400pt}{1.325pt}}
\put(1082.67,768){\rule{0.400pt}{1.204pt}}
\multiput(1082.17,768.00)(1.000,2.500){2}{\rule{0.400pt}{0.602pt}}
\put(1083.67,766){\rule{0.400pt}{1.686pt}}
\multiput(1083.17,769.50)(1.000,-3.500){2}{\rule{0.400pt}{0.843pt}}
\put(1075.0,774.0){\usebox{\plotpoint}}
\put(1085.67,766){\rule{0.400pt}{0.723pt}}
\multiput(1085.17,766.00)(1.000,1.500){2}{\rule{0.400pt}{0.361pt}}
\put(1087.17,763){\rule{0.400pt}{1.300pt}}
\multiput(1086.17,766.30)(2.000,-3.302){2}{\rule{0.400pt}{0.650pt}}
\put(1088.67,763){\rule{0.400pt}{2.409pt}}
\multiput(1088.17,763.00)(1.000,5.000){2}{\rule{0.400pt}{1.204pt}}
\put(1085.0,766.0){\usebox{\plotpoint}}
\put(1090.67,763){\rule{0.400pt}{2.409pt}}
\multiput(1090.17,768.00)(1.000,-5.000){2}{\rule{0.400pt}{1.204pt}}
\put(1091.67,763){\rule{0.400pt}{1.445pt}}
\multiput(1091.17,763.00)(1.000,3.000){2}{\rule{0.400pt}{0.723pt}}
\put(1092.67,751){\rule{0.400pt}{4.336pt}}
\multiput(1092.17,760.00)(1.000,-9.000){2}{\rule{0.400pt}{2.168pt}}
\put(1093.67,751){\rule{0.400pt}{1.445pt}}
\multiput(1093.17,751.00)(1.000,3.000){2}{\rule{0.400pt}{0.723pt}}
\put(1094.67,749){\rule{0.400pt}{1.927pt}}
\multiput(1094.17,753.00)(1.000,-4.000){2}{\rule{0.400pt}{0.964pt}}
\put(1095.67,749){\rule{0.400pt}{1.445pt}}
\multiput(1095.17,749.00)(1.000,3.000){2}{\rule{0.400pt}{0.723pt}}
\put(1096.67,755){\rule{0.400pt}{0.723pt}}
\multiput(1096.17,755.00)(1.000,1.500){2}{\rule{0.400pt}{0.361pt}}
\put(1097.67,758){\rule{0.400pt}{0.482pt}}
\multiput(1097.17,758.00)(1.000,1.000){2}{\rule{0.400pt}{0.241pt}}
\put(1098.67,751){\rule{0.400pt}{2.168pt}}
\multiput(1098.17,755.50)(1.000,-4.500){2}{\rule{0.400pt}{1.084pt}}
\put(1099.67,751){\rule{0.400pt}{5.059pt}}
\multiput(1099.17,751.00)(1.000,10.500){2}{\rule{0.400pt}{2.529pt}}
\put(1100.67,742){\rule{0.400pt}{7.227pt}}
\multiput(1100.17,757.00)(1.000,-15.000){2}{\rule{0.400pt}{3.613pt}}
\put(1101.67,742){\rule{0.400pt}{2.168pt}}
\multiput(1101.17,742.00)(1.000,4.500){2}{\rule{0.400pt}{1.084pt}}
\put(1102.67,737){\rule{0.400pt}{3.373pt}}
\multiput(1102.17,744.00)(1.000,-7.000){2}{\rule{0.400pt}{1.686pt}}
\put(1103.67,735){\rule{0.400pt}{0.482pt}}
\multiput(1103.17,736.00)(1.000,-1.000){2}{\rule{0.400pt}{0.241pt}}
\put(1104.67,735){\rule{0.400pt}{9.395pt}}
\multiput(1104.17,735.00)(1.000,19.500){2}{\rule{0.400pt}{4.698pt}}
\put(1105.67,765){\rule{0.400pt}{2.168pt}}
\multiput(1105.17,769.50)(1.000,-4.500){2}{\rule{0.400pt}{1.084pt}}
\put(1106.67,730){\rule{0.400pt}{8.432pt}}
\multiput(1106.17,747.50)(1.000,-17.500){2}{\rule{0.400pt}{4.216pt}}
\put(1107.67,727){\rule{0.400pt}{0.723pt}}
\multiput(1107.17,728.50)(1.000,-1.500){2}{\rule{0.400pt}{0.361pt}}
\put(1108.67,727){\rule{0.400pt}{5.059pt}}
\multiput(1108.17,727.00)(1.000,10.500){2}{\rule{0.400pt}{2.529pt}}
\put(1109.67,746){\rule{0.400pt}{0.482pt}}
\multiput(1109.17,747.00)(1.000,-1.000){2}{\rule{0.400pt}{0.241pt}}
\put(1110.67,722){\rule{0.400pt}{5.782pt}}
\multiput(1110.17,734.00)(1.000,-12.000){2}{\rule{0.400pt}{2.891pt}}
\put(1111.67,712){\rule{0.400pt}{2.409pt}}
\multiput(1111.17,717.00)(1.000,-5.000){2}{\rule{0.400pt}{1.204pt}}
\put(1113.17,712){\rule{0.400pt}{3.900pt}}
\multiput(1112.17,712.00)(2.000,10.905){2}{\rule{0.400pt}{1.950pt}}
\put(1114.67,731){\rule{0.400pt}{3.132pt}}
\multiput(1114.17,731.00)(1.000,6.500){2}{\rule{0.400pt}{1.566pt}}
\put(1115.67,716){\rule{0.400pt}{6.745pt}}
\multiput(1115.17,730.00)(1.000,-14.000){2}{\rule{0.400pt}{3.373pt}}
\put(1116.67,716){\rule{0.400pt}{1.927pt}}
\multiput(1116.17,716.00)(1.000,4.000){2}{\rule{0.400pt}{0.964pt}}
\put(1117.67,699){\rule{0.400pt}{6.023pt}}
\multiput(1117.17,711.50)(1.000,-12.500){2}{\rule{0.400pt}{3.011pt}}
\put(1118.67,699){\rule{0.400pt}{1.927pt}}
\multiput(1118.17,699.00)(1.000,4.000){2}{\rule{0.400pt}{0.964pt}}
\put(1119.67,677){\rule{0.400pt}{7.227pt}}
\multiput(1119.17,692.00)(1.000,-15.000){2}{\rule{0.400pt}{3.613pt}}
\put(1120.67,677){\rule{0.400pt}{3.132pt}}
\multiput(1120.17,677.00)(1.000,6.500){2}{\rule{0.400pt}{1.566pt}}
\put(1121.67,645){\rule{0.400pt}{10.840pt}}
\multiput(1121.17,667.50)(1.000,-22.500){2}{\rule{0.400pt}{5.420pt}}
\put(1122.67,638){\rule{0.400pt}{1.686pt}}
\multiput(1122.17,641.50)(1.000,-3.500){2}{\rule{0.400pt}{0.843pt}}
\put(1123.67,638){\rule{0.400pt}{5.059pt}}
\multiput(1123.17,638.00)(1.000,10.500){2}{\rule{0.400pt}{2.529pt}}
\put(1124.67,651){\rule{0.400pt}{1.927pt}}
\multiput(1124.17,655.00)(1.000,-4.000){2}{\rule{0.400pt}{0.964pt}}
\put(1125.67,651){\rule{0.400pt}{3.614pt}}
\multiput(1125.17,651.00)(1.000,7.500){2}{\rule{0.400pt}{1.807pt}}
\put(1126.67,656){\rule{0.400pt}{2.409pt}}
\multiput(1126.17,661.00)(1.000,-5.000){2}{\rule{0.400pt}{1.204pt}}
\put(1127.67,656){\rule{0.400pt}{4.095pt}}
\multiput(1127.17,656.00)(1.000,8.500){2}{\rule{0.400pt}{2.048pt}}
\put(1128.67,667){\rule{0.400pt}{1.445pt}}
\multiput(1128.17,670.00)(1.000,-3.000){2}{\rule{0.400pt}{0.723pt}}
\put(1129.67,667){\rule{0.400pt}{4.095pt}}
\multiput(1129.17,667.00)(1.000,8.500){2}{\rule{0.400pt}{2.048pt}}
\put(1130.67,684){\rule{0.400pt}{0.723pt}}
\multiput(1130.17,684.00)(1.000,1.500){2}{\rule{0.400pt}{0.361pt}}
\put(1131.67,657){\rule{0.400pt}{7.227pt}}
\multiput(1131.17,672.00)(1.000,-15.000){2}{\rule{0.400pt}{3.613pt}}
\put(1132.67,653){\rule{0.400pt}{0.964pt}}
\multiput(1132.17,655.00)(1.000,-2.000){2}{\rule{0.400pt}{0.482pt}}
\put(1133.67,653){\rule{0.400pt}{2.650pt}}
\multiput(1133.17,653.00)(1.000,5.500){2}{\rule{0.400pt}{1.325pt}}
\put(1134.67,656){\rule{0.400pt}{1.927pt}}
\multiput(1134.17,660.00)(1.000,-4.000){2}{\rule{0.400pt}{0.964pt}}
\put(1135.67,656){\rule{0.400pt}{2.409pt}}
\multiput(1135.17,656.00)(1.000,5.000){2}{\rule{0.400pt}{1.204pt}}
\put(1136.67,660){\rule{0.400pt}{1.445pt}}
\multiput(1136.17,663.00)(1.000,-3.000){2}{\rule{0.400pt}{0.723pt}}
\put(1137.67,660){\rule{0.400pt}{2.650pt}}
\multiput(1137.17,660.00)(1.000,5.500){2}{\rule{0.400pt}{1.325pt}}
\put(1139.17,671){\rule{0.400pt}{0.700pt}}
\multiput(1138.17,671.00)(2.000,1.547){2}{\rule{0.400pt}{0.350pt}}
\put(1140.67,662){\rule{0.400pt}{2.891pt}}
\multiput(1140.17,668.00)(1.000,-6.000){2}{\rule{0.400pt}{1.445pt}}
\put(1141.67,656){\rule{0.400pt}{1.445pt}}
\multiput(1141.17,659.00)(1.000,-3.000){2}{\rule{0.400pt}{0.723pt}}
\put(1142.67,656){\rule{0.400pt}{1.927pt}}
\multiput(1142.17,656.00)(1.000,4.000){2}{\rule{0.400pt}{0.964pt}}
\put(1090.0,773.0){\usebox{\plotpoint}}
\put(1144.67,649){\rule{0.400pt}{3.614pt}}
\multiput(1144.17,656.50)(1.000,-7.500){2}{\rule{0.400pt}{1.807pt}}
\put(1145.67,649){\rule{0.400pt}{0.723pt}}
\multiput(1145.17,649.00)(1.000,1.500){2}{\rule{0.400pt}{0.361pt}}
\put(1146.67,642){\rule{0.400pt}{2.409pt}}
\multiput(1146.17,647.00)(1.000,-5.000){2}{\rule{0.400pt}{1.204pt}}
\put(1147.67,639){\rule{0.400pt}{0.723pt}}
\multiput(1147.17,640.50)(1.000,-1.500){2}{\rule{0.400pt}{0.361pt}}
\put(1148.67,639){\rule{0.400pt}{1.686pt}}
\multiput(1148.17,639.00)(1.000,3.500){2}{\rule{0.400pt}{0.843pt}}
\put(1144.0,664.0){\usebox{\plotpoint}}
\put(1150.67,627){\rule{0.400pt}{4.577pt}}
\multiput(1150.17,636.50)(1.000,-9.500){2}{\rule{0.400pt}{2.289pt}}
\put(1152,625.67){\rule{0.241pt}{0.400pt}}
\multiput(1152.00,626.17)(0.500,-1.000){2}{\rule{0.120pt}{0.400pt}}
\put(1152.67,626){\rule{0.400pt}{1.445pt}}
\multiput(1152.17,626.00)(1.000,3.000){2}{\rule{0.400pt}{0.723pt}}
\put(1153.67,625){\rule{0.400pt}{1.686pt}}
\multiput(1153.17,628.50)(1.000,-3.500){2}{\rule{0.400pt}{0.843pt}}
\put(1154.67,625){\rule{0.400pt}{2.168pt}}
\multiput(1154.17,625.00)(1.000,4.500){2}{\rule{0.400pt}{1.084pt}}
\put(1155.67,627){\rule{0.400pt}{1.686pt}}
\multiput(1155.17,630.50)(1.000,-3.500){2}{\rule{0.400pt}{0.843pt}}
\put(1156.67,617){\rule{0.400pt}{2.409pt}}
\multiput(1156.17,622.00)(1.000,-5.000){2}{\rule{0.400pt}{1.204pt}}
\put(1157.67,617){\rule{0.400pt}{1.927pt}}
\multiput(1157.17,617.00)(1.000,4.000){2}{\rule{0.400pt}{0.964pt}}
\put(1158.67,609){\rule{0.400pt}{3.854pt}}
\multiput(1158.17,617.00)(1.000,-8.000){2}{\rule{0.400pt}{1.927pt}}
\put(1159.67,609){\rule{0.400pt}{4.818pt}}
\multiput(1159.17,609.00)(1.000,10.000){2}{\rule{0.400pt}{2.409pt}}
\put(1160.67,596){\rule{0.400pt}{7.950pt}}
\multiput(1160.17,612.50)(1.000,-16.500){2}{\rule{0.400pt}{3.975pt}}
\put(1161.67,596){\rule{0.400pt}{0.964pt}}
\multiput(1161.17,596.00)(1.000,2.000){2}{\rule{0.400pt}{0.482pt}}
\put(1162.67,520){\rule{0.400pt}{19.272pt}}
\multiput(1162.17,560.00)(1.000,-40.000){2}{\rule{0.400pt}{9.636pt}}
\put(1163.67,520){\rule{0.400pt}{0.482pt}}
\multiput(1163.17,520.00)(1.000,1.000){2}{\rule{0.400pt}{0.241pt}}
\put(1165.17,391){\rule{0.400pt}{26.300pt}}
\multiput(1164.17,467.41)(2.000,-76.413){2}{\rule{0.400pt}{13.150pt}}
\put(1166.67,380){\rule{0.400pt}{2.650pt}}
\multiput(1166.17,385.50)(1.000,-5.500){2}{\rule{0.400pt}{1.325pt}}
\put(1167.67,380){\rule{0.400pt}{4.336pt}}
\multiput(1167.17,380.00)(1.000,9.000){2}{\rule{0.400pt}{2.168pt}}
\put(1168.67,375){\rule{0.400pt}{5.541pt}}
\multiput(1168.17,386.50)(1.000,-11.500){2}{\rule{0.400pt}{2.770pt}}
\put(1169.67,375){\rule{0.400pt}{3.373pt}}
\multiput(1169.17,375.00)(1.000,7.000){2}{\rule{0.400pt}{1.686pt}}
\put(1170.67,383){\rule{0.400pt}{1.445pt}}
\multiput(1170.17,386.00)(1.000,-3.000){2}{\rule{0.400pt}{0.723pt}}
\put(1171.67,348){\rule{0.400pt}{8.432pt}}
\multiput(1171.17,365.50)(1.000,-17.500){2}{\rule{0.400pt}{4.216pt}}
\put(1172.67,348){\rule{0.400pt}{0.723pt}}
\multiput(1172.17,348.00)(1.000,1.500){2}{\rule{0.400pt}{0.361pt}}
\put(1173.67,329){\rule{0.400pt}{5.300pt}}
\multiput(1173.17,340.00)(1.000,-11.000){2}{\rule{0.400pt}{2.650pt}}
\put(1174.67,325){\rule{0.400pt}{0.964pt}}
\multiput(1174.17,327.00)(1.000,-2.000){2}{\rule{0.400pt}{0.482pt}}
\put(1175.67,305){\rule{0.400pt}{4.818pt}}
\multiput(1175.17,315.00)(1.000,-10.000){2}{\rule{0.400pt}{2.409pt}}
\put(1176.67,303){\rule{0.400pt}{0.482pt}}
\multiput(1176.17,304.00)(1.000,-1.000){2}{\rule{0.400pt}{0.241pt}}
\put(1177.67,286){\rule{0.400pt}{4.095pt}}
\multiput(1177.17,294.50)(1.000,-8.500){2}{\rule{0.400pt}{2.048pt}}
\put(1178.67,281){\rule{0.400pt}{1.204pt}}
\multiput(1178.17,283.50)(1.000,-2.500){2}{\rule{0.400pt}{0.602pt}}
\put(1179.67,278){\rule{0.400pt}{0.723pt}}
\multiput(1179.17,279.50)(1.000,-1.500){2}{\rule{0.400pt}{0.361pt}}
\put(1180.67,263){\rule{0.400pt}{3.614pt}}
\multiput(1180.17,270.50)(1.000,-7.500){2}{\rule{0.400pt}{1.807pt}}
\put(1182,262.67){\rule{0.241pt}{0.400pt}}
\multiput(1182.00,262.17)(0.500,1.000){2}{\rule{0.120pt}{0.400pt}}
\put(1182.67,250){\rule{0.400pt}{3.373pt}}
\multiput(1182.17,257.00)(1.000,-7.000){2}{\rule{0.400pt}{1.686pt}}
\put(1183.67,250){\rule{0.400pt}{1.686pt}}
\multiput(1183.17,250.00)(1.000,3.500){2}{\rule{0.400pt}{0.843pt}}
\put(1184.67,245){\rule{0.400pt}{2.891pt}}
\multiput(1184.17,251.00)(1.000,-6.000){2}{\rule{0.400pt}{1.445pt}}
\put(1185.67,245){\rule{0.400pt}{1.686pt}}
\multiput(1185.17,245.00)(1.000,3.500){2}{\rule{0.400pt}{0.843pt}}
\put(1186.67,241){\rule{0.400pt}{2.650pt}}
\multiput(1186.17,246.50)(1.000,-5.500){2}{\rule{0.400pt}{1.325pt}}
\put(1187.67,241){\rule{0.400pt}{1.686pt}}
\multiput(1187.17,241.00)(1.000,3.500){2}{\rule{0.400pt}{0.843pt}}
\put(1188.67,238){\rule{0.400pt}{2.409pt}}
\multiput(1188.17,243.00)(1.000,-5.000){2}{\rule{0.400pt}{1.204pt}}
\put(1189.67,238){\rule{0.400pt}{1.686pt}}
\multiput(1189.17,238.00)(1.000,3.500){2}{\rule{0.400pt}{0.843pt}}
\put(1191.17,235){\rule{0.400pt}{2.100pt}}
\multiput(1190.17,240.64)(2.000,-5.641){2}{\rule{0.400pt}{1.050pt}}
\put(1192.67,235){\rule{0.400pt}{2.409pt}}
\multiput(1192.17,235.00)(1.000,5.000){2}{\rule{0.400pt}{1.204pt}}
\put(1193.67,232){\rule{0.400pt}{3.132pt}}
\multiput(1193.17,238.50)(1.000,-6.500){2}{\rule{0.400pt}{1.566pt}}
\put(1194.67,232){\rule{0.400pt}{2.650pt}}
\multiput(1194.17,232.00)(1.000,5.500){2}{\rule{0.400pt}{1.325pt}}
\put(1195.67,233){\rule{0.400pt}{2.409pt}}
\multiput(1195.17,238.00)(1.000,-5.000){2}{\rule{0.400pt}{1.204pt}}
\put(1196.67,233){\rule{0.400pt}{1.204pt}}
\multiput(1196.17,233.00)(1.000,2.500){2}{\rule{0.400pt}{0.602pt}}
\put(1197.67,228){\rule{0.400pt}{2.409pt}}
\multiput(1197.17,233.00)(1.000,-5.000){2}{\rule{0.400pt}{1.204pt}}
\put(1198.67,228){\rule{0.400pt}{2.650pt}}
\multiput(1198.17,228.00)(1.000,5.500){2}{\rule{0.400pt}{1.325pt}}
\put(1199.67,226){\rule{0.400pt}{3.132pt}}
\multiput(1199.17,232.50)(1.000,-6.500){2}{\rule{0.400pt}{1.566pt}}
\put(1200.67,226){\rule{0.400pt}{2.891pt}}
\multiput(1200.17,226.00)(1.000,6.000){2}{\rule{0.400pt}{1.445pt}}
\put(1201.67,228){\rule{0.400pt}{2.409pt}}
\multiput(1201.17,233.00)(1.000,-5.000){2}{\rule{0.400pt}{1.204pt}}
\put(1202.67,228){\rule{0.400pt}{2.168pt}}
\multiput(1202.17,228.00)(1.000,4.500){2}{\rule{0.400pt}{1.084pt}}
\put(1203.67,228){\rule{0.400pt}{2.168pt}}
\multiput(1203.17,232.50)(1.000,-4.500){2}{\rule{0.400pt}{1.084pt}}
\put(1204.67,228){\rule{0.400pt}{1.445pt}}
\multiput(1204.17,228.00)(1.000,3.000){2}{\rule{0.400pt}{0.723pt}}
\put(1205.67,225){\rule{0.400pt}{2.168pt}}
\multiput(1205.17,229.50)(1.000,-4.500){2}{\rule{0.400pt}{1.084pt}}
\put(1206.67,225){\rule{0.400pt}{2.168pt}}
\multiput(1206.17,225.00)(1.000,4.500){2}{\rule{0.400pt}{1.084pt}}
\put(1207.67,225){\rule{0.400pt}{2.168pt}}
\multiput(1207.17,229.50)(1.000,-4.500){2}{\rule{0.400pt}{1.084pt}}
\put(1208.67,225){\rule{0.400pt}{2.409pt}}
\multiput(1208.17,225.00)(1.000,5.000){2}{\rule{0.400pt}{1.204pt}}
\put(1209.67,227){\rule{0.400pt}{1.927pt}}
\multiput(1209.17,231.00)(1.000,-4.000){2}{\rule{0.400pt}{0.964pt}}
\put(1210.67,227){\rule{0.400pt}{1.204pt}}
\multiput(1210.17,227.00)(1.000,2.500){2}{\rule{0.400pt}{0.602pt}}
\put(1211.67,221){\rule{0.400pt}{2.650pt}}
\multiput(1211.17,226.50)(1.000,-5.500){2}{\rule{0.400pt}{1.325pt}}
\put(1212.67,221){\rule{0.400pt}{2.409pt}}
\multiput(1212.17,221.00)(1.000,5.000){2}{\rule{0.400pt}{1.204pt}}
\put(1213.67,223){\rule{0.400pt}{1.927pt}}
\multiput(1213.17,227.00)(1.000,-4.000){2}{\rule{0.400pt}{0.964pt}}
\put(1214.67,223){\rule{0.400pt}{1.686pt}}
\multiput(1214.17,223.00)(1.000,3.500){2}{\rule{0.400pt}{0.843pt}}
\put(1215.67,222){\rule{0.400pt}{1.927pt}}
\multiput(1215.17,226.00)(1.000,-4.000){2}{\rule{0.400pt}{0.964pt}}
\put(1217.17,222){\rule{0.400pt}{1.500pt}}
\multiput(1216.17,222.00)(2.000,3.887){2}{\rule{0.400pt}{0.750pt}}
\put(1218.67,221){\rule{0.400pt}{1.927pt}}
\multiput(1218.17,225.00)(1.000,-4.000){2}{\rule{0.400pt}{0.964pt}}
\put(1219.67,221){\rule{0.400pt}{1.686pt}}
\multiput(1219.17,221.00)(1.000,3.500){2}{\rule{0.400pt}{0.843pt}}
\put(1220.67,218){\rule{0.400pt}{2.409pt}}
\multiput(1220.17,223.00)(1.000,-5.000){2}{\rule{0.400pt}{1.204pt}}
\put(1221.67,218){\rule{0.400pt}{2.409pt}}
\multiput(1221.17,218.00)(1.000,5.000){2}{\rule{0.400pt}{1.204pt}}
\put(1222.67,220){\rule{0.400pt}{1.927pt}}
\multiput(1222.17,224.00)(1.000,-4.000){2}{\rule{0.400pt}{0.964pt}}
\put(1223.67,220){\rule{0.400pt}{1.204pt}}
\multiput(1223.17,220.00)(1.000,2.500){2}{\rule{0.400pt}{0.602pt}}
\put(1224.67,218){\rule{0.400pt}{1.686pt}}
\multiput(1224.17,221.50)(1.000,-3.500){2}{\rule{0.400pt}{0.843pt}}
\put(1225.67,218){\rule{0.400pt}{1.927pt}}
\multiput(1225.17,218.00)(1.000,4.000){2}{\rule{0.400pt}{0.964pt}}
\put(1226.67,218){\rule{0.400pt}{1.927pt}}
\multiput(1226.17,222.00)(1.000,-4.000){2}{\rule{0.400pt}{0.964pt}}
\put(1227.67,218){\rule{0.400pt}{1.204pt}}
\multiput(1227.17,218.00)(1.000,2.500){2}{\rule{0.400pt}{0.602pt}}
\put(1228.67,216){\rule{0.400pt}{1.686pt}}
\multiput(1228.17,219.50)(1.000,-3.500){2}{\rule{0.400pt}{0.843pt}}
\put(1229.67,216){\rule{0.400pt}{2.168pt}}
\multiput(1229.17,216.00)(1.000,4.500){2}{\rule{0.400pt}{1.084pt}}
\put(1230.67,218){\rule{0.400pt}{1.686pt}}
\multiput(1230.17,221.50)(1.000,-3.500){2}{\rule{0.400pt}{0.843pt}}
\put(1231.67,218){\rule{0.400pt}{1.204pt}}
\multiput(1231.17,218.00)(1.000,2.500){2}{\rule{0.400pt}{0.602pt}}
\put(1232.67,213){\rule{0.400pt}{2.409pt}}
\multiput(1232.17,218.00)(1.000,-5.000){2}{\rule{0.400pt}{1.204pt}}
\put(1233.67,213){\rule{0.400pt}{2.891pt}}
\multiput(1233.17,213.00)(1.000,6.000){2}{\rule{0.400pt}{1.445pt}}
\put(1234.67,215){\rule{0.400pt}{2.409pt}}
\multiput(1234.17,220.00)(1.000,-5.000){2}{\rule{0.400pt}{1.204pt}}
\put(1235.67,215){\rule{0.400pt}{1.445pt}}
\multiput(1235.17,215.00)(1.000,3.000){2}{\rule{0.400pt}{0.723pt}}
\put(1236.67,215){\rule{0.400pt}{1.445pt}}
\multiput(1236.17,218.00)(1.000,-3.000){2}{\rule{0.400pt}{0.723pt}}
\put(1237.67,215){\rule{0.400pt}{1.445pt}}
\multiput(1237.17,215.00)(1.000,3.000){2}{\rule{0.400pt}{0.723pt}}
\put(1238.67,214){\rule{0.400pt}{1.686pt}}
\multiput(1238.17,217.50)(1.000,-3.500){2}{\rule{0.400pt}{0.843pt}}
\put(1239.67,214){\rule{0.400pt}{2.168pt}}
\multiput(1239.17,214.00)(1.000,4.500){2}{\rule{0.400pt}{1.084pt}}
\put(1240.67,214){\rule{0.400pt}{2.168pt}}
\multiput(1240.17,218.50)(1.000,-4.500){2}{\rule{0.400pt}{1.084pt}}
\put(1241.67,214){\rule{0.400pt}{1.686pt}}
\multiput(1241.17,214.00)(1.000,3.500){2}{\rule{0.400pt}{0.843pt}}
\put(1243.17,215){\rule{0.400pt}{1.300pt}}
\multiput(1242.17,218.30)(2.000,-3.302){2}{\rule{0.400pt}{0.650pt}}
\put(1244.67,215){\rule{0.400pt}{0.723pt}}
\multiput(1244.17,215.00)(1.000,1.500){2}{\rule{0.400pt}{0.361pt}}
\put(1245.67,212){\rule{0.400pt}{1.445pt}}
\multiput(1245.17,215.00)(1.000,-3.000){2}{\rule{0.400pt}{0.723pt}}
\put(1246.67,212){\rule{0.400pt}{2.168pt}}
\multiput(1246.17,212.00)(1.000,4.500){2}{\rule{0.400pt}{1.084pt}}
\put(1247.67,213){\rule{0.400pt}{1.927pt}}
\multiput(1247.17,217.00)(1.000,-4.000){2}{\rule{0.400pt}{0.964pt}}
\put(1248.67,213){\rule{0.400pt}{1.445pt}}
\multiput(1248.17,213.00)(1.000,3.000){2}{\rule{0.400pt}{0.723pt}}
\put(1249.67,214){\rule{0.400pt}{1.204pt}}
\multiput(1249.17,216.50)(1.000,-2.500){2}{\rule{0.400pt}{0.602pt}}
\put(1250.67,214){\rule{0.400pt}{0.723pt}}
\multiput(1250.17,214.00)(1.000,1.500){2}{\rule{0.400pt}{0.361pt}}
\put(1251.67,212){\rule{0.400pt}{1.204pt}}
\multiput(1251.17,214.50)(1.000,-2.500){2}{\rule{0.400pt}{0.602pt}}
\put(1252.67,212){\rule{0.400pt}{0.964pt}}
\multiput(1252.17,212.00)(1.000,2.000){2}{\rule{0.400pt}{0.482pt}}
\put(1253.67,210){\rule{0.400pt}{1.445pt}}
\multiput(1253.17,213.00)(1.000,-3.000){2}{\rule{0.400pt}{0.723pt}}
\put(1254.67,210){\rule{0.400pt}{1.686pt}}
\multiput(1254.17,210.00)(1.000,3.500){2}{\rule{0.400pt}{0.843pt}}
\put(1255.67,212){\rule{0.400pt}{1.204pt}}
\multiput(1255.17,214.50)(1.000,-2.500){2}{\rule{0.400pt}{0.602pt}}
\put(1256.67,212){\rule{0.400pt}{0.964pt}}
\multiput(1256.17,212.00)(1.000,2.000){2}{\rule{0.400pt}{0.482pt}}
\put(1258,215.67){\rule{0.241pt}{0.400pt}}
\multiput(1258.00,215.17)(0.500,1.000){2}{\rule{0.120pt}{0.400pt}}
\put(1258.67,209){\rule{0.400pt}{1.927pt}}
\multiput(1258.17,213.00)(1.000,-4.000){2}{\rule{0.400pt}{0.964pt}}
\put(1260,208.67){\rule{0.241pt}{0.400pt}}
\multiput(1260.00,208.17)(0.500,1.000){2}{\rule{0.120pt}{0.400pt}}
\put(1260.67,210){\rule{0.400pt}{1.204pt}}
\multiput(1260.17,210.00)(1.000,2.500){2}{\rule{0.400pt}{0.602pt}}
\put(1261.67,210){\rule{0.400pt}{1.204pt}}
\multiput(1261.17,212.50)(1.000,-2.500){2}{\rule{0.400pt}{0.602pt}}
\put(1262.67,210){\rule{0.400pt}{0.964pt}}
\multiput(1262.17,210.00)(1.000,2.000){2}{\rule{0.400pt}{0.482pt}}
\put(1263.67,209){\rule{0.400pt}{1.204pt}}
\multiput(1263.17,211.50)(1.000,-2.500){2}{\rule{0.400pt}{0.602pt}}
\put(1264.67,209){\rule{0.400pt}{1.927pt}}
\multiput(1264.17,209.00)(1.000,4.000){2}{\rule{0.400pt}{0.964pt}}
\put(1265.67,209){\rule{0.400pt}{1.927pt}}
\multiput(1265.17,213.00)(1.000,-4.000){2}{\rule{0.400pt}{0.964pt}}
\put(1266.67,209){\rule{0.400pt}{1.927pt}}
\multiput(1266.17,209.00)(1.000,4.000){2}{\rule{0.400pt}{0.964pt}}
\put(1267.67,209){\rule{0.400pt}{1.927pt}}
\multiput(1267.17,213.00)(1.000,-4.000){2}{\rule{0.400pt}{0.964pt}}
\put(1269.17,209){\rule{0.400pt}{1.700pt}}
\multiput(1268.17,209.00)(2.000,4.472){2}{\rule{0.400pt}{0.850pt}}
\put(1270.67,215){\rule{0.400pt}{0.482pt}}
\multiput(1270.17,216.00)(1.000,-1.000){2}{\rule{0.400pt}{0.241pt}}
\put(1271.67,208){\rule{0.400pt}{1.686pt}}
\multiput(1271.17,211.50)(1.000,-3.500){2}{\rule{0.400pt}{0.843pt}}
\put(1273,207.67){\rule{0.241pt}{0.400pt}}
\multiput(1273.00,207.17)(0.500,1.000){2}{\rule{0.120pt}{0.400pt}}
\put(1273.67,209){\rule{0.400pt}{1.204pt}}
\multiput(1273.17,209.00)(1.000,2.500){2}{\rule{0.400pt}{0.602pt}}
\put(1274.67,210){\rule{0.400pt}{0.964pt}}
\multiput(1274.17,212.00)(1.000,-2.000){2}{\rule{0.400pt}{0.482pt}}
\put(1275.67,210){\rule{0.400pt}{1.204pt}}
\multiput(1275.17,210.00)(1.000,2.500){2}{\rule{0.400pt}{0.602pt}}
\put(1276.67,215){\rule{0.400pt}{0.482pt}}
\multiput(1276.17,215.00)(1.000,1.000){2}{\rule{0.400pt}{0.241pt}}
\put(1277.67,207){\rule{0.400pt}{2.409pt}}
\multiput(1277.17,212.00)(1.000,-5.000){2}{\rule{0.400pt}{1.204pt}}
\put(1278.67,207){\rule{0.400pt}{0.964pt}}
\multiput(1278.17,207.00)(1.000,2.000){2}{\rule{0.400pt}{0.482pt}}
\put(1279.67,211){\rule{0.400pt}{1.204pt}}
\multiput(1279.17,211.00)(1.000,2.500){2}{\rule{0.400pt}{0.602pt}}
\put(1281,214.67){\rule{0.241pt}{0.400pt}}
\multiput(1281.00,215.17)(0.500,-1.000){2}{\rule{0.120pt}{0.400pt}}
\put(1281.67,208){\rule{0.400pt}{1.686pt}}
\multiput(1281.17,211.50)(1.000,-3.500){2}{\rule{0.400pt}{0.843pt}}
\put(1282.67,208){\rule{0.400pt}{0.482pt}}
\multiput(1282.17,208.00)(1.000,1.000){2}{\rule{0.400pt}{0.241pt}}
\put(1283.67,210){\rule{0.400pt}{1.204pt}}
\multiput(1283.17,210.00)(1.000,2.500){2}{\rule{0.400pt}{0.602pt}}
\put(1284.67,213){\rule{0.400pt}{0.482pt}}
\multiput(1284.17,214.00)(1.000,-1.000){2}{\rule{0.400pt}{0.241pt}}
\put(1285.67,207){\rule{0.400pt}{1.445pt}}
\multiput(1285.17,210.00)(1.000,-3.000){2}{\rule{0.400pt}{0.723pt}}
\put(1286.67,207){\rule{0.400pt}{0.482pt}}
\multiput(1286.17,207.00)(1.000,1.000){2}{\rule{0.400pt}{0.241pt}}
\put(1287.67,209){\rule{0.400pt}{1.204pt}}
\multiput(1287.17,209.00)(1.000,2.500){2}{\rule{0.400pt}{0.602pt}}
\put(1150.0,646.0){\usebox{\plotpoint}}
\put(1289.67,207){\rule{0.400pt}{1.686pt}}
\multiput(1289.17,210.50)(1.000,-3.500){2}{\rule{0.400pt}{0.843pt}}
\put(1290.67,207){\rule{0.400pt}{1.204pt}}
\multiput(1290.17,207.00)(1.000,2.500){2}{\rule{0.400pt}{0.602pt}}
\put(1291.67,208){\rule{0.400pt}{0.964pt}}
\multiput(1291.17,210.00)(1.000,-2.000){2}{\rule{0.400pt}{0.482pt}}
\put(1293,206.67){\rule{0.241pt}{0.400pt}}
\multiput(1293.00,207.17)(0.500,-1.000){2}{\rule{0.120pt}{0.400pt}}
\put(1293.67,207){\rule{0.400pt}{1.445pt}}
\multiput(1293.17,207.00)(1.000,3.000){2}{\rule{0.400pt}{0.723pt}}
\put(1295,211.67){\rule{0.482pt}{0.400pt}}
\multiput(1295.00,212.17)(1.000,-1.000){2}{\rule{0.241pt}{0.400pt}}
\put(1296.67,205){\rule{0.400pt}{1.686pt}}
\multiput(1296.17,208.50)(1.000,-3.500){2}{\rule{0.400pt}{0.843pt}}
\put(1297.67,205){\rule{0.400pt}{0.723pt}}
\multiput(1297.17,205.00)(1.000,1.500){2}{\rule{0.400pt}{0.361pt}}
\put(1298.67,208){\rule{0.400pt}{1.204pt}}
\multiput(1298.17,208.00)(1.000,2.500){2}{\rule{0.400pt}{0.602pt}}
\put(1299.67,213){\rule{0.400pt}{0.482pt}}
\multiput(1299.17,213.00)(1.000,1.000){2}{\rule{0.400pt}{0.241pt}}
\put(1300.67,206){\rule{0.400pt}{2.168pt}}
\multiput(1300.17,210.50)(1.000,-4.500){2}{\rule{0.400pt}{1.084pt}}
\put(1301.67,206){\rule{0.400pt}{1.927pt}}
\multiput(1301.17,206.00)(1.000,4.000){2}{\rule{0.400pt}{0.964pt}}
\put(1302.67,205){\rule{0.400pt}{2.168pt}}
\multiput(1302.17,209.50)(1.000,-4.500){2}{\rule{0.400pt}{1.084pt}}
\put(1303.67,205){\rule{0.400pt}{1.204pt}}
\multiput(1303.17,205.00)(1.000,2.500){2}{\rule{0.400pt}{0.602pt}}
\put(1304.67,210){\rule{0.400pt}{0.482pt}}
\multiput(1304.17,210.00)(1.000,1.000){2}{\rule{0.400pt}{0.241pt}}
\put(1305.67,209){\rule{0.400pt}{0.723pt}}
\multiput(1305.17,210.50)(1.000,-1.500){2}{\rule{0.400pt}{0.361pt}}
\put(1306.67,209){\rule{0.400pt}{1.204pt}}
\multiput(1306.17,209.00)(1.000,2.500){2}{\rule{0.400pt}{0.602pt}}
\put(1307.67,207){\rule{0.400pt}{1.686pt}}
\multiput(1307.17,210.50)(1.000,-3.500){2}{\rule{0.400pt}{0.843pt}}
\put(1308.67,207){\rule{0.400pt}{1.445pt}}
\multiput(1308.17,207.00)(1.000,3.000){2}{\rule{0.400pt}{0.723pt}}
\put(1289.0,214.0){\usebox{\plotpoint}}
\put(1310.67,206){\rule{0.400pt}{1.686pt}}
\multiput(1310.17,209.50)(1.000,-3.500){2}{\rule{0.400pt}{0.843pt}}
\put(1311.67,206){\rule{0.400pt}{0.723pt}}
\multiput(1311.17,206.00)(1.000,1.500){2}{\rule{0.400pt}{0.361pt}}
\put(1312.67,209){\rule{0.400pt}{0.482pt}}
\multiput(1312.17,209.00)(1.000,1.000){2}{\rule{0.400pt}{0.241pt}}
\put(1313.67,208){\rule{0.400pt}{0.723pt}}
\multiput(1313.17,209.50)(1.000,-1.500){2}{\rule{0.400pt}{0.361pt}}
\put(1314.67,208){\rule{0.400pt}{1.204pt}}
\multiput(1314.17,208.00)(1.000,2.500){2}{\rule{0.400pt}{0.602pt}}
\put(1310.0,213.0){\usebox{\plotpoint}}
\put(1316.67,207){\rule{0.400pt}{1.445pt}}
\multiput(1316.17,210.00)(1.000,-3.000){2}{\rule{0.400pt}{0.723pt}}
\put(1318,205.67){\rule{0.241pt}{0.400pt}}
\multiput(1318.00,206.17)(0.500,-1.000){2}{\rule{0.120pt}{0.400pt}}
\put(1318.67,206){\rule{0.400pt}{1.445pt}}
\multiput(1318.17,206.00)(1.000,3.000){2}{\rule{0.400pt}{0.723pt}}
\put(1320,210.67){\rule{0.241pt}{0.400pt}}
\multiput(1320.00,211.17)(0.500,-1.000){2}{\rule{0.120pt}{0.400pt}}
\put(1321.17,208){\rule{0.400pt}{0.700pt}}
\multiput(1320.17,209.55)(2.000,-1.547){2}{\rule{0.400pt}{0.350pt}}
\put(1322.67,208){\rule{0.400pt}{0.723pt}}
\multiput(1322.17,208.00)(1.000,1.500){2}{\rule{0.400pt}{0.361pt}}
\put(1323.67,208){\rule{0.400pt}{0.723pt}}
\multiput(1323.17,209.50)(1.000,-1.500){2}{\rule{0.400pt}{0.361pt}}
\put(1316.0,213.0){\usebox{\plotpoint}}
\put(1325.67,208){\rule{0.400pt}{1.204pt}}
\multiput(1325.17,208.00)(1.000,2.500){2}{\rule{0.400pt}{0.602pt}}
\put(1325.0,208.0){\usebox{\plotpoint}}
\put(1327.67,206){\rule{0.400pt}{1.686pt}}
\multiput(1327.17,209.50)(1.000,-3.500){2}{\rule{0.400pt}{0.843pt}}
\put(1328.67,206){\rule{0.400pt}{1.445pt}}
\multiput(1328.17,206.00)(1.000,3.000){2}{\rule{0.400pt}{0.723pt}}
\put(1329.67,206){\rule{0.400pt}{1.445pt}}
\multiput(1329.17,209.00)(1.000,-3.000){2}{\rule{0.400pt}{0.723pt}}
\put(1327.0,213.0){\usebox{\plotpoint}}
\put(1331.67,206){\rule{0.400pt}{1.204pt}}
\multiput(1331.17,206.00)(1.000,2.500){2}{\rule{0.400pt}{0.602pt}}
\put(1331.0,206.0){\usebox{\plotpoint}}
\put(1333.67,208){\rule{0.400pt}{0.723pt}}
\multiput(1333.17,209.50)(1.000,-1.500){2}{\rule{0.400pt}{0.361pt}}
\put(1334.67,208){\rule{0.400pt}{0.482pt}}
\multiput(1334.17,208.00)(1.000,1.000){2}{\rule{0.400pt}{0.241pt}}
\put(1335.67,207){\rule{0.400pt}{0.723pt}}
\multiput(1335.17,208.50)(1.000,-1.500){2}{\rule{0.400pt}{0.361pt}}
\put(1333.0,211.0){\usebox{\plotpoint}}
\put(1337.67,207){\rule{0.400pt}{2.168pt}}
\multiput(1337.17,207.00)(1.000,4.500){2}{\rule{0.400pt}{1.084pt}}
\put(1338.67,206){\rule{0.400pt}{2.409pt}}
\multiput(1338.17,211.00)(1.000,-5.000){2}{\rule{0.400pt}{1.204pt}}
\put(1339.67,206){\rule{0.400pt}{2.168pt}}
\multiput(1339.17,206.00)(1.000,4.500){2}{\rule{0.400pt}{1.084pt}}
\put(1340.67,206){\rule{0.400pt}{2.168pt}}
\multiput(1340.17,210.50)(1.000,-4.500){2}{\rule{0.400pt}{1.084pt}}
\put(1341.67,206){\rule{0.400pt}{2.168pt}}
\multiput(1341.17,206.00)(1.000,4.500){2}{\rule{0.400pt}{1.084pt}}
\put(1342.67,206){\rule{0.400pt}{2.168pt}}
\multiput(1342.17,210.50)(1.000,-4.500){2}{\rule{0.400pt}{1.084pt}}
\put(1343.67,206){\rule{0.400pt}{2.168pt}}
\multiput(1343.17,206.00)(1.000,4.500){2}{\rule{0.400pt}{1.084pt}}
\put(1337.0,207.0){\usebox{\plotpoint}}
\put(1345.67,205){\rule{0.400pt}{2.409pt}}
\multiput(1345.17,210.00)(1.000,-5.000){2}{\rule{0.400pt}{1.204pt}}
\put(1347,204.67){\rule{0.482pt}{0.400pt}}
\multiput(1347.00,204.17)(1.000,1.000){2}{\rule{0.241pt}{0.400pt}}
\put(1348.67,206){\rule{0.400pt}{1.927pt}}
\multiput(1348.17,206.00)(1.000,4.000){2}{\rule{0.400pt}{0.964pt}}
\put(1345.0,215.0){\usebox{\plotpoint}}
\put(1350.67,211){\rule{0.400pt}{0.723pt}}
\multiput(1350.17,212.50)(1.000,-1.500){2}{\rule{0.400pt}{0.361pt}}
\put(1350.0,214.0){\usebox{\plotpoint}}
\put(1352.67,211){\rule{0.400pt}{0.482pt}}
\multiput(1352.17,211.00)(1.000,1.000){2}{\rule{0.400pt}{0.241pt}}
\put(1353.67,210){\rule{0.400pt}{0.723pt}}
\multiput(1353.17,211.50)(1.000,-1.500){2}{\rule{0.400pt}{0.361pt}}
\put(1354.67,210){\rule{0.400pt}{0.723pt}}
\multiput(1354.17,210.00)(1.000,1.500){2}{\rule{0.400pt}{0.361pt}}
\put(1356,211.67){\rule{0.241pt}{0.400pt}}
\multiput(1356.00,212.17)(0.500,-1.000){2}{\rule{0.120pt}{0.400pt}}
\put(1356.67,209){\rule{0.400pt}{0.723pt}}
\multiput(1356.17,210.50)(1.000,-1.500){2}{\rule{0.400pt}{0.361pt}}
\put(1357.67,209){\rule{0.400pt}{1.445pt}}
\multiput(1357.17,209.00)(1.000,3.000){2}{\rule{0.400pt}{0.723pt}}
\put(1358.67,209){\rule{0.400pt}{1.445pt}}
\multiput(1358.17,212.00)(1.000,-3.000){2}{\rule{0.400pt}{0.723pt}}
\put(1352.0,211.0){\usebox{\plotpoint}}
\put(1360.67,209){\rule{0.400pt}{1.445pt}}
\multiput(1360.17,209.00)(1.000,3.000){2}{\rule{0.400pt}{0.723pt}}
\put(1360.0,209.0){\usebox{\plotpoint}}
\put(1362.67,209){\rule{0.400pt}{1.445pt}}
\multiput(1362.17,212.00)(1.000,-3.000){2}{\rule{0.400pt}{0.723pt}}
\put(1362.0,215.0){\usebox{\plotpoint}}
\put(1364.67,209){\rule{0.400pt}{1.204pt}}
\multiput(1364.17,209.00)(1.000,2.500){2}{\rule{0.400pt}{0.602pt}}
\put(1365.67,208){\rule{0.400pt}{1.445pt}}
\multiput(1365.17,211.00)(1.000,-3.000){2}{\rule{0.400pt}{0.723pt}}
\put(1366.67,208){\rule{0.400pt}{1.445pt}}
\multiput(1366.17,208.00)(1.000,3.000){2}{\rule{0.400pt}{0.723pt}}
\put(1364.0,209.0){\usebox{\plotpoint}}
\put(1368.67,208){\rule{0.400pt}{1.445pt}}
\multiput(1368.17,211.00)(1.000,-3.000){2}{\rule{0.400pt}{0.723pt}}
\put(1369.67,208){\rule{0.400pt}{0.723pt}}
\multiput(1369.17,208.00)(1.000,1.500){2}{\rule{0.400pt}{0.361pt}}
\put(1370.67,211){\rule{0.400pt}{0.723pt}}
\multiput(1370.17,211.00)(1.000,1.500){2}{\rule{0.400pt}{0.361pt}}
\put(1371.67,211){\rule{0.400pt}{0.723pt}}
\multiput(1371.17,212.50)(1.000,-1.500){2}{\rule{0.400pt}{0.361pt}}
\put(1373,211.17){\rule{0.482pt}{0.400pt}}
\multiput(1373.00,210.17)(1.000,2.000){2}{\rule{0.241pt}{0.400pt}}
\put(1374.67,211){\rule{0.400pt}{0.482pt}}
\multiput(1374.17,212.00)(1.000,-1.000){2}{\rule{0.400pt}{0.241pt}}
\put(1375.67,211){\rule{0.400pt}{0.482pt}}
\multiput(1375.17,211.00)(1.000,1.000){2}{\rule{0.400pt}{0.241pt}}
\put(1376.67,210){\rule{0.400pt}{0.723pt}}
\multiput(1376.17,211.50)(1.000,-1.500){2}{\rule{0.400pt}{0.361pt}}
\put(1377.67,210){\rule{0.400pt}{0.723pt}}
\multiput(1377.17,210.00)(1.000,1.500){2}{\rule{0.400pt}{0.361pt}}
\put(1378.67,210){\rule{0.400pt}{0.723pt}}
\multiput(1378.17,211.50)(1.000,-1.500){2}{\rule{0.400pt}{0.361pt}}
\put(1379.67,210){\rule{0.400pt}{1.445pt}}
\multiput(1379.17,210.00)(1.000,3.000){2}{\rule{0.400pt}{0.723pt}}
\put(1380.67,209){\rule{0.400pt}{1.686pt}}
\multiput(1380.17,212.50)(1.000,-3.500){2}{\rule{0.400pt}{0.843pt}}
\put(1381.67,209){\rule{0.400pt}{0.723pt}}
\multiput(1381.17,209.00)(1.000,1.500){2}{\rule{0.400pt}{0.361pt}}
\put(1382.67,210){\rule{0.400pt}{0.482pt}}
\multiput(1382.17,211.00)(1.000,-1.000){2}{\rule{0.400pt}{0.241pt}}
\put(1383.67,210){\rule{0.400pt}{1.204pt}}
\multiput(1383.17,210.00)(1.000,2.500){2}{\rule{0.400pt}{0.602pt}}
\put(1384.67,209){\rule{0.400pt}{1.445pt}}
\multiput(1384.17,212.00)(1.000,-3.000){2}{\rule{0.400pt}{0.723pt}}
\put(1385.67,209){\rule{0.400pt}{1.445pt}}
\multiput(1385.17,209.00)(1.000,3.000){2}{\rule{0.400pt}{0.723pt}}
\put(1386.67,209){\rule{0.400pt}{1.445pt}}
\multiput(1386.17,212.00)(1.000,-3.000){2}{\rule{0.400pt}{0.723pt}}
\put(1387.67,209){\rule{0.400pt}{1.445pt}}
\multiput(1387.17,209.00)(1.000,3.000){2}{\rule{0.400pt}{0.723pt}}
\put(1389,213.67){\rule{0.241pt}{0.400pt}}
\multiput(1389.00,214.17)(0.500,-1.000){2}{\rule{0.120pt}{0.400pt}}
\put(1389.67,209){\rule{0.400pt}{1.204pt}}
\multiput(1389.17,211.50)(1.000,-2.500){2}{\rule{0.400pt}{0.602pt}}
\put(1391,207.67){\rule{0.241pt}{0.400pt}}
\multiput(1391.00,208.17)(0.500,-1.000){2}{\rule{0.120pt}{0.400pt}}
\put(1391.67,208){\rule{0.400pt}{1.445pt}}
\multiput(1391.17,208.00)(1.000,3.000){2}{\rule{0.400pt}{0.723pt}}
\put(1393,213.67){\rule{0.241pt}{0.400pt}}
\multiput(1393.00,213.17)(0.500,1.000){2}{\rule{0.120pt}{0.400pt}}
\put(1393.67,208){\rule{0.400pt}{1.686pt}}
\multiput(1393.17,211.50)(1.000,-3.500){2}{\rule{0.400pt}{0.843pt}}
\put(1395,207.67){\rule{0.241pt}{0.400pt}}
\multiput(1395.00,207.17)(0.500,1.000){2}{\rule{0.120pt}{0.400pt}}
\put(1395.67,209){\rule{0.400pt}{1.204pt}}
\multiput(1395.17,209.00)(1.000,2.500){2}{\rule{0.400pt}{0.602pt}}
\put(1368.0,214.0){\usebox{\plotpoint}}
\put(1397.67,208){\rule{0.400pt}{1.445pt}}
\multiput(1397.17,211.00)(1.000,-3.000){2}{\rule{0.400pt}{0.723pt}}
\put(1397.0,214.0){\usebox{\plotpoint}}
\put(1400.67,208){\rule{0.400pt}{1.445pt}}
\multiput(1400.17,208.00)(1.000,3.000){2}{\rule{0.400pt}{0.723pt}}
\put(1401.67,211){\rule{0.400pt}{0.723pt}}
\multiput(1401.17,212.50)(1.000,-1.500){2}{\rule{0.400pt}{0.361pt}}
\put(1402.67,211){\rule{0.400pt}{0.482pt}}
\multiput(1402.17,211.00)(1.000,1.000){2}{\rule{0.400pt}{0.241pt}}
\put(1403.67,211){\rule{0.400pt}{0.482pt}}
\multiput(1403.17,212.00)(1.000,-1.000){2}{\rule{0.400pt}{0.241pt}}
\put(1404.67,211){\rule{0.400pt}{0.482pt}}
\multiput(1404.17,211.00)(1.000,1.000){2}{\rule{0.400pt}{0.241pt}}
\put(1405.67,210){\rule{0.400pt}{0.723pt}}
\multiput(1405.17,211.50)(1.000,-1.500){2}{\rule{0.400pt}{0.361pt}}
\put(1406.67,210){\rule{0.400pt}{0.723pt}}
\multiput(1406.17,210.00)(1.000,1.500){2}{\rule{0.400pt}{0.361pt}}
\put(1407.67,210){\rule{0.400pt}{0.723pt}}
\multiput(1407.17,211.50)(1.000,-1.500){2}{\rule{0.400pt}{0.361pt}}
\put(1408.67,210){\rule{0.400pt}{0.723pt}}
\multiput(1408.17,210.00)(1.000,1.500){2}{\rule{0.400pt}{0.361pt}}
\put(1409.67,210){\rule{0.400pt}{0.723pt}}
\multiput(1409.17,211.50)(1.000,-1.500){2}{\rule{0.400pt}{0.361pt}}
\put(1410.67,210){\rule{0.400pt}{0.723pt}}
\multiput(1410.17,210.00)(1.000,1.500){2}{\rule{0.400pt}{0.361pt}}
\put(1411.67,209){\rule{0.400pt}{0.964pt}}
\multiput(1411.17,211.00)(1.000,-2.000){2}{\rule{0.400pt}{0.482pt}}
\put(1412.67,209){\rule{0.400pt}{1.445pt}}
\multiput(1412.17,209.00)(1.000,3.000){2}{\rule{0.400pt}{0.723pt}}
\put(1413.67,209){\rule{0.400pt}{1.445pt}}
\multiput(1413.17,212.00)(1.000,-3.000){2}{\rule{0.400pt}{0.723pt}}
\put(1414.67,209){\rule{0.400pt}{1.445pt}}
\multiput(1414.17,209.00)(1.000,3.000){2}{\rule{0.400pt}{0.723pt}}
\put(1415.67,210){\rule{0.400pt}{1.204pt}}
\multiput(1415.17,212.50)(1.000,-2.500){2}{\rule{0.400pt}{0.602pt}}
\put(1416.67,210){\rule{0.400pt}{1.204pt}}
\multiput(1416.17,210.00)(1.000,2.500){2}{\rule{0.400pt}{0.602pt}}
\put(1417.67,209){\rule{0.400pt}{1.445pt}}
\multiput(1417.17,212.00)(1.000,-3.000){2}{\rule{0.400pt}{0.723pt}}
\put(1418.67,209){\rule{0.400pt}{1.445pt}}
\multiput(1418.17,209.00)(1.000,3.000){2}{\rule{0.400pt}{0.723pt}}
\put(1420,213.67){\rule{0.241pt}{0.400pt}}
\multiput(1420.00,214.17)(0.500,-1.000){2}{\rule{0.120pt}{0.400pt}}
\put(1420.67,209){\rule{0.400pt}{1.204pt}}
\multiput(1420.17,211.50)(1.000,-2.500){2}{\rule{0.400pt}{0.602pt}}
\put(1421.67,209){\rule{0.400pt}{1.445pt}}
\multiput(1421.17,209.00)(1.000,3.000){2}{\rule{0.400pt}{0.723pt}}
\put(1422.67,209){\rule{0.400pt}{1.445pt}}
\multiput(1422.17,212.00)(1.000,-3.000){2}{\rule{0.400pt}{0.723pt}}
\put(1423.67,209){\rule{0.400pt}{1.445pt}}
\multiput(1423.17,209.00)(1.000,3.000){2}{\rule{0.400pt}{0.723pt}}
\put(1425.17,209){\rule{0.400pt}{1.300pt}}
\multiput(1424.17,212.30)(2.000,-3.302){2}{\rule{0.400pt}{0.650pt}}
\put(1426.67,209){\rule{0.400pt}{1.204pt}}
\multiput(1426.17,209.00)(1.000,2.500){2}{\rule{0.400pt}{0.602pt}}
\put(1427.67,208){\rule{0.400pt}{1.445pt}}
\multiput(1427.17,211.00)(1.000,-3.000){2}{\rule{0.400pt}{0.723pt}}
\put(1428.67,208){\rule{0.400pt}{1.445pt}}
\multiput(1428.17,208.00)(1.000,3.000){2}{\rule{0.400pt}{0.723pt}}
\put(1429.67,208){\rule{0.400pt}{1.445pt}}
\multiput(1429.17,211.00)(1.000,-3.000){2}{\rule{0.400pt}{0.723pt}}
\put(1430.67,208){\rule{0.400pt}{1.445pt}}
\multiput(1430.17,208.00)(1.000,3.000){2}{\rule{0.400pt}{0.723pt}}
\put(1431.67,208){\rule{0.400pt}{1.445pt}}
\multiput(1431.17,211.00)(1.000,-3.000){2}{\rule{0.400pt}{0.723pt}}
\put(1432.67,208){\rule{0.400pt}{1.445pt}}
\multiput(1432.17,208.00)(1.000,3.000){2}{\rule{0.400pt}{0.723pt}}
\put(1433.67,208){\rule{0.400pt}{1.445pt}}
\multiput(1433.17,211.00)(1.000,-3.000){2}{\rule{0.400pt}{0.723pt}}
\put(1434.67,208){\rule{0.400pt}{1.445pt}}
\multiput(1434.17,208.00)(1.000,3.000){2}{\rule{0.400pt}{0.723pt}}
\put(1435.67,208){\rule{0.400pt}{1.445pt}}
\multiput(1435.17,211.00)(1.000,-3.000){2}{\rule{0.400pt}{0.723pt}}
\put(1436.67,208){\rule{0.400pt}{0.482pt}}
\multiput(1436.17,208.00)(1.000,1.000){2}{\rule{0.400pt}{0.241pt}}
\put(1437.67,210){\rule{0.400pt}{0.723pt}}
\multiput(1437.17,210.00)(1.000,1.500){2}{\rule{0.400pt}{0.361pt}}
\put(1399.0,208.0){\rule[-0.200pt]{0.482pt}{0.400pt}}
\put(191.0,131.0){\rule[-0.200pt]{0.400pt}{175.375pt}}
\put(191.0,131.0){\rule[-0.200pt]{300.643pt}{0.400pt}}
\put(1439.0,131.0){\rule[-0.200pt]{0.400pt}{175.375pt}}
\put(191.0,859.0){\rule[-0.200pt]{300.643pt}{0.400pt}}
\end{picture}

\caption{
Závislosť prúdu plazmy $I_p\(t\)$ na čase $t$ pre výstrel \#23728. S určenými časmi začiatku a konca života plazmy.
}\label{G_V-1-I}
\end{figure}

\begin{figure}
% GNUPLOT: LaTeX picture
\setlength{\unitlength}{0.240900pt}
\ifx\plotpoint\undefined\newsavebox{\plotpoint}\fi
\begin{picture}(1500,900)(0,0)
\sbox{\plotpoint}{\rule[-0.200pt]{0.400pt}{0.400pt}}%
\put(191.0,131.0){\rule[-0.200pt]{4.818pt}{0.400pt}}
\put(171,131){\makebox(0,0)[r]{-0.25}}
\put(1419.0,131.0){\rule[-0.200pt]{4.818pt}{0.400pt}}
\put(191.0,222.0){\rule[-0.200pt]{4.818pt}{0.400pt}}
\put(171,222){\makebox(0,0)[r]{-0.2}}
\put(1419.0,222.0){\rule[-0.200pt]{4.818pt}{0.400pt}}
\put(191.0,313.0){\rule[-0.200pt]{4.818pt}{0.400pt}}
\put(171,313){\makebox(0,0)[r]{-0.15}}
\put(1419.0,313.0){\rule[-0.200pt]{4.818pt}{0.400pt}}
\put(191.0,404.0){\rule[-0.200pt]{4.818pt}{0.400pt}}
\put(171,404){\makebox(0,0)[r]{-0.1}}
\put(1419.0,404.0){\rule[-0.200pt]{4.818pt}{0.400pt}}
\put(191.0,495.0){\rule[-0.200pt]{4.818pt}{0.400pt}}
\put(171,495){\makebox(0,0)[r]{-0.05}}
\put(1419.0,495.0){\rule[-0.200pt]{4.818pt}{0.400pt}}
\put(191.0,586.0){\rule[-0.200pt]{4.818pt}{0.400pt}}
\put(171,586){\makebox(0,0)[r]{ 0}}
\put(1419.0,586.0){\rule[-0.200pt]{4.818pt}{0.400pt}}
\put(191.0,677.0){\rule[-0.200pt]{4.818pt}{0.400pt}}
\put(171,677){\makebox(0,0)[r]{ 0.05}}
\put(1419.0,677.0){\rule[-0.200pt]{4.818pt}{0.400pt}}
\put(191.0,768.0){\rule[-0.200pt]{4.818pt}{0.400pt}}
\put(171,768){\makebox(0,0)[r]{ 0.1}}
\put(1419.0,768.0){\rule[-0.200pt]{4.818pt}{0.400pt}}
\put(191.0,859.0){\rule[-0.200pt]{4.818pt}{0.400pt}}
\put(171,859){\makebox(0,0)[r]{ 0.15}}
\put(1419.0,859.0){\rule[-0.200pt]{4.818pt}{0.400pt}}
\put(191.0,131.0){\rule[-0.200pt]{0.400pt}{4.818pt}}
\put(191,90){\makebox(0,0){ 0}}
\put(191.0,839.0){\rule[-0.200pt]{0.400pt}{4.818pt}}
\put(399.0,131.0){\rule[-0.200pt]{0.400pt}{4.818pt}}
\put(399,90){\makebox(0,0){ 2}}
\put(399.0,839.0){\rule[-0.200pt]{0.400pt}{4.818pt}}
\put(607.0,131.0){\rule[-0.200pt]{0.400pt}{4.818pt}}
\put(607,90){\makebox(0,0){ 4}}
\put(607.0,839.0){\rule[-0.200pt]{0.400pt}{4.818pt}}
\put(815.0,131.0){\rule[-0.200pt]{0.400pt}{4.818pt}}
\put(815,90){\makebox(0,0){ 6}}
\put(815.0,839.0){\rule[-0.200pt]{0.400pt}{4.818pt}}
\put(1023.0,131.0){\rule[-0.200pt]{0.400pt}{4.818pt}}
\put(1023,90){\makebox(0,0){ 8}}
\put(1023.0,839.0){\rule[-0.200pt]{0.400pt}{4.818pt}}
\put(1231.0,131.0){\rule[-0.200pt]{0.400pt}{4.818pt}}
\put(1231,90){\makebox(0,0){ 10}}
\put(1231.0,839.0){\rule[-0.200pt]{0.400pt}{4.818pt}}
\put(1439.0,131.0){\rule[-0.200pt]{0.400pt}{4.818pt}}
\put(1439,90){\makebox(0,0){ 12}}
\put(1439.0,839.0){\rule[-0.200pt]{0.400pt}{4.818pt}}
\put(191.0,131.0){\rule[-0.200pt]{0.400pt}{175.375pt}}
\put(191.0,131.0){\rule[-0.200pt]{300.643pt}{0.400pt}}
\put(1439.0,131.0){\rule[-0.200pt]{0.400pt}{175.375pt}}
\put(191.0,859.0){\rule[-0.200pt]{300.643pt}{0.400pt}}
\put(30,495){\makebox(0,0){\popi{U}{V}}}
\put(815,29){\makebox(0,0){\popi{t}{ms}}}
\put(538,899){\line(0,-1){899}}
\put(1194,899){\line(0,-1){899}}
\put(538,899){\line(0,-1){899}}
\put(1194,899){\line(0,-1){899}}
\put(451,172){\makebox(0,0)[r]{$H_\alpha$ radiation}}
\put(471.0,172.0){\rule[-0.200pt]{24.090pt}{0.400pt}}
\put(192,586){\usebox{\plotpoint}}
\put(191.67,571){\rule{0.400pt}{3.614pt}}
\multiput(191.17,578.50)(1.000,-7.500){2}{\rule{0.400pt}{1.807pt}}
\put(192.67,571){\rule{0.400pt}{3.614pt}}
\multiput(192.17,571.00)(1.000,7.500){2}{\rule{0.400pt}{1.807pt}}
\put(193.67,571){\rule{0.400pt}{3.614pt}}
\multiput(193.17,578.50)(1.000,-7.500){2}{\rule{0.400pt}{1.807pt}}
\put(195.67,571){\rule{0.400pt}{3.614pt}}
\multiput(195.17,571.00)(1.000,7.500){2}{\rule{0.400pt}{1.807pt}}
\put(195.0,571.0){\usebox{\plotpoint}}
\put(197.67,571){\rule{0.400pt}{3.614pt}}
\multiput(197.17,578.50)(1.000,-7.500){2}{\rule{0.400pt}{1.807pt}}
\put(197.0,586.0){\usebox{\plotpoint}}
\put(199.67,571){\rule{0.400pt}{3.614pt}}
\multiput(199.17,571.00)(1.000,7.500){2}{\rule{0.400pt}{1.807pt}}
\put(200.67,571){\rule{0.400pt}{3.614pt}}
\multiput(200.17,578.50)(1.000,-7.500){2}{\rule{0.400pt}{1.807pt}}
\put(201.67,571){\rule{0.400pt}{3.614pt}}
\multiput(201.17,571.00)(1.000,7.500){2}{\rule{0.400pt}{1.807pt}}
\put(203.17,571){\rule{0.400pt}{3.100pt}}
\multiput(202.17,579.57)(2.000,-8.566){2}{\rule{0.400pt}{1.550pt}}
\put(204.67,571){\rule{0.400pt}{3.614pt}}
\multiput(204.17,571.00)(1.000,7.500){2}{\rule{0.400pt}{1.807pt}}
\put(205.67,571){\rule{0.400pt}{3.614pt}}
\multiput(205.17,578.50)(1.000,-7.500){2}{\rule{0.400pt}{1.807pt}}
\put(206.67,571){\rule{0.400pt}{3.614pt}}
\multiput(206.17,571.00)(1.000,7.500){2}{\rule{0.400pt}{1.807pt}}
\put(199.0,571.0){\usebox{\plotpoint}}
\put(208.67,571){\rule{0.400pt}{3.614pt}}
\multiput(208.17,578.50)(1.000,-7.500){2}{\rule{0.400pt}{1.807pt}}
\put(208.0,586.0){\usebox{\plotpoint}}
\put(210.67,571){\rule{0.400pt}{3.614pt}}
\multiput(210.17,571.00)(1.000,7.500){2}{\rule{0.400pt}{1.807pt}}
\put(211.67,571){\rule{0.400pt}{3.614pt}}
\multiput(211.17,578.50)(1.000,-7.500){2}{\rule{0.400pt}{1.807pt}}
\put(212.67,571){\rule{0.400pt}{3.614pt}}
\multiput(212.17,571.00)(1.000,7.500){2}{\rule{0.400pt}{1.807pt}}
\put(213.67,571){\rule{0.400pt}{3.614pt}}
\multiput(213.17,578.50)(1.000,-7.500){2}{\rule{0.400pt}{1.807pt}}
\put(214.67,571){\rule{0.400pt}{3.614pt}}
\multiput(214.17,571.00)(1.000,7.500){2}{\rule{0.400pt}{1.807pt}}
\put(210.0,571.0){\usebox{\plotpoint}}
\put(216.67,571){\rule{0.400pt}{3.614pt}}
\multiput(216.17,578.50)(1.000,-7.500){2}{\rule{0.400pt}{1.807pt}}
\put(217.67,571){\rule{0.400pt}{3.614pt}}
\multiput(217.17,571.00)(1.000,7.500){2}{\rule{0.400pt}{1.807pt}}
\put(218.67,571){\rule{0.400pt}{3.614pt}}
\multiput(218.17,578.50)(1.000,-7.500){2}{\rule{0.400pt}{1.807pt}}
\put(219.67,571){\rule{0.400pt}{3.614pt}}
\multiput(219.17,571.00)(1.000,7.500){2}{\rule{0.400pt}{1.807pt}}
\put(220.67,571){\rule{0.400pt}{3.614pt}}
\multiput(220.17,578.50)(1.000,-7.500){2}{\rule{0.400pt}{1.807pt}}
\put(221.67,564){\rule{0.400pt}{1.686pt}}
\multiput(221.17,567.50)(1.000,-3.500){2}{\rule{0.400pt}{0.843pt}}
\put(222.67,564){\rule{0.400pt}{5.300pt}}
\multiput(222.17,564.00)(1.000,11.000){2}{\rule{0.400pt}{2.650pt}}
\put(216.0,586.0){\usebox{\plotpoint}}
\put(224.67,564){\rule{0.400pt}{5.300pt}}
\multiput(224.17,575.00)(1.000,-11.000){2}{\rule{0.400pt}{2.650pt}}
\put(225.67,564){\rule{0.400pt}{1.686pt}}
\multiput(225.17,564.00)(1.000,3.500){2}{\rule{0.400pt}{0.843pt}}
\put(226.67,571){\rule{0.400pt}{3.614pt}}
\multiput(226.17,571.00)(1.000,7.500){2}{\rule{0.400pt}{1.807pt}}
\put(224.0,586.0){\usebox{\plotpoint}}
\put(229.17,571){\rule{0.400pt}{3.100pt}}
\multiput(228.17,579.57)(2.000,-8.566){2}{\rule{0.400pt}{1.550pt}}
\put(228.0,586.0){\usebox{\plotpoint}}
\put(231.67,571){\rule{0.400pt}{3.614pt}}
\multiput(231.17,571.00)(1.000,7.500){2}{\rule{0.400pt}{1.807pt}}
\put(231.0,571.0){\usebox{\plotpoint}}
\put(233.67,571){\rule{0.400pt}{3.614pt}}
\multiput(233.17,578.50)(1.000,-7.500){2}{\rule{0.400pt}{1.807pt}}
\put(234.67,571){\rule{0.400pt}{3.614pt}}
\multiput(234.17,571.00)(1.000,7.500){2}{\rule{0.400pt}{1.807pt}}
\put(235.67,564){\rule{0.400pt}{5.300pt}}
\multiput(235.17,575.00)(1.000,-11.000){2}{\rule{0.400pt}{2.650pt}}
\put(236.67,564){\rule{0.400pt}{1.686pt}}
\multiput(236.17,564.00)(1.000,3.500){2}{\rule{0.400pt}{0.843pt}}
\put(237.67,571){\rule{0.400pt}{3.614pt}}
\multiput(237.17,571.00)(1.000,7.500){2}{\rule{0.400pt}{1.807pt}}
\put(233.0,586.0){\usebox{\plotpoint}}
\put(239.67,571){\rule{0.400pt}{3.614pt}}
\multiput(239.17,578.50)(1.000,-7.500){2}{\rule{0.400pt}{1.807pt}}
\put(239.0,586.0){\usebox{\plotpoint}}
\put(241.67,571){\rule{0.400pt}{3.614pt}}
\multiput(241.17,571.00)(1.000,7.500){2}{\rule{0.400pt}{1.807pt}}
\put(242.67,571){\rule{0.400pt}{3.614pt}}
\multiput(242.17,578.50)(1.000,-7.500){2}{\rule{0.400pt}{1.807pt}}
\put(243.67,571){\rule{0.400pt}{3.614pt}}
\multiput(243.17,571.00)(1.000,7.500){2}{\rule{0.400pt}{1.807pt}}
\put(241.0,571.0){\usebox{\plotpoint}}
\put(245.67,564){\rule{0.400pt}{5.300pt}}
\multiput(245.17,575.00)(1.000,-11.000){2}{\rule{0.400pt}{2.650pt}}
\put(246.67,564){\rule{0.400pt}{5.300pt}}
\multiput(246.17,564.00)(1.000,11.000){2}{\rule{0.400pt}{2.650pt}}
\put(247.67,564){\rule{0.400pt}{5.300pt}}
\multiput(247.17,575.00)(1.000,-11.000){2}{\rule{0.400pt}{2.650pt}}
\put(245.0,586.0){\usebox{\plotpoint}}
\put(249.67,564){\rule{0.400pt}{5.300pt}}
\multiput(249.17,564.00)(1.000,11.000){2}{\rule{0.400pt}{2.650pt}}
\put(249.0,564.0){\usebox{\plotpoint}}
\put(251.67,564){\rule{0.400pt}{5.300pt}}
\multiput(251.17,575.00)(1.000,-11.000){2}{\rule{0.400pt}{2.650pt}}
\put(252.67,564){\rule{0.400pt}{5.300pt}}
\multiput(252.17,564.00)(1.000,11.000){2}{\rule{0.400pt}{2.650pt}}
\put(253.67,564){\rule{0.400pt}{5.300pt}}
\multiput(253.17,575.00)(1.000,-11.000){2}{\rule{0.400pt}{2.650pt}}
\put(255.17,564){\rule{0.400pt}{1.500pt}}
\multiput(254.17,564.00)(2.000,3.887){2}{\rule{0.400pt}{0.750pt}}
\put(256.67,571){\rule{0.400pt}{3.614pt}}
\multiput(256.17,571.00)(1.000,7.500){2}{\rule{0.400pt}{1.807pt}}
\put(257.67,579){\rule{0.400pt}{1.686pt}}
\multiput(257.17,582.50)(1.000,-3.500){2}{\rule{0.400pt}{0.843pt}}
\put(258.67,564){\rule{0.400pt}{3.614pt}}
\multiput(258.17,571.50)(1.000,-7.500){2}{\rule{0.400pt}{1.807pt}}
\put(259.67,564){\rule{0.400pt}{3.614pt}}
\multiput(259.17,564.00)(1.000,7.500){2}{\rule{0.400pt}{1.807pt}}
\put(260.67,564){\rule{0.400pt}{3.614pt}}
\multiput(260.17,571.50)(1.000,-7.500){2}{\rule{0.400pt}{1.807pt}}
\put(261.67,564){\rule{0.400pt}{3.614pt}}
\multiput(261.17,564.00)(1.000,7.500){2}{\rule{0.400pt}{1.807pt}}
\put(262.67,564){\rule{0.400pt}{3.614pt}}
\multiput(262.17,571.50)(1.000,-7.500){2}{\rule{0.400pt}{1.807pt}}
\put(263.67,564){\rule{0.400pt}{5.300pt}}
\multiput(263.17,564.00)(1.000,11.000){2}{\rule{0.400pt}{2.650pt}}
\put(264.67,564){\rule{0.400pt}{5.300pt}}
\multiput(264.17,575.00)(1.000,-11.000){2}{\rule{0.400pt}{2.650pt}}
\put(265.67,564){\rule{0.400pt}{3.614pt}}
\multiput(265.17,564.00)(1.000,7.500){2}{\rule{0.400pt}{1.807pt}}
\put(266.67,564){\rule{0.400pt}{3.614pt}}
\multiput(266.17,571.50)(1.000,-7.500){2}{\rule{0.400pt}{1.807pt}}
\put(267.67,564){\rule{0.400pt}{3.614pt}}
\multiput(267.17,564.00)(1.000,7.500){2}{\rule{0.400pt}{1.807pt}}
\put(268.67,564){\rule{0.400pt}{3.614pt}}
\multiput(268.17,571.50)(1.000,-7.500){2}{\rule{0.400pt}{1.807pt}}
\put(269.67,564){\rule{0.400pt}{3.614pt}}
\multiput(269.17,564.00)(1.000,7.500){2}{\rule{0.400pt}{1.807pt}}
\put(270.67,564){\rule{0.400pt}{3.614pt}}
\multiput(270.17,571.50)(1.000,-7.500){2}{\rule{0.400pt}{1.807pt}}
\put(271.67,564){\rule{0.400pt}{3.614pt}}
\multiput(271.17,564.00)(1.000,7.500){2}{\rule{0.400pt}{1.807pt}}
\put(272.67,564){\rule{0.400pt}{3.614pt}}
\multiput(272.17,571.50)(1.000,-7.500){2}{\rule{0.400pt}{1.807pt}}
\put(273.67,564){\rule{0.400pt}{3.614pt}}
\multiput(273.17,564.00)(1.000,7.500){2}{\rule{0.400pt}{1.807pt}}
\put(274.67,564){\rule{0.400pt}{3.614pt}}
\multiput(274.17,571.50)(1.000,-7.500){2}{\rule{0.400pt}{1.807pt}}
\put(275.67,564){\rule{0.400pt}{3.614pt}}
\multiput(275.17,564.00)(1.000,7.500){2}{\rule{0.400pt}{1.807pt}}
\put(276.67,564){\rule{0.400pt}{3.614pt}}
\multiput(276.17,571.50)(1.000,-7.500){2}{\rule{0.400pt}{1.807pt}}
\put(277.67,564){\rule{0.400pt}{3.614pt}}
\multiput(277.17,564.00)(1.000,7.500){2}{\rule{0.400pt}{1.807pt}}
\put(278.67,564){\rule{0.400pt}{3.614pt}}
\multiput(278.17,571.50)(1.000,-7.500){2}{\rule{0.400pt}{1.807pt}}
\put(279.67,564){\rule{0.400pt}{3.614pt}}
\multiput(279.17,564.00)(1.000,7.500){2}{\rule{0.400pt}{1.807pt}}
\put(281.17,564){\rule{0.400pt}{3.100pt}}
\multiput(280.17,572.57)(2.000,-8.566){2}{\rule{0.400pt}{1.550pt}}
\put(282.67,564){\rule{0.400pt}{3.614pt}}
\multiput(282.17,564.00)(1.000,7.500){2}{\rule{0.400pt}{1.807pt}}
\put(283.67,564){\rule{0.400pt}{3.614pt}}
\multiput(283.17,571.50)(1.000,-7.500){2}{\rule{0.400pt}{1.807pt}}
\put(284.67,564){\rule{0.400pt}{3.614pt}}
\multiput(284.17,564.00)(1.000,7.500){2}{\rule{0.400pt}{1.807pt}}
\put(285.67,564){\rule{0.400pt}{3.614pt}}
\multiput(285.17,571.50)(1.000,-7.500){2}{\rule{0.400pt}{1.807pt}}
\put(286.67,564){\rule{0.400pt}{3.614pt}}
\multiput(286.17,564.00)(1.000,7.500){2}{\rule{0.400pt}{1.807pt}}
\put(287.67,564){\rule{0.400pt}{3.614pt}}
\multiput(287.17,571.50)(1.000,-7.500){2}{\rule{0.400pt}{1.807pt}}
\put(288.67,564){\rule{0.400pt}{3.614pt}}
\multiput(288.17,564.00)(1.000,7.500){2}{\rule{0.400pt}{1.807pt}}
\put(289.67,564){\rule{0.400pt}{3.614pt}}
\multiput(289.17,571.50)(1.000,-7.500){2}{\rule{0.400pt}{1.807pt}}
\put(290.67,564){\rule{0.400pt}{3.614pt}}
\multiput(290.17,564.00)(1.000,7.500){2}{\rule{0.400pt}{1.807pt}}
\put(291.67,564){\rule{0.400pt}{3.614pt}}
\multiput(291.17,571.50)(1.000,-7.500){2}{\rule{0.400pt}{1.807pt}}
\put(292.67,564){\rule{0.400pt}{3.614pt}}
\multiput(292.17,564.00)(1.000,7.500){2}{\rule{0.400pt}{1.807pt}}
\put(293.67,564){\rule{0.400pt}{3.614pt}}
\multiput(293.17,571.50)(1.000,-7.500){2}{\rule{0.400pt}{1.807pt}}
\put(294.67,564){\rule{0.400pt}{3.614pt}}
\multiput(294.17,564.00)(1.000,7.500){2}{\rule{0.400pt}{1.807pt}}
\put(295.67,564){\rule{0.400pt}{3.614pt}}
\multiput(295.17,571.50)(1.000,-7.500){2}{\rule{0.400pt}{1.807pt}}
\put(296.67,564){\rule{0.400pt}{3.614pt}}
\multiput(296.17,564.00)(1.000,7.500){2}{\rule{0.400pt}{1.807pt}}
\put(297.67,564){\rule{0.400pt}{3.614pt}}
\multiput(297.17,571.50)(1.000,-7.500){2}{\rule{0.400pt}{1.807pt}}
\put(251.0,586.0){\usebox{\plotpoint}}
\put(299.67,564){\rule{0.400pt}{3.614pt}}
\multiput(299.17,564.00)(1.000,7.500){2}{\rule{0.400pt}{1.807pt}}
\put(299.0,564.0){\usebox{\plotpoint}}
\put(301.67,564){\rule{0.400pt}{3.614pt}}
\multiput(301.17,571.50)(1.000,-7.500){2}{\rule{0.400pt}{1.807pt}}
\put(302.67,564){\rule{0.400pt}{3.614pt}}
\multiput(302.17,564.00)(1.000,7.500){2}{\rule{0.400pt}{1.807pt}}
\put(303.67,564){\rule{0.400pt}{3.614pt}}
\multiput(303.17,571.50)(1.000,-7.500){2}{\rule{0.400pt}{1.807pt}}
\put(304.67,564){\rule{0.400pt}{3.614pt}}
\multiput(304.17,564.00)(1.000,7.500){2}{\rule{0.400pt}{1.807pt}}
\put(305.67,564){\rule{0.400pt}{3.614pt}}
\multiput(305.17,571.50)(1.000,-7.500){2}{\rule{0.400pt}{1.807pt}}
\put(301.0,579.0){\usebox{\plotpoint}}
\put(308.67,564){\rule{0.400pt}{3.614pt}}
\multiput(308.17,564.00)(1.000,7.500){2}{\rule{0.400pt}{1.807pt}}
\put(309.67,564){\rule{0.400pt}{3.614pt}}
\multiput(309.17,571.50)(1.000,-7.500){2}{\rule{0.400pt}{1.807pt}}
\put(310.67,564){\rule{0.400pt}{3.614pt}}
\multiput(310.17,564.00)(1.000,7.500){2}{\rule{0.400pt}{1.807pt}}
\put(307.0,564.0){\rule[-0.200pt]{0.482pt}{0.400pt}}
\put(312.67,564){\rule{0.400pt}{3.614pt}}
\multiput(312.17,571.50)(1.000,-7.500){2}{\rule{0.400pt}{1.807pt}}
\put(312.0,579.0){\usebox{\plotpoint}}
\put(314.67,564){\rule{0.400pt}{3.614pt}}
\multiput(314.17,564.00)(1.000,7.500){2}{\rule{0.400pt}{1.807pt}}
\put(314.0,564.0){\usebox{\plotpoint}}
\put(316.67,564){\rule{0.400pt}{3.614pt}}
\multiput(316.17,571.50)(1.000,-7.500){2}{\rule{0.400pt}{1.807pt}}
\put(316.0,579.0){\usebox{\plotpoint}}
\put(318.67,564){\rule{0.400pt}{3.614pt}}
\multiput(318.17,564.00)(1.000,7.500){2}{\rule{0.400pt}{1.807pt}}
\put(319.67,564){\rule{0.400pt}{3.614pt}}
\multiput(319.17,571.50)(1.000,-7.500){2}{\rule{0.400pt}{1.807pt}}
\put(320.67,564){\rule{0.400pt}{3.614pt}}
\multiput(320.17,564.00)(1.000,7.500){2}{\rule{0.400pt}{1.807pt}}
\put(318.0,564.0){\usebox{\plotpoint}}
\put(322.67,564){\rule{0.400pt}{3.614pt}}
\multiput(322.17,571.50)(1.000,-7.500){2}{\rule{0.400pt}{1.807pt}}
\put(322.0,579.0){\usebox{\plotpoint}}
\put(324.67,564){\rule{0.400pt}{3.614pt}}
\multiput(324.17,564.00)(1.000,7.500){2}{\rule{0.400pt}{1.807pt}}
\put(325.67,564){\rule{0.400pt}{3.614pt}}
\multiput(325.17,571.50)(1.000,-7.500){2}{\rule{0.400pt}{1.807pt}}
\put(326.67,564){\rule{0.400pt}{3.614pt}}
\multiput(326.17,564.00)(1.000,7.500){2}{\rule{0.400pt}{1.807pt}}
\put(327.67,564){\rule{0.400pt}{3.614pt}}
\multiput(327.17,571.50)(1.000,-7.500){2}{\rule{0.400pt}{1.807pt}}
\put(328.67,564){\rule{0.400pt}{3.614pt}}
\multiput(328.17,564.00)(1.000,7.500){2}{\rule{0.400pt}{1.807pt}}
\put(329.67,564){\rule{0.400pt}{3.614pt}}
\multiput(329.17,571.50)(1.000,-7.500){2}{\rule{0.400pt}{1.807pt}}
\put(330.67,564){\rule{0.400pt}{3.614pt}}
\multiput(330.17,564.00)(1.000,7.500){2}{\rule{0.400pt}{1.807pt}}
\put(331.67,564){\rule{0.400pt}{3.614pt}}
\multiput(331.17,571.50)(1.000,-7.500){2}{\rule{0.400pt}{1.807pt}}
\put(333.17,564){\rule{0.400pt}{3.100pt}}
\multiput(332.17,564.00)(2.000,8.566){2}{\rule{0.400pt}{1.550pt}}
\put(334.67,564){\rule{0.400pt}{3.614pt}}
\multiput(334.17,571.50)(1.000,-7.500){2}{\rule{0.400pt}{1.807pt}}
\put(335.67,564){\rule{0.400pt}{3.614pt}}
\multiput(335.17,564.00)(1.000,7.500){2}{\rule{0.400pt}{1.807pt}}
\put(336.67,564){\rule{0.400pt}{3.614pt}}
\multiput(336.17,571.50)(1.000,-7.500){2}{\rule{0.400pt}{1.807pt}}
\put(337.67,564){\rule{0.400pt}{3.614pt}}
\multiput(337.17,564.00)(1.000,7.500){2}{\rule{0.400pt}{1.807pt}}
\put(338.67,564){\rule{0.400pt}{3.614pt}}
\multiput(338.17,571.50)(1.000,-7.500){2}{\rule{0.400pt}{1.807pt}}
\put(339.67,564){\rule{0.400pt}{3.614pt}}
\multiput(339.17,564.00)(1.000,7.500){2}{\rule{0.400pt}{1.807pt}}
\put(340.67,564){\rule{0.400pt}{3.614pt}}
\multiput(340.17,571.50)(1.000,-7.500){2}{\rule{0.400pt}{1.807pt}}
\put(341.67,564){\rule{0.400pt}{3.614pt}}
\multiput(341.17,564.00)(1.000,7.500){2}{\rule{0.400pt}{1.807pt}}
\put(342.67,564){\rule{0.400pt}{3.614pt}}
\multiput(342.17,571.50)(1.000,-7.500){2}{\rule{0.400pt}{1.807pt}}
\put(343.67,564){\rule{0.400pt}{3.614pt}}
\multiput(343.17,564.00)(1.000,7.500){2}{\rule{0.400pt}{1.807pt}}
\put(344.67,564){\rule{0.400pt}{3.614pt}}
\multiput(344.17,571.50)(1.000,-7.500){2}{\rule{0.400pt}{1.807pt}}
\put(345.67,564){\rule{0.400pt}{3.614pt}}
\multiput(345.17,564.00)(1.000,7.500){2}{\rule{0.400pt}{1.807pt}}
\put(324.0,564.0){\usebox{\plotpoint}}
\put(347.67,564){\rule{0.400pt}{3.614pt}}
\multiput(347.17,571.50)(1.000,-7.500){2}{\rule{0.400pt}{1.807pt}}
\put(347.0,579.0){\usebox{\plotpoint}}
\put(349.67,564){\rule{0.400pt}{3.614pt}}
\multiput(349.17,564.00)(1.000,7.500){2}{\rule{0.400pt}{1.807pt}}
\put(349.0,564.0){\usebox{\plotpoint}}
\put(351.67,564){\rule{0.400pt}{3.614pt}}
\multiput(351.17,571.50)(1.000,-7.500){2}{\rule{0.400pt}{1.807pt}}
\put(352.67,564){\rule{0.400pt}{3.614pt}}
\multiput(352.17,564.00)(1.000,7.500){2}{\rule{0.400pt}{1.807pt}}
\put(353.67,564){\rule{0.400pt}{3.614pt}}
\multiput(353.17,571.50)(1.000,-7.500){2}{\rule{0.400pt}{1.807pt}}
\put(351.0,579.0){\usebox{\plotpoint}}
\put(355.67,564){\rule{0.400pt}{3.614pt}}
\multiput(355.17,564.00)(1.000,7.500){2}{\rule{0.400pt}{1.807pt}}
\put(356.67,564){\rule{0.400pt}{3.614pt}}
\multiput(356.17,571.50)(1.000,-7.500){2}{\rule{0.400pt}{1.807pt}}
\put(357.67,564){\rule{0.400pt}{3.614pt}}
\multiput(357.17,564.00)(1.000,7.500){2}{\rule{0.400pt}{1.807pt}}
\put(359.17,564){\rule{0.400pt}{3.100pt}}
\multiput(358.17,572.57)(2.000,-8.566){2}{\rule{0.400pt}{1.550pt}}
\put(360.67,564){\rule{0.400pt}{3.614pt}}
\multiput(360.17,564.00)(1.000,7.500){2}{\rule{0.400pt}{1.807pt}}
\put(361.67,564){\rule{0.400pt}{3.614pt}}
\multiput(361.17,571.50)(1.000,-7.500){2}{\rule{0.400pt}{1.807pt}}
\put(362.67,564){\rule{0.400pt}{3.614pt}}
\multiput(362.17,564.00)(1.000,7.500){2}{\rule{0.400pt}{1.807pt}}
\put(355.0,564.0){\usebox{\plotpoint}}
\put(364.67,564){\rule{0.400pt}{3.614pt}}
\multiput(364.17,571.50)(1.000,-7.500){2}{\rule{0.400pt}{1.807pt}}
\put(364.0,579.0){\usebox{\plotpoint}}
\put(366.67,564){\rule{0.400pt}{3.614pt}}
\multiput(366.17,564.00)(1.000,7.500){2}{\rule{0.400pt}{1.807pt}}
\put(366.0,564.0){\usebox{\plotpoint}}
\put(368.67,564){\rule{0.400pt}{3.614pt}}
\multiput(368.17,571.50)(1.000,-7.500){2}{\rule{0.400pt}{1.807pt}}
\put(368.0,579.0){\usebox{\plotpoint}}
\put(370.67,564){\rule{0.400pt}{3.614pt}}
\multiput(370.17,564.00)(1.000,7.500){2}{\rule{0.400pt}{1.807pt}}
\put(371.67,564){\rule{0.400pt}{3.614pt}}
\multiput(371.17,571.50)(1.000,-7.500){2}{\rule{0.400pt}{1.807pt}}
\put(372.67,564){\rule{0.400pt}{3.614pt}}
\multiput(372.17,564.00)(1.000,7.500){2}{\rule{0.400pt}{1.807pt}}
\put(373.67,564){\rule{0.400pt}{3.614pt}}
\multiput(373.17,571.50)(1.000,-7.500){2}{\rule{0.400pt}{1.807pt}}
\put(374.67,564){\rule{0.400pt}{3.614pt}}
\multiput(374.17,564.00)(1.000,7.500){2}{\rule{0.400pt}{1.807pt}}
\put(370.0,564.0){\usebox{\plotpoint}}
\put(376.67,564){\rule{0.400pt}{3.614pt}}
\multiput(376.17,571.50)(1.000,-7.500){2}{\rule{0.400pt}{1.807pt}}
\put(377.67,564){\rule{0.400pt}{3.614pt}}
\multiput(377.17,564.00)(1.000,7.500){2}{\rule{0.400pt}{1.807pt}}
\put(378.67,564){\rule{0.400pt}{3.614pt}}
\multiput(378.17,571.50)(1.000,-7.500){2}{\rule{0.400pt}{1.807pt}}
\put(376.0,579.0){\usebox{\plotpoint}}
\put(380.67,564){\rule{0.400pt}{3.614pt}}
\multiput(380.17,564.00)(1.000,7.500){2}{\rule{0.400pt}{1.807pt}}
\put(381.67,564){\rule{0.400pt}{3.614pt}}
\multiput(381.17,571.50)(1.000,-7.500){2}{\rule{0.400pt}{1.807pt}}
\put(382.67,564){\rule{0.400pt}{3.614pt}}
\multiput(382.17,564.00)(1.000,7.500){2}{\rule{0.400pt}{1.807pt}}
\put(383.67,564){\rule{0.400pt}{3.614pt}}
\multiput(383.17,571.50)(1.000,-7.500){2}{\rule{0.400pt}{1.807pt}}
\put(385.17,564){\rule{0.400pt}{3.100pt}}
\multiput(384.17,564.00)(2.000,8.566){2}{\rule{0.400pt}{1.550pt}}
\put(386.67,564){\rule{0.400pt}{3.614pt}}
\multiput(386.17,571.50)(1.000,-7.500){2}{\rule{0.400pt}{1.807pt}}
\put(387.67,564){\rule{0.400pt}{3.614pt}}
\multiput(387.17,564.00)(1.000,7.500){2}{\rule{0.400pt}{1.807pt}}
\put(388.67,564){\rule{0.400pt}{3.614pt}}
\multiput(388.17,571.50)(1.000,-7.500){2}{\rule{0.400pt}{1.807pt}}
\put(389.67,564){\rule{0.400pt}{3.614pt}}
\multiput(389.17,564.00)(1.000,7.500){2}{\rule{0.400pt}{1.807pt}}
\put(390.67,564){\rule{0.400pt}{3.614pt}}
\multiput(390.17,571.50)(1.000,-7.500){2}{\rule{0.400pt}{1.807pt}}
\put(391.67,564){\rule{0.400pt}{3.614pt}}
\multiput(391.17,564.00)(1.000,7.500){2}{\rule{0.400pt}{1.807pt}}
\put(392.67,564){\rule{0.400pt}{3.614pt}}
\multiput(392.17,571.50)(1.000,-7.500){2}{\rule{0.400pt}{1.807pt}}
\put(393.67,564){\rule{0.400pt}{3.614pt}}
\multiput(393.17,564.00)(1.000,7.500){2}{\rule{0.400pt}{1.807pt}}
\put(394.67,564){\rule{0.400pt}{3.614pt}}
\multiput(394.17,571.50)(1.000,-7.500){2}{\rule{0.400pt}{1.807pt}}
\put(395.67,564){\rule{0.400pt}{3.614pt}}
\multiput(395.17,564.00)(1.000,7.500){2}{\rule{0.400pt}{1.807pt}}
\put(396.67,564){\rule{0.400pt}{3.614pt}}
\multiput(396.17,571.50)(1.000,-7.500){2}{\rule{0.400pt}{1.807pt}}
\put(397.67,564){\rule{0.400pt}{3.614pt}}
\multiput(397.17,564.00)(1.000,7.500){2}{\rule{0.400pt}{1.807pt}}
\put(380.0,564.0){\usebox{\plotpoint}}
\put(399.67,564){\rule{0.400pt}{3.614pt}}
\multiput(399.17,571.50)(1.000,-7.500){2}{\rule{0.400pt}{1.807pt}}
\put(400.67,564){\rule{0.400pt}{3.614pt}}
\multiput(400.17,564.00)(1.000,7.500){2}{\rule{0.400pt}{1.807pt}}
\put(401.67,564){\rule{0.400pt}{3.614pt}}
\multiput(401.17,571.50)(1.000,-7.500){2}{\rule{0.400pt}{1.807pt}}
\put(402.67,564){\rule{0.400pt}{3.614pt}}
\multiput(402.17,564.00)(1.000,7.500){2}{\rule{0.400pt}{1.807pt}}
\put(403.67,564){\rule{0.400pt}{3.614pt}}
\multiput(403.17,571.50)(1.000,-7.500){2}{\rule{0.400pt}{1.807pt}}
\put(404.67,564){\rule{0.400pt}{3.614pt}}
\multiput(404.17,564.00)(1.000,7.500){2}{\rule{0.400pt}{1.807pt}}
\put(405.67,564){\rule{0.400pt}{3.614pt}}
\multiput(405.17,571.50)(1.000,-7.500){2}{\rule{0.400pt}{1.807pt}}
\put(399.0,579.0){\usebox{\plotpoint}}
\put(407.67,564){\rule{0.400pt}{3.614pt}}
\multiput(407.17,564.00)(1.000,7.500){2}{\rule{0.400pt}{1.807pt}}
\put(408.67,579){\rule{0.400pt}{20.958pt}}
\multiput(408.17,579.00)(1.000,43.500){2}{\rule{0.400pt}{10.479pt}}
\put(409.67,470){\rule{0.400pt}{47.216pt}}
\multiput(409.17,568.00)(1.000,-98.000){2}{\rule{0.400pt}{23.608pt}}
\put(411.17,470){\rule{0.400pt}{64.100pt}}
\multiput(410.17,470.00)(2.000,186.957){2}{\rule{0.400pt}{32.050pt}}
\put(412.67,266){\rule{0.400pt}{126.232pt}}
\multiput(412.17,528.00)(1.000,-262.000){2}{\rule{0.400pt}{63.116pt}}
\put(413.67,266){\rule{0.400pt}{71.788pt}}
\multiput(413.17,266.00)(1.000,149.000){2}{\rule{0.400pt}{35.894pt}}
\put(414.67,564){\rule{0.400pt}{3.614pt}}
\multiput(414.17,564.00)(1.000,7.500){2}{\rule{0.400pt}{1.807pt}}
\put(415.67,564){\rule{0.400pt}{3.614pt}}
\multiput(415.17,571.50)(1.000,-7.500){2}{\rule{0.400pt}{1.807pt}}
\put(416.67,564){\rule{0.400pt}{3.614pt}}
\multiput(416.17,564.00)(1.000,7.500){2}{\rule{0.400pt}{1.807pt}}
\put(407.0,564.0){\usebox{\plotpoint}}
\put(418.67,564){\rule{0.400pt}{3.614pt}}
\multiput(418.17,571.50)(1.000,-7.500){2}{\rule{0.400pt}{1.807pt}}
\put(418.0,579.0){\usebox{\plotpoint}}
\put(420.67,564){\rule{0.400pt}{3.614pt}}
\multiput(420.17,564.00)(1.000,7.500){2}{\rule{0.400pt}{1.807pt}}
\put(421.67,564){\rule{0.400pt}{3.614pt}}
\multiput(421.17,571.50)(1.000,-7.500){2}{\rule{0.400pt}{1.807pt}}
\put(422.67,564){\rule{0.400pt}{5.300pt}}
\multiput(422.17,564.00)(1.000,11.000){2}{\rule{0.400pt}{2.650pt}}
\put(423.67,564){\rule{0.400pt}{5.300pt}}
\multiput(423.17,575.00)(1.000,-11.000){2}{\rule{0.400pt}{2.650pt}}
\put(424.67,564){\rule{0.400pt}{5.300pt}}
\multiput(424.17,564.00)(1.000,11.000){2}{\rule{0.400pt}{2.650pt}}
\put(425.67,564){\rule{0.400pt}{5.300pt}}
\multiput(425.17,575.00)(1.000,-11.000){2}{\rule{0.400pt}{2.650pt}}
\put(426.67,564){\rule{0.400pt}{5.300pt}}
\multiput(426.17,564.00)(1.000,11.000){2}{\rule{0.400pt}{2.650pt}}
\put(427.67,564){\rule{0.400pt}{5.300pt}}
\multiput(427.17,575.00)(1.000,-11.000){2}{\rule{0.400pt}{2.650pt}}
\put(428.67,564){\rule{0.400pt}{5.300pt}}
\multiput(428.17,564.00)(1.000,11.000){2}{\rule{0.400pt}{2.650pt}}
\put(429.67,579){\rule{0.400pt}{1.686pt}}
\multiput(429.17,582.50)(1.000,-3.500){2}{\rule{0.400pt}{0.843pt}}
\put(430.67,564){\rule{0.400pt}{3.614pt}}
\multiput(430.17,571.50)(1.000,-7.500){2}{\rule{0.400pt}{1.807pt}}
\put(431.67,564){\rule{0.400pt}{3.614pt}}
\multiput(431.17,564.00)(1.000,7.500){2}{\rule{0.400pt}{1.807pt}}
\put(432.67,564){\rule{0.400pt}{3.614pt}}
\multiput(432.17,571.50)(1.000,-7.500){2}{\rule{0.400pt}{1.807pt}}
\put(433.67,564){\rule{0.400pt}{3.614pt}}
\multiput(433.17,564.00)(1.000,7.500){2}{\rule{0.400pt}{1.807pt}}
\put(434.67,564){\rule{0.400pt}{3.614pt}}
\multiput(434.17,571.50)(1.000,-7.500){2}{\rule{0.400pt}{1.807pt}}
\put(435.67,564){\rule{0.400pt}{3.614pt}}
\multiput(435.17,564.00)(1.000,7.500){2}{\rule{0.400pt}{1.807pt}}
\put(437.17,564){\rule{0.400pt}{3.100pt}}
\multiput(436.17,572.57)(2.000,-8.566){2}{\rule{0.400pt}{1.550pt}}
\put(438.67,564){\rule{0.400pt}{3.614pt}}
\multiput(438.17,564.00)(1.000,7.500){2}{\rule{0.400pt}{1.807pt}}
\put(439.67,564){\rule{0.400pt}{3.614pt}}
\multiput(439.17,571.50)(1.000,-7.500){2}{\rule{0.400pt}{1.807pt}}
\put(420.0,564.0){\usebox{\plotpoint}}
\put(441.67,564){\rule{0.400pt}{5.300pt}}
\multiput(441.17,564.00)(1.000,11.000){2}{\rule{0.400pt}{2.650pt}}
\put(442.67,564){\rule{0.400pt}{5.300pt}}
\multiput(442.17,575.00)(1.000,-11.000){2}{\rule{0.400pt}{2.650pt}}
\put(443.67,564){\rule{0.400pt}{5.300pt}}
\multiput(443.17,564.00)(1.000,11.000){2}{\rule{0.400pt}{2.650pt}}
\put(444.67,564){\rule{0.400pt}{5.300pt}}
\multiput(444.17,575.00)(1.000,-11.000){2}{\rule{0.400pt}{2.650pt}}
\put(445.67,564){\rule{0.400pt}{3.614pt}}
\multiput(445.17,564.00)(1.000,7.500){2}{\rule{0.400pt}{1.807pt}}
\put(446.67,564){\rule{0.400pt}{3.614pt}}
\multiput(446.17,571.50)(1.000,-7.500){2}{\rule{0.400pt}{1.807pt}}
\put(447.67,564){\rule{0.400pt}{5.300pt}}
\multiput(447.17,564.00)(1.000,11.000){2}{\rule{0.400pt}{2.650pt}}
\put(448.67,564){\rule{0.400pt}{5.300pt}}
\multiput(448.17,575.00)(1.000,-11.000){2}{\rule{0.400pt}{2.650pt}}
\put(449.67,564){\rule{0.400pt}{5.300pt}}
\multiput(449.17,564.00)(1.000,11.000){2}{\rule{0.400pt}{2.650pt}}
\put(450.67,564){\rule{0.400pt}{5.300pt}}
\multiput(450.17,575.00)(1.000,-11.000){2}{\rule{0.400pt}{2.650pt}}
\put(451.67,564){\rule{0.400pt}{5.300pt}}
\multiput(451.17,564.00)(1.000,11.000){2}{\rule{0.400pt}{2.650pt}}
\put(452.67,564){\rule{0.400pt}{5.300pt}}
\multiput(452.17,575.00)(1.000,-11.000){2}{\rule{0.400pt}{2.650pt}}
\put(453.67,564){\rule{0.400pt}{5.300pt}}
\multiput(453.17,564.00)(1.000,11.000){2}{\rule{0.400pt}{2.650pt}}
\put(454.67,564){\rule{0.400pt}{5.300pt}}
\multiput(454.17,575.00)(1.000,-11.000){2}{\rule{0.400pt}{2.650pt}}
\put(455.67,564){\rule{0.400pt}{5.300pt}}
\multiput(455.17,564.00)(1.000,11.000){2}{\rule{0.400pt}{2.650pt}}
\put(456.67,579){\rule{0.400pt}{1.686pt}}
\multiput(456.17,582.50)(1.000,-3.500){2}{\rule{0.400pt}{0.843pt}}
\put(457.67,564){\rule{0.400pt}{3.614pt}}
\multiput(457.17,571.50)(1.000,-7.500){2}{\rule{0.400pt}{1.807pt}}
\put(458.67,564){\rule{0.400pt}{3.614pt}}
\multiput(458.17,564.00)(1.000,7.500){2}{\rule{0.400pt}{1.807pt}}
\put(459.67,564){\rule{0.400pt}{3.614pt}}
\multiput(459.17,571.50)(1.000,-7.500){2}{\rule{0.400pt}{1.807pt}}
\put(460.67,564){\rule{0.400pt}{5.300pt}}
\multiput(460.17,564.00)(1.000,11.000){2}{\rule{0.400pt}{2.650pt}}
\put(461.67,564){\rule{0.400pt}{5.300pt}}
\multiput(461.17,575.00)(1.000,-11.000){2}{\rule{0.400pt}{2.650pt}}
\put(463.17,564){\rule{0.400pt}{1.500pt}}
\multiput(462.17,564.00)(2.000,3.887){2}{\rule{0.400pt}{0.750pt}}
\put(464.67,571){\rule{0.400pt}{3.614pt}}
\multiput(464.17,571.00)(1.000,7.500){2}{\rule{0.400pt}{1.807pt}}
\put(441.0,564.0){\usebox{\plotpoint}}
\put(466.67,571){\rule{0.400pt}{3.614pt}}
\multiput(466.17,578.50)(1.000,-7.500){2}{\rule{0.400pt}{1.807pt}}
\put(467.67,571){\rule{0.400pt}{3.614pt}}
\multiput(467.17,571.00)(1.000,7.500){2}{\rule{0.400pt}{1.807pt}}
\put(468.67,571){\rule{0.400pt}{3.614pt}}
\multiput(468.17,578.50)(1.000,-7.500){2}{\rule{0.400pt}{1.807pt}}
\put(469.67,571){\rule{0.400pt}{3.614pt}}
\multiput(469.17,571.00)(1.000,7.500){2}{\rule{0.400pt}{1.807pt}}
\put(470.67,571){\rule{0.400pt}{3.614pt}}
\multiput(470.17,578.50)(1.000,-7.500){2}{\rule{0.400pt}{1.807pt}}
\put(471.67,571){\rule{0.400pt}{3.614pt}}
\multiput(471.17,571.00)(1.000,7.500){2}{\rule{0.400pt}{1.807pt}}
\put(472.67,571){\rule{0.400pt}{3.614pt}}
\multiput(472.17,578.50)(1.000,-7.500){2}{\rule{0.400pt}{1.807pt}}
\put(473.67,571){\rule{0.400pt}{3.614pt}}
\multiput(473.17,571.00)(1.000,7.500){2}{\rule{0.400pt}{1.807pt}}
\put(474.67,571){\rule{0.400pt}{3.614pt}}
\multiput(474.17,578.50)(1.000,-7.500){2}{\rule{0.400pt}{1.807pt}}
\put(475.67,571){\rule{0.400pt}{3.614pt}}
\multiput(475.17,571.00)(1.000,7.500){2}{\rule{0.400pt}{1.807pt}}
\put(476.67,571){\rule{0.400pt}{3.614pt}}
\multiput(476.17,578.50)(1.000,-7.500){2}{\rule{0.400pt}{1.807pt}}
\put(477.67,571){\rule{0.400pt}{3.614pt}}
\multiput(477.17,571.00)(1.000,7.500){2}{\rule{0.400pt}{1.807pt}}
\put(478.67,571){\rule{0.400pt}{3.614pt}}
\multiput(478.17,578.50)(1.000,-7.500){2}{\rule{0.400pt}{1.807pt}}
\put(479.67,571){\rule{0.400pt}{3.614pt}}
\multiput(479.17,571.00)(1.000,7.500){2}{\rule{0.400pt}{1.807pt}}
\put(480.67,571){\rule{0.400pt}{3.614pt}}
\multiput(480.17,578.50)(1.000,-7.500){2}{\rule{0.400pt}{1.807pt}}
\put(466.0,586.0){\usebox{\plotpoint}}
\put(482.67,571){\rule{0.400pt}{3.614pt}}
\multiput(482.17,571.00)(1.000,7.500){2}{\rule{0.400pt}{1.807pt}}
\put(483.67,571){\rule{0.400pt}{3.614pt}}
\multiput(483.17,578.50)(1.000,-7.500){2}{\rule{0.400pt}{1.807pt}}
\put(484.67,571){\rule{0.400pt}{3.614pt}}
\multiput(484.17,571.00)(1.000,7.500){2}{\rule{0.400pt}{1.807pt}}
\put(482.0,571.0){\usebox{\plotpoint}}
\put(486.67,571){\rule{0.400pt}{3.614pt}}
\multiput(486.17,578.50)(1.000,-7.500){2}{\rule{0.400pt}{1.807pt}}
\put(487.67,571){\rule{0.400pt}{3.614pt}}
\multiput(487.17,571.00)(1.000,7.500){2}{\rule{0.400pt}{1.807pt}}
\put(489.17,571){\rule{0.400pt}{3.100pt}}
\multiput(488.17,579.57)(2.000,-8.566){2}{\rule{0.400pt}{1.550pt}}
\put(490.67,571){\rule{0.400pt}{3.614pt}}
\multiput(490.17,571.00)(1.000,7.500){2}{\rule{0.400pt}{1.807pt}}
\put(491.67,571){\rule{0.400pt}{3.614pt}}
\multiput(491.17,578.50)(1.000,-7.500){2}{\rule{0.400pt}{1.807pt}}
\put(492.67,571){\rule{0.400pt}{3.614pt}}
\multiput(492.17,571.00)(1.000,7.500){2}{\rule{0.400pt}{1.807pt}}
\put(493.67,571){\rule{0.400pt}{3.614pt}}
\multiput(493.17,578.50)(1.000,-7.500){2}{\rule{0.400pt}{1.807pt}}
\put(494.67,571){\rule{0.400pt}{5.300pt}}
\multiput(494.17,571.00)(1.000,11.000){2}{\rule{0.400pt}{2.650pt}}
\put(495.67,564){\rule{0.400pt}{6.986pt}}
\multiput(495.17,578.50)(1.000,-14.500){2}{\rule{0.400pt}{3.493pt}}
\put(496.67,564){\rule{0.400pt}{5.300pt}}
\multiput(496.17,564.00)(1.000,11.000){2}{\rule{0.400pt}{2.650pt}}
\put(497.67,571){\rule{0.400pt}{3.614pt}}
\multiput(497.17,578.50)(1.000,-7.500){2}{\rule{0.400pt}{1.807pt}}
\put(498.67,571){\rule{0.400pt}{3.614pt}}
\multiput(498.17,571.00)(1.000,7.500){2}{\rule{0.400pt}{1.807pt}}
\put(499.67,571){\rule{0.400pt}{3.614pt}}
\multiput(499.17,578.50)(1.000,-7.500){2}{\rule{0.400pt}{1.807pt}}
\put(500.67,571){\rule{0.400pt}{3.614pt}}
\multiput(500.17,571.00)(1.000,7.500){2}{\rule{0.400pt}{1.807pt}}
\put(501.67,571){\rule{0.400pt}{3.614pt}}
\multiput(501.17,578.50)(1.000,-7.500){2}{\rule{0.400pt}{1.807pt}}
\put(486.0,586.0){\usebox{\plotpoint}}
\put(503.67,571){\rule{0.400pt}{3.614pt}}
\multiput(503.17,571.00)(1.000,7.500){2}{\rule{0.400pt}{1.807pt}}
\put(503.0,571.0){\usebox{\plotpoint}}
\put(505.67,571){\rule{0.400pt}{3.614pt}}
\multiput(505.17,578.50)(1.000,-7.500){2}{\rule{0.400pt}{1.807pt}}
\put(506.67,571){\rule{0.400pt}{3.614pt}}
\multiput(506.17,571.00)(1.000,7.500){2}{\rule{0.400pt}{1.807pt}}
\put(507.67,571){\rule{0.400pt}{3.614pt}}
\multiput(507.17,578.50)(1.000,-7.500){2}{\rule{0.400pt}{1.807pt}}
\put(505.0,586.0){\usebox{\plotpoint}}
\put(509.67,571){\rule{0.400pt}{3.614pt}}
\multiput(509.17,571.00)(1.000,7.500){2}{\rule{0.400pt}{1.807pt}}
\put(509.0,571.0){\usebox{\plotpoint}}
\put(511.67,571){\rule{0.400pt}{3.614pt}}
\multiput(511.17,578.50)(1.000,-7.500){2}{\rule{0.400pt}{1.807pt}}
\put(512.67,571){\rule{0.400pt}{3.614pt}}
\multiput(512.17,571.00)(1.000,7.500){2}{\rule{0.400pt}{1.807pt}}
\put(513.67,571){\rule{0.400pt}{3.614pt}}
\multiput(513.17,578.50)(1.000,-7.500){2}{\rule{0.400pt}{1.807pt}}
\put(511.0,586.0){\usebox{\plotpoint}}
\put(516.67,571){\rule{0.400pt}{3.614pt}}
\multiput(516.17,571.00)(1.000,7.500){2}{\rule{0.400pt}{1.807pt}}
\put(515.0,571.0){\rule[-0.200pt]{0.482pt}{0.400pt}}
\put(518.67,571){\rule{0.400pt}{3.614pt}}
\multiput(518.17,578.50)(1.000,-7.500){2}{\rule{0.400pt}{1.807pt}}
\put(519.67,571){\rule{0.400pt}{3.614pt}}
\multiput(519.17,571.00)(1.000,7.500){2}{\rule{0.400pt}{1.807pt}}
\put(520.67,571){\rule{0.400pt}{3.614pt}}
\multiput(520.17,578.50)(1.000,-7.500){2}{\rule{0.400pt}{1.807pt}}
\put(518.0,586.0){\usebox{\plotpoint}}
\put(522.67,571){\rule{0.400pt}{3.614pt}}
\multiput(522.17,571.00)(1.000,7.500){2}{\rule{0.400pt}{1.807pt}}
\put(522.0,571.0){\usebox{\plotpoint}}
\put(524.67,571){\rule{0.400pt}{3.614pt}}
\multiput(524.17,578.50)(1.000,-7.500){2}{\rule{0.400pt}{1.807pt}}
\put(524.0,586.0){\usebox{\plotpoint}}
\put(526.67,571){\rule{0.400pt}{3.614pt}}
\multiput(526.17,571.00)(1.000,7.500){2}{\rule{0.400pt}{1.807pt}}
\put(526.0,571.0){\usebox{\plotpoint}}
\put(528.67,571){\rule{0.400pt}{3.614pt}}
\multiput(528.17,578.50)(1.000,-7.500){2}{\rule{0.400pt}{1.807pt}}
\put(529.67,571){\rule{0.400pt}{3.614pt}}
\multiput(529.17,571.00)(1.000,7.500){2}{\rule{0.400pt}{1.807pt}}
\put(530.67,571){\rule{0.400pt}{3.614pt}}
\multiput(530.17,578.50)(1.000,-7.500){2}{\rule{0.400pt}{1.807pt}}
\put(528.0,586.0){\usebox{\plotpoint}}
\put(532.67,571){\rule{0.400pt}{5.300pt}}
\multiput(532.17,571.00)(1.000,11.000){2}{\rule{0.400pt}{2.650pt}}
\put(532.0,571.0){\usebox{\plotpoint}}
\put(534.67,571){\rule{0.400pt}{5.300pt}}
\multiput(534.17,582.00)(1.000,-11.000){2}{\rule{0.400pt}{2.650pt}}
\put(534.0,593.0){\usebox{\plotpoint}}
\put(536.67,571){\rule{0.400pt}{10.600pt}}
\multiput(536.17,571.00)(1.000,22.000){2}{\rule{0.400pt}{5.300pt}}
\put(537.67,579){\rule{0.400pt}{8.672pt}}
\multiput(537.17,597.00)(1.000,-18.000){2}{\rule{0.400pt}{4.336pt}}
\put(538.67,579){\rule{0.400pt}{12.286pt}}
\multiput(538.17,579.00)(1.000,25.500){2}{\rule{0.400pt}{6.143pt}}
\put(539.67,586){\rule{0.400pt}{10.600pt}}
\multiput(539.17,608.00)(1.000,-22.000){2}{\rule{0.400pt}{5.300pt}}
\put(541.17,586){\rule{0.400pt}{10.300pt}}
\multiput(540.17,586.00)(2.000,29.622){2}{\rule{0.400pt}{5.150pt}}
\put(542.67,586){\rule{0.400pt}{12.286pt}}
\multiput(542.17,611.50)(1.000,-25.500){2}{\rule{0.400pt}{6.143pt}}
\put(543.67,586){\rule{0.400pt}{12.286pt}}
\multiput(543.17,586.00)(1.000,25.500){2}{\rule{0.400pt}{6.143pt}}
\put(544.67,630){\rule{0.400pt}{1.686pt}}
\multiput(544.17,633.50)(1.000,-3.500){2}{\rule{0.400pt}{0.843pt}}
\put(545.67,586){\rule{0.400pt}{10.600pt}}
\multiput(545.17,608.00)(1.000,-22.000){2}{\rule{0.400pt}{5.300pt}}
\put(546.67,586){\rule{0.400pt}{12.286pt}}
\multiput(546.17,586.00)(1.000,25.500){2}{\rule{0.400pt}{6.143pt}}
\put(547.67,571){\rule{0.400pt}{15.899pt}}
\multiput(547.17,604.00)(1.000,-33.000){2}{\rule{0.400pt}{7.950pt}}
\put(548.67,571){\rule{0.400pt}{12.286pt}}
\multiput(548.17,571.00)(1.000,25.500){2}{\rule{0.400pt}{6.143pt}}
\put(549.67,579){\rule{0.400pt}{10.359pt}}
\multiput(549.17,600.50)(1.000,-21.500){2}{\rule{0.400pt}{5.179pt}}
\put(536.0,571.0){\usebox{\plotpoint}}
\put(551.67,579){\rule{0.400pt}{12.286pt}}
\multiput(551.17,579.00)(1.000,25.500){2}{\rule{0.400pt}{6.143pt}}
\put(552.67,615){\rule{0.400pt}{3.614pt}}
\multiput(552.17,622.50)(1.000,-7.500){2}{\rule{0.400pt}{1.807pt}}
\put(553.67,586){\rule{0.400pt}{6.986pt}}
\multiput(553.17,600.50)(1.000,-14.500){2}{\rule{0.400pt}{3.493pt}}
\put(554.67,571){\rule{0.400pt}{3.614pt}}
\multiput(554.17,578.50)(1.000,-7.500){2}{\rule{0.400pt}{1.807pt}}
\put(555.67,571){\rule{0.400pt}{15.899pt}}
\multiput(555.17,571.00)(1.000,33.000){2}{\rule{0.400pt}{7.950pt}}
\put(556.67,615){\rule{0.400pt}{5.300pt}}
\multiput(556.17,626.00)(1.000,-11.000){2}{\rule{0.400pt}{2.650pt}}
\put(557.67,579){\rule{0.400pt}{8.672pt}}
\multiput(557.17,597.00)(1.000,-18.000){2}{\rule{0.400pt}{4.336pt}}
\put(551.0,579.0){\usebox{\plotpoint}}
\put(559.67,579){\rule{0.400pt}{6.986pt}}
\multiput(559.17,579.00)(1.000,14.500){2}{\rule{0.400pt}{3.493pt}}
\put(560.67,571){\rule{0.400pt}{8.913pt}}
\multiput(560.17,589.50)(1.000,-18.500){2}{\rule{0.400pt}{4.457pt}}
\put(561.67,571){\rule{0.400pt}{8.913pt}}
\multiput(561.17,571.00)(1.000,18.500){2}{\rule{0.400pt}{4.457pt}}
\put(562.67,608){\rule{0.400pt}{1.686pt}}
\multiput(562.17,608.00)(1.000,3.500){2}{\rule{0.400pt}{0.843pt}}
\put(563.67,564){\rule{0.400pt}{12.286pt}}
\multiput(563.17,589.50)(1.000,-25.500){2}{\rule{0.400pt}{6.143pt}}
\put(564.67,564){\rule{0.400pt}{13.972pt}}
\multiput(564.17,564.00)(1.000,29.000){2}{\rule{0.400pt}{6.986pt}}
\put(565.67,557){\rule{0.400pt}{15.658pt}}
\multiput(565.17,589.50)(1.000,-32.500){2}{\rule{0.400pt}{7.829pt}}
\put(567.17,557){\rule{0.400pt}{2.900pt}}
\multiput(566.17,557.00)(2.000,7.981){2}{\rule{0.400pt}{1.450pt}}
\put(568.67,571){\rule{0.400pt}{8.913pt}}
\multiput(568.17,571.00)(1.000,18.500){2}{\rule{0.400pt}{4.457pt}}
\put(569.67,601){\rule{0.400pt}{1.686pt}}
\multiput(569.17,604.50)(1.000,-3.500){2}{\rule{0.400pt}{0.843pt}}
\put(570.67,557){\rule{0.400pt}{10.600pt}}
\multiput(570.17,579.00)(1.000,-22.000){2}{\rule{0.400pt}{5.300pt}}
\put(571.67,557){\rule{0.400pt}{8.672pt}}
\multiput(571.17,557.00)(1.000,18.000){2}{\rule{0.400pt}{4.336pt}}
\put(572.67,571){\rule{0.400pt}{5.300pt}}
\multiput(572.17,582.00)(1.000,-11.000){2}{\rule{0.400pt}{2.650pt}}
\put(573.67,571){\rule{0.400pt}{5.300pt}}
\multiput(573.17,571.00)(1.000,11.000){2}{\rule{0.400pt}{2.650pt}}
\put(574.67,557){\rule{0.400pt}{8.672pt}}
\multiput(574.17,575.00)(1.000,-18.000){2}{\rule{0.400pt}{4.336pt}}
\put(575.67,557){\rule{0.400pt}{10.600pt}}
\multiput(575.17,557.00)(1.000,22.000){2}{\rule{0.400pt}{5.300pt}}
\put(576.67,550){\rule{0.400pt}{12.286pt}}
\multiput(576.17,575.50)(1.000,-25.500){2}{\rule{0.400pt}{6.143pt}}
\put(577.67,550){\rule{0.400pt}{8.672pt}}
\multiput(577.17,550.00)(1.000,18.000){2}{\rule{0.400pt}{4.336pt}}
\put(578.67,564){\rule{0.400pt}{5.300pt}}
\multiput(578.17,575.00)(1.000,-11.000){2}{\rule{0.400pt}{2.650pt}}
\put(579.67,564){\rule{0.400pt}{10.600pt}}
\multiput(579.17,564.00)(1.000,22.000){2}{\rule{0.400pt}{5.300pt}}
\put(580.67,550){\rule{0.400pt}{13.972pt}}
\multiput(580.17,579.00)(1.000,-29.000){2}{\rule{0.400pt}{6.986pt}}
\put(581.67,550){\rule{0.400pt}{8.672pt}}
\multiput(581.17,550.00)(1.000,18.000){2}{\rule{0.400pt}{4.336pt}}
\put(582.67,550){\rule{0.400pt}{8.672pt}}
\multiput(582.17,568.00)(1.000,-18.000){2}{\rule{0.400pt}{4.336pt}}
\put(583.67,550){\rule{0.400pt}{8.672pt}}
\multiput(583.17,550.00)(1.000,18.000){2}{\rule{0.400pt}{4.336pt}}
\put(584.67,557){\rule{0.400pt}{6.986pt}}
\multiput(584.17,571.50)(1.000,-14.500){2}{\rule{0.400pt}{3.493pt}}
\put(585.67,557){\rule{0.400pt}{5.300pt}}
\multiput(585.17,557.00)(1.000,11.000){2}{\rule{0.400pt}{2.650pt}}
\put(586.67,557){\rule{0.400pt}{5.300pt}}
\multiput(586.17,568.00)(1.000,-11.000){2}{\rule{0.400pt}{2.650pt}}
\put(587.67,557){\rule{0.400pt}{5.300pt}}
\multiput(587.17,557.00)(1.000,11.000){2}{\rule{0.400pt}{2.650pt}}
\put(588.67,542){\rule{0.400pt}{8.913pt}}
\multiput(588.17,560.50)(1.000,-18.500){2}{\rule{0.400pt}{4.457pt}}
\put(589.67,542){\rule{0.400pt}{8.913pt}}
\multiput(589.17,542.00)(1.000,18.500){2}{\rule{0.400pt}{4.457pt}}
\put(590.67,550){\rule{0.400pt}{6.986pt}}
\multiput(590.17,564.50)(1.000,-14.500){2}{\rule{0.400pt}{3.493pt}}
\put(591.67,550){\rule{0.400pt}{5.059pt}}
\multiput(591.17,550.00)(1.000,10.500){2}{\rule{0.400pt}{2.529pt}}
\put(593.17,557){\rule{0.400pt}{2.900pt}}
\multiput(592.17,564.98)(2.000,-7.981){2}{\rule{0.400pt}{1.450pt}}
\put(594.67,557){\rule{0.400pt}{3.373pt}}
\multiput(594.17,557.00)(1.000,7.000){2}{\rule{0.400pt}{1.686pt}}
\put(595.67,542){\rule{0.400pt}{6.986pt}}
\multiput(595.17,556.50)(1.000,-14.500){2}{\rule{0.400pt}{3.493pt}}
\put(596.67,542){\rule{0.400pt}{5.300pt}}
\multiput(596.17,542.00)(1.000,11.000){2}{\rule{0.400pt}{2.650pt}}
\put(597.67,542){\rule{0.400pt}{5.300pt}}
\multiput(597.17,553.00)(1.000,-11.000){2}{\rule{0.400pt}{2.650pt}}
\put(598.67,542){\rule{0.400pt}{5.300pt}}
\multiput(598.17,542.00)(1.000,11.000){2}{\rule{0.400pt}{2.650pt}}
\put(599.67,535){\rule{0.400pt}{6.986pt}}
\multiput(599.17,549.50)(1.000,-14.500){2}{\rule{0.400pt}{3.493pt}}
\put(600.67,535){\rule{0.400pt}{3.614pt}}
\multiput(600.17,535.00)(1.000,7.500){2}{\rule{0.400pt}{1.807pt}}
\put(601.67,535){\rule{0.400pt}{3.614pt}}
\multiput(601.17,542.50)(1.000,-7.500){2}{\rule{0.400pt}{1.807pt}}
\put(602.67,535){\rule{0.400pt}{3.614pt}}
\multiput(602.17,535.00)(1.000,7.500){2}{\rule{0.400pt}{1.807pt}}
\put(603.67,528){\rule{0.400pt}{5.300pt}}
\multiput(603.17,539.00)(1.000,-11.000){2}{\rule{0.400pt}{2.650pt}}
\put(604.67,528){\rule{0.400pt}{3.373pt}}
\multiput(604.17,528.00)(1.000,7.000){2}{\rule{0.400pt}{1.686pt}}
\put(605.67,528){\rule{0.400pt}{3.373pt}}
\multiput(605.17,535.00)(1.000,-7.000){2}{\rule{0.400pt}{1.686pt}}
\put(606.67,520){\rule{0.400pt}{1.927pt}}
\multiput(606.17,524.00)(1.000,-4.000){2}{\rule{0.400pt}{0.964pt}}
\put(607.67,520){\rule{0.400pt}{3.614pt}}
\multiput(607.17,520.00)(1.000,7.500){2}{\rule{0.400pt}{1.807pt}}
\put(559.0,579.0){\usebox{\plotpoint}}
\put(609.67,513){\rule{0.400pt}{5.300pt}}
\multiput(609.17,524.00)(1.000,-11.000){2}{\rule{0.400pt}{2.650pt}}
\put(610.67,513){\rule{0.400pt}{3.614pt}}
\multiput(610.17,513.00)(1.000,7.500){2}{\rule{0.400pt}{1.807pt}}
\put(611.67,513){\rule{0.400pt}{3.614pt}}
\multiput(611.17,520.50)(1.000,-7.500){2}{\rule{0.400pt}{1.807pt}}
\put(612.67,513){\rule{0.400pt}{3.614pt}}
\multiput(612.17,513.00)(1.000,7.500){2}{\rule{0.400pt}{1.807pt}}
\put(613.67,499){\rule{0.400pt}{6.986pt}}
\multiput(613.17,513.50)(1.000,-14.500){2}{\rule{0.400pt}{3.493pt}}
\put(614.67,499){\rule{0.400pt}{5.059pt}}
\multiput(614.17,499.00)(1.000,10.500){2}{\rule{0.400pt}{2.529pt}}
\put(615.67,499){\rule{0.400pt}{5.059pt}}
\multiput(615.17,509.50)(1.000,-10.500){2}{\rule{0.400pt}{2.529pt}}
\put(616.67,499){\rule{0.400pt}{5.059pt}}
\multiput(616.17,499.00)(1.000,10.500){2}{\rule{0.400pt}{2.529pt}}
\put(617.67,491){\rule{0.400pt}{6.986pt}}
\multiput(617.17,505.50)(1.000,-14.500){2}{\rule{0.400pt}{3.493pt}}
\put(619.17,491){\rule{0.400pt}{3.100pt}}
\multiput(618.17,491.00)(2.000,8.566){2}{\rule{0.400pt}{1.550pt}}
\put(620.67,491){\rule{0.400pt}{3.614pt}}
\multiput(620.17,498.50)(1.000,-7.500){2}{\rule{0.400pt}{1.807pt}}
\put(621.67,491){\rule{0.400pt}{3.614pt}}
\multiput(621.17,491.00)(1.000,7.500){2}{\rule{0.400pt}{1.807pt}}
\put(622.67,484){\rule{0.400pt}{5.300pt}}
\multiput(622.17,495.00)(1.000,-11.000){2}{\rule{0.400pt}{2.650pt}}
\put(609.0,535.0){\usebox{\plotpoint}}
\put(624.67,484){\rule{0.400pt}{5.300pt}}
\multiput(624.17,484.00)(1.000,11.000){2}{\rule{0.400pt}{2.650pt}}
\put(625.67,484){\rule{0.400pt}{5.300pt}}
\multiput(625.17,495.00)(1.000,-11.000){2}{\rule{0.400pt}{2.650pt}}
\put(626.67,484){\rule{0.400pt}{3.614pt}}
\multiput(626.17,484.00)(1.000,7.500){2}{\rule{0.400pt}{1.807pt}}
\put(627.67,484){\rule{0.400pt}{3.614pt}}
\multiput(627.17,491.50)(1.000,-7.500){2}{\rule{0.400pt}{1.807pt}}
\put(628.67,484){\rule{0.400pt}{3.614pt}}
\multiput(628.17,484.00)(1.000,7.500){2}{\rule{0.400pt}{1.807pt}}
\put(624.0,484.0){\usebox{\plotpoint}}
\put(630.67,477){\rule{0.400pt}{5.300pt}}
\multiput(630.17,488.00)(1.000,-11.000){2}{\rule{0.400pt}{2.650pt}}
\put(631.67,477){\rule{0.400pt}{3.373pt}}
\multiput(631.17,477.00)(1.000,7.000){2}{\rule{0.400pt}{1.686pt}}
\put(632.67,477){\rule{0.400pt}{3.373pt}}
\multiput(632.17,484.00)(1.000,-7.000){2}{\rule{0.400pt}{1.686pt}}
\put(633.67,477){\rule{0.400pt}{3.373pt}}
\multiput(633.17,477.00)(1.000,7.000){2}{\rule{0.400pt}{1.686pt}}
\put(634.67,477){\rule{0.400pt}{3.373pt}}
\multiput(634.17,484.00)(1.000,-7.000){2}{\rule{0.400pt}{1.686pt}}
\put(630.0,499.0){\usebox{\plotpoint}}
\put(636.67,477){\rule{0.400pt}{3.373pt}}
\multiput(636.17,477.00)(1.000,7.000){2}{\rule{0.400pt}{1.686pt}}
\put(636.0,477.0){\usebox{\plotpoint}}
\put(638.67,470){\rule{0.400pt}{5.059pt}}
\multiput(638.17,480.50)(1.000,-10.500){2}{\rule{0.400pt}{2.529pt}}
\put(638.0,491.0){\usebox{\plotpoint}}
\put(640.67,470){\rule{0.400pt}{5.059pt}}
\multiput(640.17,470.00)(1.000,10.500){2}{\rule{0.400pt}{2.529pt}}
\put(641.67,484){\rule{0.400pt}{1.686pt}}
\multiput(641.17,487.50)(1.000,-3.500){2}{\rule{0.400pt}{0.843pt}}
\put(642.67,470){\rule{0.400pt}{3.373pt}}
\multiput(642.17,477.00)(1.000,-7.000){2}{\rule{0.400pt}{1.686pt}}
\put(640.0,470.0){\usebox{\plotpoint}}
\put(645.17,470){\rule{0.400pt}{4.300pt}}
\multiput(644.17,470.00)(2.000,12.075){2}{\rule{0.400pt}{2.150pt}}
\put(646.67,491){\rule{0.400pt}{1.927pt}}
\multiput(646.17,491.00)(1.000,4.000){2}{\rule{0.400pt}{0.964pt}}
\put(647.67,440){\rule{0.400pt}{14.213pt}}
\multiput(647.17,469.50)(1.000,-29.500){2}{\rule{0.400pt}{7.107pt}}
\put(648.67,440){\rule{0.400pt}{10.600pt}}
\multiput(648.17,440.00)(1.000,22.000){2}{\rule{0.400pt}{5.300pt}}
\put(649.67,455){\rule{0.400pt}{6.986pt}}
\multiput(649.17,469.50)(1.000,-14.500){2}{\rule{0.400pt}{3.493pt}}
\put(650.67,455){\rule{0.400pt}{6.986pt}}
\multiput(650.17,455.00)(1.000,14.500){2}{\rule{0.400pt}{3.493pt}}
\put(651.67,455){\rule{0.400pt}{6.986pt}}
\multiput(651.17,469.50)(1.000,-14.500){2}{\rule{0.400pt}{3.493pt}}
\put(652.67,455){\rule{0.400pt}{5.300pt}}
\multiput(652.17,455.00)(1.000,11.000){2}{\rule{0.400pt}{2.650pt}}
\put(653.67,448){\rule{0.400pt}{6.986pt}}
\multiput(653.17,462.50)(1.000,-14.500){2}{\rule{0.400pt}{3.493pt}}
\put(654.67,448){\rule{0.400pt}{6.986pt}}
\multiput(654.17,448.00)(1.000,14.500){2}{\rule{0.400pt}{3.493pt}}
\put(655.67,455){\rule{0.400pt}{5.300pt}}
\multiput(655.17,466.00)(1.000,-11.000){2}{\rule{0.400pt}{2.650pt}}
\put(656.67,455){\rule{0.400pt}{5.300pt}}
\multiput(656.17,455.00)(1.000,11.000){2}{\rule{0.400pt}{2.650pt}}
\put(657.67,455){\rule{0.400pt}{5.300pt}}
\multiput(657.17,466.00)(1.000,-11.000){2}{\rule{0.400pt}{2.650pt}}
\put(658.67,455){\rule{0.400pt}{5.300pt}}
\multiput(658.17,455.00)(1.000,11.000){2}{\rule{0.400pt}{2.650pt}}
\put(659.67,448){\rule{0.400pt}{6.986pt}}
\multiput(659.17,462.50)(1.000,-14.500){2}{\rule{0.400pt}{3.493pt}}
\put(644.0,470.0){\usebox{\plotpoint}}
\put(661.67,448){\rule{0.400pt}{6.986pt}}
\multiput(661.17,448.00)(1.000,14.500){2}{\rule{0.400pt}{3.493pt}}
\put(662.67,448){\rule{0.400pt}{6.986pt}}
\multiput(662.17,462.50)(1.000,-14.500){2}{\rule{0.400pt}{3.493pt}}
\put(663.67,448){\rule{0.400pt}{5.300pt}}
\multiput(663.17,448.00)(1.000,11.000){2}{\rule{0.400pt}{2.650pt}}
\put(664.67,462){\rule{0.400pt}{1.927pt}}
\multiput(664.17,466.00)(1.000,-4.000){2}{\rule{0.400pt}{0.964pt}}
\put(665.67,448){\rule{0.400pt}{3.373pt}}
\multiput(665.17,455.00)(1.000,-7.000){2}{\rule{0.400pt}{1.686pt}}
\put(666.67,448){\rule{0.400pt}{3.373pt}}
\multiput(666.17,448.00)(1.000,7.000){2}{\rule{0.400pt}{1.686pt}}
\put(667.67,440){\rule{0.400pt}{5.300pt}}
\multiput(667.17,451.00)(1.000,-11.000){2}{\rule{0.400pt}{2.650pt}}
\put(668.67,440){\rule{0.400pt}{5.300pt}}
\multiput(668.17,440.00)(1.000,11.000){2}{\rule{0.400pt}{2.650pt}}
\put(669.67,440){\rule{0.400pt}{5.300pt}}
\multiput(669.17,451.00)(1.000,-11.000){2}{\rule{0.400pt}{2.650pt}}
\put(671.17,440){\rule{0.400pt}{4.500pt}}
\multiput(670.17,440.00)(2.000,12.660){2}{\rule{0.400pt}{2.250pt}}
\put(672.67,433){\rule{0.400pt}{6.986pt}}
\multiput(672.17,447.50)(1.000,-14.500){2}{\rule{0.400pt}{3.493pt}}
\put(661.0,448.0){\usebox{\plotpoint}}
\put(674.67,433){\rule{0.400pt}{5.300pt}}
\multiput(674.17,433.00)(1.000,11.000){2}{\rule{0.400pt}{2.650pt}}
\put(674.0,433.0){\usebox{\plotpoint}}
\put(676.67,426){\rule{0.400pt}{6.986pt}}
\multiput(676.17,440.50)(1.000,-14.500){2}{\rule{0.400pt}{3.493pt}}
\put(676.0,455.0){\usebox{\plotpoint}}
\put(678.67,426){\rule{0.400pt}{6.986pt}}
\multiput(678.17,426.00)(1.000,14.500){2}{\rule{0.400pt}{3.493pt}}
\put(679.67,419){\rule{0.400pt}{8.672pt}}
\multiput(679.17,437.00)(1.000,-18.000){2}{\rule{0.400pt}{4.336pt}}
\put(680.67,419){\rule{0.400pt}{8.672pt}}
\multiput(680.17,419.00)(1.000,18.000){2}{\rule{0.400pt}{4.336pt}}
\put(681.67,426){\rule{0.400pt}{6.986pt}}
\multiput(681.17,440.50)(1.000,-14.500){2}{\rule{0.400pt}{3.493pt}}
\put(682.67,426){\rule{0.400pt}{3.373pt}}
\multiput(682.17,426.00)(1.000,7.000){2}{\rule{0.400pt}{1.686pt}}
\put(683.67,440){\rule{0.400pt}{1.927pt}}
\multiput(683.17,440.00)(1.000,4.000){2}{\rule{0.400pt}{0.964pt}}
\put(684.67,411){\rule{0.400pt}{8.913pt}}
\multiput(684.17,429.50)(1.000,-18.500){2}{\rule{0.400pt}{4.457pt}}
\put(678.0,426.0){\usebox{\plotpoint}}
\put(686.67,411){\rule{0.400pt}{6.986pt}}
\multiput(686.17,411.00)(1.000,14.500){2}{\rule{0.400pt}{3.493pt}}
\put(687.67,433){\rule{0.400pt}{1.686pt}}
\multiput(687.17,436.50)(1.000,-3.500){2}{\rule{0.400pt}{0.843pt}}
\put(688.67,411){\rule{0.400pt}{5.300pt}}
\multiput(688.17,422.00)(1.000,-11.000){2}{\rule{0.400pt}{2.650pt}}
\put(689.67,411){\rule{0.400pt}{6.986pt}}
\multiput(689.17,411.00)(1.000,14.500){2}{\rule{0.400pt}{3.493pt}}
\put(690.67,404){\rule{0.400pt}{8.672pt}}
\multiput(690.17,422.00)(1.000,-18.000){2}{\rule{0.400pt}{4.336pt}}
\put(691.67,404){\rule{0.400pt}{6.986pt}}
\multiput(691.17,404.00)(1.000,14.500){2}{\rule{0.400pt}{3.493pt}}
\put(692.67,404){\rule{0.400pt}{6.986pt}}
\multiput(692.17,418.50)(1.000,-14.500){2}{\rule{0.400pt}{3.493pt}}
\put(693.67,404){\rule{0.400pt}{3.614pt}}
\multiput(693.17,404.00)(1.000,7.500){2}{\rule{0.400pt}{1.807pt}}
\put(694.67,404){\rule{0.400pt}{3.614pt}}
\multiput(694.17,411.50)(1.000,-7.500){2}{\rule{0.400pt}{1.807pt}}
\put(695.67,404){\rule{0.400pt}{3.614pt}}
\multiput(695.17,404.00)(1.000,7.500){2}{\rule{0.400pt}{1.807pt}}
\put(697.17,397){\rule{0.400pt}{4.500pt}}
\multiput(696.17,409.66)(2.000,-12.660){2}{\rule{0.400pt}{2.250pt}}
\put(686.0,411.0){\usebox{\plotpoint}}
\put(699.67,397){\rule{0.400pt}{6.986pt}}
\multiput(699.17,397.00)(1.000,14.500){2}{\rule{0.400pt}{3.493pt}}
\put(700.67,397){\rule{0.400pt}{6.986pt}}
\multiput(700.17,411.50)(1.000,-14.500){2}{\rule{0.400pt}{3.493pt}}
\put(701.67,397){\rule{0.400pt}{5.300pt}}
\multiput(701.17,397.00)(1.000,11.000){2}{\rule{0.400pt}{2.650pt}}
\put(702.67,411){\rule{0.400pt}{1.927pt}}
\multiput(702.17,415.00)(1.000,-4.000){2}{\rule{0.400pt}{0.964pt}}
\put(703.67,397){\rule{0.400pt}{3.373pt}}
\multiput(703.17,404.00)(1.000,-7.000){2}{\rule{0.400pt}{1.686pt}}
\put(704.67,397){\rule{0.400pt}{3.373pt}}
\multiput(704.17,397.00)(1.000,7.000){2}{\rule{0.400pt}{1.686pt}}
\put(705.67,389){\rule{0.400pt}{5.300pt}}
\multiput(705.17,400.00)(1.000,-11.000){2}{\rule{0.400pt}{2.650pt}}
\put(706.67,389){\rule{0.400pt}{5.300pt}}
\multiput(706.17,389.00)(1.000,11.000){2}{\rule{0.400pt}{2.650pt}}
\put(707.67,389){\rule{0.400pt}{5.300pt}}
\multiput(707.17,400.00)(1.000,-11.000){2}{\rule{0.400pt}{2.650pt}}
\put(699.0,397.0){\usebox{\plotpoint}}
\put(709.67,389){\rule{0.400pt}{3.614pt}}
\multiput(709.17,389.00)(1.000,7.500){2}{\rule{0.400pt}{1.807pt}}
\put(709.0,389.0){\usebox{\plotpoint}}
\put(711.67,382){\rule{0.400pt}{5.300pt}}
\multiput(711.17,393.00)(1.000,-11.000){2}{\rule{0.400pt}{2.650pt}}
\put(712.67,382){\rule{0.400pt}{3.614pt}}
\multiput(712.17,382.00)(1.000,7.500){2}{\rule{0.400pt}{1.807pt}}
\put(713.67,382){\rule{0.400pt}{3.614pt}}
\multiput(713.17,389.50)(1.000,-7.500){2}{\rule{0.400pt}{1.807pt}}
\put(714.67,382){\rule{0.400pt}{3.614pt}}
\multiput(714.17,382.00)(1.000,7.500){2}{\rule{0.400pt}{1.807pt}}
\put(715.67,382){\rule{0.400pt}{3.614pt}}
\multiput(715.17,389.50)(1.000,-7.500){2}{\rule{0.400pt}{1.807pt}}
\put(716.67,382){\rule{0.400pt}{3.614pt}}
\multiput(716.17,382.00)(1.000,7.500){2}{\rule{0.400pt}{1.807pt}}
\put(717.67,375){\rule{0.400pt}{5.300pt}}
\multiput(717.17,386.00)(1.000,-11.000){2}{\rule{0.400pt}{2.650pt}}
\put(718.67,375){\rule{0.400pt}{3.373pt}}
\multiput(718.17,375.00)(1.000,7.000){2}{\rule{0.400pt}{1.686pt}}
\put(719.67,368){\rule{0.400pt}{5.059pt}}
\multiput(719.17,378.50)(1.000,-10.500){2}{\rule{0.400pt}{2.529pt}}
\put(711.0,404.0){\usebox{\plotpoint}}
\put(721.67,368){\rule{0.400pt}{5.059pt}}
\multiput(721.17,368.00)(1.000,10.500){2}{\rule{0.400pt}{2.529pt}}
\put(723.17,375){\rule{0.400pt}{2.900pt}}
\multiput(722.17,382.98)(2.000,-7.981){2}{\rule{0.400pt}{1.450pt}}
\put(724.67,360){\rule{0.400pt}{3.614pt}}
\multiput(724.17,367.50)(1.000,-7.500){2}{\rule{0.400pt}{1.807pt}}
\put(725.67,360){\rule{0.400pt}{3.614pt}}
\multiput(725.17,360.00)(1.000,7.500){2}{\rule{0.400pt}{1.807pt}}
\put(726.67,360){\rule{0.400pt}{3.614pt}}
\multiput(726.17,367.50)(1.000,-7.500){2}{\rule{0.400pt}{1.807pt}}
\put(727.67,360){\rule{0.400pt}{3.614pt}}
\multiput(727.17,360.00)(1.000,7.500){2}{\rule{0.400pt}{1.807pt}}
\put(728.67,360){\rule{0.400pt}{3.614pt}}
\multiput(728.17,367.50)(1.000,-7.500){2}{\rule{0.400pt}{1.807pt}}
\put(729.67,360){\rule{0.400pt}{3.614pt}}
\multiput(729.17,360.00)(1.000,7.500){2}{\rule{0.400pt}{1.807pt}}
\put(730.67,353){\rule{0.400pt}{5.300pt}}
\multiput(730.17,364.00)(1.000,-11.000){2}{\rule{0.400pt}{2.650pt}}
\put(731.67,353){\rule{0.400pt}{3.614pt}}
\multiput(731.17,353.00)(1.000,7.500){2}{\rule{0.400pt}{1.807pt}}
\put(732.67,353){\rule{0.400pt}{3.614pt}}
\multiput(732.17,360.50)(1.000,-7.500){2}{\rule{0.400pt}{1.807pt}}
\put(721.0,368.0){\usebox{\plotpoint}}
\put(734.67,353){\rule{0.400pt}{3.614pt}}
\multiput(734.17,353.00)(1.000,7.500){2}{\rule{0.400pt}{1.807pt}}
\put(735.67,360){\rule{0.400pt}{1.927pt}}
\multiput(735.17,364.00)(1.000,-4.000){2}{\rule{0.400pt}{0.964pt}}
\put(736.67,346){\rule{0.400pt}{3.373pt}}
\multiput(736.17,353.00)(1.000,-7.000){2}{\rule{0.400pt}{1.686pt}}
\put(737.67,346){\rule{0.400pt}{3.373pt}}
\multiput(737.17,346.00)(1.000,7.000){2}{\rule{0.400pt}{1.686pt}}
\put(738.67,346){\rule{0.400pt}{3.373pt}}
\multiput(738.17,353.00)(1.000,-7.000){2}{\rule{0.400pt}{1.686pt}}
\put(739.67,346){\rule{0.400pt}{3.373pt}}
\multiput(739.17,346.00)(1.000,7.000){2}{\rule{0.400pt}{1.686pt}}
\put(740.67,338){\rule{0.400pt}{5.300pt}}
\multiput(740.17,349.00)(1.000,-11.000){2}{\rule{0.400pt}{2.650pt}}
\put(734.0,353.0){\usebox{\plotpoint}}
\put(742.67,338){\rule{0.400pt}{3.614pt}}
\multiput(742.17,338.00)(1.000,7.500){2}{\rule{0.400pt}{1.807pt}}
\put(743.67,338){\rule{0.400pt}{3.614pt}}
\multiput(743.17,345.50)(1.000,-7.500){2}{\rule{0.400pt}{1.807pt}}
\put(744.67,338){\rule{0.400pt}{3.614pt}}
\multiput(744.17,338.00)(1.000,7.500){2}{\rule{0.400pt}{1.807pt}}
\put(742.0,338.0){\usebox{\plotpoint}}
\put(746.67,324){\rule{0.400pt}{6.986pt}}
\multiput(746.17,338.50)(1.000,-14.500){2}{\rule{0.400pt}{3.493pt}}
\put(747.67,324){\rule{0.400pt}{5.300pt}}
\multiput(747.17,324.00)(1.000,11.000){2}{\rule{0.400pt}{2.650pt}}
\put(749.17,324){\rule{0.400pt}{4.500pt}}
\multiput(748.17,336.66)(2.000,-12.660){2}{\rule{0.400pt}{2.250pt}}
\put(750.67,324){\rule{0.400pt}{5.300pt}}
\multiput(750.17,324.00)(1.000,11.000){2}{\rule{0.400pt}{2.650pt}}
\put(751.67,317){\rule{0.400pt}{6.986pt}}
\multiput(751.17,331.50)(1.000,-14.500){2}{\rule{0.400pt}{3.493pt}}
\put(752.67,317){\rule{0.400pt}{6.986pt}}
\multiput(752.17,317.00)(1.000,14.500){2}{\rule{0.400pt}{3.493pt}}
\put(753.67,317){\rule{0.400pt}{6.986pt}}
\multiput(753.17,331.50)(1.000,-14.500){2}{\rule{0.400pt}{3.493pt}}
\put(746.0,353.0){\usebox{\plotpoint}}
\put(755.67,317){\rule{0.400pt}{3.373pt}}
\multiput(755.17,317.00)(1.000,7.000){2}{\rule{0.400pt}{1.686pt}}
\put(755.0,317.0){\usebox{\plotpoint}}
\put(757.67,317){\rule{0.400pt}{3.373pt}}
\multiput(757.17,324.00)(1.000,-7.000){2}{\rule{0.400pt}{1.686pt}}
\put(758.67,317){\rule{0.400pt}{3.373pt}}
\multiput(758.17,317.00)(1.000,7.000){2}{\rule{0.400pt}{1.686pt}}
\put(759.67,309){\rule{0.400pt}{5.300pt}}
\multiput(759.17,320.00)(1.000,-11.000){2}{\rule{0.400pt}{2.650pt}}
\put(757.0,331.0){\usebox{\plotpoint}}
\put(761.67,309){\rule{0.400pt}{3.614pt}}
\multiput(761.17,309.00)(1.000,7.500){2}{\rule{0.400pt}{1.807pt}}
\put(761.0,309.0){\usebox{\plotpoint}}
\put(763.67,302){\rule{0.400pt}{5.300pt}}
\multiput(763.17,313.00)(1.000,-11.000){2}{\rule{0.400pt}{2.650pt}}
\put(764.67,302){\rule{0.400pt}{3.614pt}}
\multiput(764.17,302.00)(1.000,7.500){2}{\rule{0.400pt}{1.807pt}}
\put(765.67,302){\rule{0.400pt}{3.614pt}}
\multiput(765.17,309.50)(1.000,-7.500){2}{\rule{0.400pt}{1.807pt}}
\put(763.0,324.0){\usebox{\plotpoint}}
\put(767.67,302){\rule{0.400pt}{3.614pt}}
\multiput(767.17,302.00)(1.000,7.500){2}{\rule{0.400pt}{1.807pt}}
\put(768.67,302){\rule{0.400pt}{3.614pt}}
\multiput(768.17,309.50)(1.000,-7.500){2}{\rule{0.400pt}{1.807pt}}
\put(769.67,302){\rule{0.400pt}{3.614pt}}
\multiput(769.17,302.00)(1.000,7.500){2}{\rule{0.400pt}{1.807pt}}
\put(767.0,302.0){\usebox{\plotpoint}}
\put(771.67,295){\rule{0.400pt}{5.300pt}}
\multiput(771.17,306.00)(1.000,-11.000){2}{\rule{0.400pt}{2.650pt}}
\put(772.67,295){\rule{0.400pt}{3.373pt}}
\multiput(772.17,295.00)(1.000,7.000){2}{\rule{0.400pt}{1.686pt}}
\put(773.67,295){\rule{0.400pt}{3.373pt}}
\multiput(773.17,302.00)(1.000,-7.000){2}{\rule{0.400pt}{1.686pt}}
\put(775.17,295){\rule{0.400pt}{2.900pt}}
\multiput(774.17,295.00)(2.000,7.981){2}{\rule{0.400pt}{1.450pt}}
\put(776.67,288){\rule{0.400pt}{5.059pt}}
\multiput(776.17,298.50)(1.000,-10.500){2}{\rule{0.400pt}{2.529pt}}
\put(777.67,288){\rule{0.400pt}{5.059pt}}
\multiput(777.17,288.00)(1.000,10.500){2}{\rule{0.400pt}{2.529pt}}
\put(778.67,280){\rule{0.400pt}{6.986pt}}
\multiput(778.17,294.50)(1.000,-14.500){2}{\rule{0.400pt}{3.493pt}}
\put(771.0,317.0){\usebox{\plotpoint}}
\put(780.67,280){\rule{0.400pt}{5.300pt}}
\multiput(780.17,280.00)(1.000,11.000){2}{\rule{0.400pt}{2.650pt}}
\put(780.0,280.0){\usebox{\plotpoint}}
\put(782.67,273){\rule{0.400pt}{6.986pt}}
\multiput(782.17,287.50)(1.000,-14.500){2}{\rule{0.400pt}{3.493pt}}
\put(783.67,273){\rule{0.400pt}{5.300pt}}
\multiput(783.17,273.00)(1.000,11.000){2}{\rule{0.400pt}{2.650pt}}
\put(784.67,273){\rule{0.400pt}{5.300pt}}
\multiput(784.17,284.00)(1.000,-11.000){2}{\rule{0.400pt}{2.650pt}}
\put(785.67,273){\rule{0.400pt}{3.614pt}}
\multiput(785.17,273.00)(1.000,7.500){2}{\rule{0.400pt}{1.807pt}}
\put(786.67,273){\rule{0.400pt}{3.614pt}}
\multiput(786.17,280.50)(1.000,-7.500){2}{\rule{0.400pt}{1.807pt}}
\put(787.67,273){\rule{0.400pt}{3.614pt}}
\multiput(787.17,273.00)(1.000,7.500){2}{\rule{0.400pt}{1.807pt}}
\put(788.67,273){\rule{0.400pt}{3.614pt}}
\multiput(788.17,280.50)(1.000,-7.500){2}{\rule{0.400pt}{1.807pt}}
\put(789.67,273){\rule{0.400pt}{3.614pt}}
\multiput(789.17,273.00)(1.000,7.500){2}{\rule{0.400pt}{1.807pt}}
\put(790.67,273){\rule{0.400pt}{3.614pt}}
\multiput(790.17,280.50)(1.000,-7.500){2}{\rule{0.400pt}{1.807pt}}
\put(791.67,273){\rule{0.400pt}{1.686pt}}
\multiput(791.17,273.00)(1.000,3.500){2}{\rule{0.400pt}{0.843pt}}
\put(792.67,266){\rule{0.400pt}{3.373pt}}
\multiput(792.17,273.00)(1.000,-7.000){2}{\rule{0.400pt}{1.686pt}}
\put(793.67,266){\rule{0.400pt}{3.373pt}}
\multiput(793.17,266.00)(1.000,7.000){2}{\rule{0.400pt}{1.686pt}}
\put(794.67,266){\rule{0.400pt}{3.373pt}}
\multiput(794.17,273.00)(1.000,-7.000){2}{\rule{0.400pt}{1.686pt}}
\put(795.67,266){\rule{0.400pt}{3.373pt}}
\multiput(795.17,266.00)(1.000,7.000){2}{\rule{0.400pt}{1.686pt}}
\put(796.67,266){\rule{0.400pt}{3.373pt}}
\multiput(796.17,273.00)(1.000,-7.000){2}{\rule{0.400pt}{1.686pt}}
\put(782.0,302.0){\usebox{\plotpoint}}
\put(798.67,266){\rule{0.400pt}{3.373pt}}
\multiput(798.17,266.00)(1.000,7.000){2}{\rule{0.400pt}{1.686pt}}
\put(799.67,258){\rule{0.400pt}{5.300pt}}
\multiput(799.17,269.00)(1.000,-11.000){2}{\rule{0.400pt}{2.650pt}}
\put(801.17,258){\rule{0.400pt}{4.500pt}}
\multiput(800.17,258.00)(2.000,12.660){2}{\rule{0.400pt}{2.250pt}}
\put(802.67,258){\rule{0.400pt}{5.300pt}}
\multiput(802.17,269.00)(1.000,-11.000){2}{\rule{0.400pt}{2.650pt}}
\put(803.67,258){\rule{0.400pt}{3.614pt}}
\multiput(803.17,258.00)(1.000,7.500){2}{\rule{0.400pt}{1.807pt}}
\put(804.67,258){\rule{0.400pt}{3.614pt}}
\multiput(804.17,265.50)(1.000,-7.500){2}{\rule{0.400pt}{1.807pt}}
\put(805.67,258){\rule{0.400pt}{3.614pt}}
\multiput(805.17,258.00)(1.000,7.500){2}{\rule{0.400pt}{1.807pt}}
\put(806.67,258){\rule{0.400pt}{3.614pt}}
\multiput(806.17,265.50)(1.000,-7.500){2}{\rule{0.400pt}{1.807pt}}
\put(807.67,258){\rule{0.400pt}{3.614pt}}
\multiput(807.17,258.00)(1.000,7.500){2}{\rule{0.400pt}{1.807pt}}
\put(808.67,258){\rule{0.400pt}{3.614pt}}
\multiput(808.17,265.50)(1.000,-7.500){2}{\rule{0.400pt}{1.807pt}}
\put(809.67,258){\rule{0.400pt}{3.614pt}}
\multiput(809.17,258.00)(1.000,7.500){2}{\rule{0.400pt}{1.807pt}}
\put(810.67,258){\rule{0.400pt}{3.614pt}}
\multiput(810.17,265.50)(1.000,-7.500){2}{\rule{0.400pt}{1.807pt}}
\put(811.67,258){\rule{0.400pt}{3.614pt}}
\multiput(811.17,258.00)(1.000,7.500){2}{\rule{0.400pt}{1.807pt}}
\put(812.67,266){\rule{0.400pt}{1.686pt}}
\multiput(812.17,269.50)(1.000,-3.500){2}{\rule{0.400pt}{0.843pt}}
\put(813.67,251){\rule{0.400pt}{3.614pt}}
\multiput(813.17,258.50)(1.000,-7.500){2}{\rule{0.400pt}{1.807pt}}
\put(814.67,251){\rule{0.400pt}{3.614pt}}
\multiput(814.17,251.00)(1.000,7.500){2}{\rule{0.400pt}{1.807pt}}
\put(815.67,251){\rule{0.400pt}{3.614pt}}
\multiput(815.17,258.50)(1.000,-7.500){2}{\rule{0.400pt}{1.807pt}}
\put(816.67,251){\rule{0.400pt}{5.300pt}}
\multiput(816.17,251.00)(1.000,11.000){2}{\rule{0.400pt}{2.650pt}}
\put(817.67,251){\rule{0.400pt}{5.300pt}}
\multiput(817.17,262.00)(1.000,-11.000){2}{\rule{0.400pt}{2.650pt}}
\put(818.67,251){\rule{0.400pt}{3.614pt}}
\multiput(818.17,251.00)(1.000,7.500){2}{\rule{0.400pt}{1.807pt}}
\put(819.67,251){\rule{0.400pt}{3.614pt}}
\multiput(819.17,258.50)(1.000,-7.500){2}{\rule{0.400pt}{1.807pt}}
\put(820.67,251){\rule{0.400pt}{3.614pt}}
\multiput(820.17,251.00)(1.000,7.500){2}{\rule{0.400pt}{1.807pt}}
\put(821.67,251){\rule{0.400pt}{3.614pt}}
\multiput(821.17,258.50)(1.000,-7.500){2}{\rule{0.400pt}{1.807pt}}
\put(822.67,251){\rule{0.400pt}{3.614pt}}
\multiput(822.17,251.00)(1.000,7.500){2}{\rule{0.400pt}{1.807pt}}
\put(823.67,251){\rule{0.400pt}{3.614pt}}
\multiput(823.17,258.50)(1.000,-7.500){2}{\rule{0.400pt}{1.807pt}}
\put(824.67,251){\rule{0.400pt}{3.614pt}}
\multiput(824.17,251.00)(1.000,7.500){2}{\rule{0.400pt}{1.807pt}}
\put(825.67,251){\rule{0.400pt}{3.614pt}}
\multiput(825.17,258.50)(1.000,-7.500){2}{\rule{0.400pt}{1.807pt}}
\put(798.0,266.0){\usebox{\plotpoint}}
\put(828.67,251){\rule{0.400pt}{3.614pt}}
\multiput(828.17,251.00)(1.000,7.500){2}{\rule{0.400pt}{1.807pt}}
\put(829.67,251){\rule{0.400pt}{3.614pt}}
\multiput(829.17,258.50)(1.000,-7.500){2}{\rule{0.400pt}{1.807pt}}
\put(830.67,251){\rule{0.400pt}{3.614pt}}
\multiput(830.17,251.00)(1.000,7.500){2}{\rule{0.400pt}{1.807pt}}
\put(831.67,251){\rule{0.400pt}{3.614pt}}
\multiput(831.17,258.50)(1.000,-7.500){2}{\rule{0.400pt}{1.807pt}}
\put(832.67,251){\rule{0.400pt}{3.614pt}}
\multiput(832.17,251.00)(1.000,7.500){2}{\rule{0.400pt}{1.807pt}}
\put(833.67,251){\rule{0.400pt}{3.614pt}}
\multiput(833.17,258.50)(1.000,-7.500){2}{\rule{0.400pt}{1.807pt}}
\put(834.67,251){\rule{0.400pt}{3.614pt}}
\multiput(834.17,251.00)(1.000,7.500){2}{\rule{0.400pt}{1.807pt}}
\put(835.67,251){\rule{0.400pt}{3.614pt}}
\multiput(835.17,258.50)(1.000,-7.500){2}{\rule{0.400pt}{1.807pt}}
\put(836.67,251){\rule{0.400pt}{3.614pt}}
\multiput(836.17,251.00)(1.000,7.500){2}{\rule{0.400pt}{1.807pt}}
\put(827.0,251.0){\rule[-0.200pt]{0.482pt}{0.400pt}}
\put(838.67,251){\rule{0.400pt}{3.614pt}}
\multiput(838.17,258.50)(1.000,-7.500){2}{\rule{0.400pt}{1.807pt}}
\put(838.0,266.0){\usebox{\plotpoint}}
\put(840.67,251){\rule{0.400pt}{3.614pt}}
\multiput(840.17,251.00)(1.000,7.500){2}{\rule{0.400pt}{1.807pt}}
\put(841.67,251){\rule{0.400pt}{3.614pt}}
\multiput(841.17,258.50)(1.000,-7.500){2}{\rule{0.400pt}{1.807pt}}
\put(842.67,251){\rule{0.400pt}{3.614pt}}
\multiput(842.17,251.00)(1.000,7.500){2}{\rule{0.400pt}{1.807pt}}
\put(843.67,251){\rule{0.400pt}{3.614pt}}
\multiput(843.17,258.50)(1.000,-7.500){2}{\rule{0.400pt}{1.807pt}}
\put(844.67,251){\rule{0.400pt}{3.614pt}}
\multiput(844.17,251.00)(1.000,7.500){2}{\rule{0.400pt}{1.807pt}}
\put(840.0,251.0){\usebox{\plotpoint}}
\put(846.67,251){\rule{0.400pt}{3.614pt}}
\multiput(846.17,258.50)(1.000,-7.500){2}{\rule{0.400pt}{1.807pt}}
\put(846.0,266.0){\usebox{\plotpoint}}
\put(848.67,251){\rule{0.400pt}{3.614pt}}
\multiput(848.17,251.00)(1.000,7.500){2}{\rule{0.400pt}{1.807pt}}
\put(848.0,251.0){\usebox{\plotpoint}}
\put(850.67,251){\rule{0.400pt}{3.614pt}}
\multiput(850.17,258.50)(1.000,-7.500){2}{\rule{0.400pt}{1.807pt}}
\put(850.0,266.0){\usebox{\plotpoint}}
\put(853.17,251){\rule{0.400pt}{3.100pt}}
\multiput(852.17,251.00)(2.000,8.566){2}{\rule{0.400pt}{1.550pt}}
\put(854.67,251){\rule{0.400pt}{3.614pt}}
\multiput(854.17,258.50)(1.000,-7.500){2}{\rule{0.400pt}{1.807pt}}
\put(855.67,251){\rule{0.400pt}{3.614pt}}
\multiput(855.17,251.00)(1.000,7.500){2}{\rule{0.400pt}{1.807pt}}
\put(856.67,251){\rule{0.400pt}{3.614pt}}
\multiput(856.17,258.50)(1.000,-7.500){2}{\rule{0.400pt}{1.807pt}}
\put(857.67,251){\rule{0.400pt}{3.614pt}}
\multiput(857.17,251.00)(1.000,7.500){2}{\rule{0.400pt}{1.807pt}}
\put(852.0,251.0){\usebox{\plotpoint}}
\put(859.67,251){\rule{0.400pt}{3.614pt}}
\multiput(859.17,258.50)(1.000,-7.500){2}{\rule{0.400pt}{1.807pt}}
\put(859.0,266.0){\usebox{\plotpoint}}
\put(861.67,251){\rule{0.400pt}{3.614pt}}
\multiput(861.17,251.00)(1.000,7.500){2}{\rule{0.400pt}{1.807pt}}
\put(862.67,251){\rule{0.400pt}{3.614pt}}
\multiput(862.17,258.50)(1.000,-7.500){2}{\rule{0.400pt}{1.807pt}}
\put(863.67,251){\rule{0.400pt}{3.614pt}}
\multiput(863.17,251.00)(1.000,7.500){2}{\rule{0.400pt}{1.807pt}}
\put(864.67,251){\rule{0.400pt}{3.614pt}}
\multiput(864.17,258.50)(1.000,-7.500){2}{\rule{0.400pt}{1.807pt}}
\put(865.67,251){\rule{0.400pt}{3.614pt}}
\multiput(865.17,251.00)(1.000,7.500){2}{\rule{0.400pt}{1.807pt}}
\put(866.67,251){\rule{0.400pt}{3.614pt}}
\multiput(866.17,258.50)(1.000,-7.500){2}{\rule{0.400pt}{1.807pt}}
\put(867.67,251){\rule{0.400pt}{3.614pt}}
\multiput(867.17,251.00)(1.000,7.500){2}{\rule{0.400pt}{1.807pt}}
\put(868.67,251){\rule{0.400pt}{3.614pt}}
\multiput(868.17,258.50)(1.000,-7.500){2}{\rule{0.400pt}{1.807pt}}
\put(869.67,251){\rule{0.400pt}{1.686pt}}
\multiput(869.17,251.00)(1.000,3.500){2}{\rule{0.400pt}{0.843pt}}
\put(870.67,244){\rule{0.400pt}{3.373pt}}
\multiput(870.17,251.00)(1.000,-7.000){2}{\rule{0.400pt}{1.686pt}}
\put(871.67,244){\rule{0.400pt}{3.373pt}}
\multiput(871.17,244.00)(1.000,7.000){2}{\rule{0.400pt}{1.686pt}}
\put(861.0,251.0){\usebox{\plotpoint}}
\put(873.67,237){\rule{0.400pt}{5.059pt}}
\multiput(873.17,247.50)(1.000,-10.500){2}{\rule{0.400pt}{2.529pt}}
\put(874.67,237){\rule{0.400pt}{1.686pt}}
\multiput(874.17,237.00)(1.000,3.500){2}{\rule{0.400pt}{0.843pt}}
\put(875.67,244){\rule{0.400pt}{5.300pt}}
\multiput(875.17,244.00)(1.000,11.000){2}{\rule{0.400pt}{2.650pt}}
\put(876.67,258){\rule{0.400pt}{1.927pt}}
\multiput(876.17,262.00)(1.000,-4.000){2}{\rule{0.400pt}{0.964pt}}
\put(877.67,237){\rule{0.400pt}{5.059pt}}
\multiput(877.17,247.50)(1.000,-10.500){2}{\rule{0.400pt}{2.529pt}}
\put(879.17,237){\rule{0.400pt}{4.300pt}}
\multiput(878.17,237.00)(2.000,12.075){2}{\rule{0.400pt}{2.150pt}}
\put(880.67,237){\rule{0.400pt}{5.059pt}}
\multiput(880.17,247.50)(1.000,-10.500){2}{\rule{0.400pt}{2.529pt}}
\put(873.0,258.0){\usebox{\plotpoint}}
\put(882.67,237){\rule{0.400pt}{5.059pt}}
\multiput(882.17,237.00)(1.000,10.500){2}{\rule{0.400pt}{2.529pt}}
\put(882.0,237.0){\usebox{\plotpoint}}
\put(884.67,237){\rule{0.400pt}{5.059pt}}
\multiput(884.17,247.50)(1.000,-10.500){2}{\rule{0.400pt}{2.529pt}}
\put(885.67,237){\rule{0.400pt}{5.059pt}}
\multiput(885.17,237.00)(1.000,10.500){2}{\rule{0.400pt}{2.529pt}}
\put(886.67,237){\rule{0.400pt}{5.059pt}}
\multiput(886.17,247.50)(1.000,-10.500){2}{\rule{0.400pt}{2.529pt}}
\put(884.0,258.0){\usebox{\plotpoint}}
\put(888.67,237){\rule{0.400pt}{5.059pt}}
\multiput(888.17,237.00)(1.000,10.500){2}{\rule{0.400pt}{2.529pt}}
\put(888.0,237.0){\usebox{\plotpoint}}
\put(890.67,237){\rule{0.400pt}{5.059pt}}
\multiput(890.17,247.50)(1.000,-10.500){2}{\rule{0.400pt}{2.529pt}}
\put(890.0,258.0){\usebox{\plotpoint}}
\put(892.67,237){\rule{0.400pt}{5.059pt}}
\multiput(892.17,237.00)(1.000,10.500){2}{\rule{0.400pt}{2.529pt}}
\put(893.67,237){\rule{0.400pt}{5.059pt}}
\multiput(893.17,247.50)(1.000,-10.500){2}{\rule{0.400pt}{2.529pt}}
\put(894.67,237){\rule{0.400pt}{5.059pt}}
\multiput(894.17,237.00)(1.000,10.500){2}{\rule{0.400pt}{2.529pt}}
\put(895.67,237){\rule{0.400pt}{5.059pt}}
\multiput(895.17,247.50)(1.000,-10.500){2}{\rule{0.400pt}{2.529pt}}
\put(896.67,237){\rule{0.400pt}{5.059pt}}
\multiput(896.17,237.00)(1.000,10.500){2}{\rule{0.400pt}{2.529pt}}
\put(897.67,244){\rule{0.400pt}{3.373pt}}
\multiput(897.17,251.00)(1.000,-7.000){2}{\rule{0.400pt}{1.686pt}}
\put(898.67,244){\rule{0.400pt}{3.373pt}}
\multiput(898.17,244.00)(1.000,7.000){2}{\rule{0.400pt}{1.686pt}}
\put(899.67,237){\rule{0.400pt}{5.059pt}}
\multiput(899.17,247.50)(1.000,-10.500){2}{\rule{0.400pt}{2.529pt}}
\put(900.67,237){\rule{0.400pt}{5.059pt}}
\multiput(900.17,237.00)(1.000,10.500){2}{\rule{0.400pt}{2.529pt}}
\put(901.67,237){\rule{0.400pt}{5.059pt}}
\multiput(901.17,247.50)(1.000,-10.500){2}{\rule{0.400pt}{2.529pt}}
\put(902.67,237){\rule{0.400pt}{5.059pt}}
\multiput(902.17,237.00)(1.000,10.500){2}{\rule{0.400pt}{2.529pt}}
\put(892.0,237.0){\usebox{\plotpoint}}
\put(905.17,237){\rule{0.400pt}{4.300pt}}
\multiput(904.17,249.08)(2.000,-12.075){2}{\rule{0.400pt}{2.150pt}}
\put(904.0,258.0){\usebox{\plotpoint}}
\put(907.67,237){\rule{0.400pt}{5.059pt}}
\multiput(907.17,237.00)(1.000,10.500){2}{\rule{0.400pt}{2.529pt}}
\put(907.0,237.0){\usebox{\plotpoint}}
\put(909.67,237){\rule{0.400pt}{5.059pt}}
\multiput(909.17,247.50)(1.000,-10.500){2}{\rule{0.400pt}{2.529pt}}
\put(910.67,237){\rule{0.400pt}{5.059pt}}
\multiput(910.17,237.00)(1.000,10.500){2}{\rule{0.400pt}{2.529pt}}
\put(911.67,237){\rule{0.400pt}{5.059pt}}
\multiput(911.17,247.50)(1.000,-10.500){2}{\rule{0.400pt}{2.529pt}}
\put(909.0,258.0){\usebox{\plotpoint}}
\put(913.67,237){\rule{0.400pt}{5.059pt}}
\multiput(913.17,237.00)(1.000,10.500){2}{\rule{0.400pt}{2.529pt}}
\put(914.67,237){\rule{0.400pt}{5.059pt}}
\multiput(914.17,247.50)(1.000,-10.500){2}{\rule{0.400pt}{2.529pt}}
\put(915.67,237){\rule{0.400pt}{5.059pt}}
\multiput(915.17,237.00)(1.000,10.500){2}{\rule{0.400pt}{2.529pt}}
\put(913.0,237.0){\usebox{\plotpoint}}
\put(917.67,237){\rule{0.400pt}{5.059pt}}
\multiput(917.17,247.50)(1.000,-10.500){2}{\rule{0.400pt}{2.529pt}}
\put(917.0,258.0){\usebox{\plotpoint}}
\put(919.67,237){\rule{0.400pt}{5.059pt}}
\multiput(919.17,237.00)(1.000,10.500){2}{\rule{0.400pt}{2.529pt}}
\put(920.67,237){\rule{0.400pt}{5.059pt}}
\multiput(920.17,247.50)(1.000,-10.500){2}{\rule{0.400pt}{2.529pt}}
\put(921.67,237){\rule{0.400pt}{5.059pt}}
\multiput(921.17,237.00)(1.000,10.500){2}{\rule{0.400pt}{2.529pt}}
\put(919.0,237.0){\usebox{\plotpoint}}
\put(923.67,237){\rule{0.400pt}{5.059pt}}
\multiput(923.17,247.50)(1.000,-10.500){2}{\rule{0.400pt}{2.529pt}}
\put(923.0,258.0){\usebox{\plotpoint}}
\put(925.67,237){\rule{0.400pt}{5.059pt}}
\multiput(925.17,237.00)(1.000,10.500){2}{\rule{0.400pt}{2.529pt}}
\put(926.67,237){\rule{0.400pt}{5.059pt}}
\multiput(926.17,247.50)(1.000,-10.500){2}{\rule{0.400pt}{2.529pt}}
\put(927.67,237){\rule{0.400pt}{5.059pt}}
\multiput(927.17,237.00)(1.000,10.500){2}{\rule{0.400pt}{2.529pt}}
\put(928.67,237){\rule{0.400pt}{5.059pt}}
\multiput(928.17,247.50)(1.000,-10.500){2}{\rule{0.400pt}{2.529pt}}
\put(929.67,237){\rule{0.400pt}{5.059pt}}
\multiput(929.17,237.00)(1.000,10.500){2}{\rule{0.400pt}{2.529pt}}
\put(925.0,237.0){\usebox{\plotpoint}}
\put(932.67,237){\rule{0.400pt}{5.059pt}}
\multiput(932.17,247.50)(1.000,-10.500){2}{\rule{0.400pt}{2.529pt}}
\put(933.67,237){\rule{0.400pt}{5.059pt}}
\multiput(933.17,237.00)(1.000,10.500){2}{\rule{0.400pt}{2.529pt}}
\put(934.67,237){\rule{0.400pt}{5.059pt}}
\multiput(934.17,247.50)(1.000,-10.500){2}{\rule{0.400pt}{2.529pt}}
\put(935.67,237){\rule{0.400pt}{5.059pt}}
\multiput(935.17,237.00)(1.000,10.500){2}{\rule{0.400pt}{2.529pt}}
\put(936.67,237){\rule{0.400pt}{5.059pt}}
\multiput(936.17,247.50)(1.000,-10.500){2}{\rule{0.400pt}{2.529pt}}
\put(931.0,258.0){\rule[-0.200pt]{0.482pt}{0.400pt}}
\put(938.67,237){\rule{0.400pt}{5.059pt}}
\multiput(938.17,237.00)(1.000,10.500){2}{\rule{0.400pt}{2.529pt}}
\put(938.0,237.0){\usebox{\plotpoint}}
\put(940.67,237){\rule{0.400pt}{5.059pt}}
\multiput(940.17,247.50)(1.000,-10.500){2}{\rule{0.400pt}{2.529pt}}
\put(940.0,258.0){\usebox{\plotpoint}}
\put(942.67,237){\rule{0.400pt}{5.059pt}}
\multiput(942.17,237.00)(1.000,10.500){2}{\rule{0.400pt}{2.529pt}}
\put(943.67,237){\rule{0.400pt}{5.059pt}}
\multiput(943.17,247.50)(1.000,-10.500){2}{\rule{0.400pt}{2.529pt}}
\put(944.67,237){\rule{0.400pt}{5.059pt}}
\multiput(944.17,237.00)(1.000,10.500){2}{\rule{0.400pt}{2.529pt}}
\put(942.0,237.0){\usebox{\plotpoint}}
\put(946.67,237){\rule{0.400pt}{5.059pt}}
\multiput(946.17,247.50)(1.000,-10.500){2}{\rule{0.400pt}{2.529pt}}
\put(947.67,237){\rule{0.400pt}{5.059pt}}
\multiput(947.17,237.00)(1.000,10.500){2}{\rule{0.400pt}{2.529pt}}
\put(948.67,237){\rule{0.400pt}{5.059pt}}
\multiput(948.17,247.50)(1.000,-10.500){2}{\rule{0.400pt}{2.529pt}}
\put(946.0,258.0){\usebox{\plotpoint}}
\put(950.67,237){\rule{0.400pt}{5.059pt}}
\multiput(950.17,237.00)(1.000,10.500){2}{\rule{0.400pt}{2.529pt}}
\put(951.67,237){\rule{0.400pt}{5.059pt}}
\multiput(951.17,247.50)(1.000,-10.500){2}{\rule{0.400pt}{2.529pt}}
\put(952.67,237){\rule{0.400pt}{5.059pt}}
\multiput(952.17,237.00)(1.000,10.500){2}{\rule{0.400pt}{2.529pt}}
\put(950.0,237.0){\usebox{\plotpoint}}
\put(954.67,237){\rule{0.400pt}{5.059pt}}
\multiput(954.17,247.50)(1.000,-10.500){2}{\rule{0.400pt}{2.529pt}}
\put(954.0,258.0){\usebox{\plotpoint}}
\put(957.17,237){\rule{0.400pt}{4.300pt}}
\multiput(956.17,237.00)(2.000,12.075){2}{\rule{0.400pt}{2.150pt}}
\put(958.67,237){\rule{0.400pt}{5.059pt}}
\multiput(958.17,247.50)(1.000,-10.500){2}{\rule{0.400pt}{2.529pt}}
\put(959.67,237){\rule{0.400pt}{5.059pt}}
\multiput(959.17,237.00)(1.000,10.500){2}{\rule{0.400pt}{2.529pt}}
\put(956.0,237.0){\usebox{\plotpoint}}
\put(961.67,237){\rule{0.400pt}{5.059pt}}
\multiput(961.17,247.50)(1.000,-10.500){2}{\rule{0.400pt}{2.529pt}}
\put(962.67,237){\rule{0.400pt}{5.059pt}}
\multiput(962.17,237.00)(1.000,10.500){2}{\rule{0.400pt}{2.529pt}}
\put(963.67,237){\rule{0.400pt}{5.059pt}}
\multiput(963.17,247.50)(1.000,-10.500){2}{\rule{0.400pt}{2.529pt}}
\put(964.67,237){\rule{0.400pt}{5.059pt}}
\multiput(964.17,237.00)(1.000,10.500){2}{\rule{0.400pt}{2.529pt}}
\put(965.67,237){\rule{0.400pt}{5.059pt}}
\multiput(965.17,247.50)(1.000,-10.500){2}{\rule{0.400pt}{2.529pt}}
\put(966.67,237){\rule{0.400pt}{5.059pt}}
\multiput(966.17,237.00)(1.000,10.500){2}{\rule{0.400pt}{2.529pt}}
\put(967.67,237){\rule{0.400pt}{5.059pt}}
\multiput(967.17,247.50)(1.000,-10.500){2}{\rule{0.400pt}{2.529pt}}
\put(968.67,237){\rule{0.400pt}{5.059pt}}
\multiput(968.17,237.00)(1.000,10.500){2}{\rule{0.400pt}{2.529pt}}
\put(969.67,237){\rule{0.400pt}{5.059pt}}
\multiput(969.17,247.50)(1.000,-10.500){2}{\rule{0.400pt}{2.529pt}}
\put(961.0,258.0){\usebox{\plotpoint}}
\put(971.67,237){\rule{0.400pt}{5.059pt}}
\multiput(971.17,237.00)(1.000,10.500){2}{\rule{0.400pt}{2.529pt}}
\put(972.67,237){\rule{0.400pt}{5.059pt}}
\multiput(972.17,247.50)(1.000,-10.500){2}{\rule{0.400pt}{2.529pt}}
\put(973.67,237){\rule{0.400pt}{5.059pt}}
\multiput(973.17,237.00)(1.000,10.500){2}{\rule{0.400pt}{2.529pt}}
\put(974.67,237){\rule{0.400pt}{5.059pt}}
\multiput(974.17,247.50)(1.000,-10.500){2}{\rule{0.400pt}{2.529pt}}
\put(975.67,237){\rule{0.400pt}{5.059pt}}
\multiput(975.17,237.00)(1.000,10.500){2}{\rule{0.400pt}{2.529pt}}
\put(971.0,237.0){\usebox{\plotpoint}}
\put(977.67,237){\rule{0.400pt}{5.059pt}}
\multiput(977.17,247.50)(1.000,-10.500){2}{\rule{0.400pt}{2.529pt}}
\put(977.0,258.0){\usebox{\plotpoint}}
\put(979.67,237){\rule{0.400pt}{5.059pt}}
\multiput(979.17,237.00)(1.000,10.500){2}{\rule{0.400pt}{2.529pt}}
\put(980.67,237){\rule{0.400pt}{5.059pt}}
\multiput(980.17,247.50)(1.000,-10.500){2}{\rule{0.400pt}{2.529pt}}
\put(981.67,237){\rule{0.400pt}{5.059pt}}
\multiput(981.17,237.00)(1.000,10.500){2}{\rule{0.400pt}{2.529pt}}
\put(983.17,237){\rule{0.400pt}{4.300pt}}
\multiput(982.17,249.08)(2.000,-12.075){2}{\rule{0.400pt}{2.150pt}}
\put(984.67,237){\rule{0.400pt}{5.059pt}}
\multiput(984.17,237.00)(1.000,10.500){2}{\rule{0.400pt}{2.529pt}}
\put(985.67,237){\rule{0.400pt}{5.059pt}}
\multiput(985.17,247.50)(1.000,-10.500){2}{\rule{0.400pt}{2.529pt}}
\put(986.67,237){\rule{0.400pt}{5.059pt}}
\multiput(986.17,237.00)(1.000,10.500){2}{\rule{0.400pt}{2.529pt}}
\put(979.0,237.0){\usebox{\plotpoint}}
\put(988.67,237){\rule{0.400pt}{5.059pt}}
\multiput(988.17,247.50)(1.000,-10.500){2}{\rule{0.400pt}{2.529pt}}
\put(989.67,237){\rule{0.400pt}{5.059pt}}
\multiput(989.17,237.00)(1.000,10.500){2}{\rule{0.400pt}{2.529pt}}
\put(990.67,237){\rule{0.400pt}{5.059pt}}
\multiput(990.17,247.50)(1.000,-10.500){2}{\rule{0.400pt}{2.529pt}}
\put(991.67,237){\rule{0.400pt}{5.059pt}}
\multiput(991.17,237.00)(1.000,10.500){2}{\rule{0.400pt}{2.529pt}}
\put(992.67,237){\rule{0.400pt}{5.059pt}}
\multiput(992.17,247.50)(1.000,-10.500){2}{\rule{0.400pt}{2.529pt}}
\put(988.0,258.0){\usebox{\plotpoint}}
\put(994.67,237){\rule{0.400pt}{5.059pt}}
\multiput(994.17,237.00)(1.000,10.500){2}{\rule{0.400pt}{2.529pt}}
\put(994.0,237.0){\usebox{\plotpoint}}
\put(996.67,237){\rule{0.400pt}{5.059pt}}
\multiput(996.17,247.50)(1.000,-10.500){2}{\rule{0.400pt}{2.529pt}}
\put(997.67,237){\rule{0.400pt}{5.059pt}}
\multiput(997.17,237.00)(1.000,10.500){2}{\rule{0.400pt}{2.529pt}}
\put(998.67,237){\rule{0.400pt}{5.059pt}}
\multiput(998.17,247.50)(1.000,-10.500){2}{\rule{0.400pt}{2.529pt}}
\put(996.0,258.0){\usebox{\plotpoint}}
\put(1000.67,237){\rule{0.400pt}{5.059pt}}
\multiput(1000.17,237.00)(1.000,10.500){2}{\rule{0.400pt}{2.529pt}}
\put(1000.0,237.0){\usebox{\plotpoint}}
\put(1002.67,237){\rule{0.400pt}{5.059pt}}
\multiput(1002.17,247.50)(1.000,-10.500){2}{\rule{0.400pt}{2.529pt}}
\put(1002.0,258.0){\usebox{\plotpoint}}
\put(1004.67,237){\rule{0.400pt}{5.059pt}}
\multiput(1004.17,237.00)(1.000,10.500){2}{\rule{0.400pt}{2.529pt}}
\put(1004.0,237.0){\usebox{\plotpoint}}
\put(1006.67,237){\rule{0.400pt}{5.059pt}}
\multiput(1006.17,247.50)(1.000,-10.500){2}{\rule{0.400pt}{2.529pt}}
\put(1006.0,258.0){\usebox{\plotpoint}}
\put(1009.17,237){\rule{0.400pt}{4.300pt}}
\multiput(1008.17,237.00)(2.000,12.075){2}{\rule{0.400pt}{2.150pt}}
\put(1010.67,237){\rule{0.400pt}{5.059pt}}
\multiput(1010.17,247.50)(1.000,-10.500){2}{\rule{0.400pt}{2.529pt}}
\put(1011.67,237){\rule{0.400pt}{5.059pt}}
\multiput(1011.17,237.00)(1.000,10.500){2}{\rule{0.400pt}{2.529pt}}
\put(1012.67,237){\rule{0.400pt}{5.059pt}}
\multiput(1012.17,247.50)(1.000,-10.500){2}{\rule{0.400pt}{2.529pt}}
\put(1013.67,237){\rule{0.400pt}{5.059pt}}
\multiput(1013.17,237.00)(1.000,10.500){2}{\rule{0.400pt}{2.529pt}}
\put(1008.0,237.0){\usebox{\plotpoint}}
\put(1015.67,237){\rule{0.400pt}{5.059pt}}
\multiput(1015.17,247.50)(1.000,-10.500){2}{\rule{0.400pt}{2.529pt}}
\put(1015.0,258.0){\usebox{\plotpoint}}
\put(1017.67,237){\rule{0.400pt}{5.059pt}}
\multiput(1017.17,237.00)(1.000,10.500){2}{\rule{0.400pt}{2.529pt}}
\put(1017.0,237.0){\usebox{\plotpoint}}
\put(1019.67,237){\rule{0.400pt}{5.059pt}}
\multiput(1019.17,247.50)(1.000,-10.500){2}{\rule{0.400pt}{2.529pt}}
\put(1020.67,237){\rule{0.400pt}{5.059pt}}
\multiput(1020.17,237.00)(1.000,10.500){2}{\rule{0.400pt}{2.529pt}}
\put(1021.67,237){\rule{0.400pt}{5.059pt}}
\multiput(1021.17,247.50)(1.000,-10.500){2}{\rule{0.400pt}{2.529pt}}
\put(1022.67,237){\rule{0.400pt}{5.059pt}}
\multiput(1022.17,237.00)(1.000,10.500){2}{\rule{0.400pt}{2.529pt}}
\put(1023.67,244){\rule{0.400pt}{3.373pt}}
\multiput(1023.17,251.00)(1.000,-7.000){2}{\rule{0.400pt}{1.686pt}}
\put(1024.67,244){\rule{0.400pt}{3.373pt}}
\multiput(1024.17,244.00)(1.000,7.000){2}{\rule{0.400pt}{1.686pt}}
\put(1025.67,244){\rule{0.400pt}{3.373pt}}
\multiput(1025.17,251.00)(1.000,-7.000){2}{\rule{0.400pt}{1.686pt}}
\put(1026.67,244){\rule{0.400pt}{3.373pt}}
\multiput(1026.17,244.00)(1.000,7.000){2}{\rule{0.400pt}{1.686pt}}
\put(1027.67,237){\rule{0.400pt}{5.059pt}}
\multiput(1027.17,247.50)(1.000,-10.500){2}{\rule{0.400pt}{2.529pt}}
\put(1028.67,237){\rule{0.400pt}{1.686pt}}
\multiput(1028.17,237.00)(1.000,3.500){2}{\rule{0.400pt}{0.843pt}}
\put(1029.67,244){\rule{0.400pt}{3.373pt}}
\multiput(1029.17,244.00)(1.000,7.000){2}{\rule{0.400pt}{1.686pt}}
\put(1030.67,258){\rule{0.400pt}{1.927pt}}
\multiput(1030.17,258.00)(1.000,4.000){2}{\rule{0.400pt}{0.964pt}}
\put(1031.67,244){\rule{0.400pt}{5.300pt}}
\multiput(1031.17,255.00)(1.000,-11.000){2}{\rule{0.400pt}{2.650pt}}
\put(1032.67,244){\rule{0.400pt}{3.373pt}}
\multiput(1032.17,244.00)(1.000,7.000){2}{\rule{0.400pt}{1.686pt}}
\put(1033.67,237){\rule{0.400pt}{5.059pt}}
\multiput(1033.17,247.50)(1.000,-10.500){2}{\rule{0.400pt}{2.529pt}}
\put(1035.17,237){\rule{0.400pt}{1.500pt}}
\multiput(1034.17,237.00)(2.000,3.887){2}{\rule{0.400pt}{0.750pt}}
\put(1036.67,244){\rule{0.400pt}{3.373pt}}
\multiput(1036.17,244.00)(1.000,7.000){2}{\rule{0.400pt}{1.686pt}}
\put(1019.0,258.0){\usebox{\plotpoint}}
\put(1038.67,237){\rule{0.400pt}{5.059pt}}
\multiput(1038.17,247.50)(1.000,-10.500){2}{\rule{0.400pt}{2.529pt}}
\put(1038.0,258.0){\usebox{\plotpoint}}
\put(1040.67,237){\rule{0.400pt}{5.059pt}}
\multiput(1040.17,237.00)(1.000,10.500){2}{\rule{0.400pt}{2.529pt}}
\put(1041.67,237){\rule{0.400pt}{5.059pt}}
\multiput(1041.17,247.50)(1.000,-10.500){2}{\rule{0.400pt}{2.529pt}}
\put(1042.67,237){\rule{0.400pt}{5.059pt}}
\multiput(1042.17,237.00)(1.000,10.500){2}{\rule{0.400pt}{2.529pt}}
\put(1043.67,237){\rule{0.400pt}{5.059pt}}
\multiput(1043.17,247.50)(1.000,-10.500){2}{\rule{0.400pt}{2.529pt}}
\put(1044.67,237){\rule{0.400pt}{5.059pt}}
\multiput(1044.17,237.00)(1.000,10.500){2}{\rule{0.400pt}{2.529pt}}
\put(1045.67,237){\rule{0.400pt}{5.059pt}}
\multiput(1045.17,247.50)(1.000,-10.500){2}{\rule{0.400pt}{2.529pt}}
\put(1046.67,237){\rule{0.400pt}{6.986pt}}
\multiput(1046.17,237.00)(1.000,14.500){2}{\rule{0.400pt}{3.493pt}}
\put(1047.67,258){\rule{0.400pt}{1.927pt}}
\multiput(1047.17,262.00)(1.000,-4.000){2}{\rule{0.400pt}{0.964pt}}
\put(1048.67,237){\rule{0.400pt}{5.059pt}}
\multiput(1048.17,247.50)(1.000,-10.500){2}{\rule{0.400pt}{2.529pt}}
\put(1040.0,237.0){\usebox{\plotpoint}}
\put(1050.67,237){\rule{0.400pt}{5.059pt}}
\multiput(1050.17,237.00)(1.000,10.500){2}{\rule{0.400pt}{2.529pt}}
\put(1051.67,237){\rule{0.400pt}{5.059pt}}
\multiput(1051.17,247.50)(1.000,-10.500){2}{\rule{0.400pt}{2.529pt}}
\put(1052.67,237){\rule{0.400pt}{5.059pt}}
\multiput(1052.17,237.00)(1.000,10.500){2}{\rule{0.400pt}{2.529pt}}
\put(1050.0,237.0){\usebox{\plotpoint}}
\put(1054.67,237){\rule{0.400pt}{5.059pt}}
\multiput(1054.17,247.50)(1.000,-10.500){2}{\rule{0.400pt}{2.529pt}}
\put(1055.67,237){\rule{0.400pt}{5.059pt}}
\multiput(1055.17,237.00)(1.000,10.500){2}{\rule{0.400pt}{2.529pt}}
\put(1056.67,237){\rule{0.400pt}{5.059pt}}
\multiput(1056.17,247.50)(1.000,-10.500){2}{\rule{0.400pt}{2.529pt}}
\put(1057.67,237){\rule{0.400pt}{5.059pt}}
\multiput(1057.17,237.00)(1.000,10.500){2}{\rule{0.400pt}{2.529pt}}
\put(1058.67,237){\rule{0.400pt}{5.059pt}}
\multiput(1058.17,247.50)(1.000,-10.500){2}{\rule{0.400pt}{2.529pt}}
\put(1059.67,237){\rule{0.400pt}{5.059pt}}
\multiput(1059.17,237.00)(1.000,10.500){2}{\rule{0.400pt}{2.529pt}}
\put(1061.17,237){\rule{0.400pt}{4.300pt}}
\multiput(1060.17,249.08)(2.000,-12.075){2}{\rule{0.400pt}{2.150pt}}
\put(1054.0,258.0){\usebox{\plotpoint}}
\put(1063.67,237){\rule{0.400pt}{5.059pt}}
\multiput(1063.17,237.00)(1.000,10.500){2}{\rule{0.400pt}{2.529pt}}
\put(1064.67,237){\rule{0.400pt}{5.059pt}}
\multiput(1064.17,247.50)(1.000,-10.500){2}{\rule{0.400pt}{2.529pt}}
\put(1065.67,237){\rule{0.400pt}{5.059pt}}
\multiput(1065.17,237.00)(1.000,10.500){2}{\rule{0.400pt}{2.529pt}}
\put(1066.67,237){\rule{0.400pt}{5.059pt}}
\multiput(1066.17,247.50)(1.000,-10.500){2}{\rule{0.400pt}{2.529pt}}
\put(1067.67,237){\rule{0.400pt}{5.059pt}}
\multiput(1067.17,237.00)(1.000,10.500){2}{\rule{0.400pt}{2.529pt}}
\put(1068.67,237){\rule{0.400pt}{5.059pt}}
\multiput(1068.17,247.50)(1.000,-10.500){2}{\rule{0.400pt}{2.529pt}}
\put(1069.67,237){\rule{0.400pt}{5.059pt}}
\multiput(1069.17,237.00)(1.000,10.500){2}{\rule{0.400pt}{2.529pt}}
\put(1070.67,237){\rule{0.400pt}{5.059pt}}
\multiput(1070.17,247.50)(1.000,-10.500){2}{\rule{0.400pt}{2.529pt}}
\put(1071.67,237){\rule{0.400pt}{5.059pt}}
\multiput(1071.17,237.00)(1.000,10.500){2}{\rule{0.400pt}{2.529pt}}
\put(1072.67,237){\rule{0.400pt}{5.059pt}}
\multiput(1072.17,247.50)(1.000,-10.500){2}{\rule{0.400pt}{2.529pt}}
\put(1073.67,237){\rule{0.400pt}{5.059pt}}
\multiput(1073.17,237.00)(1.000,10.500){2}{\rule{0.400pt}{2.529pt}}
\put(1063.0,237.0){\usebox{\plotpoint}}
\put(1075.67,237){\rule{0.400pt}{5.059pt}}
\multiput(1075.17,247.50)(1.000,-10.500){2}{\rule{0.400pt}{2.529pt}}
\put(1076.67,237){\rule{0.400pt}{5.059pt}}
\multiput(1076.17,237.00)(1.000,10.500){2}{\rule{0.400pt}{2.529pt}}
\put(1077.67,237){\rule{0.400pt}{5.059pt}}
\multiput(1077.17,247.50)(1.000,-10.500){2}{\rule{0.400pt}{2.529pt}}
\put(1075.0,258.0){\usebox{\plotpoint}}
\put(1079.67,237){\rule{0.400pt}{5.059pt}}
\multiput(1079.17,237.00)(1.000,10.500){2}{\rule{0.400pt}{2.529pt}}
\put(1080.67,237){\rule{0.400pt}{5.059pt}}
\multiput(1080.17,247.50)(1.000,-10.500){2}{\rule{0.400pt}{2.529pt}}
\put(1081.67,237){\rule{0.400pt}{3.373pt}}
\multiput(1081.17,237.00)(1.000,7.000){2}{\rule{0.400pt}{1.686pt}}
\put(1082.67,251){\rule{0.400pt}{1.686pt}}
\multiput(1082.17,251.00)(1.000,3.500){2}{\rule{0.400pt}{0.843pt}}
\put(1083.67,237){\rule{0.400pt}{5.059pt}}
\multiput(1083.17,247.50)(1.000,-10.500){2}{\rule{0.400pt}{2.529pt}}
\put(1084.67,237){\rule{0.400pt}{5.059pt}}
\multiput(1084.17,237.00)(1.000,10.500){2}{\rule{0.400pt}{2.529pt}}
\put(1085.67,237){\rule{0.400pt}{5.059pt}}
\multiput(1085.17,247.50)(1.000,-10.500){2}{\rule{0.400pt}{2.529pt}}
\put(1087.17,237){\rule{0.400pt}{4.300pt}}
\multiput(1086.17,237.00)(2.000,12.075){2}{\rule{0.400pt}{2.150pt}}
\put(1088.67,237){\rule{0.400pt}{5.059pt}}
\multiput(1088.17,247.50)(1.000,-10.500){2}{\rule{0.400pt}{2.529pt}}
\put(1079.0,237.0){\usebox{\plotpoint}}
\put(1090.67,237){\rule{0.400pt}{5.059pt}}
\multiput(1090.17,237.00)(1.000,10.500){2}{\rule{0.400pt}{2.529pt}}
\put(1091.67,237){\rule{0.400pt}{5.059pt}}
\multiput(1091.17,247.50)(1.000,-10.500){2}{\rule{0.400pt}{2.529pt}}
\put(1092.67,237){\rule{0.400pt}{5.059pt}}
\multiput(1092.17,237.00)(1.000,10.500){2}{\rule{0.400pt}{2.529pt}}
\put(1090.0,237.0){\usebox{\plotpoint}}
\put(1094.67,229){\rule{0.400pt}{6.986pt}}
\multiput(1094.17,243.50)(1.000,-14.500){2}{\rule{0.400pt}{3.493pt}}
\put(1094.0,258.0){\usebox{\plotpoint}}
\put(1096.67,229){\rule{0.400pt}{5.300pt}}
\multiput(1096.17,229.00)(1.000,11.000){2}{\rule{0.400pt}{2.650pt}}
\put(1097.67,229){\rule{0.400pt}{5.300pt}}
\multiput(1097.17,240.00)(1.000,-11.000){2}{\rule{0.400pt}{2.650pt}}
\put(1098.67,229){\rule{0.400pt}{6.986pt}}
\multiput(1098.17,229.00)(1.000,14.500){2}{\rule{0.400pt}{3.493pt}}
\put(1099.67,244){\rule{0.400pt}{3.373pt}}
\multiput(1099.17,251.00)(1.000,-7.000){2}{\rule{0.400pt}{1.686pt}}
\put(1100.67,229){\rule{0.400pt}{3.614pt}}
\multiput(1100.17,236.50)(1.000,-7.500){2}{\rule{0.400pt}{1.807pt}}
\put(1101.67,229){\rule{0.400pt}{3.614pt}}
\multiput(1101.17,229.00)(1.000,7.500){2}{\rule{0.400pt}{1.807pt}}
\put(1102.67,229){\rule{0.400pt}{3.614pt}}
\multiput(1102.17,236.50)(1.000,-7.500){2}{\rule{0.400pt}{1.807pt}}
\put(1096.0,229.0){\usebox{\plotpoint}}
\put(1104.67,229){\rule{0.400pt}{3.614pt}}
\multiput(1104.17,229.00)(1.000,7.500){2}{\rule{0.400pt}{1.807pt}}
\put(1104.0,229.0){\usebox{\plotpoint}}
\put(1106.67,229){\rule{0.400pt}{3.614pt}}
\multiput(1106.17,236.50)(1.000,-7.500){2}{\rule{0.400pt}{1.807pt}}
\put(1107.67,229){\rule{0.400pt}{3.614pt}}
\multiput(1107.17,229.00)(1.000,7.500){2}{\rule{0.400pt}{1.807pt}}
\put(1108.67,229){\rule{0.400pt}{3.614pt}}
\multiput(1108.17,236.50)(1.000,-7.500){2}{\rule{0.400pt}{1.807pt}}
\put(1109.67,229){\rule{0.400pt}{3.614pt}}
\multiput(1109.17,229.00)(1.000,7.500){2}{\rule{0.400pt}{1.807pt}}
\put(1110.67,222){\rule{0.400pt}{5.300pt}}
\multiput(1110.17,233.00)(1.000,-11.000){2}{\rule{0.400pt}{2.650pt}}
\put(1106.0,244.0){\usebox{\plotpoint}}
\put(1113.17,222){\rule{0.400pt}{4.500pt}}
\multiput(1112.17,222.00)(2.000,12.660){2}{\rule{0.400pt}{2.250pt}}
\put(1112.0,222.0){\usebox{\plotpoint}}
\put(1115.67,222){\rule{0.400pt}{5.300pt}}
\multiput(1115.17,233.00)(1.000,-11.000){2}{\rule{0.400pt}{2.650pt}}
\put(1116.67,222){\rule{0.400pt}{3.614pt}}
\multiput(1116.17,222.00)(1.000,7.500){2}{\rule{0.400pt}{1.807pt}}
\put(1117.67,222){\rule{0.400pt}{3.614pt}}
\multiput(1117.17,229.50)(1.000,-7.500){2}{\rule{0.400pt}{1.807pt}}
\put(1118.67,222){\rule{0.400pt}{5.300pt}}
\multiput(1118.17,222.00)(1.000,11.000){2}{\rule{0.400pt}{2.650pt}}
\put(1119.67,215){\rule{0.400pt}{6.986pt}}
\multiput(1119.17,229.50)(1.000,-14.500){2}{\rule{0.400pt}{3.493pt}}
\put(1120.67,215){\rule{0.400pt}{1.686pt}}
\multiput(1120.17,215.00)(1.000,3.500){2}{\rule{0.400pt}{0.843pt}}
\put(1121.67,222){\rule{0.400pt}{3.614pt}}
\multiput(1121.17,222.00)(1.000,7.500){2}{\rule{0.400pt}{1.807pt}}
\put(1115.0,244.0){\usebox{\plotpoint}}
\put(1123.67,215){\rule{0.400pt}{5.300pt}}
\multiput(1123.17,226.00)(1.000,-11.000){2}{\rule{0.400pt}{2.650pt}}
\put(1124.67,215){\rule{0.400pt}{3.373pt}}
\multiput(1124.17,215.00)(1.000,7.000){2}{\rule{0.400pt}{1.686pt}}
\put(1125.67,215){\rule{0.400pt}{3.373pt}}
\multiput(1125.17,222.00)(1.000,-7.000){2}{\rule{0.400pt}{1.686pt}}
\put(1123.0,237.0){\usebox{\plotpoint}}
\put(1127.67,215){\rule{0.400pt}{5.300pt}}
\multiput(1127.17,215.00)(1.000,11.000){2}{\rule{0.400pt}{2.650pt}}
\put(1127.0,215.0){\usebox{\plotpoint}}
\put(1129.67,215){\rule{0.400pt}{5.300pt}}
\multiput(1129.17,226.00)(1.000,-11.000){2}{\rule{0.400pt}{2.650pt}}
\put(1130.67,215){\rule{0.400pt}{5.300pt}}
\multiput(1130.17,215.00)(1.000,11.000){2}{\rule{0.400pt}{2.650pt}}
\put(1131.67,215){\rule{0.400pt}{5.300pt}}
\multiput(1131.17,226.00)(1.000,-11.000){2}{\rule{0.400pt}{2.650pt}}
\put(1132.67,215){\rule{0.400pt}{1.686pt}}
\multiput(1132.17,215.00)(1.000,3.500){2}{\rule{0.400pt}{0.843pt}}
\put(1133.67,222){\rule{0.400pt}{3.614pt}}
\multiput(1133.17,222.00)(1.000,7.500){2}{\rule{0.400pt}{1.807pt}}
\put(1134.67,222){\rule{0.400pt}{3.614pt}}
\multiput(1134.17,229.50)(1.000,-7.500){2}{\rule{0.400pt}{1.807pt}}
\put(1135.67,222){\rule{0.400pt}{3.614pt}}
\multiput(1135.17,222.00)(1.000,7.500){2}{\rule{0.400pt}{1.807pt}}
\put(1129.0,237.0){\usebox{\plotpoint}}
\put(1137.67,222){\rule{0.400pt}{3.614pt}}
\multiput(1137.17,229.50)(1.000,-7.500){2}{\rule{0.400pt}{1.807pt}}
\put(1137.0,237.0){\usebox{\plotpoint}}
\put(1140.67,222){\rule{0.400pt}{5.300pt}}
\multiput(1140.17,222.00)(1.000,11.000){2}{\rule{0.400pt}{2.650pt}}
\put(1141.67,229){\rule{0.400pt}{3.614pt}}
\multiput(1141.17,236.50)(1.000,-7.500){2}{\rule{0.400pt}{1.807pt}}
\put(1142.67,229){\rule{0.400pt}{3.614pt}}
\multiput(1142.17,229.00)(1.000,7.500){2}{\rule{0.400pt}{1.807pt}}
\put(1143.67,229){\rule{0.400pt}{3.614pt}}
\multiput(1143.17,236.50)(1.000,-7.500){2}{\rule{0.400pt}{1.807pt}}
\put(1144.67,229){\rule{0.400pt}{3.614pt}}
\multiput(1144.17,229.00)(1.000,7.500){2}{\rule{0.400pt}{1.807pt}}
\put(1139.0,222.0){\rule[-0.200pt]{0.482pt}{0.400pt}}
\put(1146.67,229){\rule{0.400pt}{3.614pt}}
\multiput(1146.17,236.50)(1.000,-7.500){2}{\rule{0.400pt}{1.807pt}}
\put(1147.67,229){\rule{0.400pt}{3.614pt}}
\multiput(1147.17,229.00)(1.000,7.500){2}{\rule{0.400pt}{1.807pt}}
\put(1148.67,229){\rule{0.400pt}{3.614pt}}
\multiput(1148.17,236.50)(1.000,-7.500){2}{\rule{0.400pt}{1.807pt}}
\put(1149.67,229){\rule{0.400pt}{3.614pt}}
\multiput(1149.17,229.00)(1.000,7.500){2}{\rule{0.400pt}{1.807pt}}
\put(1150.67,229){\rule{0.400pt}{3.614pt}}
\multiput(1150.17,236.50)(1.000,-7.500){2}{\rule{0.400pt}{1.807pt}}
\put(1151.67,229){\rule{0.400pt}{3.614pt}}
\multiput(1151.17,229.00)(1.000,7.500){2}{\rule{0.400pt}{1.807pt}}
\put(1152.67,229){\rule{0.400pt}{3.614pt}}
\multiput(1152.17,236.50)(1.000,-7.500){2}{\rule{0.400pt}{1.807pt}}
\put(1146.0,244.0){\usebox{\plotpoint}}
\put(1154.67,229){\rule{0.400pt}{3.614pt}}
\multiput(1154.17,229.00)(1.000,7.500){2}{\rule{0.400pt}{1.807pt}}
\put(1155.67,229){\rule{0.400pt}{3.614pt}}
\multiput(1155.17,236.50)(1.000,-7.500){2}{\rule{0.400pt}{1.807pt}}
\put(1156.67,229){\rule{0.400pt}{5.300pt}}
\multiput(1156.17,229.00)(1.000,11.000){2}{\rule{0.400pt}{2.650pt}}
\put(1157.67,229){\rule{0.400pt}{5.300pt}}
\multiput(1157.17,240.00)(1.000,-11.000){2}{\rule{0.400pt}{2.650pt}}
\put(1158.67,229){\rule{0.400pt}{5.300pt}}
\multiput(1158.17,229.00)(1.000,11.000){2}{\rule{0.400pt}{2.650pt}}
\put(1154.0,229.0){\usebox{\plotpoint}}
\put(1160.67,222){\rule{0.400pt}{6.986pt}}
\multiput(1160.17,236.50)(1.000,-14.500){2}{\rule{0.400pt}{3.493pt}}
\put(1161.67,222){\rule{0.400pt}{5.300pt}}
\multiput(1161.17,222.00)(1.000,11.000){2}{\rule{0.400pt}{2.650pt}}
\put(1162.67,222){\rule{0.400pt}{5.300pt}}
\multiput(1162.17,233.00)(1.000,-11.000){2}{\rule{0.400pt}{2.650pt}}
\put(1163.67,222){\rule{0.400pt}{5.300pt}}
\multiput(1163.17,222.00)(1.000,11.000){2}{\rule{0.400pt}{2.650pt}}
\put(1165.17,215){\rule{0.400pt}{5.900pt}}
\multiput(1164.17,231.75)(2.000,-16.754){2}{\rule{0.400pt}{2.950pt}}
\put(1166.67,215){\rule{0.400pt}{5.300pt}}
\multiput(1166.17,215.00)(1.000,11.000){2}{\rule{0.400pt}{2.650pt}}
\put(1167.67,207){\rule{0.400pt}{7.227pt}}
\multiput(1167.17,222.00)(1.000,-15.000){2}{\rule{0.400pt}{3.613pt}}
\put(1168.67,207){\rule{0.400pt}{5.300pt}}
\multiput(1168.17,207.00)(1.000,11.000){2}{\rule{0.400pt}{2.650pt}}
\put(1169.67,207){\rule{0.400pt}{5.300pt}}
\multiput(1169.17,218.00)(1.000,-11.000){2}{\rule{0.400pt}{2.650pt}}
\put(1170.67,207){\rule{0.400pt}{5.300pt}}
\multiput(1170.17,207.00)(1.000,11.000){2}{\rule{0.400pt}{2.650pt}}
\put(1171.67,200){\rule{0.400pt}{6.986pt}}
\multiput(1171.17,214.50)(1.000,-14.500){2}{\rule{0.400pt}{3.493pt}}
\put(1172.67,200){\rule{0.400pt}{1.686pt}}
\multiput(1172.17,200.00)(1.000,3.500){2}{\rule{0.400pt}{0.843pt}}
\put(1173.67,207){\rule{0.400pt}{5.300pt}}
\multiput(1173.17,207.00)(1.000,11.000){2}{\rule{0.400pt}{2.650pt}}
\put(1174.67,207){\rule{0.400pt}{5.300pt}}
\multiput(1174.17,218.00)(1.000,-11.000){2}{\rule{0.400pt}{2.650pt}}
\put(1175.67,207){\rule{0.400pt}{5.300pt}}
\multiput(1175.17,207.00)(1.000,11.000){2}{\rule{0.400pt}{2.650pt}}
\put(1176.67,215){\rule{0.400pt}{3.373pt}}
\multiput(1176.17,222.00)(1.000,-7.000){2}{\rule{0.400pt}{1.686pt}}
\put(1177.67,215){\rule{0.400pt}{5.300pt}}
\multiput(1177.17,215.00)(1.000,11.000){2}{\rule{0.400pt}{2.650pt}}
\put(1178.67,222){\rule{0.400pt}{3.614pt}}
\multiput(1178.17,229.50)(1.000,-7.500){2}{\rule{0.400pt}{1.807pt}}
\put(1179.67,222){\rule{0.400pt}{5.300pt}}
\multiput(1179.17,222.00)(1.000,11.000){2}{\rule{0.400pt}{2.650pt}}
\put(1180.67,229){\rule{0.400pt}{3.614pt}}
\multiput(1180.17,236.50)(1.000,-7.500){2}{\rule{0.400pt}{1.807pt}}
\put(1181.67,229){\rule{0.400pt}{6.986pt}}
\multiput(1181.17,229.00)(1.000,14.500){2}{\rule{0.400pt}{3.493pt}}
\put(1182.67,244){\rule{0.400pt}{3.373pt}}
\multiput(1182.17,251.00)(1.000,-7.000){2}{\rule{0.400pt}{1.686pt}}
\put(1183.67,244){\rule{0.400pt}{6.986pt}}
\multiput(1183.17,244.00)(1.000,14.500){2}{\rule{0.400pt}{3.493pt}}
\put(1184.67,258){\rule{0.400pt}{3.614pt}}
\multiput(1184.17,265.50)(1.000,-7.500){2}{\rule{0.400pt}{1.807pt}}
\put(1185.67,258){\rule{0.400pt}{5.300pt}}
\multiput(1185.17,258.00)(1.000,11.000){2}{\rule{0.400pt}{2.650pt}}
\put(1186.67,266){\rule{0.400pt}{3.373pt}}
\multiput(1186.17,273.00)(1.000,-7.000){2}{\rule{0.400pt}{1.686pt}}
\put(1187.67,266){\rule{0.400pt}{6.986pt}}
\multiput(1187.17,266.00)(1.000,14.500){2}{\rule{0.400pt}{3.493pt}}
\put(1188.67,273){\rule{0.400pt}{5.300pt}}
\multiput(1188.17,284.00)(1.000,-11.000){2}{\rule{0.400pt}{2.650pt}}
\put(1189.67,273){\rule{0.400pt}{8.672pt}}
\multiput(1189.17,273.00)(1.000,18.000){2}{\rule{0.400pt}{4.336pt}}
\put(1191.17,295){\rule{0.400pt}{2.900pt}}
\multiput(1190.17,302.98)(2.000,-7.981){2}{\rule{0.400pt}{1.450pt}}
\put(1192.67,295){\rule{0.400pt}{5.300pt}}
\multiput(1192.17,295.00)(1.000,11.000){2}{\rule{0.400pt}{2.650pt}}
\put(1193.67,302){\rule{0.400pt}{3.614pt}}
\multiput(1193.17,309.50)(1.000,-7.500){2}{\rule{0.400pt}{1.807pt}}
\put(1194.67,302){\rule{0.400pt}{5.300pt}}
\multiput(1194.17,302.00)(1.000,11.000){2}{\rule{0.400pt}{2.650pt}}
\put(1195.67,309){\rule{0.400pt}{3.614pt}}
\multiput(1195.17,316.50)(1.000,-7.500){2}{\rule{0.400pt}{1.807pt}}
\put(1196.67,309){\rule{0.400pt}{5.300pt}}
\multiput(1196.17,309.00)(1.000,11.000){2}{\rule{0.400pt}{2.650pt}}
\put(1197.67,317){\rule{0.400pt}{3.373pt}}
\multiput(1197.17,324.00)(1.000,-7.000){2}{\rule{0.400pt}{1.686pt}}
\put(1198.67,317){\rule{0.400pt}{6.986pt}}
\multiput(1198.17,317.00)(1.000,14.500){2}{\rule{0.400pt}{3.493pt}}
\put(1199.67,338){\rule{0.400pt}{1.927pt}}
\multiput(1199.17,342.00)(1.000,-4.000){2}{\rule{0.400pt}{0.964pt}}
\put(1200.67,338){\rule{0.400pt}{3.614pt}}
\multiput(1200.17,338.00)(1.000,7.500){2}{\rule{0.400pt}{1.807pt}}
\put(1201.67,346){\rule{0.400pt}{1.686pt}}
\multiput(1201.17,349.50)(1.000,-3.500){2}{\rule{0.400pt}{0.843pt}}
\put(1202.67,346){\rule{0.400pt}{3.373pt}}
\multiput(1202.17,346.00)(1.000,7.000){2}{\rule{0.400pt}{1.686pt}}
\put(1203.67,353){\rule{0.400pt}{1.686pt}}
\multiput(1203.17,356.50)(1.000,-3.500){2}{\rule{0.400pt}{0.843pt}}
\put(1204.67,353){\rule{0.400pt}{3.614pt}}
\multiput(1204.17,353.00)(1.000,7.500){2}{\rule{0.400pt}{1.807pt}}
\put(1205.67,360){\rule{0.400pt}{1.927pt}}
\multiput(1205.17,364.00)(1.000,-4.000){2}{\rule{0.400pt}{0.964pt}}
\put(1206.67,360){\rule{0.400pt}{3.614pt}}
\multiput(1206.17,360.00)(1.000,7.500){2}{\rule{0.400pt}{1.807pt}}
\put(1207.67,368){\rule{0.400pt}{1.686pt}}
\multiput(1207.17,371.50)(1.000,-3.500){2}{\rule{0.400pt}{0.843pt}}
\put(1208.67,368){\rule{0.400pt}{5.059pt}}
\multiput(1208.17,368.00)(1.000,10.500){2}{\rule{0.400pt}{2.529pt}}
\put(1209.67,368){\rule{0.400pt}{5.059pt}}
\multiput(1209.17,378.50)(1.000,-10.500){2}{\rule{0.400pt}{2.529pt}}
\put(1210.67,368){\rule{0.400pt}{6.986pt}}
\multiput(1210.17,368.00)(1.000,14.500){2}{\rule{0.400pt}{3.493pt}}
\put(1211.67,382){\rule{0.400pt}{3.614pt}}
\multiput(1211.17,389.50)(1.000,-7.500){2}{\rule{0.400pt}{1.807pt}}
\put(1212.67,382){\rule{0.400pt}{5.300pt}}
\multiput(1212.17,382.00)(1.000,11.000){2}{\rule{0.400pt}{2.650pt}}
\put(1160.0,251.0){\usebox{\plotpoint}}
\put(1214.67,389){\rule{0.400pt}{3.614pt}}
\multiput(1214.17,396.50)(1.000,-7.500){2}{\rule{0.400pt}{1.807pt}}
\put(1215.67,389){\rule{0.400pt}{1.927pt}}
\multiput(1215.17,389.00)(1.000,4.000){2}{\rule{0.400pt}{0.964pt}}
\put(1217.17,397){\rule{0.400pt}{2.900pt}}
\multiput(1216.17,397.00)(2.000,7.981){2}{\rule{0.400pt}{1.450pt}}
\put(1218.67,397){\rule{0.400pt}{3.373pt}}
\multiput(1218.17,404.00)(1.000,-7.000){2}{\rule{0.400pt}{1.686pt}}
\put(1219.67,397){\rule{0.400pt}{5.300pt}}
\multiput(1219.17,397.00)(1.000,11.000){2}{\rule{0.400pt}{2.650pt}}
\put(1214.0,404.0){\usebox{\plotpoint}}
\put(1221.67,404){\rule{0.400pt}{3.614pt}}
\multiput(1221.17,411.50)(1.000,-7.500){2}{\rule{0.400pt}{1.807pt}}
\put(1222.67,404){\rule{0.400pt}{1.686pt}}
\multiput(1222.17,404.00)(1.000,3.500){2}{\rule{0.400pt}{0.843pt}}
\put(1223.67,411){\rule{0.400pt}{5.300pt}}
\multiput(1223.17,411.00)(1.000,11.000){2}{\rule{0.400pt}{2.650pt}}
\put(1224.67,411){\rule{0.400pt}{5.300pt}}
\multiput(1224.17,422.00)(1.000,-11.000){2}{\rule{0.400pt}{2.650pt}}
\put(1225.67,411){\rule{0.400pt}{6.986pt}}
\multiput(1225.17,411.00)(1.000,14.500){2}{\rule{0.400pt}{3.493pt}}
\put(1226.67,426){\rule{0.400pt}{3.373pt}}
\multiput(1226.17,433.00)(1.000,-7.000){2}{\rule{0.400pt}{1.686pt}}
\put(1227.67,426){\rule{0.400pt}{3.373pt}}
\multiput(1227.17,426.00)(1.000,7.000){2}{\rule{0.400pt}{1.686pt}}
\put(1228.67,426){\rule{0.400pt}{3.373pt}}
\multiput(1228.17,433.00)(1.000,-7.000){2}{\rule{0.400pt}{1.686pt}}
\put(1229.67,426){\rule{0.400pt}{5.300pt}}
\multiput(1229.17,426.00)(1.000,11.000){2}{\rule{0.400pt}{2.650pt}}
\put(1230.67,433){\rule{0.400pt}{3.614pt}}
\multiput(1230.17,440.50)(1.000,-7.500){2}{\rule{0.400pt}{1.807pt}}
\put(1231.67,433){\rule{0.400pt}{5.300pt}}
\multiput(1231.17,433.00)(1.000,11.000){2}{\rule{0.400pt}{2.650pt}}
\put(1232.67,440){\rule{0.400pt}{3.614pt}}
\multiput(1232.17,447.50)(1.000,-7.500){2}{\rule{0.400pt}{1.807pt}}
\put(1233.67,440){\rule{0.400pt}{3.614pt}}
\multiput(1233.17,440.00)(1.000,7.500){2}{\rule{0.400pt}{1.807pt}}
\put(1234.67,440){\rule{0.400pt}{3.614pt}}
\multiput(1234.17,447.50)(1.000,-7.500){2}{\rule{0.400pt}{1.807pt}}
\put(1235.67,440){\rule{0.400pt}{3.614pt}}
\multiput(1235.17,440.00)(1.000,7.500){2}{\rule{0.400pt}{1.807pt}}
\put(1236.67,455){\rule{0.400pt}{1.686pt}}
\multiput(1236.17,455.00)(1.000,3.500){2}{\rule{0.400pt}{0.843pt}}
\put(1237.67,448){\rule{0.400pt}{3.373pt}}
\multiput(1237.17,455.00)(1.000,-7.000){2}{\rule{0.400pt}{1.686pt}}
\put(1238.67,448){\rule{0.400pt}{3.373pt}}
\multiput(1238.17,448.00)(1.000,7.000){2}{\rule{0.400pt}{1.686pt}}
\put(1239.67,448){\rule{0.400pt}{3.373pt}}
\multiput(1239.17,455.00)(1.000,-7.000){2}{\rule{0.400pt}{1.686pt}}
\put(1240.67,448){\rule{0.400pt}{1.686pt}}
\multiput(1240.17,448.00)(1.000,3.500){2}{\rule{0.400pt}{0.843pt}}
\put(1241.67,455){\rule{0.400pt}{5.300pt}}
\multiput(1241.17,455.00)(1.000,11.000){2}{\rule{0.400pt}{2.650pt}}
\put(1221.0,419.0){\usebox{\plotpoint}}
\put(1244.67,455){\rule{0.400pt}{5.300pt}}
\multiput(1244.17,466.00)(1.000,-11.000){2}{\rule{0.400pt}{2.650pt}}
\put(1245.67,455){\rule{0.400pt}{3.614pt}}
\multiput(1245.17,455.00)(1.000,7.500){2}{\rule{0.400pt}{1.807pt}}
\put(1246.67,470){\rule{0.400pt}{3.373pt}}
\multiput(1246.17,470.00)(1.000,7.000){2}{\rule{0.400pt}{1.686pt}}
\put(1247.67,470){\rule{0.400pt}{3.373pt}}
\multiput(1247.17,477.00)(1.000,-7.000){2}{\rule{0.400pt}{1.686pt}}
\put(1248.67,470){\rule{0.400pt}{3.373pt}}
\multiput(1248.17,470.00)(1.000,7.000){2}{\rule{0.400pt}{1.686pt}}
\put(1249.67,470){\rule{0.400pt}{3.373pt}}
\multiput(1249.17,477.00)(1.000,-7.000){2}{\rule{0.400pt}{1.686pt}}
\put(1250.67,470){\rule{0.400pt}{5.059pt}}
\multiput(1250.17,470.00)(1.000,10.500){2}{\rule{0.400pt}{2.529pt}}
\put(1251.67,477){\rule{0.400pt}{3.373pt}}
\multiput(1251.17,484.00)(1.000,-7.000){2}{\rule{0.400pt}{1.686pt}}
\put(1252.67,477){\rule{0.400pt}{3.373pt}}
\multiput(1252.17,477.00)(1.000,7.000){2}{\rule{0.400pt}{1.686pt}}
\put(1253.67,477){\rule{0.400pt}{3.373pt}}
\multiput(1253.17,484.00)(1.000,-7.000){2}{\rule{0.400pt}{1.686pt}}
\put(1254.67,477){\rule{0.400pt}{5.300pt}}
\multiput(1254.17,477.00)(1.000,11.000){2}{\rule{0.400pt}{2.650pt}}
\put(1255.67,484){\rule{0.400pt}{3.614pt}}
\multiput(1255.17,491.50)(1.000,-7.500){2}{\rule{0.400pt}{1.807pt}}
\put(1256.67,484){\rule{0.400pt}{3.614pt}}
\multiput(1256.17,484.00)(1.000,7.500){2}{\rule{0.400pt}{1.807pt}}
\put(1243.0,477.0){\rule[-0.200pt]{0.482pt}{0.400pt}}
\put(1258.67,484){\rule{0.400pt}{3.614pt}}
\multiput(1258.17,491.50)(1.000,-7.500){2}{\rule{0.400pt}{1.807pt}}
\put(1259.67,484){\rule{0.400pt}{3.614pt}}
\multiput(1259.17,484.00)(1.000,7.500){2}{\rule{0.400pt}{1.807pt}}
\put(1260.67,484){\rule{0.400pt}{3.614pt}}
\multiput(1260.17,491.50)(1.000,-7.500){2}{\rule{0.400pt}{1.807pt}}
\put(1261.67,484){\rule{0.400pt}{5.300pt}}
\multiput(1261.17,484.00)(1.000,11.000){2}{\rule{0.400pt}{2.650pt}}
\put(1262.67,491){\rule{0.400pt}{3.614pt}}
\multiput(1262.17,498.50)(1.000,-7.500){2}{\rule{0.400pt}{1.807pt}}
\put(1258.0,499.0){\usebox{\plotpoint}}
\put(1264.67,491){\rule{0.400pt}{3.614pt}}
\multiput(1264.17,491.00)(1.000,7.500){2}{\rule{0.400pt}{1.807pt}}
\put(1265.67,491){\rule{0.400pt}{3.614pt}}
\multiput(1265.17,498.50)(1.000,-7.500){2}{\rule{0.400pt}{1.807pt}}
\put(1266.67,491){\rule{0.400pt}{6.986pt}}
\multiput(1266.17,491.00)(1.000,14.500){2}{\rule{0.400pt}{3.493pt}}
\put(1267.67,491){\rule{0.400pt}{6.986pt}}
\multiput(1267.17,505.50)(1.000,-14.500){2}{\rule{0.400pt}{3.493pt}}
\put(1269.17,491){\rule{0.400pt}{5.900pt}}
\multiput(1268.17,491.00)(2.000,16.754){2}{\rule{0.400pt}{2.950pt}}
\put(1270.67,499){\rule{0.400pt}{5.059pt}}
\multiput(1270.17,509.50)(1.000,-10.500){2}{\rule{0.400pt}{2.529pt}}
\put(1271.67,499){\rule{0.400pt}{5.059pt}}
\multiput(1271.17,499.00)(1.000,10.500){2}{\rule{0.400pt}{2.529pt}}
\put(1264.0,491.0){\usebox{\plotpoint}}
\put(1273.67,499){\rule{0.400pt}{5.059pt}}
\multiput(1273.17,509.50)(1.000,-10.500){2}{\rule{0.400pt}{2.529pt}}
\put(1274.67,499){\rule{0.400pt}{6.986pt}}
\multiput(1274.17,499.00)(1.000,14.500){2}{\rule{0.400pt}{3.493pt}}
\put(1275.67,499){\rule{0.400pt}{6.986pt}}
\multiput(1275.17,513.50)(1.000,-14.500){2}{\rule{0.400pt}{3.493pt}}
\put(1276.67,499){\rule{0.400pt}{1.686pt}}
\multiput(1276.17,499.00)(1.000,3.500){2}{\rule{0.400pt}{0.843pt}}
\put(1277.67,506){\rule{0.400pt}{5.300pt}}
\multiput(1277.17,506.00)(1.000,11.000){2}{\rule{0.400pt}{2.650pt}}
\put(1273.0,520.0){\usebox{\plotpoint}}
\put(1279.67,513){\rule{0.400pt}{3.614pt}}
\multiput(1279.17,520.50)(1.000,-7.500){2}{\rule{0.400pt}{1.807pt}}
\put(1279.0,528.0){\usebox{\plotpoint}}
\put(1281.67,513){\rule{0.400pt}{3.614pt}}
\multiput(1281.17,513.00)(1.000,7.500){2}{\rule{0.400pt}{1.807pt}}
\put(1282.67,513){\rule{0.400pt}{3.614pt}}
\multiput(1282.17,520.50)(1.000,-7.500){2}{\rule{0.400pt}{1.807pt}}
\put(1283.67,513){\rule{0.400pt}{5.300pt}}
\multiput(1283.17,513.00)(1.000,11.000){2}{\rule{0.400pt}{2.650pt}}
\put(1281.0,513.0){\usebox{\plotpoint}}
\put(1285.67,513){\rule{0.400pt}{5.300pt}}
\multiput(1285.17,524.00)(1.000,-11.000){2}{\rule{0.400pt}{2.650pt}}
\put(1286.67,513){\rule{0.400pt}{1.686pt}}
\multiput(1286.17,513.00)(1.000,3.500){2}{\rule{0.400pt}{0.843pt}}
\put(1287.67,520){\rule{0.400pt}{3.614pt}}
\multiput(1287.17,520.00)(1.000,7.500){2}{\rule{0.400pt}{1.807pt}}
\put(1285.0,535.0){\usebox{\plotpoint}}
\put(1289.67,520){\rule{0.400pt}{3.614pt}}
\multiput(1289.17,527.50)(1.000,-7.500){2}{\rule{0.400pt}{1.807pt}}
\put(1289.0,535.0){\usebox{\plotpoint}}
\put(1291.67,520){\rule{0.400pt}{3.614pt}}
\multiput(1291.17,520.00)(1.000,7.500){2}{\rule{0.400pt}{1.807pt}}
\put(1292.67,520){\rule{0.400pt}{3.614pt}}
\multiput(1292.17,527.50)(1.000,-7.500){2}{\rule{0.400pt}{1.807pt}}
\put(1293.67,520){\rule{0.400pt}{5.300pt}}
\multiput(1293.17,520.00)(1.000,11.000){2}{\rule{0.400pt}{2.650pt}}
\put(1295.17,520){\rule{0.400pt}{4.500pt}}
\multiput(1294.17,532.66)(2.000,-12.660){2}{\rule{0.400pt}{2.250pt}}
\put(1296.67,520){\rule{0.400pt}{5.300pt}}
\multiput(1296.17,520.00)(1.000,11.000){2}{\rule{0.400pt}{2.650pt}}
\put(1297.67,528){\rule{0.400pt}{3.373pt}}
\multiput(1297.17,535.00)(1.000,-7.000){2}{\rule{0.400pt}{1.686pt}}
\put(1298.67,528){\rule{0.400pt}{3.373pt}}
\multiput(1298.17,528.00)(1.000,7.000){2}{\rule{0.400pt}{1.686pt}}
\put(1291.0,520.0){\usebox{\plotpoint}}
\put(1300.67,528){\rule{0.400pt}{3.373pt}}
\multiput(1300.17,535.00)(1.000,-7.000){2}{\rule{0.400pt}{1.686pt}}
\put(1301.67,528){\rule{0.400pt}{3.373pt}}
\multiput(1301.17,528.00)(1.000,7.000){2}{\rule{0.400pt}{1.686pt}}
\put(1302.67,528){\rule{0.400pt}{3.373pt}}
\multiput(1302.17,535.00)(1.000,-7.000){2}{\rule{0.400pt}{1.686pt}}
\put(1303.67,528){\rule{0.400pt}{3.373pt}}
\multiput(1303.17,528.00)(1.000,7.000){2}{\rule{0.400pt}{1.686pt}}
\put(1304.67,528){\rule{0.400pt}{3.373pt}}
\multiput(1304.17,535.00)(1.000,-7.000){2}{\rule{0.400pt}{1.686pt}}
\put(1300.0,542.0){\usebox{\plotpoint}}
\put(1306.67,528){\rule{0.400pt}{5.300pt}}
\multiput(1306.17,528.00)(1.000,11.000){2}{\rule{0.400pt}{2.650pt}}
\put(1306.0,528.0){\usebox{\plotpoint}}
\put(1308.67,528){\rule{0.400pt}{5.300pt}}
\multiput(1308.17,539.00)(1.000,-11.000){2}{\rule{0.400pt}{2.650pt}}
\put(1309.67,528){\rule{0.400pt}{5.300pt}}
\multiput(1309.17,528.00)(1.000,11.000){2}{\rule{0.400pt}{2.650pt}}
\put(1310.67,528){\rule{0.400pt}{5.300pt}}
\multiput(1310.17,539.00)(1.000,-11.000){2}{\rule{0.400pt}{2.650pt}}
\put(1311.67,528){\rule{0.400pt}{5.300pt}}
\multiput(1311.17,528.00)(1.000,11.000){2}{\rule{0.400pt}{2.650pt}}
\put(1312.67,535){\rule{0.400pt}{3.614pt}}
\multiput(1312.17,542.50)(1.000,-7.500){2}{\rule{0.400pt}{1.807pt}}
\put(1313.67,535){\rule{0.400pt}{3.614pt}}
\multiput(1313.17,535.00)(1.000,7.500){2}{\rule{0.400pt}{1.807pt}}
\put(1314.67,535){\rule{0.400pt}{3.614pt}}
\multiput(1314.17,542.50)(1.000,-7.500){2}{\rule{0.400pt}{1.807pt}}
\put(1315.67,535){\rule{0.400pt}{3.614pt}}
\multiput(1315.17,535.00)(1.000,7.500){2}{\rule{0.400pt}{1.807pt}}
\put(1316.67,535){\rule{0.400pt}{3.614pt}}
\multiput(1316.17,542.50)(1.000,-7.500){2}{\rule{0.400pt}{1.807pt}}
\put(1317.67,535){\rule{0.400pt}{3.614pt}}
\multiput(1317.17,535.00)(1.000,7.500){2}{\rule{0.400pt}{1.807pt}}
\put(1318.67,535){\rule{0.400pt}{3.614pt}}
\multiput(1318.17,542.50)(1.000,-7.500){2}{\rule{0.400pt}{1.807pt}}
\put(1319.67,535){\rule{0.400pt}{3.614pt}}
\multiput(1319.17,535.00)(1.000,7.500){2}{\rule{0.400pt}{1.807pt}}
\put(1321.17,535){\rule{0.400pt}{3.100pt}}
\multiput(1320.17,543.57)(2.000,-8.566){2}{\rule{0.400pt}{1.550pt}}
\put(1322.67,535){\rule{0.400pt}{3.614pt}}
\multiput(1322.17,535.00)(1.000,7.500){2}{\rule{0.400pt}{1.807pt}}
\put(1323.67,535){\rule{0.400pt}{3.614pt}}
\multiput(1323.17,542.50)(1.000,-7.500){2}{\rule{0.400pt}{1.807pt}}
\put(1324.67,535){\rule{0.400pt}{3.614pt}}
\multiput(1324.17,535.00)(1.000,7.500){2}{\rule{0.400pt}{1.807pt}}
\put(1325.67,542){\rule{0.400pt}{1.927pt}}
\multiput(1325.17,546.00)(1.000,-4.000){2}{\rule{0.400pt}{0.964pt}}
\put(1326.67,542){\rule{0.400pt}{5.300pt}}
\multiput(1326.17,542.00)(1.000,11.000){2}{\rule{0.400pt}{2.650pt}}
\put(1327.67,542){\rule{0.400pt}{5.300pt}}
\multiput(1327.17,553.00)(1.000,-11.000){2}{\rule{0.400pt}{2.650pt}}
\put(1308.0,550.0){\usebox{\plotpoint}}
\put(1329.67,542){\rule{0.400pt}{5.300pt}}
\multiput(1329.17,542.00)(1.000,11.000){2}{\rule{0.400pt}{2.650pt}}
\put(1330.67,542){\rule{0.400pt}{5.300pt}}
\multiput(1330.17,553.00)(1.000,-11.000){2}{\rule{0.400pt}{2.650pt}}
\put(1331.67,542){\rule{0.400pt}{5.300pt}}
\multiput(1331.17,542.00)(1.000,11.000){2}{\rule{0.400pt}{2.650pt}}
\put(1329.0,542.0){\usebox{\plotpoint}}
\put(1333.67,542){\rule{0.400pt}{5.300pt}}
\multiput(1333.17,553.00)(1.000,-11.000){2}{\rule{0.400pt}{2.650pt}}
\put(1334.67,542){\rule{0.400pt}{5.300pt}}
\multiput(1334.17,542.00)(1.000,11.000){2}{\rule{0.400pt}{2.650pt}}
\put(1335.67,542){\rule{0.400pt}{5.300pt}}
\multiput(1335.17,553.00)(1.000,-11.000){2}{\rule{0.400pt}{2.650pt}}
\put(1336.67,542){\rule{0.400pt}{5.300pt}}
\multiput(1336.17,542.00)(1.000,11.000){2}{\rule{0.400pt}{2.650pt}}
\put(1337.67,542){\rule{0.400pt}{5.300pt}}
\multiput(1337.17,553.00)(1.000,-11.000){2}{\rule{0.400pt}{2.650pt}}
\put(1333.0,564.0){\usebox{\plotpoint}}
\put(1339.67,542){\rule{0.400pt}{5.300pt}}
\multiput(1339.17,542.00)(1.000,11.000){2}{\rule{0.400pt}{2.650pt}}
\put(1339.0,542.0){\usebox{\plotpoint}}
\put(1341.67,542){\rule{0.400pt}{5.300pt}}
\multiput(1341.17,553.00)(1.000,-11.000){2}{\rule{0.400pt}{2.650pt}}
\put(1342.67,542){\rule{0.400pt}{5.300pt}}
\multiput(1342.17,542.00)(1.000,11.000){2}{\rule{0.400pt}{2.650pt}}
\put(1343.67,542){\rule{0.400pt}{5.300pt}}
\multiput(1343.17,553.00)(1.000,-11.000){2}{\rule{0.400pt}{2.650pt}}
\put(1344.67,542){\rule{0.400pt}{5.300pt}}
\multiput(1344.17,542.00)(1.000,11.000){2}{\rule{0.400pt}{2.650pt}}
\put(1345.67,542){\rule{0.400pt}{5.300pt}}
\multiput(1345.17,553.00)(1.000,-11.000){2}{\rule{0.400pt}{2.650pt}}
\put(1347.17,542){\rule{0.400pt}{4.500pt}}
\multiput(1346.17,542.00)(2.000,12.660){2}{\rule{0.400pt}{2.250pt}}
\put(1348.67,542){\rule{0.400pt}{5.300pt}}
\multiput(1348.17,553.00)(1.000,-11.000){2}{\rule{0.400pt}{2.650pt}}
\put(1349.67,542){\rule{0.400pt}{5.300pt}}
\multiput(1349.17,542.00)(1.000,11.000){2}{\rule{0.400pt}{2.650pt}}
\put(1350.67,550){\rule{0.400pt}{3.373pt}}
\multiput(1350.17,557.00)(1.000,-7.000){2}{\rule{0.400pt}{1.686pt}}
\put(1351.67,550){\rule{0.400pt}{3.373pt}}
\multiput(1351.17,550.00)(1.000,7.000){2}{\rule{0.400pt}{1.686pt}}
\put(1352.67,550){\rule{0.400pt}{3.373pt}}
\multiput(1352.17,557.00)(1.000,-7.000){2}{\rule{0.400pt}{1.686pt}}
\put(1353.67,550){\rule{0.400pt}{1.686pt}}
\multiput(1353.17,550.00)(1.000,3.500){2}{\rule{0.400pt}{0.843pt}}
\put(1354.67,557){\rule{0.400pt}{1.686pt}}
\multiput(1354.17,557.00)(1.000,3.500){2}{\rule{0.400pt}{0.843pt}}
\put(1355.67,557){\rule{0.400pt}{1.686pt}}
\multiput(1355.17,560.50)(1.000,-3.500){2}{\rule{0.400pt}{0.843pt}}
\put(1356.67,557){\rule{0.400pt}{3.373pt}}
\multiput(1356.17,557.00)(1.000,7.000){2}{\rule{0.400pt}{1.686pt}}
\put(1357.67,557){\rule{0.400pt}{3.373pt}}
\multiput(1357.17,564.00)(1.000,-7.000){2}{\rule{0.400pt}{1.686pt}}
\put(1358.67,557){\rule{0.400pt}{3.373pt}}
\multiput(1358.17,557.00)(1.000,7.000){2}{\rule{0.400pt}{1.686pt}}
\put(1359.67,557){\rule{0.400pt}{3.373pt}}
\multiput(1359.17,564.00)(1.000,-7.000){2}{\rule{0.400pt}{1.686pt}}
\put(1360.67,557){\rule{0.400pt}{3.373pt}}
\multiput(1360.17,557.00)(1.000,7.000){2}{\rule{0.400pt}{1.686pt}}
\put(1361.67,557){\rule{0.400pt}{3.373pt}}
\multiput(1361.17,564.00)(1.000,-7.000){2}{\rule{0.400pt}{1.686pt}}
\put(1362.67,557){\rule{0.400pt}{3.373pt}}
\multiput(1362.17,557.00)(1.000,7.000){2}{\rule{0.400pt}{1.686pt}}
\put(1363.67,557){\rule{0.400pt}{3.373pt}}
\multiput(1363.17,564.00)(1.000,-7.000){2}{\rule{0.400pt}{1.686pt}}
\put(1364.67,557){\rule{0.400pt}{3.373pt}}
\multiput(1364.17,557.00)(1.000,7.000){2}{\rule{0.400pt}{1.686pt}}
\put(1365.67,557){\rule{0.400pt}{3.373pt}}
\multiput(1365.17,564.00)(1.000,-7.000){2}{\rule{0.400pt}{1.686pt}}
\put(1366.67,557){\rule{0.400pt}{3.373pt}}
\multiput(1366.17,557.00)(1.000,7.000){2}{\rule{0.400pt}{1.686pt}}
\put(1367.67,557){\rule{0.400pt}{3.373pt}}
\multiput(1367.17,564.00)(1.000,-7.000){2}{\rule{0.400pt}{1.686pt}}
\put(1368.67,557){\rule{0.400pt}{3.373pt}}
\multiput(1368.17,557.00)(1.000,7.000){2}{\rule{0.400pt}{1.686pt}}
\put(1369.67,557){\rule{0.400pt}{3.373pt}}
\multiput(1369.17,564.00)(1.000,-7.000){2}{\rule{0.400pt}{1.686pt}}
\put(1370.67,557){\rule{0.400pt}{3.373pt}}
\multiput(1370.17,557.00)(1.000,7.000){2}{\rule{0.400pt}{1.686pt}}
\put(1371.67,557){\rule{0.400pt}{3.373pt}}
\multiput(1371.17,564.00)(1.000,-7.000){2}{\rule{0.400pt}{1.686pt}}
\put(1373.17,557){\rule{0.400pt}{2.900pt}}
\multiput(1372.17,557.00)(2.000,7.981){2}{\rule{0.400pt}{1.450pt}}
\put(1374.67,557){\rule{0.400pt}{3.373pt}}
\multiput(1374.17,564.00)(1.000,-7.000){2}{\rule{0.400pt}{1.686pt}}
\put(1375.67,557){\rule{0.400pt}{3.373pt}}
\multiput(1375.17,557.00)(1.000,7.000){2}{\rule{0.400pt}{1.686pt}}
\put(1376.67,557){\rule{0.400pt}{3.373pt}}
\multiput(1376.17,564.00)(1.000,-7.000){2}{\rule{0.400pt}{1.686pt}}
\put(1377.67,557){\rule{0.400pt}{3.373pt}}
\multiput(1377.17,557.00)(1.000,7.000){2}{\rule{0.400pt}{1.686pt}}
\put(1341.0,564.0){\usebox{\plotpoint}}
\put(1379.67,557){\rule{0.400pt}{3.373pt}}
\multiput(1379.17,564.00)(1.000,-7.000){2}{\rule{0.400pt}{1.686pt}}
\put(1379.0,571.0){\usebox{\plotpoint}}
\put(1381.67,557){\rule{0.400pt}{3.373pt}}
\multiput(1381.17,557.00)(1.000,7.000){2}{\rule{0.400pt}{1.686pt}}
\put(1382.67,557){\rule{0.400pt}{3.373pt}}
\multiput(1382.17,564.00)(1.000,-7.000){2}{\rule{0.400pt}{1.686pt}}
\put(1383.67,557){\rule{0.400pt}{3.373pt}}
\multiput(1383.17,557.00)(1.000,7.000){2}{\rule{0.400pt}{1.686pt}}
\put(1381.0,557.0){\usebox{\plotpoint}}
\put(1385.67,557){\rule{0.400pt}{3.373pt}}
\multiput(1385.17,564.00)(1.000,-7.000){2}{\rule{0.400pt}{1.686pt}}
\put(1385.0,571.0){\usebox{\plotpoint}}
\put(1387.67,557){\rule{0.400pt}{3.373pt}}
\multiput(1387.17,557.00)(1.000,7.000){2}{\rule{0.400pt}{1.686pt}}
\put(1387.0,557.0){\usebox{\plotpoint}}
\put(1389.67,557){\rule{0.400pt}{3.373pt}}
\multiput(1389.17,564.00)(1.000,-7.000){2}{\rule{0.400pt}{1.686pt}}
\put(1389.0,571.0){\usebox{\plotpoint}}
\put(1391.67,557){\rule{0.400pt}{3.373pt}}
\multiput(1391.17,557.00)(1.000,7.000){2}{\rule{0.400pt}{1.686pt}}
\put(1392.67,557){\rule{0.400pt}{3.373pt}}
\multiput(1392.17,564.00)(1.000,-7.000){2}{\rule{0.400pt}{1.686pt}}
\put(1393.67,557){\rule{0.400pt}{3.373pt}}
\multiput(1393.17,557.00)(1.000,7.000){2}{\rule{0.400pt}{1.686pt}}
\put(1391.0,557.0){\usebox{\plotpoint}}
\put(1395.67,557){\rule{0.400pt}{3.373pt}}
\multiput(1395.17,564.00)(1.000,-7.000){2}{\rule{0.400pt}{1.686pt}}
\put(1396.67,557){\rule{0.400pt}{3.373pt}}
\multiput(1396.17,557.00)(1.000,7.000){2}{\rule{0.400pt}{1.686pt}}
\put(1397.67,557){\rule{0.400pt}{3.373pt}}
\multiput(1397.17,564.00)(1.000,-7.000){2}{\rule{0.400pt}{1.686pt}}
\put(1399.17,557){\rule{0.400pt}{2.900pt}}
\multiput(1398.17,557.00)(2.000,7.981){2}{\rule{0.400pt}{1.450pt}}
\put(1400.67,557){\rule{0.400pt}{3.373pt}}
\multiput(1400.17,564.00)(1.000,-7.000){2}{\rule{0.400pt}{1.686pt}}
\put(1401.67,557){\rule{0.400pt}{3.373pt}}
\multiput(1401.17,557.00)(1.000,7.000){2}{\rule{0.400pt}{1.686pt}}
\put(1402.67,557){\rule{0.400pt}{3.373pt}}
\multiput(1402.17,564.00)(1.000,-7.000){2}{\rule{0.400pt}{1.686pt}}
\put(1403.67,557){\rule{0.400pt}{3.373pt}}
\multiput(1403.17,557.00)(1.000,7.000){2}{\rule{0.400pt}{1.686pt}}
\put(1404.67,557){\rule{0.400pt}{3.373pt}}
\multiput(1404.17,564.00)(1.000,-7.000){2}{\rule{0.400pt}{1.686pt}}
\put(1405.67,557){\rule{0.400pt}{5.300pt}}
\multiput(1405.17,557.00)(1.000,11.000){2}{\rule{0.400pt}{2.650pt}}
\put(1406.67,557){\rule{0.400pt}{5.300pt}}
\multiput(1406.17,568.00)(1.000,-11.000){2}{\rule{0.400pt}{2.650pt}}
\put(1407.67,557){\rule{0.400pt}{3.373pt}}
\multiput(1407.17,557.00)(1.000,7.000){2}{\rule{0.400pt}{1.686pt}}
\put(1408.67,564){\rule{0.400pt}{1.686pt}}
\multiput(1408.17,567.50)(1.000,-3.500){2}{\rule{0.400pt}{0.843pt}}
\put(1395.0,571.0){\usebox{\plotpoint}}
\put(1410.67,564){\rule{0.400pt}{1.686pt}}
\multiput(1410.17,564.00)(1.000,3.500){2}{\rule{0.400pt}{0.843pt}}
\put(1411.67,564){\rule{0.400pt}{1.686pt}}
\multiput(1411.17,567.50)(1.000,-3.500){2}{\rule{0.400pt}{0.843pt}}
\put(1412.67,564){\rule{0.400pt}{1.686pt}}
\multiput(1412.17,564.00)(1.000,3.500){2}{\rule{0.400pt}{0.843pt}}
\put(1413.67,564){\rule{0.400pt}{1.686pt}}
\multiput(1413.17,567.50)(1.000,-3.500){2}{\rule{0.400pt}{0.843pt}}
\put(1414.67,564){\rule{0.400pt}{1.686pt}}
\multiput(1414.17,564.00)(1.000,3.500){2}{\rule{0.400pt}{0.843pt}}
\put(1415.67,564){\rule{0.400pt}{1.686pt}}
\multiput(1415.17,567.50)(1.000,-3.500){2}{\rule{0.400pt}{0.843pt}}
\put(1416.67,564){\rule{0.400pt}{3.614pt}}
\multiput(1416.17,564.00)(1.000,7.500){2}{\rule{0.400pt}{1.807pt}}
\put(1417.67,564){\rule{0.400pt}{3.614pt}}
\multiput(1417.17,571.50)(1.000,-7.500){2}{\rule{0.400pt}{1.807pt}}
\put(1418.67,564){\rule{0.400pt}{3.614pt}}
\multiput(1418.17,564.00)(1.000,7.500){2}{\rule{0.400pt}{1.807pt}}
\put(1419.67,564){\rule{0.400pt}{3.614pt}}
\multiput(1419.17,571.50)(1.000,-7.500){2}{\rule{0.400pt}{1.807pt}}
\put(1420.67,564){\rule{0.400pt}{3.614pt}}
\multiput(1420.17,564.00)(1.000,7.500){2}{\rule{0.400pt}{1.807pt}}
\put(1421.67,564){\rule{0.400pt}{3.614pt}}
\multiput(1421.17,571.50)(1.000,-7.500){2}{\rule{0.400pt}{1.807pt}}
\put(1422.67,564){\rule{0.400pt}{3.614pt}}
\multiput(1422.17,564.00)(1.000,7.500){2}{\rule{0.400pt}{1.807pt}}
\put(1423.67,564){\rule{0.400pt}{3.614pt}}
\multiput(1423.17,571.50)(1.000,-7.500){2}{\rule{0.400pt}{1.807pt}}
\put(1425.17,564){\rule{0.400pt}{3.100pt}}
\multiput(1424.17,564.00)(2.000,8.566){2}{\rule{0.400pt}{1.550pt}}
\put(1410.0,564.0){\usebox{\plotpoint}}
\put(1427.67,564){\rule{0.400pt}{3.614pt}}
\multiput(1427.17,571.50)(1.000,-7.500){2}{\rule{0.400pt}{1.807pt}}
\put(1427.0,579.0){\usebox{\plotpoint}}
\put(1429.67,564){\rule{0.400pt}{3.614pt}}
\multiput(1429.17,564.00)(1.000,7.500){2}{\rule{0.400pt}{1.807pt}}
\put(1430.67,564){\rule{0.400pt}{3.614pt}}
\multiput(1430.17,571.50)(1.000,-7.500){2}{\rule{0.400pt}{1.807pt}}
\put(1431.67,564){\rule{0.400pt}{3.614pt}}
\multiput(1431.17,564.00)(1.000,7.500){2}{\rule{0.400pt}{1.807pt}}
\put(1432.67,564){\rule{0.400pt}{3.614pt}}
\multiput(1432.17,571.50)(1.000,-7.500){2}{\rule{0.400pt}{1.807pt}}
\put(1433.67,564){\rule{0.400pt}{3.614pt}}
\multiput(1433.17,564.00)(1.000,7.500){2}{\rule{0.400pt}{1.807pt}}
\put(1434.67,564){\rule{0.400pt}{3.614pt}}
\multiput(1434.17,571.50)(1.000,-7.500){2}{\rule{0.400pt}{1.807pt}}
\put(1435.67,564){\rule{0.400pt}{3.614pt}}
\multiput(1435.17,564.00)(1.000,7.500){2}{\rule{0.400pt}{1.807pt}}
\put(1429.0,564.0){\usebox{\plotpoint}}
\put(1437.67,564){\rule{0.400pt}{3.614pt}}
\multiput(1437.17,571.50)(1.000,-7.500){2}{\rule{0.400pt}{1.807pt}}
\put(1437.0,579.0){\usebox{\plotpoint}}
\put(191.0,131.0){\rule[-0.200pt]{0.400pt}{175.375pt}}
\put(191.0,131.0){\rule[-0.200pt]{300.643pt}{0.400pt}}
\put(1439.0,131.0){\rule[-0.200pt]{0.400pt}{175.375pt}}
\put(191.0,859.0){\rule[-0.200pt]{300.643pt}{0.400pt}}
\end{picture}

\caption{
Závislosť napätia na $H_\alpha\(t\)$ dióde na čase $t$ pre výstrel \#23728. S určenými časmi začiatku a konca života plazmy.
}\label{G_V-1-H}
\end{figure}


\begin{figure}
% GNUPLOT: LaTeX picture
\setlength{\unitlength}{0.240900pt}
\ifx\plotpoint\undefined\newsavebox{\plotpoint}\fi
\begin{picture}(1500,900)(0,0)
\sbox{\plotpoint}{\rule[-0.200pt]{0.400pt}{0.400pt}}%
\put(151.0,131.0){\rule[-0.200pt]{4.818pt}{0.400pt}}
\put(131,131){\makebox(0,0)[r]{ 0}}
\put(1419.0,131.0){\rule[-0.200pt]{4.818pt}{0.400pt}}
\put(151.0,222.0){\rule[-0.200pt]{4.818pt}{0.400pt}}
\put(131,222){\makebox(0,0)[r]{ 5}}
\put(1419.0,222.0){\rule[-0.200pt]{4.818pt}{0.400pt}}
\put(151.0,313.0){\rule[-0.200pt]{4.818pt}{0.400pt}}
\put(131,313){\makebox(0,0)[r]{ 10}}
\put(1419.0,313.0){\rule[-0.200pt]{4.818pt}{0.400pt}}
\put(151.0,404.0){\rule[-0.200pt]{4.818pt}{0.400pt}}
\put(131,404){\makebox(0,0)[r]{ 15}}
\put(1419.0,404.0){\rule[-0.200pt]{4.818pt}{0.400pt}}
\put(151.0,495.0){\rule[-0.200pt]{4.818pt}{0.400pt}}
\put(131,495){\makebox(0,0)[r]{ 20}}
\put(1419.0,495.0){\rule[-0.200pt]{4.818pt}{0.400pt}}
\put(151.0,586.0){\rule[-0.200pt]{4.818pt}{0.400pt}}
\put(131,586){\makebox(0,0)[r]{ 25}}
\put(1419.0,586.0){\rule[-0.200pt]{4.818pt}{0.400pt}}
\put(151.0,677.0){\rule[-0.200pt]{4.818pt}{0.400pt}}
\put(131,677){\makebox(0,0)[r]{ 30}}
\put(1419.0,677.0){\rule[-0.200pt]{4.818pt}{0.400pt}}
\put(151.0,768.0){\rule[-0.200pt]{4.818pt}{0.400pt}}
\put(131,768){\makebox(0,0)[r]{ 35}}
\put(1419.0,768.0){\rule[-0.200pt]{4.818pt}{0.400pt}}
\put(151.0,859.0){\rule[-0.200pt]{4.818pt}{0.400pt}}
\put(131,859){\makebox(0,0)[r]{ 40}}
\put(1419.0,859.0){\rule[-0.200pt]{4.818pt}{0.400pt}}
\put(151.0,131.0){\rule[-0.200pt]{0.400pt}{4.818pt}}
\put(151,90){\makebox(0,0){ 0}}
\put(151.0,839.0){\rule[-0.200pt]{0.400pt}{4.818pt}}
\put(366.0,131.0){\rule[-0.200pt]{0.400pt}{4.818pt}}
\put(366,90){\makebox(0,0){ 2}}
\put(366.0,839.0){\rule[-0.200pt]{0.400pt}{4.818pt}}
\put(580.0,131.0){\rule[-0.200pt]{0.400pt}{4.818pt}}
\put(580,90){\makebox(0,0){ 4}}
\put(580.0,839.0){\rule[-0.200pt]{0.400pt}{4.818pt}}
\put(795.0,131.0){\rule[-0.200pt]{0.400pt}{4.818pt}}
\put(795,90){\makebox(0,0){ 6}}
\put(795.0,839.0){\rule[-0.200pt]{0.400pt}{4.818pt}}
\put(1010.0,131.0){\rule[-0.200pt]{0.400pt}{4.818pt}}
\put(1010,90){\makebox(0,0){ 8}}
\put(1010.0,839.0){\rule[-0.200pt]{0.400pt}{4.818pt}}
\put(1224.0,131.0){\rule[-0.200pt]{0.400pt}{4.818pt}}
\put(1224,90){\makebox(0,0){ 10}}
\put(1224.0,839.0){\rule[-0.200pt]{0.400pt}{4.818pt}}
\put(1439.0,131.0){\rule[-0.200pt]{0.400pt}{4.818pt}}
\put(1439,90){\makebox(0,0){ 12}}
\put(1439.0,839.0){\rule[-0.200pt]{0.400pt}{4.818pt}}
\put(151.0,131.0){\rule[-0.200pt]{0.400pt}{175.375pt}}
\put(151.0,131.0){\rule[-0.200pt]{310.279pt}{0.400pt}}
\put(1439.0,131.0){\rule[-0.200pt]{0.400pt}{175.375pt}}
\put(151.0,859.0){\rule[-0.200pt]{310.279pt}{0.400pt}}
\put(30,495){\makebox(0,0){\popi{T}{eV}}}
\put(795,29){\makebox(0,0){\popi{t}{ms}}}
\put(509,0){\line(0,1){899}}
\put(1186,0){\line(0,1){899}}
\put(509,0){\line(0,1){899}}
\put(1186,0){\line(0,1){899}}
\put(509,0){\line(0,1){899}}
\put(1186,0){\line(0,1){899}}
\put(571,172){\makebox(0,0)[r]{Electron temperature}}
\put(591.0,172.0){\rule[-0.200pt]{24.090pt}{0.400pt}}
\put(152,131){\usebox{\plotpoint}}
\put(445.67,131){\rule{0.400pt}{120.932pt}}
\multiput(445.17,131.00)(1.000,251.000){2}{\rule{0.400pt}{60.466pt}}
\put(446.67,633){\rule{0.400pt}{6.986pt}}
\multiput(446.17,633.00)(1.000,14.500){2}{\rule{0.400pt}{3.493pt}}
\put(447.67,609){\rule{0.400pt}{12.768pt}}
\multiput(447.17,635.50)(1.000,-26.500){2}{\rule{0.400pt}{6.384pt}}
\put(448.67,609){\rule{0.400pt}{11.081pt}}
\multiput(448.17,609.00)(1.000,23.000){2}{\rule{0.400pt}{5.541pt}}
\put(450.17,588){\rule{0.400pt}{13.500pt}}
\multiput(449.17,626.98)(2.000,-38.980){2}{\rule{0.400pt}{6.750pt}}
\put(451.67,588){\rule{0.400pt}{5.541pt}}
\multiput(451.17,588.00)(1.000,11.500){2}{\rule{0.400pt}{2.770pt}}
\put(452.67,568){\rule{0.400pt}{10.359pt}}
\multiput(452.17,589.50)(1.000,-21.500){2}{\rule{0.400pt}{5.179pt}}
\put(453.67,568){\rule{0.400pt}{5.300pt}}
\multiput(453.17,568.00)(1.000,11.000){2}{\rule{0.400pt}{2.650pt}}
\put(454.67,551){\rule{0.400pt}{9.395pt}}
\multiput(454.17,570.50)(1.000,-19.500){2}{\rule{0.400pt}{4.698pt}}
\put(455.67,551){\rule{0.400pt}{8.191pt}}
\multiput(455.17,551.00)(1.000,17.000){2}{\rule{0.400pt}{4.095pt}}
\put(456.67,535){\rule{0.400pt}{12.045pt}}
\multiput(456.17,560.00)(1.000,-25.000){2}{\rule{0.400pt}{6.022pt}}
\put(457.67,535){\rule{0.400pt}{7.227pt}}
\multiput(457.17,535.00)(1.000,15.000){2}{\rule{0.400pt}{3.613pt}}
\put(458.67,531){\rule{0.400pt}{8.191pt}}
\multiput(458.17,548.00)(1.000,-17.000){2}{\rule{0.400pt}{4.095pt}}
\put(459.67,531){\rule{0.400pt}{4.095pt}}
\multiput(459.17,531.00)(1.000,8.500){2}{\rule{0.400pt}{2.048pt}}
\put(460.67,516){\rule{0.400pt}{7.709pt}}
\multiput(460.17,532.00)(1.000,-16.000){2}{\rule{0.400pt}{3.854pt}}
\put(461.67,516){\rule{0.400pt}{6.504pt}}
\multiput(461.17,516.00)(1.000,13.500){2}{\rule{0.400pt}{3.252pt}}
\put(462.67,502){\rule{0.400pt}{9.877pt}}
\multiput(462.17,522.50)(1.000,-20.500){2}{\rule{0.400pt}{4.938pt}}
\put(463.67,502){\rule{0.400pt}{6.023pt}}
\multiput(463.17,502.00)(1.000,12.500){2}{\rule{0.400pt}{3.011pt}}
\put(465.17,498){\rule{0.400pt}{5.900pt}}
\multiput(464.17,514.75)(2.000,-16.754){2}{\rule{0.400pt}{2.950pt}}
\put(466.67,498){\rule{0.400pt}{3.373pt}}
\multiput(466.17,498.00)(1.000,7.000){2}{\rule{0.400pt}{1.686pt}}
\put(467.67,495){\rule{0.400pt}{4.095pt}}
\multiput(467.17,503.50)(1.000,-8.500){2}{\rule{0.400pt}{2.048pt}}
\put(468.67,495){\rule{0.400pt}{3.132pt}}
\multiput(468.17,495.00)(1.000,6.500){2}{\rule{0.400pt}{1.566pt}}
\put(469.67,491){\rule{0.400pt}{4.095pt}}
\multiput(469.17,499.50)(1.000,-8.500){2}{\rule{0.400pt}{2.048pt}}
\put(470.67,491){\rule{0.400pt}{2.891pt}}
\multiput(470.17,491.00)(1.000,6.000){2}{\rule{0.400pt}{1.445pt}}
\put(471.67,487){\rule{0.400pt}{3.854pt}}
\multiput(471.17,495.00)(1.000,-8.000){2}{\rule{0.400pt}{1.927pt}}
\put(472.67,487){\rule{0.400pt}{2.891pt}}
\multiput(472.17,487.00)(1.000,6.000){2}{\rule{0.400pt}{1.445pt}}
\put(473.67,474){\rule{0.400pt}{6.023pt}}
\multiput(473.17,486.50)(1.000,-12.500){2}{\rule{0.400pt}{3.011pt}}
\put(474.67,468){\rule{0.400pt}{1.445pt}}
\multiput(474.17,471.00)(1.000,-3.000){2}{\rule{0.400pt}{0.723pt}}
\put(475.67,468){\rule{0.400pt}{4.336pt}}
\multiput(475.17,468.00)(1.000,9.000){2}{\rule{0.400pt}{2.168pt}}
\put(476.67,480){\rule{0.400pt}{1.445pt}}
\multiput(476.17,483.00)(1.000,-3.000){2}{\rule{0.400pt}{0.723pt}}
\put(477.67,465){\rule{0.400pt}{3.614pt}}
\multiput(477.17,472.50)(1.000,-7.500){2}{\rule{0.400pt}{1.807pt}}
\put(479.17,458){\rule{0.400pt}{1.500pt}}
\multiput(478.17,461.89)(2.000,-3.887){2}{\rule{0.400pt}{0.750pt}}
\put(480.67,458){\rule{0.400pt}{4.577pt}}
\multiput(480.17,458.00)(1.000,9.500){2}{\rule{0.400pt}{2.289pt}}
\put(481.67,469){\rule{0.400pt}{1.927pt}}
\multiput(481.17,473.00)(1.000,-4.000){2}{\rule{0.400pt}{0.964pt}}
\put(482.67,448){\rule{0.400pt}{5.059pt}}
\multiput(482.17,458.50)(1.000,-10.500){2}{\rule{0.400pt}{2.529pt}}
\put(483.67,448){\rule{0.400pt}{2.168pt}}
\multiput(483.17,448.00)(1.000,4.500){2}{\rule{0.400pt}{1.084pt}}
\put(484.67,444){\rule{0.400pt}{3.132pt}}
\multiput(484.17,450.50)(1.000,-6.500){2}{\rule{0.400pt}{1.566pt}}
\put(485.67,444){\rule{0.400pt}{1.927pt}}
\multiput(485.17,444.00)(1.000,4.000){2}{\rule{0.400pt}{0.964pt}}
\put(486.67,439){\rule{0.400pt}{3.132pt}}
\multiput(486.17,445.50)(1.000,-6.500){2}{\rule{0.400pt}{1.566pt}}
\put(487.67,439){\rule{0.400pt}{3.614pt}}
\multiput(487.17,439.00)(1.000,7.500){2}{\rule{0.400pt}{1.807pt}}
\put(488.67,434){\rule{0.400pt}{4.818pt}}
\multiput(488.17,444.00)(1.000,-10.000){2}{\rule{0.400pt}{2.409pt}}
\put(489.67,434){\rule{0.400pt}{3.614pt}}
\multiput(489.17,434.00)(1.000,7.500){2}{\rule{0.400pt}{1.807pt}}
\put(490.67,436){\rule{0.400pt}{3.132pt}}
\multiput(490.17,442.50)(1.000,-6.500){2}{\rule{0.400pt}{1.566pt}}
\put(491.67,436){\rule{0.400pt}{1.927pt}}
\multiput(491.17,436.00)(1.000,4.000){2}{\rule{0.400pt}{0.964pt}}
\put(492.67,431){\rule{0.400pt}{3.132pt}}
\multiput(492.17,437.50)(1.000,-6.500){2}{\rule{0.400pt}{1.566pt}}
\put(494.17,431){\rule{0.400pt}{1.500pt}}
\multiput(493.17,431.00)(2.000,3.887){2}{\rule{0.400pt}{0.750pt}}
\put(495.67,419){\rule{0.400pt}{4.577pt}}
\multiput(495.17,428.50)(1.000,-9.500){2}{\rule{0.400pt}{2.289pt}}
\put(496.67,419){\rule{0.400pt}{3.373pt}}
\multiput(496.17,419.00)(1.000,7.000){2}{\rule{0.400pt}{1.686pt}}
\put(497.67,414){\rule{0.400pt}{4.577pt}}
\multiput(497.17,423.50)(1.000,-9.500){2}{\rule{0.400pt}{2.289pt}}
\put(498.67,414){\rule{0.400pt}{3.132pt}}
\multiput(498.17,414.00)(1.000,6.500){2}{\rule{0.400pt}{1.566pt}}
\put(499.67,415){\rule{0.400pt}{2.891pt}}
\multiput(499.17,421.00)(1.000,-6.000){2}{\rule{0.400pt}{1.445pt}}
\put(500.67,415){\rule{0.400pt}{1.445pt}}
\multiput(500.17,415.00)(1.000,3.000){2}{\rule{0.400pt}{0.723pt}}
\put(501.67,410){\rule{0.400pt}{2.650pt}}
\multiput(501.17,415.50)(1.000,-5.500){2}{\rule{0.400pt}{1.325pt}}
\put(152.0,131.0){\rule[-0.200pt]{70.825pt}{0.400pt}}
\put(503.67,410){\rule{0.400pt}{3.132pt}}
\multiput(503.17,410.00)(1.000,6.500){2}{\rule{0.400pt}{1.566pt}}
\put(504.67,417){\rule{0.400pt}{1.445pt}}
\multiput(504.17,420.00)(1.000,-3.000){2}{\rule{0.400pt}{0.723pt}}
\put(505.67,405){\rule{0.400pt}{2.891pt}}
\multiput(505.17,411.00)(1.000,-6.000){2}{\rule{0.400pt}{1.445pt}}
\put(506.67,405){\rule{0.400pt}{2.891pt}}
\multiput(506.17,405.00)(1.000,6.000){2}{\rule{0.400pt}{1.445pt}}
\put(507.67,399){\rule{0.400pt}{4.336pt}}
\multiput(507.17,408.00)(1.000,-9.000){2}{\rule{0.400pt}{2.168pt}}
\put(509.17,399){\rule{0.400pt}{2.500pt}}
\multiput(508.17,399.00)(2.000,6.811){2}{\rule{0.400pt}{1.250pt}}
\put(510.67,393){\rule{0.400pt}{4.336pt}}
\multiput(510.17,402.00)(1.000,-9.000){2}{\rule{0.400pt}{2.168pt}}
\put(511.67,393){\rule{0.400pt}{4.336pt}}
\multiput(511.17,393.00)(1.000,9.000){2}{\rule{0.400pt}{2.168pt}}
\put(512.67,399){\rule{0.400pt}{2.891pt}}
\multiput(512.17,405.00)(1.000,-6.000){2}{\rule{0.400pt}{1.445pt}}
\put(513.67,393){\rule{0.400pt}{1.445pt}}
\multiput(513.17,396.00)(1.000,-3.000){2}{\rule{0.400pt}{0.723pt}}
\put(514.67,393){\rule{0.400pt}{4.336pt}}
\multiput(514.17,393.00)(1.000,9.000){2}{\rule{0.400pt}{2.168pt}}
\put(515.67,405){\rule{0.400pt}{1.445pt}}
\multiput(515.17,408.00)(1.000,-3.000){2}{\rule{0.400pt}{0.723pt}}
\put(516.67,405){\rule{0.400pt}{6.023pt}}
\multiput(516.17,405.00)(1.000,12.500){2}{\rule{0.400pt}{3.011pt}}
\put(503.0,410.0){\usebox{\plotpoint}}
\put(518.67,430){\rule{0.400pt}{8.672pt}}
\multiput(518.17,430.00)(1.000,18.000){2}{\rule{0.400pt}{4.336pt}}
\put(519.67,459){\rule{0.400pt}{1.686pt}}
\multiput(519.17,462.50)(1.000,-3.500){2}{\rule{0.400pt}{0.843pt}}
\put(520.67,459){\rule{0.400pt}{5.541pt}}
\multiput(520.17,459.00)(1.000,11.500){2}{\rule{0.400pt}{2.770pt}}
\put(521.67,473){\rule{0.400pt}{2.168pt}}
\multiput(521.17,477.50)(1.000,-4.500){2}{\rule{0.400pt}{1.084pt}}
\put(523.17,439){\rule{0.400pt}{6.900pt}}
\multiput(522.17,458.68)(2.000,-19.679){2}{\rule{0.400pt}{3.450pt}}
\put(524.67,439){\rule{0.400pt}{4.818pt}}
\multiput(524.17,439.00)(1.000,10.000){2}{\rule{0.400pt}{2.409pt}}
\put(525.67,459){\rule{0.400pt}{9.877pt}}
\multiput(525.17,459.00)(1.000,20.500){2}{\rule{0.400pt}{4.938pt}}
\put(527,499.67){\rule{0.241pt}{0.400pt}}
\multiput(527.00,499.17)(0.500,1.000){2}{\rule{0.120pt}{0.400pt}}
\put(527.67,433){\rule{0.400pt}{16.381pt}}
\multiput(527.17,467.00)(1.000,-34.000){2}{\rule{0.400pt}{8.191pt}}
\put(528.67,419){\rule{0.400pt}{3.373pt}}
\multiput(528.17,426.00)(1.000,-7.000){2}{\rule{0.400pt}{1.686pt}}
\put(529.67,419){\rule{0.400pt}{12.045pt}}
\multiput(529.17,419.00)(1.000,25.000){2}{\rule{0.400pt}{6.022pt}}
\put(530.67,461){\rule{0.400pt}{1.927pt}}
\multiput(530.17,465.00)(1.000,-4.000){2}{\rule{0.400pt}{0.964pt}}
\put(531.67,461){\rule{0.400pt}{12.045pt}}
\multiput(531.17,461.00)(1.000,25.000){2}{\rule{0.400pt}{6.022pt}}
\put(532.67,511){\rule{0.400pt}{2.409pt}}
\multiput(532.17,511.00)(1.000,5.000){2}{\rule{0.400pt}{1.204pt}}
\put(533.67,503){\rule{0.400pt}{4.336pt}}
\multiput(533.17,512.00)(1.000,-9.000){2}{\rule{0.400pt}{2.168pt}}
\put(534.67,503){\rule{0.400pt}{6.263pt}}
\multiput(534.17,503.00)(1.000,13.000){2}{\rule{0.400pt}{3.132pt}}
\put(535.67,479){\rule{0.400pt}{12.045pt}}
\multiput(535.17,504.00)(1.000,-25.000){2}{\rule{0.400pt}{6.022pt}}
\put(536.67,479){\rule{0.400pt}{1.927pt}}
\multiput(536.17,479.00)(1.000,4.000){2}{\rule{0.400pt}{0.964pt}}
\put(538.17,456){\rule{0.400pt}{6.300pt}}
\multiput(537.17,473.92)(2.000,-17.924){2}{\rule{0.400pt}{3.150pt}}
\put(539.67,456){\rule{0.400pt}{3.854pt}}
\multiput(539.17,456.00)(1.000,8.000){2}{\rule{0.400pt}{1.927pt}}
\put(540.67,416){\rule{0.400pt}{13.490pt}}
\multiput(540.17,444.00)(1.000,-28.000){2}{\rule{0.400pt}{6.745pt}}
\put(541.67,416){\rule{0.400pt}{1.204pt}}
\multiput(541.17,416.00)(1.000,2.500){2}{\rule{0.400pt}{0.602pt}}
\put(542.67,392){\rule{0.400pt}{6.986pt}}
\multiput(542.17,406.50)(1.000,-14.500){2}{\rule{0.400pt}{3.493pt}}
\put(543.67,392){\rule{0.400pt}{5.541pt}}
\multiput(543.17,392.00)(1.000,11.500){2}{\rule{0.400pt}{2.770pt}}
\put(544.67,415){\rule{0.400pt}{11.322pt}}
\multiput(544.17,415.00)(1.000,23.500){2}{\rule{0.400pt}{5.661pt}}
\put(545.67,462){\rule{0.400pt}{1.927pt}}
\multiput(545.17,462.00)(1.000,4.000){2}{\rule{0.400pt}{0.964pt}}
\put(546.67,448){\rule{0.400pt}{5.300pt}}
\multiput(546.17,459.00)(1.000,-11.000){2}{\rule{0.400pt}{2.650pt}}
\put(547.67,448){\rule{0.400pt}{4.818pt}}
\multiput(547.17,448.00)(1.000,10.000){2}{\rule{0.400pt}{2.409pt}}
\put(548.67,427){\rule{0.400pt}{9.877pt}}
\multiput(548.17,447.50)(1.000,-20.500){2}{\rule{0.400pt}{4.938pt}}
\put(549.67,427){\rule{0.400pt}{5.059pt}}
\multiput(549.17,427.00)(1.000,10.500){2}{\rule{0.400pt}{2.529pt}}
\put(550.67,448){\rule{0.400pt}{5.059pt}}
\multiput(550.17,448.00)(1.000,10.500){2}{\rule{0.400pt}{2.529pt}}
\put(552.17,461){\rule{0.400pt}{1.700pt}}
\multiput(551.17,465.47)(2.000,-4.472){2}{\rule{0.400pt}{0.850pt}}
\put(553.67,461){\rule{0.400pt}{11.804pt}}
\multiput(553.17,461.00)(1.000,24.500){2}{\rule{0.400pt}{5.902pt}}
\put(555,509.67){\rule{0.241pt}{0.400pt}}
\multiput(555.00,509.17)(0.500,1.000){2}{\rule{0.120pt}{0.400pt}}
\put(555.67,485){\rule{0.400pt}{6.263pt}}
\multiput(555.17,498.00)(1.000,-13.000){2}{\rule{0.400pt}{3.132pt}}
\put(556.67,485){\rule{0.400pt}{3.854pt}}
\multiput(556.17,485.00)(1.000,8.000){2}{\rule{0.400pt}{1.927pt}}
\put(557.67,477){\rule{0.400pt}{5.782pt}}
\multiput(557.17,489.00)(1.000,-12.000){2}{\rule{0.400pt}{2.891pt}}
\put(558.67,477){\rule{0.400pt}{8.191pt}}
\multiput(558.17,477.00)(1.000,17.000){2}{\rule{0.400pt}{4.095pt}}
\put(559.67,485){\rule{0.400pt}{6.263pt}}
\multiput(559.17,498.00)(1.000,-13.000){2}{\rule{0.400pt}{3.132pt}}
\put(560.67,485){\rule{0.400pt}{2.168pt}}
\multiput(560.17,485.00)(1.000,4.500){2}{\rule{0.400pt}{1.084pt}}
\put(561.67,494){\rule{0.400pt}{6.263pt}}
\multiput(561.17,494.00)(1.000,13.000){2}{\rule{0.400pt}{3.132pt}}
\put(518.0,430.0){\usebox{\plotpoint}}
\put(563.67,494){\rule{0.400pt}{6.263pt}}
\multiput(563.17,507.00)(1.000,-13.000){2}{\rule{0.400pt}{3.132pt}}
\put(564.67,494){\rule{0.400pt}{2.168pt}}
\multiput(564.17,494.00)(1.000,4.500){2}{\rule{0.400pt}{1.084pt}}
\put(565.67,503){\rule{0.400pt}{11.322pt}}
\multiput(565.17,503.00)(1.000,23.500){2}{\rule{0.400pt}{5.661pt}}
\put(567.17,530){\rule{0.400pt}{4.100pt}}
\multiput(566.17,541.49)(2.000,-11.490){2}{\rule{0.400pt}{2.050pt}}
\put(568.67,530){\rule{0.400pt}{7.709pt}}
\multiput(568.17,530.00)(1.000,16.000){2}{\rule{0.400pt}{3.854pt}}
\put(563.0,520.0){\usebox{\plotpoint}}
\put(570.67,532){\rule{0.400pt}{7.227pt}}
\multiput(570.17,547.00)(1.000,-15.000){2}{\rule{0.400pt}{3.613pt}}
\put(571.67,532){\rule{0.400pt}{2.409pt}}
\multiput(571.17,532.00)(1.000,5.000){2}{\rule{0.400pt}{1.204pt}}
\put(572.67,542){\rule{0.400pt}{16.622pt}}
\multiput(572.17,542.00)(1.000,34.500){2}{\rule{0.400pt}{8.311pt}}
\put(573.67,587){\rule{0.400pt}{5.782pt}}
\multiput(573.17,599.00)(1.000,-12.000){2}{\rule{0.400pt}{2.891pt}}
\put(574.67,587){\rule{0.400pt}{9.395pt}}
\multiput(574.17,587.00)(1.000,19.500){2}{\rule{0.400pt}{4.698pt}}
\put(575.67,602){\rule{0.400pt}{5.782pt}}
\multiput(575.17,614.00)(1.000,-12.000){2}{\rule{0.400pt}{2.891pt}}
\put(576.67,602){\rule{0.400pt}{13.009pt}}
\multiput(576.17,602.00)(1.000,27.000){2}{\rule{0.400pt}{6.504pt}}
\put(578,655.67){\rule{0.241pt}{0.400pt}}
\multiput(578.00,655.17)(0.500,1.000){2}{\rule{0.120pt}{0.400pt}}
\put(578.67,630){\rule{0.400pt}{6.504pt}}
\multiput(578.17,643.50)(1.000,-13.500){2}{\rule{0.400pt}{3.252pt}}
\put(579.67,630){\rule{0.400pt}{10.840pt}}
\multiput(579.17,630.00)(1.000,22.500){2}{\rule{0.400pt}{5.420pt}}
\put(580.67,632){\rule{0.400pt}{10.359pt}}
\multiput(580.17,653.50)(1.000,-21.500){2}{\rule{0.400pt}{5.179pt}}
\put(582,631.67){\rule{0.482pt}{0.400pt}}
\multiput(582.00,631.17)(1.000,1.000){2}{\rule{0.241pt}{0.400pt}}
\put(583.67,633){\rule{0.400pt}{6.986pt}}
\multiput(583.17,633.00)(1.000,14.500){2}{\rule{0.400pt}{3.493pt}}
\put(584.67,649){\rule{0.400pt}{3.132pt}}
\multiput(584.17,655.50)(1.000,-6.500){2}{\rule{0.400pt}{1.566pt}}
\put(585.67,649){\rule{0.400pt}{11.322pt}}
\multiput(585.17,649.00)(1.000,23.500){2}{\rule{0.400pt}{5.661pt}}
\put(586.67,650){\rule{0.400pt}{11.081pt}}
\multiput(586.17,673.00)(1.000,-23.000){2}{\rule{0.400pt}{5.541pt}}
\put(587.67,650){\rule{0.400pt}{15.899pt}}
\multiput(587.17,650.00)(1.000,33.000){2}{\rule{0.400pt}{7.950pt}}
\put(588.67,700){\rule{0.400pt}{3.854pt}}
\multiput(588.17,708.00)(1.000,-8.000){2}{\rule{0.400pt}{1.927pt}}
\put(589.67,654){\rule{0.400pt}{11.081pt}}
\multiput(589.17,677.00)(1.000,-23.000){2}{\rule{0.400pt}{5.541pt}}
\put(590.67,654){\rule{0.400pt}{7.709pt}}
\multiput(590.17,654.00)(1.000,16.000){2}{\rule{0.400pt}{3.854pt}}
\put(591.67,686){\rule{0.400pt}{8.432pt}}
\multiput(591.17,686.00)(1.000,17.500){2}{\rule{0.400pt}{4.216pt}}
\put(592.67,689){\rule{0.400pt}{7.709pt}}
\multiput(592.17,705.00)(1.000,-16.000){2}{\rule{0.400pt}{3.854pt}}
\put(593.67,689){\rule{0.400pt}{8.191pt}}
\multiput(593.17,689.00)(1.000,17.000){2}{\rule{0.400pt}{4.095pt}}
\put(594.67,691){\rule{0.400pt}{7.709pt}}
\multiput(594.17,707.00)(1.000,-16.000){2}{\rule{0.400pt}{3.854pt}}
\put(596.17,691){\rule{0.400pt}{6.900pt}}
\multiput(595.17,691.00)(2.000,19.679){2}{\rule{0.400pt}{3.450pt}}
\put(597.67,692){\rule{0.400pt}{7.950pt}}
\multiput(597.17,708.50)(1.000,-16.500){2}{\rule{0.400pt}{3.975pt}}
\put(598.67,692){\rule{0.400pt}{8.672pt}}
\multiput(598.17,692.00)(1.000,18.000){2}{\rule{0.400pt}{4.336pt}}
\put(599.67,711){\rule{0.400pt}{4.095pt}}
\multiput(599.17,719.50)(1.000,-8.500){2}{\rule{0.400pt}{2.048pt}}
\put(600.67,665){\rule{0.400pt}{11.081pt}}
\multiput(600.17,688.00)(1.000,-23.000){2}{\rule{0.400pt}{5.541pt}}
\put(601.67,665){\rule{0.400pt}{0.482pt}}
\multiput(601.17,665.00)(1.000,1.000){2}{\rule{0.400pt}{0.241pt}}
\put(602.67,667){\rule{0.400pt}{15.658pt}}
\multiput(602.17,667.00)(1.000,32.500){2}{\rule{0.400pt}{7.829pt}}
\put(603.67,668){\rule{0.400pt}{15.418pt}}
\multiput(603.17,700.00)(1.000,-32.000){2}{\rule{0.400pt}{7.709pt}}
\put(604.67,668){\rule{0.400pt}{11.804pt}}
\multiput(604.17,668.00)(1.000,24.500){2}{\rule{0.400pt}{5.902pt}}
\put(605.67,670){\rule{0.400pt}{11.322pt}}
\multiput(605.17,693.50)(1.000,-23.500){2}{\rule{0.400pt}{5.661pt}}
\put(606.67,670){\rule{0.400pt}{11.804pt}}
\multiput(606.17,670.00)(1.000,24.500){2}{\rule{0.400pt}{5.902pt}}
\put(570.0,562.0){\usebox{\plotpoint}}
\put(608.67,673){\rule{0.400pt}{11.081pt}}
\multiput(608.17,696.00)(1.000,-23.000){2}{\rule{0.400pt}{5.541pt}}
\put(609.67,660){\rule{0.400pt}{3.132pt}}
\multiput(609.17,666.50)(1.000,-6.500){2}{\rule{0.400pt}{1.566pt}}
\put(611.17,660){\rule{0.400pt}{12.700pt}}
\multiput(610.17,660.00)(2.000,36.641){2}{\rule{0.400pt}{6.350pt}}
\put(612.67,723){\rule{0.400pt}{4.818pt}}
\multiput(612.17,723.00)(1.000,10.000){2}{\rule{0.400pt}{2.409pt}}
\put(613.67,693){\rule{0.400pt}{12.045pt}}
\multiput(613.17,718.00)(1.000,-25.000){2}{\rule{0.400pt}{6.022pt}}
\put(614.67,678){\rule{0.400pt}{3.614pt}}
\multiput(614.17,685.50)(1.000,-7.500){2}{\rule{0.400pt}{1.807pt}}
\put(615.67,678){\rule{0.400pt}{12.045pt}}
\multiput(615.17,678.00)(1.000,25.000){2}{\rule{0.400pt}{6.022pt}}
\put(617,727.67){\rule{0.241pt}{0.400pt}}
\multiput(617.00,727.17)(0.500,1.000){2}{\rule{0.120pt}{0.400pt}}
\put(617.67,682){\rule{0.400pt}{11.322pt}}
\multiput(617.17,705.50)(1.000,-23.500){2}{\rule{0.400pt}{5.661pt}}
\put(618.67,682){\rule{0.400pt}{12.045pt}}
\multiput(618.17,682.00)(1.000,25.000){2}{\rule{0.400pt}{6.022pt}}
\put(619.67,685){\rule{0.400pt}{11.322pt}}
\multiput(619.17,708.50)(1.000,-23.500){2}{\rule{0.400pt}{5.661pt}}
\put(620.67,685){\rule{0.400pt}{12.045pt}}
\multiput(620.17,685.00)(1.000,25.000){2}{\rule{0.400pt}{6.022pt}}
\put(621.67,672){\rule{0.400pt}{15.177pt}}
\multiput(621.17,703.50)(1.000,-31.500){2}{\rule{0.400pt}{7.588pt}}
\put(622.67,672){\rule{0.400pt}{7.709pt}}
\multiput(622.17,672.00)(1.000,16.000){2}{\rule{0.400pt}{3.854pt}}
\put(623.67,704){\rule{0.400pt}{12.527pt}}
\multiput(623.17,704.00)(1.000,26.000){2}{\rule{0.400pt}{6.263pt}}
\put(625,755.67){\rule{0.241pt}{0.400pt}}
\multiput(625.00,755.17)(0.500,1.000){2}{\rule{0.120pt}{0.400pt}}
\put(626.17,741){\rule{0.400pt}{3.300pt}}
\multiput(625.17,750.15)(2.000,-9.151){2}{\rule{0.400pt}{1.650pt}}
\put(627.67,741){\rule{0.400pt}{4.818pt}}
\multiput(627.17,741.00)(1.000,10.000){2}{\rule{0.400pt}{2.409pt}}
\put(628.67,679){\rule{0.400pt}{19.754pt}}
\multiput(628.17,720.00)(1.000,-41.000){2}{\rule{0.400pt}{9.877pt}}
\put(629.67,679){\rule{0.400pt}{7.709pt}}
\multiput(629.17,679.00)(1.000,16.000){2}{\rule{0.400pt}{3.854pt}}
\put(630.67,667){\rule{0.400pt}{10.600pt}}
\multiput(630.17,689.00)(1.000,-22.000){2}{\rule{0.400pt}{5.300pt}}
\put(631.67,667){\rule{0.400pt}{19.272pt}}
\multiput(631.17,667.00)(1.000,40.000){2}{\rule{0.400pt}{9.636pt}}
\put(632.67,684){\rule{0.400pt}{15.177pt}}
\multiput(632.17,715.50)(1.000,-31.500){2}{\rule{0.400pt}{7.588pt}}
\put(633.67,684){\rule{0.400pt}{19.754pt}}
\multiput(633.17,684.00)(1.000,41.000){2}{\rule{0.400pt}{9.877pt}}
\put(634.67,686){\rule{0.400pt}{19.272pt}}
\multiput(634.17,726.00)(1.000,-40.000){2}{\rule{0.400pt}{9.636pt}}
\put(635.67,686){\rule{0.400pt}{7.950pt}}
\multiput(635.17,686.00)(1.000,16.500){2}{\rule{0.400pt}{3.975pt}}
\put(636.67,674){\rule{0.400pt}{10.840pt}}
\multiput(636.17,696.50)(1.000,-22.500){2}{\rule{0.400pt}{5.420pt}}
\put(637.67,674){\rule{0.400pt}{19.513pt}}
\multiput(637.17,674.00)(1.000,40.500){2}{\rule{0.400pt}{9.756pt}}
\put(638.67,663){\rule{0.400pt}{22.163pt}}
\multiput(638.17,709.00)(1.000,-46.000){2}{\rule{0.400pt}{11.081pt}}
\put(640,663.17){\rule{0.482pt}{0.400pt}}
\multiput(640.00,662.17)(1.000,2.000){2}{\rule{0.241pt}{0.400pt}}
\put(641.67,665){\rule{0.400pt}{22.163pt}}
\multiput(641.17,665.00)(1.000,46.000){2}{\rule{0.400pt}{11.081pt}}
\put(642.67,757){\rule{0.400pt}{5.059pt}}
\multiput(642.17,757.00)(1.000,10.500){2}{\rule{0.400pt}{2.529pt}}
\put(643.67,696){\rule{0.400pt}{19.754pt}}
\multiput(643.17,737.00)(1.000,-41.000){2}{\rule{0.400pt}{9.877pt}}
\put(644.67,696){\rule{0.400pt}{3.854pt}}
\multiput(644.17,696.00)(1.000,8.000){2}{\rule{0.400pt}{1.927pt}}
\put(645.67,669){\rule{0.400pt}{10.359pt}}
\multiput(645.17,690.50)(1.000,-21.500){2}{\rule{0.400pt}{5.179pt}}
\put(646.67,669){\rule{0.400pt}{7.227pt}}
\multiput(646.17,669.00)(1.000,15.000){2}{\rule{0.400pt}{3.613pt}}
\put(647.67,699){\rule{0.400pt}{11.563pt}}
\multiput(647.17,699.00)(1.000,24.000){2}{\rule{0.400pt}{5.782pt}}
\put(648.67,701){\rule{0.400pt}{11.081pt}}
\multiput(648.17,724.00)(1.000,-23.000){2}{\rule{0.400pt}{5.541pt}}
\put(649.67,701){\rule{0.400pt}{16.140pt}}
\multiput(649.17,701.00)(1.000,33.500){2}{\rule{0.400pt}{8.070pt}}
\put(650.67,734){\rule{0.400pt}{8.191pt}}
\multiput(650.17,751.00)(1.000,-17.000){2}{\rule{0.400pt}{4.095pt}}
\put(651.67,691){\rule{0.400pt}{10.359pt}}
\multiput(651.17,712.50)(1.000,-21.500){2}{\rule{0.400pt}{5.179pt}}
\put(652.67,691){\rule{0.400pt}{19.754pt}}
\multiput(652.17,691.00)(1.000,41.000){2}{\rule{0.400pt}{9.877pt}}
\put(653.67,693){\rule{0.400pt}{19.272pt}}
\multiput(653.17,733.00)(1.000,-40.000){2}{\rule{0.400pt}{9.636pt}}
\put(655.17,693){\rule{0.400pt}{16.500pt}}
\multiput(654.17,693.00)(2.000,47.753){2}{\rule{0.400pt}{8.250pt}}
\put(656.67,710){\rule{0.400pt}{15.658pt}}
\multiput(656.17,742.50)(1.000,-32.500){2}{\rule{0.400pt}{7.829pt}}
\put(657.67,710){\rule{0.400pt}{7.950pt}}
\multiput(657.17,710.00)(1.000,16.500){2}{\rule{0.400pt}{3.975pt}}
\put(658.67,697){\rule{0.400pt}{11.081pt}}
\multiput(658.17,720.00)(1.000,-23.000){2}{\rule{0.400pt}{5.541pt}}
\put(659.67,697){\rule{0.400pt}{11.563pt}}
\multiput(659.17,697.00)(1.000,24.000){2}{\rule{0.400pt}{5.782pt}}
\put(660.67,700){\rule{0.400pt}{10.840pt}}
\multiput(660.17,722.50)(1.000,-22.500){2}{\rule{0.400pt}{5.420pt}}
\put(661.67,700){\rule{0.400pt}{11.081pt}}
\multiput(661.17,700.00)(1.000,23.000){2}{\rule{0.400pt}{5.541pt}}
\put(662.67,702){\rule{0.400pt}{10.600pt}}
\multiput(662.17,724.00)(1.000,-22.000){2}{\rule{0.400pt}{5.300pt}}
\put(663.67,702){\rule{0.400pt}{11.804pt}}
\multiput(663.17,702.00)(1.000,24.500){2}{\rule{0.400pt}{5.902pt}}
\put(664.67,720){\rule{0.400pt}{7.468pt}}
\multiput(664.17,735.50)(1.000,-15.500){2}{\rule{0.400pt}{3.734pt}}
\put(665.67,720){\rule{0.400pt}{26.258pt}}
\multiput(665.17,720.00)(1.000,54.500){2}{\rule{0.400pt}{13.129pt}}
\put(666.67,707){\rule{0.400pt}{29.390pt}}
\multiput(666.17,768.00)(1.000,-61.000){2}{\rule{0.400pt}{14.695pt}}
\put(667.67,707){\rule{0.400pt}{7.950pt}}
\multiput(667.17,707.00)(1.000,16.500){2}{\rule{0.400pt}{3.975pt}}
\put(668.67,656){\rule{0.400pt}{20.236pt}}
\multiput(668.17,698.00)(1.000,-42.000){2}{\rule{0.400pt}{10.118pt}}
\put(670.17,656){\rule{0.400pt}{10.900pt}}
\multiput(669.17,656.00)(2.000,31.377){2}{\rule{0.400pt}{5.450pt}}
\put(671.67,710){\rule{0.400pt}{11.563pt}}
\multiput(671.17,710.00)(1.000,24.000){2}{\rule{0.400pt}{5.782pt}}
\put(672.67,727){\rule{0.400pt}{7.468pt}}
\multiput(672.17,742.50)(1.000,-15.500){2}{\rule{0.400pt}{3.734pt}}
\put(673.67,727){\rule{0.400pt}{16.863pt}}
\multiput(673.17,727.00)(1.000,35.000){2}{\rule{0.400pt}{8.431pt}}
\put(674.67,781){\rule{0.400pt}{3.854pt}}
\multiput(674.17,789.00)(1.000,-8.000){2}{\rule{0.400pt}{1.927pt}}
\put(675.67,664){\rule{0.400pt}{28.185pt}}
\multiput(675.17,722.50)(1.000,-58.500){2}{\rule{0.400pt}{14.093pt}}
\put(608.0,719.0){\usebox{\plotpoint}}
\put(677.67,664){\rule{0.400pt}{17.104pt}}
\multiput(677.17,664.00)(1.000,35.500){2}{\rule{0.400pt}{8.552pt}}
\put(678.67,721){\rule{0.400pt}{3.373pt}}
\multiput(678.17,728.00)(1.000,-7.000){2}{\rule{0.400pt}{1.686pt}}
\put(679.67,721){\rule{0.400pt}{7.709pt}}
\multiput(679.17,721.00)(1.000,16.000){2}{\rule{0.400pt}{3.854pt}}
\put(680.67,753){\rule{0.400pt}{4.095pt}}
\multiput(680.17,753.00)(1.000,8.500){2}{\rule{0.400pt}{2.048pt}}
\put(681.67,725){\rule{0.400pt}{10.840pt}}
\multiput(681.17,747.50)(1.000,-22.500){2}{\rule{0.400pt}{5.420pt}}
\put(682.67,725){\rule{0.400pt}{12.045pt}}
\multiput(682.17,725.00)(1.000,25.000){2}{\rule{0.400pt}{6.022pt}}
\put(684.17,673){\rule{0.400pt}{20.500pt}}
\multiput(683.17,732.45)(2.000,-59.451){2}{\rule{0.400pt}{10.250pt}}
\put(686,672.67){\rule{0.241pt}{0.400pt}}
\multiput(686.00,672.17)(0.500,1.000){2}{\rule{0.120pt}{0.400pt}}
\put(686.67,674){\rule{0.400pt}{13.490pt}}
\multiput(686.17,674.00)(1.000,28.000){2}{\rule{0.400pt}{6.745pt}}
\put(687.67,690){\rule{0.400pt}{9.636pt}}
\multiput(687.17,710.00)(1.000,-20.000){2}{\rule{0.400pt}{4.818pt}}
\put(688.67,690){\rule{0.400pt}{13.972pt}}
\multiput(688.17,690.00)(1.000,29.000){2}{\rule{0.400pt}{6.986pt}}
\put(689.67,691){\rule{0.400pt}{13.731pt}}
\multiput(689.17,719.50)(1.000,-28.500){2}{\rule{0.400pt}{6.866pt}}
\put(690.67,691){\rule{0.400pt}{14.213pt}}
\multiput(690.17,691.00)(1.000,29.500){2}{\rule{0.400pt}{7.107pt}}
\put(691.67,750){\rule{0.400pt}{0.482pt}}
\multiput(691.17,750.00)(1.000,1.000){2}{\rule{0.400pt}{0.241pt}}
\put(692.67,709){\rule{0.400pt}{10.359pt}}
\multiput(692.17,730.50)(1.000,-21.500){2}{\rule{0.400pt}{5.179pt}}
\put(693.67,696){\rule{0.400pt}{3.132pt}}
\multiput(693.17,702.50)(1.000,-6.500){2}{\rule{0.400pt}{1.566pt}}
\put(694.67,696){\rule{0.400pt}{10.600pt}}
\multiput(694.17,696.00)(1.000,22.000){2}{\rule{0.400pt}{5.300pt}}
\put(695.67,687){\rule{0.400pt}{12.768pt}}
\multiput(695.17,713.50)(1.000,-26.500){2}{\rule{0.400pt}{6.384pt}}
\put(696.67,687){\rule{0.400pt}{20.958pt}}
\multiput(696.17,687.00)(1.000,43.500){2}{\rule{0.400pt}{10.479pt}}
\put(697.67,716){\rule{0.400pt}{13.972pt}}
\multiput(697.17,745.00)(1.000,-29.000){2}{\rule{0.400pt}{6.986pt}}
\put(699.17,665){\rule{0.400pt}{10.300pt}}
\multiput(698.17,694.62)(2.000,-29.622){2}{\rule{0.400pt}{5.150pt}}
\put(700.67,665){\rule{0.400pt}{3.132pt}}
\multiput(700.17,665.00)(1.000,6.500){2}{\rule{0.400pt}{1.566pt}}
\put(701.67,678){\rule{0.400pt}{9.636pt}}
\multiput(701.17,678.00)(1.000,20.000){2}{\rule{0.400pt}{4.818pt}}
\put(702.67,707){\rule{0.400pt}{2.650pt}}
\multiput(702.17,712.50)(1.000,-5.500){2}{\rule{0.400pt}{1.325pt}}
\put(703.67,707){\rule{0.400pt}{10.359pt}}
\multiput(703.17,707.00)(1.000,21.500){2}{\rule{0.400pt}{5.179pt}}
\put(704.67,750){\rule{0.400pt}{0.482pt}}
\multiput(704.17,750.00)(1.000,1.000){2}{\rule{0.400pt}{0.241pt}}
\put(705.67,684){\rule{0.400pt}{16.381pt}}
\multiput(705.17,718.00)(1.000,-34.000){2}{\rule{0.400pt}{8.191pt}}
\put(706.67,684){\rule{0.400pt}{0.482pt}}
\multiput(706.17,684.00)(1.000,1.000){2}{\rule{0.400pt}{0.241pt}}
\put(707.67,686){\rule{0.400pt}{6.263pt}}
\multiput(707.17,686.00)(1.000,13.000){2}{\rule{0.400pt}{3.132pt}}
\put(708.67,687){\rule{0.400pt}{6.023pt}}
\multiput(708.17,699.50)(1.000,-12.500){2}{\rule{0.400pt}{3.011pt}}
\put(709.67,687){\rule{0.400pt}{9.877pt}}
\multiput(709.17,687.00)(1.000,20.500){2}{\rule{0.400pt}{4.938pt}}
\put(710.67,716){\rule{0.400pt}{2.891pt}}
\multiput(710.17,722.00)(1.000,-6.000){2}{\rule{0.400pt}{1.445pt}}
\put(711.67,691){\rule{0.400pt}{6.023pt}}
\multiput(711.17,703.50)(1.000,-12.500){2}{\rule{0.400pt}{3.011pt}}
\put(713,690.67){\rule{0.482pt}{0.400pt}}
\multiput(713.00,690.17)(1.000,1.000){2}{\rule{0.241pt}{0.400pt}}
\put(714.67,692){\rule{0.400pt}{6.504pt}}
\multiput(714.17,692.00)(1.000,13.500){2}{\rule{0.400pt}{3.252pt}}
\put(715.67,682){\rule{0.400pt}{8.913pt}}
\multiput(715.17,700.50)(1.000,-18.500){2}{\rule{0.400pt}{4.457pt}}
\put(716.67,682){\rule{0.400pt}{13.009pt}}
\multiput(716.17,682.00)(1.000,27.000){2}{\rule{0.400pt}{6.504pt}}
\put(717.67,723){\rule{0.400pt}{3.132pt}}
\multiput(717.17,729.50)(1.000,-6.500){2}{\rule{0.400pt}{1.566pt}}
\put(718.67,686){\rule{0.400pt}{8.913pt}}
\multiput(718.17,704.50)(1.000,-18.500){2}{\rule{0.400pt}{4.457pt}}
\put(719.67,686){\rule{0.400pt}{6.263pt}}
\multiput(719.17,686.00)(1.000,13.000){2}{\rule{0.400pt}{3.132pt}}
\put(720.67,677){\rule{0.400pt}{8.432pt}}
\multiput(720.17,694.50)(1.000,-17.500){2}{\rule{0.400pt}{4.216pt}}
\put(677.0,664.0){\usebox{\plotpoint}}
\put(722.67,677){\rule{0.400pt}{6.263pt}}
\multiput(722.17,677.00)(1.000,13.000){2}{\rule{0.400pt}{3.132pt}}
\put(724,702.67){\rule{0.241pt}{0.400pt}}
\multiput(724.00,702.17)(0.500,1.000){2}{\rule{0.120pt}{0.400pt}}
\put(724.67,681){\rule{0.400pt}{5.541pt}}
\multiput(724.17,692.50)(1.000,-11.500){2}{\rule{0.400pt}{2.770pt}}
\put(726,680.67){\rule{0.241pt}{0.400pt}}
\multiput(726.00,680.17)(0.500,1.000){2}{\rule{0.120pt}{0.400pt}}
\put(726.67,682){\rule{0.400pt}{9.154pt}}
\multiput(726.17,682.00)(1.000,19.000){2}{\rule{0.400pt}{4.577pt}}
\put(728.17,709){\rule{0.400pt}{2.300pt}}
\multiput(727.17,715.23)(2.000,-6.226){2}{\rule{0.400pt}{1.150pt}}
\put(729.67,685){\rule{0.400pt}{5.782pt}}
\multiput(729.17,697.00)(1.000,-12.000){2}{\rule{0.400pt}{2.891pt}}
\put(731,684.67){\rule{0.241pt}{0.400pt}}
\multiput(731.00,684.17)(0.500,1.000){2}{\rule{0.120pt}{0.400pt}}
\put(731.67,686){\rule{0.400pt}{6.023pt}}
\multiput(731.17,686.00)(1.000,12.500){2}{\rule{0.400pt}{3.011pt}}
\put(732.67,689){\rule{0.400pt}{5.300pt}}
\multiput(732.17,700.00)(1.000,-11.000){2}{\rule{0.400pt}{2.650pt}}
\put(733.67,689){\rule{0.400pt}{6.023pt}}
\multiput(733.17,689.00)(1.000,12.500){2}{\rule{0.400pt}{3.011pt}}
\put(734.67,691){\rule{0.400pt}{5.541pt}}
\multiput(734.17,702.50)(1.000,-11.500){2}{\rule{0.400pt}{2.770pt}}
\put(735.67,691){\rule{0.400pt}{6.023pt}}
\multiput(735.17,691.00)(1.000,12.500){2}{\rule{0.400pt}{3.011pt}}
\put(737,715.67){\rule{0.241pt}{0.400pt}}
\multiput(737.00,715.17)(0.500,1.000){2}{\rule{0.120pt}{0.400pt}}
\put(737.67,671){\rule{0.400pt}{11.081pt}}
\multiput(737.17,694.00)(1.000,-23.000){2}{\rule{0.400pt}{5.541pt}}
\put(722.0,677.0){\usebox{\plotpoint}}
\put(739.67,671){\rule{0.400pt}{8.672pt}}
\multiput(739.17,671.00)(1.000,18.000){2}{\rule{0.400pt}{4.336pt}}
\put(740.67,707){\rule{0.400pt}{0.482pt}}
\multiput(740.17,707.00)(1.000,1.000){2}{\rule{0.400pt}{0.241pt}}
\put(741.67,675){\rule{0.400pt}{8.191pt}}
\multiput(741.17,692.00)(1.000,-17.000){2}{\rule{0.400pt}{4.095pt}}
\put(743.17,654){\rule{0.400pt}{4.300pt}}
\multiput(742.17,666.08)(2.000,-12.075){2}{\rule{0.400pt}{2.150pt}}
\put(744.67,654){\rule{0.400pt}{8.191pt}}
\multiput(744.17,654.00)(1.000,17.000){2}{\rule{0.400pt}{4.095pt}}
\put(745.67,677){\rule{0.400pt}{2.650pt}}
\multiput(745.17,682.50)(1.000,-5.500){2}{\rule{0.400pt}{1.325pt}}
\put(746.67,677){\rule{0.400pt}{8.672pt}}
\multiput(746.17,677.00)(1.000,18.000){2}{\rule{0.400pt}{4.336pt}}
\put(748,712.67){\rule{0.241pt}{0.400pt}}
\multiput(748.00,712.17)(0.500,1.000){2}{\rule{0.120pt}{0.400pt}}
\put(748.67,681){\rule{0.400pt}{7.950pt}}
\multiput(748.17,697.50)(1.000,-16.500){2}{\rule{0.400pt}{3.975pt}}
\put(739.0,671.0){\usebox{\plotpoint}}
\put(750.67,681){\rule{0.400pt}{8.913pt}}
\multiput(750.17,681.00)(1.000,18.500){2}{\rule{0.400pt}{4.457pt}}
\put(751.67,718){\rule{0.400pt}{0.482pt}}
\multiput(751.17,718.00)(1.000,1.000){2}{\rule{0.400pt}{0.241pt}}
\put(752.67,674){\rule{0.400pt}{11.081pt}}
\multiput(752.17,697.00)(1.000,-23.000){2}{\rule{0.400pt}{5.541pt}}
\put(753.67,665){\rule{0.400pt}{2.168pt}}
\multiput(753.17,669.50)(1.000,-4.500){2}{\rule{0.400pt}{1.084pt}}
\put(754.67,665){\rule{0.400pt}{7.950pt}}
\multiput(754.17,665.00)(1.000,16.500){2}{\rule{0.400pt}{3.975pt}}
\put(755.67,698){\rule{0.400pt}{2.891pt}}
\multiput(755.17,698.00)(1.000,6.000){2}{\rule{0.400pt}{1.445pt}}
\put(757.17,667){\rule{0.400pt}{8.700pt}}
\multiput(756.17,691.94)(2.000,-24.943){2}{\rule{0.400pt}{4.350pt}}
\put(758.67,667){\rule{0.400pt}{14.213pt}}
\multiput(758.17,667.00)(1.000,29.500){2}{\rule{0.400pt}{7.107pt}}
\put(759.67,680){\rule{0.400pt}{11.081pt}}
\multiput(759.17,703.00)(1.000,-23.000){2}{\rule{0.400pt}{5.541pt}}
\put(760.67,680){\rule{0.400pt}{5.782pt}}
\multiput(760.17,680.00)(1.000,12.000){2}{\rule{0.400pt}{2.891pt}}
\put(761.67,671){\rule{0.400pt}{7.950pt}}
\multiput(761.17,687.50)(1.000,-16.500){2}{\rule{0.400pt}{3.975pt}}
\put(762.67,671){\rule{0.400pt}{11.322pt}}
\multiput(762.17,671.00)(1.000,23.500){2}{\rule{0.400pt}{5.661pt}}
\put(763.67,684){\rule{0.400pt}{8.191pt}}
\multiput(763.17,701.00)(1.000,-17.000){2}{\rule{0.400pt}{4.095pt}}
\put(764.67,684){\rule{0.400pt}{11.563pt}}
\multiput(764.17,684.00)(1.000,24.000){2}{\rule{0.400pt}{5.782pt}}
\put(765.67,686){\rule{0.400pt}{11.081pt}}
\multiput(765.17,709.00)(1.000,-23.000){2}{\rule{0.400pt}{5.541pt}}
\put(767,685.67){\rule{0.241pt}{0.400pt}}
\multiput(767.00,685.17)(0.500,1.000){2}{\rule{0.120pt}{0.400pt}}
\put(767.67,687){\rule{0.400pt}{14.695pt}}
\multiput(767.17,687.00)(1.000,30.500){2}{\rule{0.400pt}{7.347pt}}
\put(768.67,688){\rule{0.400pt}{14.454pt}}
\multiput(768.17,718.00)(1.000,-30.000){2}{\rule{0.400pt}{7.227pt}}
\put(769.67,688){\rule{0.400pt}{11.804pt}}
\multiput(769.17,688.00)(1.000,24.500){2}{\rule{0.400pt}{5.902pt}}
\put(770.67,726){\rule{0.400pt}{2.650pt}}
\multiput(770.17,731.50)(1.000,-5.500){2}{\rule{0.400pt}{1.325pt}}
\put(772.17,692){\rule{0.400pt}{6.900pt}}
\multiput(771.17,711.68)(2.000,-19.679){2}{\rule{0.400pt}{3.450pt}}
\put(773.67,692){\rule{0.400pt}{11.563pt}}
\multiput(773.17,692.00)(1.000,24.000){2}{\rule{0.400pt}{5.782pt}}
\put(774.67,694){\rule{0.400pt}{11.081pt}}
\multiput(774.17,717.00)(1.000,-23.000){2}{\rule{0.400pt}{5.541pt}}
\put(775.67,694){\rule{0.400pt}{5.541pt}}
\multiput(775.17,694.00)(1.000,11.500){2}{\rule{0.400pt}{2.770pt}}
\put(776.67,696){\rule{0.400pt}{5.059pt}}
\multiput(776.17,706.50)(1.000,-10.500){2}{\rule{0.400pt}{2.529pt}}
\put(777.67,696){\rule{0.400pt}{5.782pt}}
\multiput(777.17,696.00)(1.000,12.000){2}{\rule{0.400pt}{2.891pt}}
\put(778.67,698){\rule{0.400pt}{5.300pt}}
\multiput(778.17,709.00)(1.000,-11.000){2}{\rule{0.400pt}{2.650pt}}
\put(779.67,688){\rule{0.400pt}{2.409pt}}
\multiput(779.17,693.00)(1.000,-5.000){2}{\rule{0.400pt}{1.204pt}}
\put(780.67,688){\rule{0.400pt}{14.454pt}}
\multiput(780.17,688.00)(1.000,30.000){2}{\rule{0.400pt}{7.227pt}}
\put(781.67,748){\rule{0.400pt}{6.745pt}}
\multiput(781.17,748.00)(1.000,14.000){2}{\rule{0.400pt}{3.373pt}}
\put(782.67,714){\rule{0.400pt}{14.936pt}}
\multiput(782.17,745.00)(1.000,-31.000){2}{\rule{0.400pt}{7.468pt}}
\put(783.67,714){\rule{0.400pt}{12.286pt}}
\multiput(783.17,714.00)(1.000,25.500){2}{\rule{0.400pt}{6.143pt}}
\put(784.67,705){\rule{0.400pt}{14.454pt}}
\multiput(784.17,735.00)(1.000,-30.000){2}{\rule{0.400pt}{7.227pt}}
\put(786,704.67){\rule{0.241pt}{0.400pt}}
\multiput(786.00,704.17)(0.500,1.000){2}{\rule{0.120pt}{0.400pt}}
\put(787.17,706){\rule{0.400pt}{12.500pt}}
\multiput(786.17,706.00)(2.000,36.056){2}{\rule{0.400pt}{6.250pt}}
\put(788.67,768){\rule{0.400pt}{0.482pt}}
\multiput(788.17,768.00)(1.000,1.000){2}{\rule{0.400pt}{0.241pt}}
\put(789.67,745){\rule{0.400pt}{6.023pt}}
\multiput(789.17,757.50)(1.000,-12.500){2}{\rule{0.400pt}{3.011pt}}
\put(790.67,745){\rule{0.400pt}{6.504pt}}
\multiput(790.17,745.00)(1.000,13.500){2}{\rule{0.400pt}{3.252pt}}
\put(791.67,700){\rule{0.400pt}{17.345pt}}
\multiput(791.17,736.00)(1.000,-36.000){2}{\rule{0.400pt}{8.672pt}}
\put(792.67,700){\rule{0.400pt}{2.891pt}}
\multiput(792.17,700.00)(1.000,6.000){2}{\rule{0.400pt}{1.445pt}}
\put(793.67,712){\rule{0.400pt}{11.804pt}}
\multiput(793.17,712.00)(1.000,24.500){2}{\rule{0.400pt}{5.902pt}}
\put(794.67,761){\rule{0.400pt}{0.482pt}}
\multiput(794.17,761.00)(1.000,1.000){2}{\rule{0.400pt}{0.241pt}}
\put(795.67,727){\rule{0.400pt}{8.672pt}}
\multiput(795.17,745.00)(1.000,-18.000){2}{\rule{0.400pt}{4.336pt}}
\put(796.67,727){\rule{0.400pt}{6.023pt}}
\multiput(796.17,727.00)(1.000,12.500){2}{\rule{0.400pt}{3.011pt}}
\put(797.67,752){\rule{0.400pt}{9.877pt}}
\multiput(797.17,752.00)(1.000,20.500){2}{\rule{0.400pt}{4.938pt}}
\put(798.67,793){\rule{0.400pt}{0.482pt}}
\multiput(798.17,793.00)(1.000,1.000){2}{\rule{0.400pt}{0.241pt}}
\put(799.67,708){\rule{0.400pt}{20.958pt}}
\multiput(799.17,751.50)(1.000,-43.500){2}{\rule{0.400pt}{10.479pt}}
\put(801.17,708){\rule{0.400pt}{2.700pt}}
\multiput(800.17,708.00)(2.000,7.396){2}{\rule{0.400pt}{1.350pt}}
\put(802.67,721){\rule{0.400pt}{8.672pt}}
\multiput(802.17,721.00)(1.000,18.000){2}{\rule{0.400pt}{4.336pt}}
\put(803.67,734){\rule{0.400pt}{5.541pt}}
\multiput(803.17,745.50)(1.000,-11.500){2}{\rule{0.400pt}{2.770pt}}
\put(804.67,734){\rule{0.400pt}{12.527pt}}
\multiput(804.17,734.00)(1.000,26.000){2}{\rule{0.400pt}{6.263pt}}
\put(806,785.67){\rule{0.241pt}{0.400pt}}
\multiput(806.00,785.17)(0.500,1.000){2}{\rule{0.120pt}{0.400pt}}
\put(806.67,762){\rule{0.400pt}{6.023pt}}
\multiput(806.17,774.50)(1.000,-12.500){2}{\rule{0.400pt}{3.011pt}}
\put(808,761.67){\rule{0.241pt}{0.400pt}}
\multiput(808.00,761.17)(0.500,1.000){2}{\rule{0.120pt}{0.400pt}}
\put(808.67,727){\rule{0.400pt}{8.672pt}}
\multiput(808.17,745.00)(1.000,-18.000){2}{\rule{0.400pt}{4.336pt}}
\put(809.67,727){\rule{0.400pt}{8.913pt}}
\multiput(809.17,727.00)(1.000,18.500){2}{\rule{0.400pt}{4.457pt}}
\put(810.67,729){\rule{0.400pt}{8.432pt}}
\multiput(810.17,746.50)(1.000,-17.500){2}{\rule{0.400pt}{4.216pt}}
\put(811.67,729){\rule{0.400pt}{6.023pt}}
\multiput(811.17,729.00)(1.000,12.500){2}{\rule{0.400pt}{3.011pt}}
\put(812.67,731){\rule{0.400pt}{5.541pt}}
\multiput(812.17,742.50)(1.000,-11.500){2}{\rule{0.400pt}{2.770pt}}
\put(813.67,731){\rule{0.400pt}{6.023pt}}
\multiput(813.17,731.00)(1.000,12.500){2}{\rule{0.400pt}{3.011pt}}
\put(814.67,722){\rule{0.400pt}{8.191pt}}
\multiput(814.17,739.00)(1.000,-17.000){2}{\rule{0.400pt}{4.095pt}}
\put(816.17,722){\rule{0.400pt}{2.500pt}}
\multiput(815.17,722.00)(2.000,6.811){2}{\rule{0.400pt}{1.250pt}}
\put(817.67,734){\rule{0.400pt}{12.286pt}}
\multiput(817.17,734.00)(1.000,25.500){2}{\rule{0.400pt}{6.143pt}}
\put(818.67,785){\rule{0.400pt}{3.614pt}}
\multiput(818.17,785.00)(1.000,7.500){2}{\rule{0.400pt}{1.807pt}}
\put(819.67,749){\rule{0.400pt}{12.286pt}}
\multiput(819.17,774.50)(1.000,-25.500){2}{\rule{0.400pt}{6.143pt}}
\put(820.67,749){\rule{0.400pt}{9.395pt}}
\multiput(820.17,749.00)(1.000,19.500){2}{\rule{0.400pt}{4.698pt}}
\put(821.67,739){\rule{0.400pt}{11.804pt}}
\multiput(821.17,763.50)(1.000,-24.500){2}{\rule{0.400pt}{5.902pt}}
\put(822.67,739){\rule{0.400pt}{2.891pt}}
\multiput(822.17,739.00)(1.000,6.000){2}{\rule{0.400pt}{1.445pt}}
\put(823.67,751){\rule{0.400pt}{13.009pt}}
\multiput(823.17,751.00)(1.000,27.000){2}{\rule{0.400pt}{6.504pt}}
\put(824.67,779){\rule{0.400pt}{6.263pt}}
\multiput(824.17,792.00)(1.000,-13.000){2}{\rule{0.400pt}{3.132pt}}
\put(825.67,779){\rule{0.400pt}{6.745pt}}
\multiput(825.17,779.00)(1.000,14.000){2}{\rule{0.400pt}{3.373pt}}
\put(827,806.67){\rule{0.241pt}{0.400pt}}
\multiput(827.00,806.17)(0.500,1.000){2}{\rule{0.120pt}{0.400pt}}
\put(827.67,782){\rule{0.400pt}{6.263pt}}
\multiput(827.17,795.00)(1.000,-13.000){2}{\rule{0.400pt}{3.132pt}}
\put(828.67,782){\rule{0.400pt}{6.986pt}}
\multiput(828.17,782.00)(1.000,14.500){2}{\rule{0.400pt}{3.493pt}}
\put(829.67,747){\rule{0.400pt}{15.418pt}}
\multiput(829.17,779.00)(1.000,-32.000){2}{\rule{0.400pt}{7.709pt}}
\put(831,746.67){\rule{0.482pt}{0.400pt}}
\multiput(831.00,746.17)(1.000,1.000){2}{\rule{0.241pt}{0.400pt}}
\put(832.67,748){\rule{0.400pt}{12.286pt}}
\multiput(832.17,748.00)(1.000,25.500){2}{\rule{0.400pt}{6.143pt}}
\put(833.67,773){\rule{0.400pt}{6.263pt}}
\multiput(833.17,786.00)(1.000,-13.000){2}{\rule{0.400pt}{3.132pt}}
\put(834.67,739){\rule{0.400pt}{8.191pt}}
\multiput(834.17,756.00)(1.000,-17.000){2}{\rule{0.400pt}{4.095pt}}
\put(750.0,681.0){\usebox{\plotpoint}}
\put(836.67,739){\rule{0.400pt}{6.023pt}}
\multiput(836.17,739.00)(1.000,12.500){2}{\rule{0.400pt}{3.011pt}}
\put(837.67,754){\rule{0.400pt}{2.409pt}}
\multiput(837.17,759.00)(1.000,-5.000){2}{\rule{0.400pt}{1.204pt}}
\put(838.67,754){\rule{0.400pt}{12.286pt}}
\multiput(838.17,754.00)(1.000,25.500){2}{\rule{0.400pt}{6.143pt}}
\put(836.0,739.0){\usebox{\plotpoint}}
\put(840.67,756){\rule{0.400pt}{11.804pt}}
\multiput(840.17,780.50)(1.000,-24.500){2}{\rule{0.400pt}{5.902pt}}
\put(841.67,745){\rule{0.400pt}{2.650pt}}
\multiput(841.17,750.50)(1.000,-5.500){2}{\rule{0.400pt}{1.325pt}}
\put(842.67,745){\rule{0.400pt}{8.913pt}}
\multiput(842.17,745.00)(1.000,18.500){2}{\rule{0.400pt}{4.457pt}}
\put(844,781.67){\rule{0.241pt}{0.400pt}}
\multiput(844.00,781.17)(0.500,1.000){2}{\rule{0.120pt}{0.400pt}}
\put(845.17,759){\rule{0.400pt}{4.900pt}}
\multiput(844.17,772.83)(2.000,-13.830){2}{\rule{0.400pt}{2.450pt}}
\put(846.67,759){\rule{0.400pt}{6.023pt}}
\multiput(846.17,759.00)(1.000,12.500){2}{\rule{0.400pt}{3.011pt}}
\put(847.67,749){\rule{0.400pt}{8.432pt}}
\multiput(847.17,766.50)(1.000,-17.500){2}{\rule{0.400pt}{4.216pt}}
\put(849,748.67){\rule{0.241pt}{0.400pt}}
\multiput(849.00,748.17)(0.500,1.000){2}{\rule{0.120pt}{0.400pt}}
\put(849.67,750){\rule{0.400pt}{8.913pt}}
\multiput(849.17,750.00)(1.000,18.500){2}{\rule{0.400pt}{4.457pt}}
\put(851,786.67){\rule{0.241pt}{0.400pt}}
\multiput(851.00,786.17)(0.500,1.000){2}{\rule{0.120pt}{0.400pt}}
\put(851.67,752){\rule{0.400pt}{8.672pt}}
\multiput(851.17,770.00)(1.000,-18.000){2}{\rule{0.400pt}{4.336pt}}
\put(852.67,752){\rule{0.400pt}{0.482pt}}
\multiput(852.17,752.00)(1.000,1.000){2}{\rule{0.400pt}{0.241pt}}
\put(853.67,754){\rule{0.400pt}{5.782pt}}
\multiput(853.17,754.00)(1.000,12.000){2}{\rule{0.400pt}{2.891pt}}
\put(854.67,755){\rule{0.400pt}{5.541pt}}
\multiput(854.17,766.50)(1.000,-11.500){2}{\rule{0.400pt}{2.770pt}}
\put(855.67,755){\rule{0.400pt}{5.782pt}}
\multiput(855.17,755.00)(1.000,12.000){2}{\rule{0.400pt}{2.891pt}}
\put(856.67,756){\rule{0.400pt}{5.541pt}}
\multiput(856.17,767.50)(1.000,-11.500){2}{\rule{0.400pt}{2.770pt}}
\put(857.67,756){\rule{0.400pt}{9.154pt}}
\multiput(857.17,756.00)(1.000,19.000){2}{\rule{0.400pt}{4.577pt}}
\put(840.0,805.0){\usebox{\plotpoint}}
\put(860.17,758){\rule{0.400pt}{7.300pt}}
\multiput(859.17,778.85)(2.000,-20.848){2}{\rule{0.400pt}{3.650pt}}
\put(861.67,758){\rule{0.400pt}{3.132pt}}
\multiput(861.17,758.00)(1.000,6.500){2}{\rule{0.400pt}{1.566pt}}
\put(862.67,771){\rule{0.400pt}{6.263pt}}
\multiput(862.17,771.00)(1.000,13.000){2}{\rule{0.400pt}{3.132pt}}
\put(863.67,773){\rule{0.400pt}{5.782pt}}
\multiput(863.17,785.00)(1.000,-12.000){2}{\rule{0.400pt}{2.891pt}}
\put(864.67,773){\rule{0.400pt}{6.023pt}}
\multiput(864.17,773.00)(1.000,12.500){2}{\rule{0.400pt}{3.011pt}}
\put(865.67,762){\rule{0.400pt}{8.672pt}}
\multiput(865.17,780.00)(1.000,-18.000){2}{\rule{0.400pt}{4.336pt}}
\put(866.67,762){\rule{0.400pt}{9.154pt}}
\multiput(866.17,762.00)(1.000,19.000){2}{\rule{0.400pt}{4.577pt}}
\put(867.67,787){\rule{0.400pt}{3.132pt}}
\multiput(867.17,793.50)(1.000,-6.500){2}{\rule{0.400pt}{1.566pt}}
\put(868.67,765){\rule{0.400pt}{5.300pt}}
\multiput(868.17,776.00)(1.000,-11.000){2}{\rule{0.400pt}{2.650pt}}
\put(869.67,765){\rule{0.400pt}{6.023pt}}
\multiput(869.17,765.00)(1.000,12.500){2}{\rule{0.400pt}{3.011pt}}
\put(870.67,766){\rule{0.400pt}{5.782pt}}
\multiput(870.17,778.00)(1.000,-12.000){2}{\rule{0.400pt}{2.891pt}}
\put(871.67,766){\rule{0.400pt}{6.023pt}}
\multiput(871.17,766.00)(1.000,12.500){2}{\rule{0.400pt}{3.011pt}}
\put(872.67,756){\rule{0.400pt}{8.432pt}}
\multiput(872.17,773.50)(1.000,-17.500){2}{\rule{0.400pt}{4.216pt}}
\put(873.67,756){\rule{0.400pt}{8.672pt}}
\multiput(873.17,756.00)(1.000,18.000){2}{\rule{0.400pt}{4.336pt}}
\put(875.17,769){\rule{0.400pt}{4.700pt}}
\multiput(874.17,782.24)(2.000,-13.245){2}{\rule{0.400pt}{2.350pt}}
\put(876.67,769){\rule{0.400pt}{5.782pt}}
\multiput(876.17,769.00)(1.000,12.000){2}{\rule{0.400pt}{2.891pt}}
\put(877.67,770){\rule{0.400pt}{5.541pt}}
\multiput(877.17,781.50)(1.000,-11.500){2}{\rule{0.400pt}{2.770pt}}
\put(878.67,770){\rule{0.400pt}{6.023pt}}
\multiput(878.17,770.00)(1.000,12.500){2}{\rule{0.400pt}{3.011pt}}
\put(879.67,760){\rule{0.400pt}{8.432pt}}
\multiput(879.17,777.50)(1.000,-17.500){2}{\rule{0.400pt}{4.216pt}}
\put(859.0,794.0){\usebox{\plotpoint}}
\put(881.67,760){\rule{0.400pt}{8.913pt}}
\multiput(881.17,760.00)(1.000,18.500){2}{\rule{0.400pt}{4.457pt}}
\put(882.67,797){\rule{0.400pt}{3.373pt}}
\multiput(882.17,797.00)(1.000,7.000){2}{\rule{0.400pt}{1.686pt}}
\put(883.67,774){\rule{0.400pt}{8.913pt}}
\multiput(883.17,792.50)(1.000,-18.500){2}{\rule{0.400pt}{4.457pt}}
\put(884.67,774){\rule{0.400pt}{5.782pt}}
\multiput(884.17,774.00)(1.000,12.000){2}{\rule{0.400pt}{2.891pt}}
\put(885.67,763){\rule{0.400pt}{8.432pt}}
\multiput(885.17,780.50)(1.000,-17.500){2}{\rule{0.400pt}{4.216pt}}
\put(886.67,763){\rule{0.400pt}{8.913pt}}
\multiput(886.17,763.00)(1.000,18.500){2}{\rule{0.400pt}{4.457pt}}
\put(887.67,764){\rule{0.400pt}{8.672pt}}
\multiput(887.17,782.00)(1.000,-18.000){2}{\rule{0.400pt}{4.336pt}}
\put(889.17,764){\rule{0.400pt}{2.500pt}}
\multiput(888.17,764.00)(2.000,6.811){2}{\rule{0.400pt}{1.250pt}}
\put(890.67,776){\rule{0.400pt}{6.263pt}}
\multiput(890.17,776.00)(1.000,13.000){2}{\rule{0.400pt}{3.132pt}}
\put(891.67,790){\rule{0.400pt}{2.891pt}}
\multiput(891.17,796.00)(1.000,-6.000){2}{\rule{0.400pt}{1.445pt}}
\put(892.67,778){\rule{0.400pt}{2.891pt}}
\multiput(892.17,784.00)(1.000,-6.000){2}{\rule{0.400pt}{1.445pt}}
\put(894,777.67){\rule{0.241pt}{0.400pt}}
\multiput(894.00,777.17)(0.500,1.000){2}{\rule{0.120pt}{0.400pt}}
\put(894.67,779){\rule{0.400pt}{6.023pt}}
\multiput(894.17,779.00)(1.000,12.500){2}{\rule{0.400pt}{3.011pt}}
\put(896,803.67){\rule{0.241pt}{0.400pt}}
\multiput(896.00,803.17)(0.500,1.000){2}{\rule{0.120pt}{0.400pt}}
\put(896.67,769){\rule{0.400pt}{8.672pt}}
\multiput(896.17,787.00)(1.000,-18.000){2}{\rule{0.400pt}{4.336pt}}
\put(897.67,769){\rule{0.400pt}{5.782pt}}
\multiput(897.17,769.00)(1.000,12.000){2}{\rule{0.400pt}{2.891pt}}
\put(898.67,748){\rule{0.400pt}{10.840pt}}
\multiput(898.17,770.50)(1.000,-22.500){2}{\rule{0.400pt}{5.420pt}}
\put(899.67,748){\rule{0.400pt}{2.409pt}}
\multiput(899.17,748.00)(1.000,5.000){2}{\rule{0.400pt}{1.204pt}}
\put(900.67,758){\rule{0.400pt}{8.913pt}}
\multiput(900.17,758.00)(1.000,18.500){2}{\rule{0.400pt}{4.457pt}}
\put(901.67,772){\rule{0.400pt}{5.541pt}}
\multiput(901.17,783.50)(1.000,-11.500){2}{\rule{0.400pt}{2.770pt}}
\put(902.67,772){\rule{0.400pt}{5.782pt}}
\multiput(902.17,772.00)(1.000,12.000){2}{\rule{0.400pt}{2.891pt}}
\put(881.0,760.0){\usebox{\plotpoint}}
\put(905.67,761){\rule{0.400pt}{8.432pt}}
\multiput(905.17,778.50)(1.000,-17.500){2}{\rule{0.400pt}{4.216pt}}
\put(906.67,761){\rule{0.400pt}{5.782pt}}
\multiput(906.17,761.00)(1.000,12.000){2}{\rule{0.400pt}{2.891pt}}
\put(907.67,751){\rule{0.400pt}{8.191pt}}
\multiput(907.17,768.00)(1.000,-17.000){2}{\rule{0.400pt}{4.095pt}}
\put(908.67,751){\rule{0.400pt}{2.650pt}}
\multiput(908.17,751.00)(1.000,5.500){2}{\rule{0.400pt}{1.325pt}}
\put(909.67,762){\rule{0.400pt}{8.672pt}}
\multiput(909.17,762.00)(1.000,18.000){2}{\rule{0.400pt}{4.336pt}}
\put(910.67,798){\rule{0.400pt}{3.373pt}}
\multiput(910.17,798.00)(1.000,7.000){2}{\rule{0.400pt}{1.686pt}}
\put(911.67,775){\rule{0.400pt}{8.913pt}}
\multiput(911.17,793.50)(1.000,-18.500){2}{\rule{0.400pt}{4.457pt}}
\put(912.67,764){\rule{0.400pt}{2.650pt}}
\multiput(912.17,769.50)(1.000,-5.500){2}{\rule{0.400pt}{1.325pt}}
\put(913.67,764){\rule{0.400pt}{8.672pt}}
\multiput(913.17,764.00)(1.000,18.000){2}{\rule{0.400pt}{4.336pt}}
\put(914.67,789){\rule{0.400pt}{2.650pt}}
\multiput(914.17,794.50)(1.000,-5.500){2}{\rule{0.400pt}{1.325pt}}
\put(915.67,755){\rule{0.400pt}{8.191pt}}
\multiput(915.17,772.00)(1.000,-17.000){2}{\rule{0.400pt}{4.095pt}}
\put(904.0,796.0){\rule[-0.200pt]{0.482pt}{0.400pt}}
\put(918.17,755){\rule{0.400pt}{9.300pt}}
\multiput(917.17,755.00)(2.000,26.697){2}{\rule{0.400pt}{4.650pt}}
\put(920,800.67){\rule{0.241pt}{0.400pt}}
\multiput(920.00,800.17)(0.500,1.000){2}{\rule{0.120pt}{0.400pt}}
\put(920.67,778){\rule{0.400pt}{5.782pt}}
\multiput(920.17,790.00)(1.000,-12.000){2}{\rule{0.400pt}{2.891pt}}
\put(921.67,778){\rule{0.400pt}{5.782pt}}
\multiput(921.17,778.00)(1.000,12.000){2}{\rule{0.400pt}{2.891pt}}
\put(922.67,768){\rule{0.400pt}{8.191pt}}
\multiput(922.17,785.00)(1.000,-17.000){2}{\rule{0.400pt}{4.095pt}}
\put(923.67,768){\rule{0.400pt}{11.804pt}}
\multiput(923.17,768.00)(1.000,24.500){2}{\rule{0.400pt}{5.902pt}}
\put(924.67,780){\rule{0.400pt}{8.913pt}}
\multiput(924.17,798.50)(1.000,-18.500){2}{\rule{0.400pt}{4.457pt}}
\put(925.67,780){\rule{0.400pt}{5.541pt}}
\multiput(925.17,780.00)(1.000,11.500){2}{\rule{0.400pt}{2.770pt}}
\put(926.67,780){\rule{0.400pt}{5.541pt}}
\multiput(926.17,791.50)(1.000,-11.500){2}{\rule{0.400pt}{2.770pt}}
\put(928,779.67){\rule{0.241pt}{0.400pt}}
\multiput(928.00,779.17)(0.500,1.000){2}{\rule{0.120pt}{0.400pt}}
\put(928.67,781){\rule{0.400pt}{5.782pt}}
\multiput(928.17,781.00)(1.000,12.000){2}{\rule{0.400pt}{2.891pt}}
\put(917.0,755.0){\usebox{\plotpoint}}
\put(930.67,759){\rule{0.400pt}{11.081pt}}
\multiput(930.17,782.00)(1.000,-23.000){2}{\rule{0.400pt}{5.541pt}}
\put(931.67,748){\rule{0.400pt}{2.650pt}}
\multiput(931.17,753.50)(1.000,-5.500){2}{\rule{0.400pt}{1.325pt}}
\put(933.17,748){\rule{0.400pt}{11.700pt}}
\multiput(932.17,748.00)(2.000,33.716){2}{\rule{0.400pt}{5.850pt}}
\put(934.67,783){\rule{0.400pt}{5.541pt}}
\multiput(934.17,794.50)(1.000,-11.500){2}{\rule{0.400pt}{2.770pt}}
\put(935.67,783){\rule{0.400pt}{15.418pt}}
\multiput(935.17,783.00)(1.000,32.000){2}{\rule{0.400pt}{7.709pt}}
\put(936.67,808){\rule{0.400pt}{9.395pt}}
\multiput(936.17,827.50)(1.000,-19.500){2}{\rule{0.400pt}{4.698pt}}
\put(937.67,751){\rule{0.400pt}{13.731pt}}
\multiput(937.17,779.50)(1.000,-28.500){2}{\rule{0.400pt}{6.866pt}}
\put(938.67,751){\rule{0.400pt}{5.300pt}}
\multiput(938.17,751.00)(1.000,11.000){2}{\rule{0.400pt}{2.650pt}}
\put(939.67,751){\rule{0.400pt}{5.300pt}}
\multiput(939.17,762.00)(1.000,-11.000){2}{\rule{0.400pt}{2.650pt}}
\put(941,749.67){\rule{0.241pt}{0.400pt}}
\multiput(941.00,750.17)(0.500,-1.000){2}{\rule{0.120pt}{0.400pt}}
\put(941.67,750){\rule{0.400pt}{11.081pt}}
\multiput(941.17,750.00)(1.000,23.000){2}{\rule{0.400pt}{5.541pt}}
\put(930.0,805.0){\usebox{\plotpoint}}
\put(943.67,762){\rule{0.400pt}{8.191pt}}
\multiput(943.17,779.00)(1.000,-17.000){2}{\rule{0.400pt}{4.095pt}}
\put(944.67,762){\rule{0.400pt}{8.191pt}}
\multiput(944.17,762.00)(1.000,17.000){2}{\rule{0.400pt}{4.095pt}}
\put(945.67,763){\rule{0.400pt}{7.950pt}}
\multiput(945.17,779.50)(1.000,-16.500){2}{\rule{0.400pt}{3.975pt}}
\put(946.67,763){\rule{0.400pt}{2.650pt}}
\multiput(946.17,763.00)(1.000,5.500){2}{\rule{0.400pt}{1.325pt}}
\put(948.17,774){\rule{0.400pt}{7.300pt}}
\multiput(947.17,774.00)(2.000,20.848){2}{\rule{0.400pt}{3.650pt}}
\put(949.67,798){\rule{0.400pt}{2.891pt}}
\multiput(949.17,804.00)(1.000,-6.000){2}{\rule{0.400pt}{1.445pt}}
\put(950.67,798){\rule{0.400pt}{9.395pt}}
\multiput(950.17,798.00)(1.000,19.500){2}{\rule{0.400pt}{4.698pt}}
\put(943.0,796.0){\usebox{\plotpoint}}
\put(952.67,787){\rule{0.400pt}{12.045pt}}
\multiput(952.17,812.00)(1.000,-25.000){2}{\rule{0.400pt}{6.022pt}}
\put(953.67,787){\rule{0.400pt}{2.891pt}}
\multiput(953.17,787.00)(1.000,6.000){2}{\rule{0.400pt}{1.445pt}}
\put(954.67,754){\rule{0.400pt}{10.840pt}}
\multiput(954.17,776.50)(1.000,-22.500){2}{\rule{0.400pt}{5.420pt}}
\put(956,752.67){\rule{0.241pt}{0.400pt}}
\multiput(956.00,753.17)(0.500,-1.000){2}{\rule{0.120pt}{0.400pt}}
\put(956.67,753){\rule{0.400pt}{17.104pt}}
\multiput(956.17,753.00)(1.000,35.500){2}{\rule{0.400pt}{8.552pt}}
\put(958,823.67){\rule{0.241pt}{0.400pt}}
\multiput(958.00,823.17)(0.500,1.000){2}{\rule{0.120pt}{0.400pt}}
\put(958.67,744){\rule{0.400pt}{19.513pt}}
\multiput(958.17,784.50)(1.000,-40.500){2}{\rule{0.400pt}{9.756pt}}
\put(960,742.67){\rule{0.241pt}{0.400pt}}
\multiput(960.00,743.17)(0.500,-1.000){2}{\rule{0.120pt}{0.400pt}}
\put(960.67,743){\rule{0.400pt}{7.950pt}}
\multiput(960.17,743.00)(1.000,16.500){2}{\rule{0.400pt}{3.975pt}}
\put(962.17,754){\rule{0.400pt}{4.500pt}}
\multiput(961.17,766.66)(2.000,-12.660){2}{\rule{0.400pt}{2.250pt}}
\put(963.67,754){\rule{0.400pt}{17.104pt}}
\multiput(963.17,754.00)(1.000,35.500){2}{\rule{0.400pt}{8.552pt}}
\put(964.67,812){\rule{0.400pt}{3.132pt}}
\multiput(964.17,818.50)(1.000,-6.500){2}{\rule{0.400pt}{1.566pt}}
\put(965.67,765){\rule{0.400pt}{11.322pt}}
\multiput(965.17,788.50)(1.000,-23.500){2}{\rule{0.400pt}{5.661pt}}
\put(966.67,765){\rule{0.400pt}{2.891pt}}
\multiput(966.17,765.00)(1.000,6.000){2}{\rule{0.400pt}{1.445pt}}
\put(967.67,744){\rule{0.400pt}{7.950pt}}
\multiput(967.17,760.50)(1.000,-16.500){2}{\rule{0.400pt}{3.975pt}}
\put(968.67,744){\rule{0.400pt}{13.490pt}}
\multiput(968.17,744.00)(1.000,28.000){2}{\rule{0.400pt}{6.745pt}}
\put(969.67,755){\rule{0.400pt}{10.840pt}}
\multiput(969.17,777.50)(1.000,-22.500){2}{\rule{0.400pt}{5.420pt}}
\put(971,753.67){\rule{0.241pt}{0.400pt}}
\multiput(971.00,754.17)(0.500,-1.000){2}{\rule{0.120pt}{0.400pt}}
\put(971.67,754){\rule{0.400pt}{11.081pt}}
\multiput(971.17,754.00)(1.000,23.000){2}{\rule{0.400pt}{5.541pt}}
\put(952.0,837.0){\usebox{\plotpoint}}
\put(973.67,765){\rule{0.400pt}{8.432pt}}
\multiput(973.17,782.50)(1.000,-17.500){2}{\rule{0.400pt}{4.216pt}}
\put(975,764.67){\rule{0.241pt}{0.400pt}}
\multiput(975.00,764.17)(0.500,1.000){2}{\rule{0.120pt}{0.400pt}}
\put(975.67,766){\rule{0.400pt}{11.081pt}}
\multiput(975.17,766.00)(1.000,23.000){2}{\rule{0.400pt}{5.541pt}}
\put(973.0,800.0){\usebox{\plotpoint}}
\put(978.67,766){\rule{0.400pt}{11.081pt}}
\multiput(978.17,789.00)(1.000,-23.000){2}{\rule{0.400pt}{5.541pt}}
\put(979.67,766){\rule{0.400pt}{8.432pt}}
\multiput(979.17,766.00)(1.000,17.500){2}{\rule{0.400pt}{4.216pt}}
\put(980.67,766){\rule{0.400pt}{8.432pt}}
\multiput(980.17,783.50)(1.000,-17.500){2}{\rule{0.400pt}{4.216pt}}
\put(981.67,766){\rule{0.400pt}{5.300pt}}
\multiput(981.17,766.00)(1.000,11.000){2}{\rule{0.400pt}{2.650pt}}
\put(982.67,755){\rule{0.400pt}{7.950pt}}
\multiput(982.17,771.50)(1.000,-16.500){2}{\rule{0.400pt}{3.975pt}}
\put(983.67,755){\rule{0.400pt}{5.059pt}}
\multiput(983.17,755.00)(1.000,10.500){2}{\rule{0.400pt}{2.529pt}}
\put(984.67,744){\rule{0.400pt}{7.709pt}}
\multiput(984.17,760.00)(1.000,-16.000){2}{\rule{0.400pt}{3.854pt}}
\put(986,743.67){\rule{0.241pt}{0.400pt}}
\multiput(986.00,743.17)(0.500,1.000){2}{\rule{0.120pt}{0.400pt}}
\put(986.67,745){\rule{0.400pt}{7.468pt}}
\multiput(986.17,745.00)(1.000,15.500){2}{\rule{0.400pt}{3.734pt}}
\put(987.67,755){\rule{0.400pt}{5.059pt}}
\multiput(987.17,765.50)(1.000,-10.500){2}{\rule{0.400pt}{2.529pt}}
\put(988.67,755){\rule{0.400pt}{10.840pt}}
\multiput(988.17,755.00)(1.000,22.500){2}{\rule{0.400pt}{5.420pt}}
\put(989.67,765){\rule{0.400pt}{8.432pt}}
\multiput(989.17,782.50)(1.000,-17.500){2}{\rule{0.400pt}{4.216pt}}
\put(990.67,765){\rule{0.400pt}{8.432pt}}
\multiput(990.17,765.00)(1.000,17.500){2}{\rule{0.400pt}{4.216pt}}
\put(977.0,812.0){\rule[-0.200pt]{0.482pt}{0.400pt}}
\put(993.67,754){\rule{0.400pt}{11.081pt}}
\multiput(993.17,777.00)(1.000,-23.000){2}{\rule{0.400pt}{5.541pt}}
\put(994.67,754){\rule{0.400pt}{5.300pt}}
\multiput(994.17,754.00)(1.000,11.000){2}{\rule{0.400pt}{2.650pt}}
\put(995.67,733){\rule{0.400pt}{10.359pt}}
\multiput(995.17,754.50)(1.000,-21.500){2}{\rule{0.400pt}{5.179pt}}
\put(996.67,733){\rule{0.400pt}{2.650pt}}
\multiput(996.17,733.00)(1.000,5.500){2}{\rule{0.400pt}{1.325pt}}
\put(997.67,744){\rule{0.400pt}{7.468pt}}
\multiput(997.17,744.00)(1.000,15.500){2}{\rule{0.400pt}{3.734pt}}
\put(998.67,743){\rule{0.400pt}{7.709pt}}
\multiput(998.17,759.00)(1.000,-16.000){2}{\rule{0.400pt}{3.854pt}}
\put(999.67,743){\rule{0.400pt}{7.709pt}}
\multiput(999.17,743.00)(1.000,16.000){2}{\rule{0.400pt}{3.854pt}}
\put(1000.67,753){\rule{0.400pt}{5.300pt}}
\multiput(1000.17,764.00)(1.000,-11.000){2}{\rule{0.400pt}{2.650pt}}
\put(1001.67,753){\rule{0.400pt}{10.840pt}}
\multiput(1001.17,753.00)(1.000,22.500){2}{\rule{0.400pt}{5.420pt}}
\put(1002.67,775){\rule{0.400pt}{5.541pt}}
\multiput(1002.17,786.50)(1.000,-11.500){2}{\rule{0.400pt}{2.770pt}}
\put(1003.67,742){\rule{0.400pt}{7.950pt}}
\multiput(1003.17,758.50)(1.000,-16.500){2}{\rule{0.400pt}{3.975pt}}
\put(1004.67,742){\rule{0.400pt}{7.709pt}}
\multiput(1004.17,742.00)(1.000,16.000){2}{\rule{0.400pt}{3.854pt}}
\put(1006.17,752){\rule{0.400pt}{4.500pt}}
\multiput(1005.17,764.66)(2.000,-12.660){2}{\rule{0.400pt}{2.250pt}}
\put(1007.67,752){\rule{0.400pt}{5.059pt}}
\multiput(1007.17,752.00)(1.000,10.500){2}{\rule{0.400pt}{2.529pt}}
\put(1008.67,752){\rule{0.400pt}{5.059pt}}
\multiput(1008.17,762.50)(1.000,-10.500){2}{\rule{0.400pt}{2.529pt}}
\put(1009.67,752){\rule{0.400pt}{10.840pt}}
\multiput(1009.17,752.00)(1.000,22.500){2}{\rule{0.400pt}{5.420pt}}
\put(1010.67,751){\rule{0.400pt}{11.081pt}}
\multiput(1010.17,774.00)(1.000,-23.000){2}{\rule{0.400pt}{5.541pt}}
\put(992.0,800.0){\rule[-0.200pt]{0.482pt}{0.400pt}}
\put(1012.67,751){\rule{0.400pt}{5.059pt}}
\multiput(1012.17,751.00)(1.000,10.500){2}{\rule{0.400pt}{2.529pt}}
\put(1012.0,751.0){\usebox{\plotpoint}}
\put(1014.67,740){\rule{0.400pt}{7.709pt}}
\multiput(1014.17,756.00)(1.000,-16.000){2}{\rule{0.400pt}{3.854pt}}
\put(1015.67,740){\rule{0.400pt}{2.409pt}}
\multiput(1015.17,740.00)(1.000,5.000){2}{\rule{0.400pt}{1.204pt}}
\put(1016.67,750){\rule{0.400pt}{5.059pt}}
\multiput(1016.17,750.00)(1.000,10.500){2}{\rule{0.400pt}{2.529pt}}
\put(1014.0,772.0){\usebox{\plotpoint}}
\put(1018.67,738){\rule{0.400pt}{7.950pt}}
\multiput(1018.17,754.50)(1.000,-16.500){2}{\rule{0.400pt}{3.975pt}}
\put(1019.67,738){\rule{0.400pt}{7.950pt}}
\multiput(1019.17,738.00)(1.000,16.500){2}{\rule{0.400pt}{3.975pt}}
\put(1021.17,727){\rule{0.400pt}{8.900pt}}
\multiput(1020.17,752.53)(2.000,-25.528){2}{\rule{0.400pt}{4.450pt}}
\put(1018.0,771.0){\usebox{\plotpoint}}
\put(1023.67,727){\rule{0.400pt}{7.468pt}}
\multiput(1023.17,727.00)(1.000,15.500){2}{\rule{0.400pt}{3.734pt}}
\put(1024.67,747){\rule{0.400pt}{2.650pt}}
\multiput(1024.17,752.50)(1.000,-5.500){2}{\rule{0.400pt}{1.325pt}}
\put(1025.67,726){\rule{0.400pt}{5.059pt}}
\multiput(1025.17,736.50)(1.000,-10.500){2}{\rule{0.400pt}{2.529pt}}
\put(1026.67,726){\rule{0.400pt}{4.818pt}}
\multiput(1026.17,726.00)(1.000,10.000){2}{\rule{0.400pt}{2.409pt}}
\put(1027.67,715){\rule{0.400pt}{7.468pt}}
\multiput(1027.17,730.50)(1.000,-15.500){2}{\rule{0.400pt}{3.734pt}}
\put(1029,713.67){\rule{0.241pt}{0.400pt}}
\multiput(1029.00,714.17)(0.500,-1.000){2}{\rule{0.120pt}{0.400pt}}
\put(1029.67,714){\rule{0.400pt}{4.818pt}}
\multiput(1029.17,714.00)(1.000,10.000){2}{\rule{0.400pt}{2.409pt}}
\put(1030.67,734){\rule{0.400pt}{2.409pt}}
\multiput(1030.17,734.00)(1.000,5.000){2}{\rule{0.400pt}{1.204pt}}
\put(1031.67,713){\rule{0.400pt}{7.468pt}}
\multiput(1031.17,728.50)(1.000,-15.500){2}{\rule{0.400pt}{3.734pt}}
\put(1023.0,727.0){\usebox{\plotpoint}}
\put(1033.67,713){\rule{0.400pt}{6.986pt}}
\multiput(1033.17,713.00)(1.000,14.500){2}{\rule{0.400pt}{3.493pt}}
\put(1034.67,721){\rule{0.400pt}{5.059pt}}
\multiput(1034.17,731.50)(1.000,-10.500){2}{\rule{0.400pt}{2.529pt}}
\put(1036.17,721){\rule{0.400pt}{4.100pt}}
\multiput(1035.17,721.00)(2.000,11.490){2}{\rule{0.400pt}{2.050pt}}
\put(1037.67,741){\rule{0.400pt}{2.409pt}}
\multiput(1037.17,741.00)(1.000,5.000){2}{\rule{0.400pt}{1.204pt}}
\put(1038.67,720){\rule{0.400pt}{7.468pt}}
\multiput(1038.17,735.50)(1.000,-15.500){2}{\rule{0.400pt}{3.734pt}}
\put(1033.0,713.0){\usebox{\plotpoint}}
\put(1040.67,720){\rule{0.400pt}{7.227pt}}
\multiput(1040.17,720.00)(1.000,15.000){2}{\rule{0.400pt}{3.613pt}}
\put(1041.67,729){\rule{0.400pt}{5.059pt}}
\multiput(1041.17,739.50)(1.000,-10.500){2}{\rule{0.400pt}{2.529pt}}
\put(1042.67,729){\rule{0.400pt}{7.468pt}}
\multiput(1042.17,729.00)(1.000,15.500){2}{\rule{0.400pt}{3.734pt}}
\put(1043.67,749){\rule{0.400pt}{2.650pt}}
\multiput(1043.17,754.50)(1.000,-5.500){2}{\rule{0.400pt}{1.325pt}}
\put(1044.67,718){\rule{0.400pt}{7.468pt}}
\multiput(1044.17,733.50)(1.000,-15.500){2}{\rule{0.400pt}{3.734pt}}
\put(1045.67,718){\rule{0.400pt}{1.927pt}}
\multiput(1045.17,718.00)(1.000,4.000){2}{\rule{0.400pt}{0.964pt}}
\put(1046.67,689){\rule{0.400pt}{8.913pt}}
\multiput(1046.17,707.50)(1.000,-18.500){2}{\rule{0.400pt}{4.457pt}}
\put(1047.67,689){\rule{0.400pt}{6.504pt}}
\multiput(1047.17,689.00)(1.000,13.500){2}{\rule{0.400pt}{3.252pt}}
\put(1048.67,716){\rule{0.400pt}{9.877pt}}
\multiput(1048.17,716.00)(1.000,20.500){2}{\rule{0.400pt}{4.938pt}}
\put(1050.17,735){\rule{0.400pt}{4.500pt}}
\multiput(1049.17,747.66)(2.000,-12.660){2}{\rule{0.400pt}{2.250pt}}
\put(1051.67,687){\rule{0.400pt}{11.563pt}}
\multiput(1051.17,711.00)(1.000,-24.000){2}{\rule{0.400pt}{5.782pt}}
\put(1052.67,687){\rule{0.400pt}{1.927pt}}
\multiput(1052.17,687.00)(1.000,4.000){2}{\rule{0.400pt}{0.964pt}}
\put(1053.67,695){\rule{0.400pt}{9.154pt}}
\multiput(1053.17,695.00)(1.000,19.000){2}{\rule{0.400pt}{4.577pt}}
\put(1054.67,733){\rule{0.400pt}{2.409pt}}
\multiput(1054.17,733.00)(1.000,5.000){2}{\rule{0.400pt}{1.204pt}}
\put(1055.67,712){\rule{0.400pt}{7.468pt}}
\multiput(1055.17,727.50)(1.000,-15.500){2}{\rule{0.400pt}{3.734pt}}
\put(1056.67,712){\rule{0.400pt}{4.818pt}}
\multiput(1056.17,712.00)(1.000,10.000){2}{\rule{0.400pt}{2.409pt}}
\put(1057.67,711){\rule{0.400pt}{5.059pt}}
\multiput(1057.17,721.50)(1.000,-10.500){2}{\rule{0.400pt}{2.529pt}}
\put(1059,709.67){\rule{0.241pt}{0.400pt}}
\multiput(1059.00,710.17)(0.500,-1.000){2}{\rule{0.120pt}{0.400pt}}
\put(1059.67,710){\rule{0.400pt}{4.818pt}}
\multiput(1059.17,710.00)(1.000,10.000){2}{\rule{0.400pt}{2.409pt}}
\put(1060.67,730){\rule{0.400pt}{2.409pt}}
\multiput(1060.17,730.00)(1.000,5.000){2}{\rule{0.400pt}{1.204pt}}
\put(1061.67,699){\rule{0.400pt}{9.877pt}}
\multiput(1061.17,719.50)(1.000,-20.500){2}{\rule{0.400pt}{4.938pt}}
\put(1040.0,720.0){\usebox{\plotpoint}}
\put(1063.67,699){\rule{0.400pt}{11.804pt}}
\multiput(1063.17,699.00)(1.000,24.500){2}{\rule{0.400pt}{5.902pt}}
\put(1065.17,738){\rule{0.400pt}{2.100pt}}
\multiput(1064.17,743.64)(2.000,-5.641){2}{\rule{0.400pt}{1.050pt}}
\put(1066.67,706){\rule{0.400pt}{7.709pt}}
\multiput(1066.17,722.00)(1.000,-16.000){2}{\rule{0.400pt}{3.854pt}}
\put(1067.67,706){\rule{0.400pt}{4.577pt}}
\multiput(1067.17,706.00)(1.000,9.500){2}{\rule{0.400pt}{2.289pt}}
\put(1068.67,696){\rule{0.400pt}{6.986pt}}
\multiput(1068.17,710.50)(1.000,-14.500){2}{\rule{0.400pt}{3.493pt}}
\put(1069.67,696){\rule{0.400pt}{6.986pt}}
\multiput(1069.17,696.00)(1.000,14.500){2}{\rule{0.400pt}{3.493pt}}
\put(1070.67,695){\rule{0.400pt}{7.227pt}}
\multiput(1070.17,710.00)(1.000,-15.000){2}{\rule{0.400pt}{3.613pt}}
\put(1071.67,695){\rule{0.400pt}{4.577pt}}
\multiput(1071.17,695.00)(1.000,9.500){2}{\rule{0.400pt}{2.289pt}}
\put(1072.67,693){\rule{0.400pt}{5.059pt}}
\multiput(1072.17,703.50)(1.000,-10.500){2}{\rule{0.400pt}{2.529pt}}
\put(1063.0,699.0){\usebox{\plotpoint}}
\put(1074.67,693){\rule{0.400pt}{4.336pt}}
\multiput(1074.17,693.00)(1.000,9.000){2}{\rule{0.400pt}{2.168pt}}
\put(1075.67,692){\rule{0.400pt}{4.577pt}}
\multiput(1075.17,701.50)(1.000,-9.500){2}{\rule{0.400pt}{2.289pt}}
\put(1076.67,692){\rule{0.400pt}{9.154pt}}
\multiput(1076.17,692.00)(1.000,19.000){2}{\rule{0.400pt}{4.577pt}}
\put(1074.0,693.0){\usebox{\plotpoint}}
\put(1079.17,698){\rule{0.400pt}{6.500pt}}
\multiput(1078.17,716.51)(2.000,-18.509){2}{\rule{0.400pt}{3.250pt}}
\put(1080.67,698){\rule{0.400pt}{4.818pt}}
\multiput(1080.17,698.00)(1.000,10.000){2}{\rule{0.400pt}{2.409pt}}
\put(1081.67,669){\rule{0.400pt}{11.804pt}}
\multiput(1081.17,693.50)(1.000,-24.500){2}{\rule{0.400pt}{5.902pt}}
\put(1082.67,669){\rule{0.400pt}{6.023pt}}
\multiput(1082.17,669.00)(1.000,12.500){2}{\rule{0.400pt}{3.011pt}}
\put(1083.67,668){\rule{0.400pt}{6.263pt}}
\multiput(1083.17,681.00)(1.000,-13.000){2}{\rule{0.400pt}{3.132pt}}
\put(1084.67,668){\rule{0.400pt}{4.336pt}}
\multiput(1084.17,668.00)(1.000,9.000){2}{\rule{0.400pt}{2.168pt}}
\put(1085.67,686){\rule{0.400pt}{4.095pt}}
\multiput(1085.17,686.00)(1.000,8.500){2}{\rule{0.400pt}{2.048pt}}
\put(1086.67,703){\rule{0.400pt}{1.927pt}}
\multiput(1086.17,703.00)(1.000,4.000){2}{\rule{0.400pt}{0.964pt}}
\put(1087.67,683){\rule{0.400pt}{6.745pt}}
\multiput(1087.17,697.00)(1.000,-14.000){2}{\rule{0.400pt}{3.373pt}}
\put(1088.67,683){\rule{0.400pt}{17.345pt}}
\multiput(1088.17,683.00)(1.000,36.000){2}{\rule{0.400pt}{8.672pt}}
\put(1089.67,663){\rule{0.400pt}{22.163pt}}
\multiput(1089.17,709.00)(1.000,-46.000){2}{\rule{0.400pt}{11.081pt}}
\put(1090.67,663){\rule{0.400pt}{6.504pt}}
\multiput(1090.17,663.00)(1.000,13.500){2}{\rule{0.400pt}{3.252pt}}
\put(1091.67,653){\rule{0.400pt}{8.913pt}}
\multiput(1091.17,671.50)(1.000,-18.500){2}{\rule{0.400pt}{4.457pt}}
\put(1092.67,651){\rule{0.400pt}{0.482pt}}
\multiput(1092.17,652.00)(1.000,-1.000){2}{\rule{0.400pt}{0.241pt}}
\put(1094.17,651){\rule{0.400pt}{27.100pt}}
\multiput(1093.17,651.00)(2.000,78.753){2}{\rule{0.400pt}{13.550pt}}
\put(1095.67,759){\rule{0.400pt}{6.504pt}}
\multiput(1095.17,772.50)(1.000,-13.500){2}{\rule{0.400pt}{3.252pt}}
\put(1096.67,642){\rule{0.400pt}{28.185pt}}
\multiput(1096.17,700.50)(1.000,-58.500){2}{\rule{0.400pt}{14.093pt}}
\put(1097.67,640){\rule{0.400pt}{0.482pt}}
\multiput(1097.17,641.00)(1.000,-1.000){2}{\rule{0.400pt}{0.241pt}}
\put(1098.67,640){\rule{0.400pt}{15.177pt}}
\multiput(1098.17,640.00)(1.000,31.500){2}{\rule{0.400pt}{7.588pt}}
\put(1099.67,694){\rule{0.400pt}{2.168pt}}
\multiput(1099.17,698.50)(1.000,-4.500){2}{\rule{0.400pt}{1.084pt}}
\put(1100.67,630){\rule{0.400pt}{15.418pt}}
\multiput(1100.17,662.00)(1.000,-32.000){2}{\rule{0.400pt}{7.709pt}}
\put(1101.67,612){\rule{0.400pt}{4.336pt}}
\multiput(1101.17,621.00)(1.000,-9.000){2}{\rule{0.400pt}{2.168pt}}
\put(1102.67,612){\rule{0.400pt}{12.045pt}}
\multiput(1102.17,612.00)(1.000,25.000){2}{\rule{0.400pt}{6.022pt}}
\put(1103.67,662){\rule{0.400pt}{9.154pt}}
\multiput(1103.17,662.00)(1.000,19.000){2}{\rule{0.400pt}{4.577pt}}
\put(1104.67,627){\rule{0.400pt}{17.586pt}}
\multiput(1104.17,663.50)(1.000,-36.500){2}{\rule{0.400pt}{8.793pt}}
\put(1105.67,627){\rule{0.400pt}{7.227pt}}
\multiput(1105.17,627.00)(1.000,15.000){2}{\rule{0.400pt}{3.613pt}}
\put(1106.67,588){\rule{0.400pt}{16.622pt}}
\multiput(1106.17,622.50)(1.000,-34.500){2}{\rule{0.400pt}{8.311pt}}
\put(1107.67,588){\rule{0.400pt}{6.504pt}}
\multiput(1107.17,588.00)(1.000,13.500){2}{\rule{0.400pt}{3.252pt}}
\put(1109.17,548){\rule{0.400pt}{13.500pt}}
\multiput(1108.17,586.98)(2.000,-38.980){2}{\rule{0.400pt}{6.750pt}}
\put(1110.67,548){\rule{0.400pt}{6.023pt}}
\multiput(1110.17,548.00)(1.000,12.500){2}{\rule{0.400pt}{3.011pt}}
\put(1111.67,495){\rule{0.400pt}{18.790pt}}
\multiput(1111.17,534.00)(1.000,-39.000){2}{\rule{0.400pt}{9.395pt}}
\put(1112.67,491){\rule{0.400pt}{0.964pt}}
\multiput(1112.17,493.00)(1.000,-2.000){2}{\rule{0.400pt}{0.482pt}}
\put(1113.67,491){\rule{0.400pt}{8.913pt}}
\multiput(1113.17,491.00)(1.000,18.500){2}{\rule{0.400pt}{4.457pt}}
\put(1114.67,512){\rule{0.400pt}{3.854pt}}
\multiput(1114.17,520.00)(1.000,-8.000){2}{\rule{0.400pt}{1.927pt}}
\put(1115.67,512){\rule{0.400pt}{8.672pt}}
\multiput(1115.17,512.00)(1.000,18.000){2}{\rule{0.400pt}{4.336pt}}
\put(1116.67,534){\rule{0.400pt}{3.373pt}}
\multiput(1116.17,541.00)(1.000,-7.000){2}{\rule{0.400pt}{1.686pt}}
\put(1117.67,534){\rule{0.400pt}{8.913pt}}
\multiput(1117.17,534.00)(1.000,18.500){2}{\rule{0.400pt}{4.457pt}}
\put(1118.67,562){\rule{0.400pt}{2.168pt}}
\multiput(1118.17,566.50)(1.000,-4.500){2}{\rule{0.400pt}{1.084pt}}
\put(1119.67,562){\rule{0.400pt}{10.118pt}}
\multiput(1119.17,562.00)(1.000,21.000){2}{\rule{0.400pt}{5.059pt}}
\put(1120.67,604){\rule{0.400pt}{1.927pt}}
\multiput(1120.17,604.00)(1.000,4.000){2}{\rule{0.400pt}{0.964pt}}
\put(1121.67,547){\rule{0.400pt}{15.658pt}}
\multiput(1121.17,579.50)(1.000,-32.500){2}{\rule{0.400pt}{7.829pt}}
\put(1123,545.17){\rule{0.482pt}{0.400pt}}
\multiput(1123.00,546.17)(1.000,-2.000){2}{\rule{0.241pt}{0.400pt}}
\put(1124.67,545){\rule{0.400pt}{6.023pt}}
\multiput(1124.17,545.00)(1.000,12.500){2}{\rule{0.400pt}{3.011pt}}
\put(1125.67,555){\rule{0.400pt}{3.614pt}}
\multiput(1125.17,562.50)(1.000,-7.500){2}{\rule{0.400pt}{1.807pt}}
\put(1126.67,555){\rule{0.400pt}{6.263pt}}
\multiput(1126.17,555.00)(1.000,13.000){2}{\rule{0.400pt}{3.132pt}}
\put(1127.67,572){\rule{0.400pt}{2.168pt}}
\multiput(1127.17,576.50)(1.000,-4.500){2}{\rule{0.400pt}{1.084pt}}
\put(1128.67,572){\rule{0.400pt}{6.745pt}}
\multiput(1128.17,572.00)(1.000,14.000){2}{\rule{0.400pt}{3.373pt}}
\put(1129.67,600){\rule{0.400pt}{3.373pt}}
\multiput(1129.17,600.00)(1.000,7.000){2}{\rule{0.400pt}{1.686pt}}
\put(1130.67,583){\rule{0.400pt}{7.468pt}}
\multiput(1130.17,598.50)(1.000,-15.500){2}{\rule{0.400pt}{3.734pt}}
\put(1131.67,574){\rule{0.400pt}{2.168pt}}
\multiput(1131.17,578.50)(1.000,-4.500){2}{\rule{0.400pt}{1.084pt}}
\put(1132.67,574){\rule{0.400pt}{5.300pt}}
\multiput(1132.17,574.00)(1.000,11.000){2}{\rule{0.400pt}{2.650pt}}
\put(1134,594.67){\rule{0.241pt}{0.400pt}}
\multiput(1134.00,595.17)(0.500,-1.000){2}{\rule{0.120pt}{0.400pt}}
\put(1134.67,564){\rule{0.400pt}{7.468pt}}
\multiput(1134.17,579.50)(1.000,-15.500){2}{\rule{0.400pt}{3.734pt}}
\put(1135.67,564){\rule{0.400pt}{2.891pt}}
\multiput(1135.17,564.00)(1.000,6.000){2}{\rule{0.400pt}{1.445pt}}
\put(1136.67,554){\rule{0.400pt}{5.300pt}}
\multiput(1136.17,565.00)(1.000,-11.000){2}{\rule{0.400pt}{2.650pt}}
\put(1138,552.17){\rule{0.482pt}{0.400pt}}
\multiput(1138.00,553.17)(1.000,-2.000){2}{\rule{0.241pt}{0.400pt}}
\put(1139.67,552){\rule{0.400pt}{4.818pt}}
\multiput(1139.17,552.00)(1.000,10.000){2}{\rule{0.400pt}{2.409pt}}
\put(1078.0,730.0){\usebox{\plotpoint}}
\put(1141.67,536){\rule{0.400pt}{8.672pt}}
\multiput(1141.17,554.00)(1.000,-18.000){2}{\rule{0.400pt}{4.336pt}}
\put(1143,534.67){\rule{0.241pt}{0.400pt}}
\multiput(1143.00,535.17)(0.500,-1.000){2}{\rule{0.120pt}{0.400pt}}
\put(1143.67,535){\rule{0.400pt}{4.095pt}}
\multiput(1143.17,535.00)(1.000,8.500){2}{\rule{0.400pt}{2.048pt}}
\put(1144.67,538){\rule{0.400pt}{3.373pt}}
\multiput(1144.17,545.00)(1.000,-7.000){2}{\rule{0.400pt}{1.686pt}}
\put(1145.67,538){\rule{0.400pt}{6.023pt}}
\multiput(1145.17,538.00)(1.000,12.500){2}{\rule{0.400pt}{3.011pt}}
\put(1146.67,554){\rule{0.400pt}{2.168pt}}
\multiput(1146.17,558.50)(1.000,-4.500){2}{\rule{0.400pt}{1.084pt}}
\put(1147.67,533){\rule{0.400pt}{5.059pt}}
\multiput(1147.17,543.50)(1.000,-10.500){2}{\rule{0.400pt}{2.529pt}}
\put(1148.67,533){\rule{0.400pt}{4.577pt}}
\multiput(1148.17,533.00)(1.000,9.500){2}{\rule{0.400pt}{2.289pt}}
\put(1149.67,523){\rule{0.400pt}{6.986pt}}
\multiput(1149.17,537.50)(1.000,-14.500){2}{\rule{0.400pt}{3.493pt}}
\put(1150.67,523){\rule{0.400pt}{13.972pt}}
\multiput(1150.17,523.00)(1.000,29.000){2}{\rule{0.400pt}{6.986pt}}
\put(1151.67,502){\rule{0.400pt}{19.031pt}}
\multiput(1151.17,541.50)(1.000,-39.500){2}{\rule{0.400pt}{9.516pt}}
\put(1153.17,502){\rule{0.400pt}{1.500pt}}
\multiput(1152.17,502.00)(2.000,3.887){2}{\rule{0.400pt}{0.750pt}}
\put(1154.67,391){\rule{0.400pt}{28.426pt}}
\multiput(1154.17,450.00)(1.000,-59.000){2}{\rule{0.400pt}{14.213pt}}
\put(1155.67,391){\rule{0.400pt}{0.723pt}}
\multiput(1155.17,391.00)(1.000,1.500){2}{\rule{0.400pt}{0.361pt}}
\put(1156.67,267){\rule{0.400pt}{30.594pt}}
\multiput(1156.17,330.50)(1.000,-63.500){2}{\rule{0.400pt}{15.297pt}}
\put(1157.67,260){\rule{0.400pt}{1.686pt}}
\multiput(1157.17,263.50)(1.000,-3.500){2}{\rule{0.400pt}{0.843pt}}
\put(1158.67,260){\rule{0.400pt}{4.095pt}}
\multiput(1158.17,260.00)(1.000,8.500){2}{\rule{0.400pt}{2.048pt}}
\put(1159.67,259){\rule{0.400pt}{4.336pt}}
\multiput(1159.17,268.00)(1.000,-9.000){2}{\rule{0.400pt}{2.168pt}}
\put(1160.67,259){\rule{0.400pt}{3.373pt}}
\multiput(1160.17,259.00)(1.000,7.000){2}{\rule{0.400pt}{1.686pt}}
\put(1161.67,269){\rule{0.400pt}{0.964pt}}
\multiput(1161.17,271.00)(1.000,-2.000){2}{\rule{0.400pt}{0.482pt}}
\put(1162.67,241){\rule{0.400pt}{6.745pt}}
\multiput(1162.17,255.00)(1.000,-14.000){2}{\rule{0.400pt}{3.373pt}}
\put(1163.67,241){\rule{0.400pt}{0.964pt}}
\multiput(1163.17,241.00)(1.000,2.000){2}{\rule{0.400pt}{0.482pt}}
\put(1164.67,228){\rule{0.400pt}{4.095pt}}
\multiput(1164.17,236.50)(1.000,-8.500){2}{\rule{0.400pt}{2.048pt}}
\put(1165.67,226){\rule{0.400pt}{0.482pt}}
\multiput(1165.17,227.00)(1.000,-1.000){2}{\rule{0.400pt}{0.241pt}}
\put(1167.17,209){\rule{0.400pt}{3.500pt}}
\multiput(1166.17,218.74)(2.000,-9.736){2}{\rule{0.400pt}{1.750pt}}
\put(1141.0,572.0){\usebox{\plotpoint}}
\put(1169.67,194){\rule{0.400pt}{3.614pt}}
\multiput(1169.17,201.50)(1.000,-7.500){2}{\rule{0.400pt}{1.807pt}}
\put(1170.67,190){\rule{0.400pt}{0.964pt}}
\multiput(1170.17,192.00)(1.000,-2.000){2}{\rule{0.400pt}{0.482pt}}
\put(1171.67,188){\rule{0.400pt}{0.482pt}}
\multiput(1171.17,189.00)(1.000,-1.000){2}{\rule{0.400pt}{0.241pt}}
\put(1172.67,174){\rule{0.400pt}{3.373pt}}
\multiput(1172.17,181.00)(1.000,-7.000){2}{\rule{0.400pt}{1.686pt}}
\put(1174,173.67){\rule{0.241pt}{0.400pt}}
\multiput(1174.00,173.17)(0.500,1.000){2}{\rule{0.120pt}{0.400pt}}
\put(1174.67,159){\rule{0.400pt}{3.854pt}}
\multiput(1174.17,167.00)(1.000,-8.000){2}{\rule{0.400pt}{1.927pt}}
\put(1175.67,159){\rule{0.400pt}{2.168pt}}
\multiput(1175.17,159.00)(1.000,4.500){2}{\rule{0.400pt}{1.084pt}}
\put(1176.67,154){\rule{0.400pt}{3.373pt}}
\multiput(1176.17,161.00)(1.000,-7.000){2}{\rule{0.400pt}{1.686pt}}
\put(1177.67,154){\rule{0.400pt}{2.409pt}}
\multiput(1177.17,154.00)(1.000,5.000){2}{\rule{0.400pt}{1.204pt}}
\put(1178.67,148){\rule{0.400pt}{3.854pt}}
\multiput(1178.17,156.00)(1.000,-8.000){2}{\rule{0.400pt}{1.927pt}}
\put(1179.67,148){\rule{0.400pt}{2.650pt}}
\multiput(1179.17,148.00)(1.000,5.500){2}{\rule{0.400pt}{1.325pt}}
\put(1180.67,141){\rule{0.400pt}{4.336pt}}
\multiput(1180.17,150.00)(1.000,-9.000){2}{\rule{0.400pt}{2.168pt}}
\put(1182.17,141){\rule{0.400pt}{2.900pt}}
\multiput(1181.17,141.00)(2.000,7.981){2}{\rule{0.400pt}{1.450pt}}
\put(1184,154.67){\rule{0.482pt}{0.400pt}}
\multiput(1184.00,154.17)(1.000,1.000){2}{\rule{0.241pt}{0.400pt}}
\put(1186.17,152){\rule{0.400pt}{0.900pt}}
\multiput(1185.17,154.13)(2.000,-2.132){2}{\rule{0.400pt}{0.450pt}}
\put(1188.17,142){\rule{0.400pt}{2.100pt}}
\multiput(1187.17,147.64)(2.000,-5.641){2}{\rule{0.400pt}{1.050pt}}
\put(1190.17,142){\rule{0.400pt}{0.700pt}}
\multiput(1189.17,142.00)(2.000,1.547){2}{\rule{0.400pt}{0.350pt}}
\put(1192,143.17){\rule{0.482pt}{0.400pt}}
\multiput(1192.00,144.17)(1.000,-2.000){2}{\rule{0.241pt}{0.400pt}}
\put(1194,141.17){\rule{0.482pt}{0.400pt}}
\multiput(1194.00,142.17)(1.000,-2.000){2}{\rule{0.241pt}{0.400pt}}
\multiput(1196.00,139.92)(3.779,-0.491){17}{\rule{3.020pt}{0.118pt}}
\multiput(1196.00,140.17)(66.732,-10.000){2}{\rule{1.510pt}{0.400pt}}
\put(1169.0,209.0){\usebox{\plotpoint}}
\put(1269.0,131.0){\rule[-0.200pt]{40.953pt}{0.400pt}}
\put(151.0,131.0){\rule[-0.200pt]{0.400pt}{175.375pt}}
\put(151.0,131.0){\rule[-0.200pt]{310.279pt}{0.400pt}}
\put(1439.0,131.0){\rule[-0.200pt]{0.400pt}{175.375pt}}
\put(151.0,859.0){\rule[-0.200pt]{310.279pt}{0.400pt}}
\end{picture}

\caption{
Závislosť teploty plazmy $T\(t\)$ na čase $t$ pre výstrel \#1. S určenými časmi začiatku a konca života plazmy.
}\label{G_V-1-T}
\end{figure}





