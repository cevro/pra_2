
\section{Pracovní úkoly}
\begin{enumerate}
    \setlength{\itemsep}{-2pt}
    \item \underline{V domácí přípravě} se naučte manipulaci se vzdálenými datovými soubory  \href{http://golem.fjfi.cvut.cz/wiki/TrainingCourses/KFpract/index}{\cite{gw:KFpraktdopr}}. Pokud máte 
    možnost, přineste si s sebou na měření notebook, na kterém máte tyto funkce manipulace se vzdálenými soubory dat zprovozněné. Na stejné stránce najdete přidělené web rozhraní, ze kterého budete ovládat tokamak. Seznamte se s ním.
    \item \underline{V laboratoři tokamaku} se seznamte fyzicky s tokamakem GOLEM a zmapujte na něm jeho základní prvky: komoru, transformátorové jádro, cívky toroidálního magnetického pole, primární cívky, čerpací systém, energetický zdroj, kondenzátorové baterie, systém napouštění pracovního plynu, řídící systémy, datový sběr a server.
    S pomocí asistenta prověřte funkci jednotlivých komponent infrastruktury tokamaku:
    \vspace{-2mm}
    \begin{enumerate}
	\setlength{\itemsep}{-2pt}
	\item vypněte a zapněte čerpání tokamaku,
	\item napusťte do tokamaku pracovní plyn,
	\item vyzkoušejte předionizační trysku.
	\end{enumerate}
    \item \underline{V laboratoři tokamaku} osaďte tokamak základními diagnostickými prostředky (drát na měření napětí na závit $U_l$, cívečka měření toroidálního magnetického pole $B_t$, Rogowského pásek pro měření $I_p$ a fotodiodu s H$_\alpha$ filtrem), napojte vše na 4-kanálový osciloskop Tektronix (resp. na datové sběry) a zaznamenávejte časové vývoje signálů jednotlivých diagnostik. Proveďte následující seznamovací experimenty (pro přístup k datům na vzdáleném serveru použijte metodu z pracovního úkolu č.1):
    \vspace{-2mm} 
    \begin{itemize}
    \item Vygenerujte na tokamaku samostatné toroidální elektrické pole $E_t$ a  zaznamenejte časový průběh napětí na závit $U_l(t)$. Z jeho průběhu a signálu z Rogowského pásku $I_{tot}(t)$  odhadněte z Ohmova zákona v prvním přiblížení odpor komory $R_{ch}$ se zanedbáním její indukčnosti.
    \item Vygenerujte na tokamaku samostatné toroidální magnetické pole $B_t$ a  zaznamenejte časový průběh napětí na měřící cívce $U_B(t)$.
    \item Vytvořte komplexní zadání pro výboj (pracovní plyn + předionizace + toroidální elektrické pole + toroidální magnetické pole) v tokamaku a zadejte k provedení.  Z napětí na závit $U_l(t)$ a průběhu proudu na Rogowského pásku $I_{tot}(t)$ vypočítejte časový vývoj proudu plazmatem $I_p(t)$ se zanedbáním jeho indukčnosti. Následně znázorněte časový vývoj elektronové teploty $T_e(t)$.
    \end{itemize}
    \vspace{-1mm}
    Všechny závislosti získané z improvizované diagnostiky srovnávejte s původním diagnostickým osazením tokamaku GOLEM.
    \item  \underline{Vzdáleným řízením} proveďte 10 výbojů, ve kterých se budete snažit pokrýt maximálně  prostor parametrů (zadávejte co nejpestřejší spektrum parametrů výbojů), přičemž se pokuste  dosáhnout co nejvyšší elektronové teploty. 
    \item \underline{Doma, při zpracovávání výsledků} vytvořte tabulku 5 výstřelů s nejvyšší $T_e$ a u každého uveďte  vámi vypočtené parametry: délku výboje, maximální proud plazmatem, maximální elektronovou teplotu, maximální ohmický příkon, maximální energii plazmatu a dobu udržení v době maxima energie plazmatu.
   
\end{enumerate}

   
\section{Pomůcky}
\noindent {\bf{Pomůcky:}} Zařízení pro generaci a udržení vysokoteplotního
plazmatu -- tokamak GOLEM, pracovní plyn -- vodík, $U_{l}$ cívka,
$B_t$ cívka, Rogowského pásek, fotodioda, H$_\alpha$ filtr, měrka vakua, systémy datových sběrů, osciloskop Tektronix.

\section{Teoretický úvod}
Kompltný teoretický úvod je dostupný na \cite{C_1}\footnote{Nepodarilo sa mi rozbehnuť preklad toho zdrojáku.}.


%%%%%%%%%%%%%%%%%%%%%%%%%%%%%%%%%%%%%%%
\subsubsection{Spracovanie chýb merania}

Označme $\mean{t}$ aritmetický priemer nameraných hodnôt $t_i$, a $\Delta t$ hodnotu $\mean{t}-t$, pričom 
\eq{
\mean{t} = \frac{1}{n}\sum_{i=1}^n t_i \,, \lbl{SCH_1}
}  
a chybu aritmetického priemeru 
\eq{
  \sigma_0=\sqrt{\frac{\sum_{i=1}^n \(t_i - \mean{t}\)^2}{n\(n-1\)}}\,, \lbl{SCH_2}
}
pričom $n$ je počet meraní.



