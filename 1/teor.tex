\section{Teória}
Odvodenie vzťahov nájdete na \cite{C_1}.

Rýchlosť zvuku $v$ od teploty $t$ môžeme spočítať ako
\eq{
v= "\(331.57+0,607t\) m/s"\,. \lbl{R_1}
}

Pre strunu s dĺžkovou hustotou $\mu$ dlžkou $L$ a pri zaťažení silou $F$ vypočítame frekvenciu kmitov $f$ ako
\eq{
f = \frac{n}{2L} \sqrt{\frac{F}{\mu}} = C n \,, \lbl{R_2}
}
kde $n$ je počet kmitní a $\mu$ je dlžková hustota.


Rezonančnú frekvenciu Helmholtzového razonátoru môžeme vypočítať podľa
\eq{
f = \frac{v}{2\pi}\sqrt{\frac{\pi r^2}{l+1.4r}} \frac{1}{\sqrt{V}}\,, \lbl{R_30}
}
kde $v$ je rýchlosť zvuku, $l$ je dlžka a $r$ polomer hrdla banky a $V$ je objem banky.

%%%%%%%%%%%%%%%%%%%%%%%%%%%%%%%%%%%%%%%=
\subsubsection{Spracovanie chýb merania}

Označme $\mean{t}$ aritmetický priemer nameraných hodnôt $t_i$, a $\Delta t$ hodnotu $\mean{t}-t$, pričom 
\eq{
\mean{t} = \frac{1}{n}\sum_{i=1}^n t_i \,, \lbl{SCH_1}
}  
a chybu aritmetického priemeru 
\eq{
  \sigma_0=\sqrt{\frac{\sum_{i=1}^n \(t_i - \mean{t}\)^2}{n\(n-1\)}}\,, \lbl{SCH_2}
}
pričom $n$ je počet meraní.



