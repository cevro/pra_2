
\section{Výsledky merania}
Nazačitaku bola digitálnym teplomerom určená teplota $t="23.1 \C"$.
Z nej podľa vzťahu \ref{R_1} bola dopočítaná odpovedajúca rýchlosť zvuku
\eq{
v = "345.6 m\cdot s^{-1}"\,.
}

\subsection{Stojaté vlněnie na strune}

Dĺžka struny je $L=" \(1.176\pm0.01\) m"$. Z tejto dĺžky bola vypočítaná podľa vzťahu \ref{R_2} predpokladaná frekvencia prvého módu, $f ="23.3 Hz"$.

Namerané hodnoty vyšších módov v závislosti od frekvenciu sú v tabuľke Tab. \ref{T_1}.

Následne z grafu Obr. \ref{G_1} bola určená smernica $C="\(23.38\pm0.03\)n"$ a pomocou vzťahu \ref{R_2} a \ref{SCH_1} hodnota 
\eq{
\mu = "\(0.0163\pm0.0001\) kg\cdot m^{-1}"\,.
}

\begin{table}[h]
\begin{center}
\begin{tabular}{| c | c |}
\hline
\popi{n}{-} & \popi{f}{Hz} \\
\hline
1  & $ 23.30 $\\
2  & $ 46.80 $\\
3  & $ 69.90 $\\
4  & $ 93.00 $\\
5  & $ 116.1 $\\
6  & $ 140.5 $\\
7  & $ 162.7 $\\
8  & $ 187.6 $\\
9  & $ 210.4 $\\
10 & $ 234.9 $\\
\hline
\end{tabular}
\caption{Závislosť frekvencie $f$ od módu kmitania struny $n$.
} \label{T_1}
\end{center}
\end{table}


\begin{figure}
% GNUPLOT: LaTeX picture
\setlength{\unitlength}{0.240900pt}
\ifx\plotpoint\undefined\newsavebox{\plotpoint}\fi
\begin{picture}(1500,900)(0,0)
\sbox{\plotpoint}{\rule[-0.200pt]{0.400pt}{0.400pt}}%
\put(171.0,131.0){\rule[-0.200pt]{4.818pt}{0.400pt}}
\put(151,131){\makebox(0,0)[r]{ 0.5}}
\put(1419.0,131.0){\rule[-0.200pt]{4.818pt}{0.400pt}}
\put(171.0,235.0){\rule[-0.200pt]{4.818pt}{0.400pt}}
\put(151,235){\makebox(0,0)[r]{ 1}}
\put(1419.0,235.0){\rule[-0.200pt]{4.818pt}{0.400pt}}
\put(171.0,339.0){\rule[-0.200pt]{4.818pt}{0.400pt}}
\put(151,339){\makebox(0,0)[r]{ 1.5}}
\put(1419.0,339.0){\rule[-0.200pt]{4.818pt}{0.400pt}}
\put(171.0,443.0){\rule[-0.200pt]{4.818pt}{0.400pt}}
\put(151,443){\makebox(0,0)[r]{ 2}}
\put(1419.0,443.0){\rule[-0.200pt]{4.818pt}{0.400pt}}
\put(171.0,547.0){\rule[-0.200pt]{4.818pt}{0.400pt}}
\put(151,547){\makebox(0,0)[r]{ 2.5}}
\put(1419.0,547.0){\rule[-0.200pt]{4.818pt}{0.400pt}}
\put(171.0,651.0){\rule[-0.200pt]{4.818pt}{0.400pt}}
\put(151,651){\makebox(0,0)[r]{ 3}}
\put(1419.0,651.0){\rule[-0.200pt]{4.818pt}{0.400pt}}
\put(171.0,755.0){\rule[-0.200pt]{4.818pt}{0.400pt}}
\put(151,755){\makebox(0,0)[r]{ 3.5}}
\put(1419.0,755.0){\rule[-0.200pt]{4.818pt}{0.400pt}}
\put(171.0,859.0){\rule[-0.200pt]{4.818pt}{0.400pt}}
\put(151,859){\makebox(0,0)[r]{ 4}}
\put(1419.0,859.0){\rule[-0.200pt]{4.818pt}{0.400pt}}
\put(171.0,131.0){\rule[-0.200pt]{0.400pt}{4.818pt}}
\put(171,90){\makebox(0,0){ 0}}
\put(171.0,839.0){\rule[-0.200pt]{0.400pt}{4.818pt}}
\put(382.0,131.0){\rule[-0.200pt]{0.400pt}{4.818pt}}
\put(382,90){\makebox(0,0){ 2}}
\put(382.0,839.0){\rule[-0.200pt]{0.400pt}{4.818pt}}
\put(594.0,131.0){\rule[-0.200pt]{0.400pt}{4.818pt}}
\put(594,90){\makebox(0,0){ 4}}
\put(594.0,839.0){\rule[-0.200pt]{0.400pt}{4.818pt}}
\put(805.0,131.0){\rule[-0.200pt]{0.400pt}{4.818pt}}
\put(805,90){\makebox(0,0){ 6}}
\put(805.0,839.0){\rule[-0.200pt]{0.400pt}{4.818pt}}
\put(1016.0,131.0){\rule[-0.200pt]{0.400pt}{4.818pt}}
\put(1016,90){\makebox(0,0){ 8}}
\put(1016.0,839.0){\rule[-0.200pt]{0.400pt}{4.818pt}}
\put(1228.0,131.0){\rule[-0.200pt]{0.400pt}{4.818pt}}
\put(1228,90){\makebox(0,0){ 10}}
\put(1228.0,839.0){\rule[-0.200pt]{0.400pt}{4.818pt}}
\put(1439.0,131.0){\rule[-0.200pt]{0.400pt}{4.818pt}}
\put(1439,90){\makebox(0,0){ 12}}
\put(1439.0,839.0){\rule[-0.200pt]{0.400pt}{4.818pt}}
\put(171.0,131.0){\rule[-0.200pt]{0.400pt}{175.375pt}}
\put(171.0,131.0){\rule[-0.200pt]{305.461pt}{0.400pt}}
\put(1439.0,131.0){\rule[-0.200pt]{0.400pt}{175.375pt}}
\put(171.0,859.0){\rule[-0.200pt]{305.461pt}{0.400pt}}
\put(30,495){\makebox(0,0){\popi{I}{A}}}
\put(805,29){\makebox(0,0){\popi{U}{V}}}
\put(1279,172){\makebox(0,0)[r]{Namerané hodnoty}}
\put(1397,792){\makebox(0,0){$+$}}
\put(1192,772){\makebox(0,0){$+$}}
\put(1068,721){\makebox(0,0){$+$}}
\put(653,519){\makebox(0,0){$+$}}
\put(541,454){\makebox(0,0){$+$}}
\put(382,345){\makebox(0,0){$+$}}
\put(296,266){\makebox(0,0){$+$}}
\put(248,218){\makebox(0,0){$+$}}
\put(748,573){\makebox(0,0){$+$}}
\put(837,618){\makebox(0,0){$+$}}
\put(916,655){\makebox(0,0){$+$}}
\put(1047,713){\makebox(0,0){$+$}}
\put(1110,740){\makebox(0,0){$+$}}
\put(1349,172){\makebox(0,0){$+$}}
\put(171.0,131.0){\rule[-0.200pt]{0.400pt}{175.375pt}}
\put(171.0,131.0){\rule[-0.200pt]{305.461pt}{0.400pt}}
\put(1439.0,131.0){\rule[-0.200pt]{0.400pt}{175.375pt}}
\put(171.0,859.0){\rule[-0.200pt]{305.461pt}{0.400pt}}
\end{picture}

\caption{Závislosť frekvencie $f$ od módu kmitania struny $n$, preložená lineárnou závislosťou
}  \label{G_1}
\end{figure}

\subsection{Quinckova trubice}

Namerané hodnoty polohy minim vlnenia, pre jednotlivé polohy, viď priložené pracovné papiere.
Z nich boli pre jednotlivé frekvencie vypočítané podľa \ref{SCH_1} priemer a chyba. viď Tab. \ref{T_2}.

Do grafu Obr. \ref{G_2} bola vynesená závislosť vlnovej dĺžky $\lambda\equiv 2\Delta$ od doby kmitu $T$. 
Zo smernice preloženej závislosť bola určená rýchlosť zvuku
\eq{
v = "\(348.7\pm6.7\) m\cdot s^{-1}"\,.
}

\begin{table}[h]
\begin{center}
\begin{tabular}{ | c | c |}
\hline
\popi{f}{kHz} & \popi{\mean{\Delta d_i}}{m} \\
\hline
$2$ & $0.080\pm0.004$\\
$2.5$ & $0.070\pm0.001$\\
$3$ & $0.058\pm0.002$\\
$3.5$ & $0.050\pm0.001$\\
$4$ & $0.041\pm0.005$\\
$4.5$ & $0.039\pm0.001$\\
$5$ & $0.034\pm0.008$\\
$5.5$ & $0.031\pm0.001$\\
$6$ & $0.029\pm0.001$\\
$4.25$& $0.040\pm0.001$\\
\hline
\end{tabular}
\caption{Vypočítané hodnoty priemernej dĺžky vzdialenosti minim $\Delta d_i$ od frekvencie $f$
} \label{T_2}
\end{center}
\end{table}



\begin{figure}
% GNUPLOT: LaTeX picture
\setlength{\unitlength}{0.240900pt}
\ifx\plotpoint\undefined\newsavebox{\plotpoint}\fi
\begin{picture}(1500,900)(0,0)
\sbox{\plotpoint}{\rule[-0.200pt]{0.400pt}{0.400pt}}%
\put(191.0,131.0){\rule[-0.200pt]{4.818pt}{0.400pt}}
\put(171,131){\makebox(0,0)[r]{ 0}}
\put(1419.0,131.0){\rule[-0.200pt]{4.818pt}{0.400pt}}
\put(191.0,212.0){\rule[-0.200pt]{4.818pt}{0.400pt}}
\put(171,212){\makebox(0,0)[r]{ 0.02}}
\put(1419.0,212.0){\rule[-0.200pt]{4.818pt}{0.400pt}}
\put(191.0,293.0){\rule[-0.200pt]{4.818pt}{0.400pt}}
\put(171,293){\makebox(0,0)[r]{ 0.04}}
\put(1419.0,293.0){\rule[-0.200pt]{4.818pt}{0.400pt}}
\put(191.0,374.0){\rule[-0.200pt]{4.818pt}{0.400pt}}
\put(171,374){\makebox(0,0)[r]{ 0.06}}
\put(1419.0,374.0){\rule[-0.200pt]{4.818pt}{0.400pt}}
\put(191.0,455.0){\rule[-0.200pt]{4.818pt}{0.400pt}}
\put(171,455){\makebox(0,0)[r]{ 0.08}}
\put(1419.0,455.0){\rule[-0.200pt]{4.818pt}{0.400pt}}
\put(191.0,535.0){\rule[-0.200pt]{4.818pt}{0.400pt}}
\put(171,535){\makebox(0,0)[r]{ 0.1}}
\put(1419.0,535.0){\rule[-0.200pt]{4.818pt}{0.400pt}}
\put(191.0,616.0){\rule[-0.200pt]{4.818pt}{0.400pt}}
\put(171,616){\makebox(0,0)[r]{ 0.12}}
\put(1419.0,616.0){\rule[-0.200pt]{4.818pt}{0.400pt}}
\put(191.0,697.0){\rule[-0.200pt]{4.818pt}{0.400pt}}
\put(171,697){\makebox(0,0)[r]{ 0.14}}
\put(1419.0,697.0){\rule[-0.200pt]{4.818pt}{0.400pt}}
\put(191.0,778.0){\rule[-0.200pt]{4.818pt}{0.400pt}}
\put(171,778){\makebox(0,0)[r]{ 0.16}}
\put(1419.0,778.0){\rule[-0.200pt]{4.818pt}{0.400pt}}
\put(191.0,859.0){\rule[-0.200pt]{4.818pt}{0.400pt}}
\put(171,859){\makebox(0,0)[r]{ 0.18}}
\put(1419.0,859.0){\rule[-0.200pt]{4.818pt}{0.400pt}}
\put(191.0,131.0){\rule[-0.200pt]{0.400pt}{4.818pt}}
\put(191,90){\makebox(0,0){ 0}}
\put(191.0,839.0){\rule[-0.200pt]{0.400pt}{4.818pt}}
\put(330.0,131.0){\rule[-0.200pt]{0.400pt}{4.818pt}}
\put(330,90){\makebox(0,0){ 10}}
\put(330.0,839.0){\rule[-0.200pt]{0.400pt}{4.818pt}}
\put(468.0,131.0){\rule[-0.200pt]{0.400pt}{4.818pt}}
\put(468,90){\makebox(0,0){ 20}}
\put(468.0,839.0){\rule[-0.200pt]{0.400pt}{4.818pt}}
\put(607.0,131.0){\rule[-0.200pt]{0.400pt}{4.818pt}}
\put(607,90){\makebox(0,0){ 30}}
\put(607.0,839.0){\rule[-0.200pt]{0.400pt}{4.818pt}}
\put(746.0,131.0){\rule[-0.200pt]{0.400pt}{4.818pt}}
\put(746,90){\makebox(0,0){ 40}}
\put(746.0,839.0){\rule[-0.200pt]{0.400pt}{4.818pt}}
\put(884.0,131.0){\rule[-0.200pt]{0.400pt}{4.818pt}}
\put(884,90){\makebox(0,0){ 50}}
\put(884.0,839.0){\rule[-0.200pt]{0.400pt}{4.818pt}}
\put(1023.0,131.0){\rule[-0.200pt]{0.400pt}{4.818pt}}
\put(1023,90){\makebox(0,0){ 60}}
\put(1023.0,839.0){\rule[-0.200pt]{0.400pt}{4.818pt}}
\put(1162.0,131.0){\rule[-0.200pt]{0.400pt}{4.818pt}}
\put(1162,90){\makebox(0,0){ 70}}
\put(1162.0,839.0){\rule[-0.200pt]{0.400pt}{4.818pt}}
\put(1300.0,131.0){\rule[-0.200pt]{0.400pt}{4.818pt}}
\put(1300,90){\makebox(0,0){ 80}}
\put(1300.0,839.0){\rule[-0.200pt]{0.400pt}{4.818pt}}
\put(1439.0,131.0){\rule[-0.200pt]{0.400pt}{4.818pt}}
\put(1439,90){\makebox(0,0){ 90}}
\put(1439.0,839.0){\rule[-0.200pt]{0.400pt}{4.818pt}}
\put(191.0,131.0){\rule[-0.200pt]{0.400pt}{175.375pt}}
\put(191.0,131.0){\rule[-0.200pt]{300.643pt}{0.400pt}}
\put(1439.0,131.0){\rule[-0.200pt]{0.400pt}{175.375pt}}
\put(191.0,859.0){\rule[-0.200pt]{300.643pt}{0.400pt}}
\put(30,495){\makebox(0,0){\popi{U}{V}}}
\put(815,29){\makebox(0,0){\popi{\phi}{\dg}}}
\put(1131,213){\makebox(0,0)[r]{namerané hodnoty}}
\put(1151.0,213.0){\rule[-0.200pt]{24.090pt}{0.400pt}}
\put(1151.0,203.0){\rule[-0.200pt]{0.400pt}{4.818pt}}
\put(1251.0,203.0){\rule[-0.200pt]{0.400pt}{4.818pt}}
\put(1439.0,147.0){\rule[-0.200pt]{0.400pt}{19.513pt}}
\put(1429.0,147.0){\rule[-0.200pt]{2.409pt}{0.400pt}}
\put(1429.0,228.0){\rule[-0.200pt]{2.409pt}{0.400pt}}
\put(1370.0,224.0){\rule[-0.200pt]{0.400pt}{19.513pt}}
\put(1360.0,224.0){\rule[-0.200pt]{4.818pt}{0.400pt}}
\put(1360.0,305.0){\rule[-0.200pt]{4.818pt}{0.400pt}}
\put(1300.0,317.0){\rule[-0.200pt]{0.400pt}{19.513pt}}
\put(1290.0,317.0){\rule[-0.200pt]{4.818pt}{0.400pt}}
\put(1290.0,398.0){\rule[-0.200pt]{4.818pt}{0.400pt}}
\put(1231.0,398.0){\rule[-0.200pt]{0.400pt}{19.513pt}}
\put(1221.0,398.0){\rule[-0.200pt]{4.818pt}{0.400pt}}
\put(1221.0,479.0){\rule[-0.200pt]{4.818pt}{0.400pt}}
\put(1162.0,467.0){\rule[-0.200pt]{0.400pt}{19.513pt}}
\put(1152.0,467.0){\rule[-0.200pt]{4.818pt}{0.400pt}}
\put(1152.0,548.0){\rule[-0.200pt]{4.818pt}{0.400pt}}
\put(1092.0,519.0){\rule[-0.200pt]{0.400pt}{19.513pt}}
\put(1082.0,519.0){\rule[-0.200pt]{4.818pt}{0.400pt}}
\put(1082.0,600.0){\rule[-0.200pt]{4.818pt}{0.400pt}}
\put(1023.0,572.0){\rule[-0.200pt]{0.400pt}{19.513pt}}
\put(1013.0,572.0){\rule[-0.200pt]{4.818pt}{0.400pt}}
\put(1013.0,653.0){\rule[-0.200pt]{4.818pt}{0.400pt}}
\put(954.0,608.0){\rule[-0.200pt]{0.400pt}{19.513pt}}
\put(944.0,608.0){\rule[-0.200pt]{4.818pt}{0.400pt}}
\put(944.0,689.0){\rule[-0.200pt]{4.818pt}{0.400pt}}
\put(884.0,641.0){\rule[-0.200pt]{0.400pt}{19.272pt}}
\put(874.0,641.0){\rule[-0.200pt]{4.818pt}{0.400pt}}
\put(874.0,721.0){\rule[-0.200pt]{4.818pt}{0.400pt}}
\put(815.0,665.0){\rule[-0.200pt]{0.400pt}{19.513pt}}
\put(805.0,665.0){\rule[-0.200pt]{4.818pt}{0.400pt}}
\put(805.0,746.0){\rule[-0.200pt]{4.818pt}{0.400pt}}
\put(746.0,689.0){\rule[-0.200pt]{0.400pt}{19.513pt}}
\put(736.0,689.0){\rule[-0.200pt]{4.818pt}{0.400pt}}
\put(736.0,770.0){\rule[-0.200pt]{4.818pt}{0.400pt}}
\put(676.0,709.0){\rule[-0.200pt]{0.400pt}{19.513pt}}
\put(666.0,709.0){\rule[-0.200pt]{4.818pt}{0.400pt}}
\put(666.0,790.0){\rule[-0.200pt]{4.818pt}{0.400pt}}
\put(607.0,726.0){\rule[-0.200pt]{0.400pt}{19.272pt}}
\put(597.0,726.0){\rule[-0.200pt]{4.818pt}{0.400pt}}
\put(597.0,806.0){\rule[-0.200pt]{4.818pt}{0.400pt}}
\put(538.0,734.0){\rule[-0.200pt]{0.400pt}{19.513pt}}
\put(528.0,734.0){\rule[-0.200pt]{4.818pt}{0.400pt}}
\put(528.0,815.0){\rule[-0.200pt]{4.818pt}{0.400pt}}
\put(468.0,746.0){\rule[-0.200pt]{0.400pt}{19.513pt}}
\put(458.0,746.0){\rule[-0.200pt]{4.818pt}{0.400pt}}
\put(458.0,827.0){\rule[-0.200pt]{4.818pt}{0.400pt}}
\put(399.0,750.0){\rule[-0.200pt]{0.400pt}{19.513pt}}
\put(389.0,750.0){\rule[-0.200pt]{4.818pt}{0.400pt}}
\put(389.0,831.0){\rule[-0.200pt]{4.818pt}{0.400pt}}
\put(330.0,754.0){\rule[-0.200pt]{0.400pt}{19.513pt}}
\put(320.0,754.0){\rule[-0.200pt]{4.818pt}{0.400pt}}
\put(320.0,835.0){\rule[-0.200pt]{4.818pt}{0.400pt}}
\put(260.0,758.0){\rule[-0.200pt]{0.400pt}{19.513pt}}
\put(250.0,758.0){\rule[-0.200pt]{4.818pt}{0.400pt}}
\put(250.0,839.0){\rule[-0.200pt]{4.818pt}{0.400pt}}
\put(191.0,758.0){\rule[-0.200pt]{0.400pt}{19.513pt}}
\put(191.0,758.0){\rule[-0.200pt]{2.409pt}{0.400pt}}
\put(191.0,839.0){\rule[-0.200pt]{2.409pt}{0.400pt}}
\put(1425.0,188.0){\rule[-0.200pt]{3.373pt}{0.400pt}}
\put(1425.0,178.0){\rule[-0.200pt]{0.400pt}{4.818pt}}
\put(1439.0,178.0){\rule[-0.200pt]{0.400pt}{4.818pt}}
\put(1356.0,264.0){\rule[-0.200pt]{6.745pt}{0.400pt}}
\put(1356.0,254.0){\rule[-0.200pt]{0.400pt}{4.818pt}}
\put(1384.0,254.0){\rule[-0.200pt]{0.400pt}{4.818pt}}
\put(1286.0,357.0){\rule[-0.200pt]{6.745pt}{0.400pt}}
\put(1286.0,347.0){\rule[-0.200pt]{0.400pt}{4.818pt}}
\put(1314.0,347.0){\rule[-0.200pt]{0.400pt}{4.818pt}}
\put(1217.0,438.0){\rule[-0.200pt]{6.745pt}{0.400pt}}
\put(1217.0,428.0){\rule[-0.200pt]{0.400pt}{4.818pt}}
\put(1245.0,428.0){\rule[-0.200pt]{0.400pt}{4.818pt}}
\put(1148.0,507.0){\rule[-0.200pt]{6.745pt}{0.400pt}}
\put(1148.0,497.0){\rule[-0.200pt]{0.400pt}{4.818pt}}
\put(1176.0,497.0){\rule[-0.200pt]{0.400pt}{4.818pt}}
\put(1078.0,560.0){\rule[-0.200pt]{6.745pt}{0.400pt}}
\put(1078.0,550.0){\rule[-0.200pt]{0.400pt}{4.818pt}}
\put(1106.0,550.0){\rule[-0.200pt]{0.400pt}{4.818pt}}
\put(1009.0,612.0){\rule[-0.200pt]{6.745pt}{0.400pt}}
\put(1009.0,602.0){\rule[-0.200pt]{0.400pt}{4.818pt}}
\put(1037.0,602.0){\rule[-0.200pt]{0.400pt}{4.818pt}}
\put(940.0,649.0){\rule[-0.200pt]{6.745pt}{0.400pt}}
\put(940.0,639.0){\rule[-0.200pt]{0.400pt}{4.818pt}}
\put(968.0,639.0){\rule[-0.200pt]{0.400pt}{4.818pt}}
\put(870.0,681.0){\rule[-0.200pt]{6.745pt}{0.400pt}}
\put(870.0,671.0){\rule[-0.200pt]{0.400pt}{4.818pt}}
\put(898.0,671.0){\rule[-0.200pt]{0.400pt}{4.818pt}}
\put(801.0,705.0){\rule[-0.200pt]{6.745pt}{0.400pt}}
\put(801.0,695.0){\rule[-0.200pt]{0.400pt}{4.818pt}}
\put(829.0,695.0){\rule[-0.200pt]{0.400pt}{4.818pt}}
\put(732.0,730.0){\rule[-0.200pt]{6.745pt}{0.400pt}}
\put(732.0,720.0){\rule[-0.200pt]{0.400pt}{4.818pt}}
\put(760.0,720.0){\rule[-0.200pt]{0.400pt}{4.818pt}}
\put(662.0,750.0){\rule[-0.200pt]{6.745pt}{0.400pt}}
\put(662.0,740.0){\rule[-0.200pt]{0.400pt}{4.818pt}}
\put(690.0,740.0){\rule[-0.200pt]{0.400pt}{4.818pt}}
\put(593.0,766.0){\rule[-0.200pt]{6.745pt}{0.400pt}}
\put(593.0,756.0){\rule[-0.200pt]{0.400pt}{4.818pt}}
\put(621.0,756.0){\rule[-0.200pt]{0.400pt}{4.818pt}}
\put(524.0,774.0){\rule[-0.200pt]{6.745pt}{0.400pt}}
\put(524.0,764.0){\rule[-0.200pt]{0.400pt}{4.818pt}}
\put(552.0,764.0){\rule[-0.200pt]{0.400pt}{4.818pt}}
\put(454.0,786.0){\rule[-0.200pt]{6.745pt}{0.400pt}}
\put(454.0,776.0){\rule[-0.200pt]{0.400pt}{4.818pt}}
\put(482.0,776.0){\rule[-0.200pt]{0.400pt}{4.818pt}}
\put(385.0,790.0){\rule[-0.200pt]{6.745pt}{0.400pt}}
\put(385.0,780.0){\rule[-0.200pt]{0.400pt}{4.818pt}}
\put(413.0,780.0){\rule[-0.200pt]{0.400pt}{4.818pt}}
\put(316.0,794.0){\rule[-0.200pt]{6.745pt}{0.400pt}}
\put(316.0,784.0){\rule[-0.200pt]{0.400pt}{4.818pt}}
\put(344.0,784.0){\rule[-0.200pt]{0.400pt}{4.818pt}}
\put(246.0,798.0){\rule[-0.200pt]{6.745pt}{0.400pt}}
\put(246.0,788.0){\rule[-0.200pt]{0.400pt}{4.818pt}}
\put(274.0,788.0){\rule[-0.200pt]{0.400pt}{4.818pt}}
\put(191.0,798.0){\rule[-0.200pt]{3.373pt}{0.400pt}}
\put(191.0,788.0){\rule[-0.200pt]{0.400pt}{4.818pt}}
\put(1439,188){\makebox(0,0){$+$}}
\put(1370,264){\makebox(0,0){$+$}}
\put(1300,357){\makebox(0,0){$+$}}
\put(1231,438){\makebox(0,0){$+$}}
\put(1162,507){\makebox(0,0){$+$}}
\put(1092,560){\makebox(0,0){$+$}}
\put(1023,612){\makebox(0,0){$+$}}
\put(954,649){\makebox(0,0){$+$}}
\put(884,681){\makebox(0,0){$+$}}
\put(815,705){\makebox(0,0){$+$}}
\put(746,730){\makebox(0,0){$+$}}
\put(676,750){\makebox(0,0){$+$}}
\put(607,766){\makebox(0,0){$+$}}
\put(538,774){\makebox(0,0){$+$}}
\put(468,786){\makebox(0,0){$+$}}
\put(399,790){\makebox(0,0){$+$}}
\put(330,794){\makebox(0,0){$+$}}
\put(260,798){\makebox(0,0){$+$}}
\put(191,798){\makebox(0,0){$+$}}
\put(1201,213){\makebox(0,0){$+$}}
\put(205.0,788.0){\rule[-0.200pt]{0.400pt}{4.818pt}}
\put(1131,172){\makebox(0,0)[r]{$f\(\phi\)=\(0.14\pm0.07\)\mathrm{cos}^2\(\phi\)+\(0.034\pm0.005\)$}}
\multiput(1151,172)(20.756,0.000){5}{\usebox{\plotpoint}}
\put(1251,172){\usebox{\plotpoint}}
\put(191,839){\usebox{\plotpoint}}
\put(191.00,839.00){\usebox{\plotpoint}}
\put(211.76,839.00){\usebox{\plotpoint}}
\put(232.47,838.00){\usebox{\plotpoint}}
\put(253.19,837.06){\usebox{\plotpoint}}
\put(273.92,836.42){\usebox{\plotpoint}}
\put(294.59,834.57){\usebox{\plotpoint}}
\put(315.18,832.14){\usebox{\plotpoint}}
\put(335.75,829.52){\usebox{\plotpoint}}
\put(356.32,826.78){\usebox{\plotpoint}}
\put(376.81,823.49){\usebox{\plotpoint}}
\put(397.32,820.28){\usebox{\plotpoint}}
\put(417.63,816.08){\usebox{\plotpoint}}
\put(438.01,812.25){\usebox{\plotpoint}}
\put(458.20,807.45){\usebox{\plotpoint}}
\put(478.38,802.60){\usebox{\plotpoint}}
\put(498.34,796.92){\usebox{\plotpoint}}
\put(518.53,792.11){\usebox{\plotpoint}}
\put(538.28,785.76){\usebox{\plotpoint}}
\put(558.11,779.63){\usebox{\plotpoint}}
\put(577.87,773.27){\usebox{\plotpoint}}
\put(597.53,766.64){\usebox{\plotpoint}}
\put(617.14,759.88){\usebox{\plotpoint}}
\put(636.45,752.29){\usebox{\plotpoint}}
\put(655.70,744.54){\usebox{\plotpoint}}
\put(675.06,737.05){\usebox{\plotpoint}}
\put(694.30,729.29){\usebox{\plotpoint}}
\put(713.25,720.81){\usebox{\plotpoint}}
\put(732.20,712.37){\usebox{\plotpoint}}
\put(751.32,704.34){\usebox{\plotpoint}}
\put(770.07,695.43){\usebox{\plotpoint}}
\put(788.73,686.36){\usebox{\plotpoint}}
\put(807.22,676.96){\usebox{\plotpoint}}
\put(825.69,667.48){\usebox{\plotpoint}}
\put(844.27,658.26){\usebox{\plotpoint}}
\put(862.50,648.38){\usebox{\plotpoint}}
\put(880.89,638.81){\usebox{\plotpoint}}
\put(899.11,628.87){\usebox{\plotpoint}}
\put(917.24,618.78){\usebox{\plotpoint}}
\put(934.99,608.01){\usebox{\plotpoint}}
\put(953.03,597.75){\usebox{\plotpoint}}
\put(970.93,587.27){\usebox{\plotpoint}}
\put(988.78,576.68){\usebox{\plotpoint}}
\put(1006.57,566.00){\usebox{\plotpoint}}
\put(1024.30,555.20){\usebox{\plotpoint}}
\put(1041.84,544.11){\usebox{\plotpoint}}
\put(1059.37,533.00){\usebox{\plotpoint}}
\put(1076.76,521.68){\usebox{\plotpoint}}
\put(1094.44,510.81){\usebox{\plotpoint}}
\put(1111.80,499.44){\usebox{\plotpoint}}
\put(1128.93,487.71){\usebox{\plotpoint}}
\put(1146.44,476.58){\usebox{\plotpoint}}
\put(1163.61,464.93){\usebox{\plotpoint}}
\put(1180.80,453.29){\usebox{\plotpoint}}
\put(1197.99,441.67){\usebox{\plotpoint}}
\put(1215.07,429.87){\usebox{\plotpoint}}
\put(1232.22,418.19){\usebox{\plotpoint}}
\put(1249.34,406.45){\usebox{\plotpoint}}
\put(1266.45,394.70){\usebox{\plotpoint}}
\put(1283.62,383.04){\usebox{\plotpoint}}
\put(1300.35,370.76){\usebox{\plotpoint}}
\put(1317.41,358.95){\usebox{\plotpoint}}
\put(1334.25,346.81){\usebox{\plotpoint}}
\put(1351.20,334.85){\usebox{\plotpoint}}
\put(1367.94,322.58){\usebox{\plotpoint}}
\put(1385.01,310.76){\usebox{\plotpoint}}
\put(1401.74,298.49){\usebox{\plotpoint}}
\put(1418.67,286.49){\usebox{\plotpoint}}
\put(1435.54,274.40){\usebox{\plotpoint}}
\put(1439,272){\usebox{\plotpoint}}
\put(191.0,131.0){\rule[-0.200pt]{0.400pt}{175.375pt}}
\put(191.0,131.0){\rule[-0.200pt]{300.643pt}{0.400pt}}
\put(1439.0,131.0){\rule[-0.200pt]{0.400pt}{175.375pt}}
\put(191.0,859.0){\rule[-0.200pt]{300.643pt}{0.400pt}}
\end{picture}

\caption{Závislosť vlnovej dĺžky $\lambda$ od doby kmitu $T$ preložená funkciou $\(348.7\pm6.7\)T - 0.001$.
}  \label{G_2}
\end{figure}

\subsection{Helmholtzovy rezonátory}
V tabuľke Tab \ref{T_3} sa nachádzajú, a sú vynesené do grafu Obr. \ref{G_3}.
Z neho bola preložením určená hodnota rýchlosti zvuku $v="\(293.2\pm9.0\) m/s"$.


\begin{table}[h]
\begin{center}
\begin{tabular}{ | c | c | c | c |}
\hline
\popi{f_n}{Hz} & \popi{V_v}{l} & \popi{V}{l} & \popi{f_v}{Hz}\\
\hline
$175.8 $ & $ 0.05 $ & $ 0.98 $ & $ 187.78 $\\
$175.8 $ & $ 0.1  $ & $ 0.93 $ & $ 192.76 $\\
$177.8 $ & $ 0.15 $ & $ 0.88 $ & $ 198.16 $\\
$179.7 $ & $ 0.2  $ & $ 0.83 $ & $ 204.04 $\\
$179.7 $ & $ 0.25 $ & $ 0.78 $ & $ 210.48 $\\
$179.7 $ & $ 0.3  $ & $ 0.73 $ & $ 217.57 $\\
$179.7 $ & $ 0.35 $ & $ 0.68 $ & $ 225.42 $\\
$179.7 $ & $ 0.4  $ & $ 0.63 $ & $ 234.20 $\\
$179.7 $ & $ 0.45 $ & $ 0.58 $ & $ 244.09 $\\
$179.8 $ & $ 0.5  $ & $ 0.53 $ & $ 255.34 $\\
\hline
\end{tabular}
\caption{Namerané hodnoty frekvencie rezonancie $f_n$, 
vypočítané hodnoty rezonancie $f_v$ podla \ref{R_30} pri objeme vody $V_v$ a z neho dopočítaný objem $V$.
} \label{T_3}
\end{center}
\end{table}




\begin{figure}
% GNUPLOT: LaTeX picture
\setlength{\unitlength}{0.240900pt}
\ifx\plotpoint\undefined\newsavebox{\plotpoint}\fi
\begin{picture}(1500,900)(0,0)
\sbox{\plotpoint}{\rule[-0.200pt]{0.400pt}{0.400pt}}%
\put(171.0,131.0){\rule[-0.200pt]{4.818pt}{0.400pt}}
\put(151,131){\makebox(0,0)[r]{ 140}}
\put(1419.0,131.0){\rule[-0.200pt]{4.818pt}{0.400pt}}
\put(171.0,252.0){\rule[-0.200pt]{4.818pt}{0.400pt}}
\put(151,252){\makebox(0,0)[r]{ 160}}
\put(1419.0,252.0){\rule[-0.200pt]{4.818pt}{0.400pt}}
\put(171.0,374.0){\rule[-0.200pt]{4.818pt}{0.400pt}}
\put(151,374){\makebox(0,0)[r]{ 180}}
\put(1419.0,374.0){\rule[-0.200pt]{4.818pt}{0.400pt}}
\put(171.0,495.0){\rule[-0.200pt]{4.818pt}{0.400pt}}
\put(151,495){\makebox(0,0)[r]{ 200}}
\put(1419.0,495.0){\rule[-0.200pt]{4.818pt}{0.400pt}}
\put(171.0,616.0){\rule[-0.200pt]{4.818pt}{0.400pt}}
\put(151,616){\makebox(0,0)[r]{ 220}}
\put(1419.0,616.0){\rule[-0.200pt]{4.818pt}{0.400pt}}
\put(171.0,738.0){\rule[-0.200pt]{4.818pt}{0.400pt}}
\put(151,738){\makebox(0,0)[r]{ 240}}
\put(1419.0,738.0){\rule[-0.200pt]{4.818pt}{0.400pt}}
\put(171.0,859.0){\rule[-0.200pt]{4.818pt}{0.400pt}}
\put(151,859){\makebox(0,0)[r]{ 260}}
\put(1419.0,859.0){\rule[-0.200pt]{4.818pt}{0.400pt}}
\put(171.0,131.0){\rule[-0.200pt]{0.400pt}{4.818pt}}
\put(171,90){\makebox(0,0){ 30}}
\put(171.0,839.0){\rule[-0.200pt]{0.400pt}{4.818pt}}
\put(352.0,131.0){\rule[-0.200pt]{0.400pt}{4.818pt}}
\put(352,90){\makebox(0,0){ 32}}
\put(352.0,839.0){\rule[-0.200pt]{0.400pt}{4.818pt}}
\put(533.0,131.0){\rule[-0.200pt]{0.400pt}{4.818pt}}
\put(533,90){\makebox(0,0){ 34}}
\put(533.0,839.0){\rule[-0.200pt]{0.400pt}{4.818pt}}
\put(714.0,131.0){\rule[-0.200pt]{0.400pt}{4.818pt}}
\put(714,90){\makebox(0,0){ 36}}
\put(714.0,839.0){\rule[-0.200pt]{0.400pt}{4.818pt}}
\put(896.0,131.0){\rule[-0.200pt]{0.400pt}{4.818pt}}
\put(896,90){\makebox(0,0){ 38}}
\put(896.0,839.0){\rule[-0.200pt]{0.400pt}{4.818pt}}
\put(1077.0,131.0){\rule[-0.200pt]{0.400pt}{4.818pt}}
\put(1077,90){\makebox(0,0){ 40}}
\put(1077.0,839.0){\rule[-0.200pt]{0.400pt}{4.818pt}}
\put(1258.0,131.0){\rule[-0.200pt]{0.400pt}{4.818pt}}
\put(1258,90){\makebox(0,0){ 42}}
\put(1258.0,839.0){\rule[-0.200pt]{0.400pt}{4.818pt}}
\put(1439.0,131.0){\rule[-0.200pt]{0.400pt}{4.818pt}}
\put(1439,90){\makebox(0,0){ 44}}
\put(1439.0,839.0){\rule[-0.200pt]{0.400pt}{4.818pt}}
\put(171.0,131.0){\rule[-0.200pt]{0.400pt}{175.375pt}}
\put(171.0,131.0){\rule[-0.200pt]{305.461pt}{0.400pt}}
\put(1439.0,131.0){\rule[-0.200pt]{0.400pt}{175.375pt}}
\put(171.0,859.0){\rule[-0.200pt]{305.461pt}{0.400pt}}
\put(30,495){\makebox(0,0){\popi{f}{Hz}}}
\put(805,29){\makebox(0,0){\popi{\frac{1}{\sqrt{V}}}{\sqrt{m^{-3}}}}}
\put(1279,254){\makebox(0,0)[r]{namerané hodnoty}}
\put(347,348){\makebox(0,0){$+$}}
\put(424,348){\makebox(0,0){$+$}}
\put(507,360){\makebox(0,0){$+$}}
\put(598,372){\makebox(0,0){$+$}}
\put(697,372){\makebox(0,0){$+$}}
\put(806,372){\makebox(0,0){$+$}}
\put(927,372){\makebox(0,0){$+$}}
\put(1062,372){\makebox(0,0){$+$}}
\put(1215,372){\makebox(0,0){$+$}}
\put(1388,372){\makebox(0,0){$+$}}
\put(1349,254){\makebox(0,0){$+$}}
\put(1279,213){\makebox(0,0)[r]{preložená závylosť}}
\multiput(1299,213)(20.756,0.000){5}{\usebox{\plotpoint}}
\put(1399,213){\usebox{\plotpoint}}
\put(347,212){\usebox{\plotpoint}}
\put(347.00,212.00){\usebox{\plotpoint}}
\put(366.96,217.69){\usebox{\plotpoint}}
\put(386.63,224.29){\usebox{\plotpoint}}
\put(406.30,230.89){\usebox{\plotpoint}}
\put(426.10,237.04){\usebox{\plotpoint}}
\put(445.90,243.17){\usebox{\plotpoint}}
\put(465.57,249.77){\usebox{\plotpoint}}
\put(485.24,256.37){\usebox{\plotpoint}}
\put(505.19,262.08){\usebox{\plotpoint}}
\put(524.83,268.68){\usebox{\plotpoint}}
\put(544.56,275.11){\usebox{\plotpoint}}
\put(564.25,281.64){\usebox{\plotpoint}}
\put(584.10,287.66){\usebox{\plotpoint}}
\put(603.74,294.29){\usebox{\plotpoint}}
\put(623.37,300.92){\usebox{\plotpoint}}
\put(643.26,306.82){\usebox{\plotpoint}}
\put(663.05,313.02){\usebox{\plotpoint}}
\put(682.75,319.54){\usebox{\plotpoint}}
\put(702.44,326.07){\usebox{\plotpoint}}
\put(722.32,332.00){\usebox{\plotpoint}}
\put(741.96,338.62){\usebox{\plotpoint}}
\put(761.78,344.74){\usebox{\plotpoint}}
\put(781.48,351.26){\usebox{\plotpoint}}
\put(801.23,357.61){\usebox{\plotpoint}}
\put(820.86,364.23){\usebox{\plotpoint}}
\put(840.81,369.94){\usebox{\plotpoint}}
\put(860.48,376.54){\usebox{\plotpoint}}
\put(880.15,383.15){\usebox{\plotpoint}}
\put(899.92,389.37){\usebox{\plotpoint}}
\put(919.61,395.84){\usebox{\plotpoint}}
\put(939.40,402.02){\usebox{\plotpoint}}
\put(959.07,408.62){\usebox{\plotpoint}}
\put(978.99,414.40){\usebox{\plotpoint}}
\put(998.67,420.91){\usebox{\plotpoint}}
\put(1018.30,427.54){\usebox{\plotpoint}}
\put(1038.05,433.82){\usebox{\plotpoint}}
\put(1057.91,439.77){\usebox{\plotpoint}}
\put(1077.58,446.38){\usebox{\plotpoint}}
\put(1097.25,452.98){\usebox{\plotpoint}}
\put(1117.14,458.86){\usebox{\plotpoint}}
\put(1136.86,465.26){\usebox{\plotpoint}}
\put(1156.53,471.83){\usebox{\plotpoint}}
\put(1176.23,478.36){\usebox{\plotpoint}}
\put(1196.13,484.22){\usebox{\plotpoint}}
\put(1215.77,490.85){\usebox{\plotpoint}}
\put(1235.40,497.47){\usebox{\plotpoint}}
\put(1255.24,503.54){\usebox{\plotpoint}}
\put(1275.01,509.82){\usebox{\plotpoint}}
\put(1294.64,516.45){\usebox{\plotpoint}}
\put(1314.28,523.08){\usebox{\plotpoint}}
\put(1334.23,528.77){\usebox{\plotpoint}}
\put(1353.90,535.37){\usebox{\plotpoint}}
\put(1373.73,541.45){\usebox{\plotpoint}}
\put(1388,546){\usebox{\plotpoint}}
\sbox{\plotpoint}{\rule[-0.400pt]{0.800pt}{0.800pt}}%
\sbox{\plotpoint}{\rule[-0.200pt]{0.400pt}{0.400pt}}%
\put(1279,172){\makebox(0,0)[r]{vypočítané hodnoty}}
\sbox{\plotpoint}{\rule[-0.400pt]{0.800pt}{0.800pt}}%
\put(347,421){\makebox(0,0){$\ast$}}
\put(424,451){\makebox(0,0){$\ast$}}
\put(507,484){\makebox(0,0){$\ast$}}
\put(598,520){\makebox(0,0){$\ast$}}
\put(697,559){\makebox(0,0){$\ast$}}
\put(806,602){\makebox(0,0){$\ast$}}
\put(927,649){\makebox(0,0){$\ast$}}
\put(1062,702){\makebox(0,0){$\ast$}}
\put(1215,762){\makebox(0,0){$\ast$}}
\put(1388,831){\makebox(0,0){$\ast$}}
\put(1349,172){\makebox(0,0){$\ast$}}
\sbox{\plotpoint}{\rule[-0.200pt]{0.400pt}{0.400pt}}%
\put(171.0,131.0){\rule[-0.200pt]{0.400pt}{175.375pt}}
\put(171.0,131.0){\rule[-0.200pt]{305.461pt}{0.400pt}}
\put(1439.0,131.0){\rule[-0.200pt]{0.400pt}{175.375pt}}
\put(171.0,859.0){\rule[-0.200pt]{305.461pt}{0.400pt}}
\end{picture}

\caption{Teoretická a nameraná závislosť rezonančnej frekvencie $f_n$ a $f_v$ od objemu $V$, preložená funkciou $f=\frac{293.2\pm9.0}{2\pi}\sqrt{\frac{\pi r^2}{l+1.4r}} \frac{1}{\sqrt{V}}$.
}  \label{G_3}
\end{figure}




