
\section{Výsledky merania}
Nazačitaku bola digitálnym teplomerom určená teplota $t="23.1 \C"$.
Z nej podľa vzťahu \ref{R_1} bola dopočítaná odpovedajúca rýchlosť zvuku
\eq{
v = "345.6 m\cdot s^{-1}"\,.
}

\subsection{Stojaté vlněnie na strune}

Dĺžka struny je $L=" \(1.176\pm0.01\) m"$. Z tejto dĺžky bola vypočítaná podľa vzťahu \ref{R_2} predpokladaná frekvencia prvého módu, $f ="23.3 Hz"$.

Namerané hodnoty vyšších módov v závislosti od frekvenciu sú v tabuľke Tab. \ref{T_1}.

Následne z grafu Obr. \ref{G_1} bola určená smernica $C="\(23.38\pm0.03\)n"$ a pomocou vzťahu \ref{R_2} a \ref{SCH_1} hodnota 
\eq{
\mu = "\(0.0163\pm0.0001\) kg\cdot m^{-1}"\,.
}

\begin{table}[h]
\begin{center}
\begin{tabular}{| c | c |}
\hline
\popi{n}{-} & \popi{f}{Hz} \\
\hline
1  & $ 23.30 $\\
2  & $ 46.80 $\\
3  & $ 69.90 $\\
4  & $ 93.00 $\\
5  & $ 116.1 $\\
6  & $ 140.5 $\\
7  & $ 162.7 $\\
8  & $ 187.6 $\\
9  & $ 210.4 $\\
10 & $ 234.9 $\\
\hline
\end{tabular}
\caption{Závislosť frekvencie $f$ od módu kmitania struny $n$.
} \label{T_1}
\end{center}
\end{table}


\begin{figure}
% GNUPLOT: LaTeX picture
\setlength{\unitlength}{0.240900pt}
\ifx\plotpoint\undefined\newsavebox{\plotpoint}\fi
\sbox{\plotpoint}{\rule[-0.200pt]{0.400pt}{0.400pt}}%
\begin{picture}(1500,900)(0,0)
\sbox{\plotpoint}{\rule[-0.200pt]{0.400pt}{0.400pt}}%
\put(171.0,131.0){\rule[-0.200pt]{4.818pt}{0.400pt}}
\put(151,131){\makebox(0,0)[r]{-0.3}}
\put(1419.0,131.0){\rule[-0.200pt]{4.818pt}{0.400pt}}
\put(171.0,252.0){\rule[-0.200pt]{4.818pt}{0.400pt}}
\put(151,252){\makebox(0,0)[r]{-0.2}}
\put(1419.0,252.0){\rule[-0.200pt]{4.818pt}{0.400pt}}
\put(171.0,374.0){\rule[-0.200pt]{4.818pt}{0.400pt}}
\put(151,374){\makebox(0,0)[r]{-0.1}}
\put(1419.0,374.0){\rule[-0.200pt]{4.818pt}{0.400pt}}
\put(171.0,495.0){\rule[-0.200pt]{4.818pt}{0.400pt}}
\put(151,495){\makebox(0,0)[r]{ 0}}
\put(1419.0,495.0){\rule[-0.200pt]{4.818pt}{0.400pt}}
\put(171.0,616.0){\rule[-0.200pt]{4.818pt}{0.400pt}}
\put(151,616){\makebox(0,0)[r]{ 0.1}}
\put(1419.0,616.0){\rule[-0.200pt]{4.818pt}{0.400pt}}
\put(171.0,738.0){\rule[-0.200pt]{4.818pt}{0.400pt}}
\put(151,738){\makebox(0,0)[r]{ 0.2}}
\put(1419.0,738.0){\rule[-0.200pt]{4.818pt}{0.400pt}}
\put(171.0,859.0){\rule[-0.200pt]{4.818pt}{0.400pt}}
\put(151,859){\makebox(0,0)[r]{ 0.3}}
\put(1419.0,859.0){\rule[-0.200pt]{4.818pt}{0.400pt}}
\put(171.0,131.0){\rule[-0.200pt]{0.400pt}{4.818pt}}
\put(171,90){\makebox(0,0){-150}}
\put(171.0,839.0){\rule[-0.200pt]{0.400pt}{4.818pt}}
\put(382.0,131.0){\rule[-0.200pt]{0.400pt}{4.818pt}}
\put(382,90){\makebox(0,0){-100}}
\put(382.0,839.0){\rule[-0.200pt]{0.400pt}{4.818pt}}
\put(594.0,131.0){\rule[-0.200pt]{0.400pt}{4.818pt}}
\put(594,90){\makebox(0,0){-50}}
\put(594.0,839.0){\rule[-0.200pt]{0.400pt}{4.818pt}}
\put(805.0,131.0){\rule[-0.200pt]{0.400pt}{4.818pt}}
\put(805,90){\makebox(0,0){ 0}}
\put(805.0,839.0){\rule[-0.200pt]{0.400pt}{4.818pt}}
\put(1016.0,131.0){\rule[-0.200pt]{0.400pt}{4.818pt}}
\put(1016,90){\makebox(0,0){ 50}}
\put(1016.0,839.0){\rule[-0.200pt]{0.400pt}{4.818pt}}
\put(1228.0,131.0){\rule[-0.200pt]{0.400pt}{4.818pt}}
\put(1228,90){\makebox(0,0){ 100}}
\put(1228.0,839.0){\rule[-0.200pt]{0.400pt}{4.818pt}}
\put(1439.0,131.0){\rule[-0.200pt]{0.400pt}{4.818pt}}
\put(1439,90){\makebox(0,0){ 150}}
\put(1439.0,839.0){\rule[-0.200pt]{0.400pt}{4.818pt}}
\put(171.0,495.0){\rule[-0.200pt]{305.461pt}{0.400pt}}
\put(805.0,131.0){\rule[-0.200pt]{0.400pt}{175.375pt}}
\put(171.0,131.0){\rule[-0.200pt]{0.400pt}{175.375pt}}
\put(171.0,131.0){\rule[-0.200pt]{305.461pt}{0.400pt}}
\put(1439.0,131.0){\rule[-0.200pt]{0.400pt}{175.375pt}}
\put(171.0,859.0){\rule[-0.200pt]{305.461pt}{0.400pt}}
\put(30,495){\makebox(0,0){\popi{B}{H}}}
\put(805,29){\makebox(0,0){\popi{H}{A\cdot m^{-1}}}}
\put(1279,295){\makebox(0,0)[r]{namerané dáta}}
\put(988,669){\makebox(0,0){$+$}}
\put(1020,694){\makebox(0,0){$+$}}
\put(1061,723){\makebox(0,0){$+$}}
\put(1100,730){\makebox(0,0){$+$}}
\put(1125,734){\makebox(0,0){$+$}}
\put(1185,745){\makebox(0,0){$+$}}
\put(1229,738){\makebox(0,0){$+$}}
\put(1281,759){\makebox(0,0){$+$}}
\put(929,658){\makebox(0,0){$+$}}
\put(881,607){\makebox(0,0){$+$}}
\put(866,600){\makebox(0,0){$+$}}
\put(842,600){\makebox(0,0){$+$}}
\put(805,600){\makebox(0,0){$+$}}
\put(729,462){\makebox(0,0){$+$}}
\put(712,458){\makebox(0,0){$+$}}
\put(681,273){\makebox(0,0){$+$}}
\put(659,255){\makebox(0,0){$+$}}
\put(622,234){\makebox(0,0){$+$}}
\put(566,230){\makebox(0,0){$+$}}
\put(510,201){\makebox(0,0){$+$}}
\put(456,205){\makebox(0,0){$+$}}
\put(383,190){\makebox(0,0){$+$}}
\put(322,168){\makebox(0,0){$+$}}
\put(1349,295){\makebox(0,0){$+$}}
\put(1279,254){\makebox(0,0)[r]{namerané dáta}}
\put(1000,746){\makebox(0,0){$\times$}}
\put(1025,749){\makebox(0,0){$\times$}}
\put(829,387){\makebox(0,0){$\times$}}
\put(1149,785){\makebox(0,0){$\times$}}
\put(1225,840){\makebox(0,0){$\times$}}
\put(964,713){\makebox(0,0){$\times$}}
\put(929,749){\makebox(0,0){$\times$}}
\put(881,706){\makebox(0,0){$\times$}}
\put(866,463){\makebox(0,0){$\times$}}
\put(805,354){\makebox(0,0){$\times$}}
\put(712,296){\makebox(0,0){$\times$}}
\put(681,296){\makebox(0,0){$\times$}}
\put(659,263){\makebox(0,0){$\times$}}
\put(625,278){\makebox(0,0){$\times$}}
\put(566,270){\makebox(0,0){$\times$}}
\put(325,256){\makebox(0,0){$\times$}}
\put(1349,254){\makebox(0,0){$\times$}}
\sbox{\plotpoint}{\rule[-0.400pt]{0.800pt}{0.800pt}}%
\sbox{\plotpoint}{\rule[-0.200pt]{0.400pt}{0.400pt}}%
\put(1279,213){\makebox(0,0)[r]{Aproximácia dát pomocou $erf(x)$}}
\sbox{\plotpoint}{\rule[-0.400pt]{0.800pt}{0.800pt}}%
\put(1299.0,213.0){\rule[-0.400pt]{24.090pt}{0.800pt}}
\put(322,224){\usebox{\plotpoint}}
\put(390,222.84){\rule{2.409pt}{0.800pt}}
\multiput(390.00,222.34)(5.000,1.000){2}{\rule{1.204pt}{0.800pt}}
\put(322.0,224.0){\rule[-0.400pt]{16.381pt}{0.800pt}}
\put(409,223.84){\rule{2.409pt}{0.800pt}}
\multiput(409.00,223.34)(5.000,1.000){2}{\rule{1.204pt}{0.800pt}}
\put(400.0,225.0){\rule[-0.400pt]{2.168pt}{0.800pt}}
\put(429,224.84){\rule{2.168pt}{0.800pt}}
\multiput(429.00,224.34)(4.500,1.000){2}{\rule{1.084pt}{0.800pt}}
\put(438,225.84){\rule{2.409pt}{0.800pt}}
\multiput(438.00,225.34)(5.000,1.000){2}{\rule{1.204pt}{0.800pt}}
\put(448,226.84){\rule{2.409pt}{0.800pt}}
\multiput(448.00,226.34)(5.000,1.000){2}{\rule{1.204pt}{0.800pt}}
\put(458,227.84){\rule{2.168pt}{0.800pt}}
\multiput(458.00,227.34)(4.500,1.000){2}{\rule{1.084pt}{0.800pt}}
\put(467,228.84){\rule{2.409pt}{0.800pt}}
\multiput(467.00,228.34)(5.000,1.000){2}{\rule{1.204pt}{0.800pt}}
\put(477,230.34){\rule{2.409pt}{0.800pt}}
\multiput(477.00,229.34)(5.000,2.000){2}{\rule{1.204pt}{0.800pt}}
\put(487,232.34){\rule{2.168pt}{0.800pt}}
\multiput(487.00,231.34)(4.500,2.000){2}{\rule{1.084pt}{0.800pt}}
\put(496,234.34){\rule{2.409pt}{0.800pt}}
\multiput(496.00,233.34)(5.000,2.000){2}{\rule{1.204pt}{0.800pt}}
\put(506,236.34){\rule{2.409pt}{0.800pt}}
\multiput(506.00,235.34)(5.000,2.000){2}{\rule{1.204pt}{0.800pt}}
\put(516,238.84){\rule{2.168pt}{0.800pt}}
\multiput(516.00,237.34)(4.500,3.000){2}{\rule{1.084pt}{0.800pt}}
\put(525,242.34){\rule{2.200pt}{0.800pt}}
\multiput(525.00,240.34)(5.434,4.000){2}{\rule{1.100pt}{0.800pt}}
\put(535,245.84){\rule{2.409pt}{0.800pt}}
\multiput(535.00,244.34)(5.000,3.000){2}{\rule{1.204pt}{0.800pt}}
\multiput(545.00,250.38)(1.096,0.560){3}{\rule{1.640pt}{0.135pt}}
\multiput(545.00,247.34)(5.596,5.000){2}{\rule{0.820pt}{0.800pt}}
\multiput(554.00,255.38)(1.264,0.560){3}{\rule{1.800pt}{0.135pt}}
\multiput(554.00,252.34)(6.264,5.000){2}{\rule{0.900pt}{0.800pt}}
\multiput(564.00,260.38)(1.264,0.560){3}{\rule{1.800pt}{0.135pt}}
\multiput(564.00,257.34)(6.264,5.000){2}{\rule{0.900pt}{0.800pt}}
\multiput(574.00,265.39)(0.797,0.536){5}{\rule{1.400pt}{0.129pt}}
\multiput(574.00,262.34)(6.094,6.000){2}{\rule{0.700pt}{0.800pt}}
\multiput(583.00,271.40)(0.738,0.526){7}{\rule{1.343pt}{0.127pt}}
\multiput(583.00,268.34)(7.213,7.000){2}{\rule{0.671pt}{0.800pt}}
\multiput(593.00,278.40)(0.738,0.526){7}{\rule{1.343pt}{0.127pt}}
\multiput(593.00,275.34)(7.213,7.000){2}{\rule{0.671pt}{0.800pt}}
\multiput(603.00,285.40)(0.548,0.516){11}{\rule{1.089pt}{0.124pt}}
\multiput(603.00,282.34)(7.740,9.000){2}{\rule{0.544pt}{0.800pt}}
\multiput(613.00,294.40)(0.554,0.520){9}{\rule{1.100pt}{0.125pt}}
\multiput(613.00,291.34)(6.717,8.000){2}{\rule{0.550pt}{0.800pt}}
\multiput(622.00,302.40)(0.487,0.514){13}{\rule{1.000pt}{0.124pt}}
\multiput(622.00,299.34)(7.924,10.000){2}{\rule{0.500pt}{0.800pt}}
\multiput(632.00,312.40)(0.487,0.514){13}{\rule{1.000pt}{0.124pt}}
\multiput(632.00,309.34)(7.924,10.000){2}{\rule{0.500pt}{0.800pt}}
\multiput(643.40,321.00)(0.516,0.674){11}{\rule{0.124pt}{1.267pt}}
\multiput(640.34,321.00)(9.000,9.371){2}{\rule{0.800pt}{0.633pt}}
\multiput(652.40,333.00)(0.514,0.543){13}{\rule{0.124pt}{1.080pt}}
\multiput(649.34,333.00)(10.000,8.758){2}{\rule{0.800pt}{0.540pt}}
\multiput(662.40,344.00)(0.514,0.654){13}{\rule{0.124pt}{1.240pt}}
\multiput(659.34,344.00)(10.000,10.426){2}{\rule{0.800pt}{0.620pt}}
\multiput(672.40,357.00)(0.516,0.737){11}{\rule{0.124pt}{1.356pt}}
\multiput(669.34,357.00)(9.000,10.186){2}{\rule{0.800pt}{0.678pt}}
\multiput(681.40,370.00)(0.514,0.710){13}{\rule{0.124pt}{1.320pt}}
\multiput(678.34,370.00)(10.000,11.260){2}{\rule{0.800pt}{0.660pt}}
\multiput(691.40,384.00)(0.514,0.710){13}{\rule{0.124pt}{1.320pt}}
\multiput(688.34,384.00)(10.000,11.260){2}{\rule{0.800pt}{0.660pt}}
\multiput(701.40,398.00)(0.516,0.863){11}{\rule{0.124pt}{1.533pt}}
\multiput(698.34,398.00)(9.000,11.817){2}{\rule{0.800pt}{0.767pt}}
\multiput(710.40,413.00)(0.514,0.766){13}{\rule{0.124pt}{1.400pt}}
\multiput(707.34,413.00)(10.000,12.094){2}{\rule{0.800pt}{0.700pt}}
\multiput(720.40,428.00)(0.514,0.821){13}{\rule{0.124pt}{1.480pt}}
\multiput(717.34,428.00)(10.000,12.928){2}{\rule{0.800pt}{0.740pt}}
\multiput(730.40,444.00)(0.516,0.927){11}{\rule{0.124pt}{1.622pt}}
\multiput(727.34,444.00)(9.000,12.633){2}{\rule{0.800pt}{0.811pt}}
\multiput(739.40,460.00)(0.514,0.821){13}{\rule{0.124pt}{1.480pt}}
\multiput(736.34,460.00)(10.000,12.928){2}{\rule{0.800pt}{0.740pt}}
\multiput(749.40,476.00)(0.514,0.821){13}{\rule{0.124pt}{1.480pt}}
\multiput(746.34,476.00)(10.000,12.928){2}{\rule{0.800pt}{0.740pt}}
\multiput(759.40,492.00)(0.516,0.990){11}{\rule{0.124pt}{1.711pt}}
\multiput(756.34,492.00)(9.000,13.449){2}{\rule{0.800pt}{0.856pt}}
\multiput(768.40,509.00)(0.514,0.821){13}{\rule{0.124pt}{1.480pt}}
\multiput(765.34,509.00)(10.000,12.928){2}{\rule{0.800pt}{0.740pt}}
\multiput(778.40,525.00)(0.514,0.821){13}{\rule{0.124pt}{1.480pt}}
\multiput(775.34,525.00)(10.000,12.928){2}{\rule{0.800pt}{0.740pt}}
\multiput(788.40,541.00)(0.514,0.821){13}{\rule{0.124pt}{1.480pt}}
\multiput(785.34,541.00)(10.000,12.928){2}{\rule{0.800pt}{0.740pt}}
\multiput(798.40,557.00)(0.516,0.863){11}{\rule{0.124pt}{1.533pt}}
\multiput(795.34,557.00)(9.000,11.817){2}{\rule{0.800pt}{0.767pt}}
\multiput(807.40,572.00)(0.514,0.766){13}{\rule{0.124pt}{1.400pt}}
\multiput(804.34,572.00)(10.000,12.094){2}{\rule{0.800pt}{0.700pt}}
\multiput(817.40,587.00)(0.514,0.766){13}{\rule{0.124pt}{1.400pt}}
\multiput(814.34,587.00)(10.000,12.094){2}{\rule{0.800pt}{0.700pt}}
\multiput(827.40,602.00)(0.516,0.800){11}{\rule{0.124pt}{1.444pt}}
\multiput(824.34,602.00)(9.000,11.002){2}{\rule{0.800pt}{0.722pt}}
\multiput(836.40,616.00)(0.514,0.654){13}{\rule{0.124pt}{1.240pt}}
\multiput(833.34,616.00)(10.000,10.426){2}{\rule{0.800pt}{0.620pt}}
\multiput(846.40,629.00)(0.514,0.654){13}{\rule{0.124pt}{1.240pt}}
\multiput(843.34,629.00)(10.000,10.426){2}{\rule{0.800pt}{0.620pt}}
\multiput(856.40,642.00)(0.516,0.674){11}{\rule{0.124pt}{1.267pt}}
\multiput(853.34,642.00)(9.000,9.371){2}{\rule{0.800pt}{0.633pt}}
\multiput(865.40,654.00)(0.514,0.543){13}{\rule{0.124pt}{1.080pt}}
\multiput(862.34,654.00)(10.000,8.758){2}{\rule{0.800pt}{0.540pt}}
\multiput(875.40,665.00)(0.514,0.543){13}{\rule{0.124pt}{1.080pt}}
\multiput(872.34,665.00)(10.000,8.758){2}{\rule{0.800pt}{0.540pt}}
\multiput(885.40,676.00)(0.516,0.548){11}{\rule{0.124pt}{1.089pt}}
\multiput(882.34,676.00)(9.000,7.740){2}{\rule{0.800pt}{0.544pt}}
\multiput(893.00,687.40)(0.548,0.516){11}{\rule{1.089pt}{0.124pt}}
\multiput(893.00,684.34)(7.740,9.000){2}{\rule{0.544pt}{0.800pt}}
\multiput(903.00,696.40)(0.627,0.520){9}{\rule{1.200pt}{0.125pt}}
\multiput(903.00,693.34)(7.509,8.000){2}{\rule{0.600pt}{0.800pt}}
\multiput(913.00,704.40)(0.554,0.520){9}{\rule{1.100pt}{0.125pt}}
\multiput(913.00,701.34)(6.717,8.000){2}{\rule{0.550pt}{0.800pt}}
\multiput(922.00,712.40)(0.738,0.526){7}{\rule{1.343pt}{0.127pt}}
\multiput(922.00,709.34)(7.213,7.000){2}{\rule{0.671pt}{0.800pt}}
\multiput(932.00,719.39)(0.909,0.536){5}{\rule{1.533pt}{0.129pt}}
\multiput(932.00,716.34)(6.817,6.000){2}{\rule{0.767pt}{0.800pt}}
\multiput(942.00,725.39)(0.797,0.536){5}{\rule{1.400pt}{0.129pt}}
\multiput(942.00,722.34)(6.094,6.000){2}{\rule{0.700pt}{0.800pt}}
\multiput(951.00,731.38)(1.264,0.560){3}{\rule{1.800pt}{0.135pt}}
\multiput(951.00,728.34)(6.264,5.000){2}{\rule{0.900pt}{0.800pt}}
\put(961,735.34){\rule{2.200pt}{0.800pt}}
\multiput(961.00,733.34)(5.434,4.000){2}{\rule{1.100pt}{0.800pt}}
\put(971,739.34){\rule{2.000pt}{0.800pt}}
\multiput(971.00,737.34)(4.849,4.000){2}{\rule{1.000pt}{0.800pt}}
\put(980,743.34){\rule{2.200pt}{0.800pt}}
\multiput(980.00,741.34)(5.434,4.000){2}{\rule{1.100pt}{0.800pt}}
\put(990,746.84){\rule{2.409pt}{0.800pt}}
\multiput(990.00,745.34)(5.000,3.000){2}{\rule{1.204pt}{0.800pt}}
\put(1000,749.34){\rule{2.409pt}{0.800pt}}
\multiput(1000.00,748.34)(5.000,2.000){2}{\rule{1.204pt}{0.800pt}}
\put(1010,751.84){\rule{2.168pt}{0.800pt}}
\multiput(1010.00,750.34)(4.500,3.000){2}{\rule{1.084pt}{0.800pt}}
\put(1019,754.34){\rule{2.409pt}{0.800pt}}
\multiput(1019.00,753.34)(5.000,2.000){2}{\rule{1.204pt}{0.800pt}}
\put(1029,755.84){\rule{2.409pt}{0.800pt}}
\multiput(1029.00,755.34)(5.000,1.000){2}{\rule{1.204pt}{0.800pt}}
\put(1039,757.34){\rule{2.168pt}{0.800pt}}
\multiput(1039.00,756.34)(4.500,2.000){2}{\rule{1.084pt}{0.800pt}}
\put(1048,758.84){\rule{2.409pt}{0.800pt}}
\multiput(1048.00,758.34)(5.000,1.000){2}{\rule{1.204pt}{0.800pt}}
\put(1058,759.84){\rule{2.409pt}{0.800pt}}
\multiput(1058.00,759.34)(5.000,1.000){2}{\rule{1.204pt}{0.800pt}}
\put(1068,760.84){\rule{2.168pt}{0.800pt}}
\multiput(1068.00,760.34)(4.500,1.000){2}{\rule{1.084pt}{0.800pt}}
\put(1077,761.84){\rule{2.409pt}{0.800pt}}
\multiput(1077.00,761.34)(5.000,1.000){2}{\rule{1.204pt}{0.800pt}}
\put(419.0,226.0){\rule[-0.400pt]{2.409pt}{0.800pt}}
\put(1097,762.84){\rule{2.168pt}{0.800pt}}
\multiput(1097.00,762.34)(4.500,1.000){2}{\rule{1.084pt}{0.800pt}}
\put(1087.0,764.0){\rule[-0.400pt]{2.409pt}{0.800pt}}
\put(1126,763.84){\rule{2.168pt}{0.800pt}}
\multiput(1126.00,763.34)(4.500,1.000){2}{\rule{1.084pt}{0.800pt}}
\put(1106.0,765.0){\rule[-0.400pt]{4.818pt}{0.800pt}}
\put(1193,764.84){\rule{2.409pt}{0.800pt}}
\multiput(1193.00,764.34)(5.000,1.000){2}{\rule{1.204pt}{0.800pt}}
\put(1135.0,766.0){\rule[-0.400pt]{13.972pt}{0.800pt}}
\put(1203.0,767.0){\rule[-0.400pt]{18.790pt}{0.800pt}}
\sbox{\plotpoint}{\rule[-0.500pt]{1.000pt}{1.000pt}}%
\sbox{\plotpoint}{\rule[-0.200pt]{0.400pt}{0.400pt}}%
\put(1279,172){\makebox(0,0)[r]{Aproximácia dát pomocou $erf(x)$}}
\sbox{\plotpoint}{\rule[-0.500pt]{1.000pt}{1.000pt}}%
\multiput(1299,172)(20.756,0.000){5}{\usebox{\plotpoint}}
\put(1399,172){\usebox{\plotpoint}}
\put(322,223){\usebox{\plotpoint}}
\put(322.00,223.00){\usebox{\plotpoint}}
\put(342.76,223.00){\usebox{\plotpoint}}
\put(363.51,223.00){\usebox{\plotpoint}}
\put(384.27,223.00){\usebox{\plotpoint}}
\put(405.02,223.00){\usebox{\plotpoint}}
\put(425.73,224.00){\usebox{\plotpoint}}
\put(446.48,224.00){\usebox{\plotpoint}}
\put(467.24,224.00){\usebox{\plotpoint}}
\put(487.99,224.11){\usebox{\plotpoint}}
\put(508.69,225.00){\usebox{\plotpoint}}
\put(529.39,226.00){\usebox{\plotpoint}}
\put(550.07,227.56){\usebox{\plotpoint}}
\put(570.72,229.67){\usebox{\plotpoint}}
\put(591.08,233.62){\usebox{\plotpoint}}
\put(611.43,237.69){\usebox{\plotpoint}}
\put(631.26,243.78){\usebox{\plotpoint}}
\put(650.72,250.88){\usebox{\plotpoint}}
\put(669.29,260.14){\usebox{\plotpoint}}
\put(686.88,271.13){\usebox{\plotpoint}}
\put(703.67,283.26){\usebox{\plotpoint}}
\put(719.13,297.12){\usebox{\plotpoint}}
\put(734.00,311.56){\usebox{\plotpoint}}
\put(747.94,326.93){\usebox{\plotpoint}}
\put(761.48,342.64){\usebox{\plotpoint}}
\put(774.05,359.16){\usebox{\plotpoint}}
\put(786.25,375.95){\usebox{\plotpoint}}
\put(798.22,392.90){\usebox{\plotpoint}}
\put(809.53,410.30){\usebox{\plotpoint}}
\put(820.82,427.71){\usebox{\plotpoint}}
\put(831.38,445.57){\usebox{\plotpoint}}
\put(842.09,463.35){\usebox{\plotpoint}}
\put(853.09,480.95){\usebox{\plotpoint}}
\put(863.41,498.95){\usebox{\plotpoint}}
\put(874.35,516.59){\usebox{\plotpoint}}
\put(884.84,534.50){\usebox{\plotpoint}}
\put(895.28,552.43){\usebox{\plotpoint}}
\put(906.80,569.70){\usebox{\plotpoint}}
\put(917.93,587.21){\usebox{\plotpoint}}
\put(929.46,604.44){\usebox{\plotpoint}}
\put(941.52,621.33){\usebox{\plotpoint}}
\put(953.52,638.27){\usebox{\plotpoint}}
\put(966.70,654.27){\usebox{\plotpoint}}
\put(980.11,670.12){\usebox{\plotpoint}}
\put(994.49,685.04){\usebox{\plotpoint}}
\put(1009.92,698.93){\usebox{\plotpoint}}
\put(1026.05,711.94){\usebox{\plotpoint}}
\put(1043.12,723.75){\usebox{\plotpoint}}
\put(1061.32,733.66){\usebox{\plotpoint}}
\put(1080.19,742.28){\usebox{\plotpoint}}
\put(1099.83,748.94){\usebox{\plotpoint}}
\put(1119.73,754.75){\usebox{\plotpoint}}
\put(1140.12,758.51){\usebox{\plotpoint}}
\put(1160.66,761.26){\usebox{\plotpoint}}
\put(1181.29,763.00){\usebox{\plotpoint}}
\put(1201.99,764.00){\usebox{\plotpoint}}
\put(1222.69,765.00){\usebox{\plotpoint}}
\put(1243.39,766.00){\usebox{\plotpoint}}
\put(1264.15,766.00){\usebox{\plotpoint}}
\put(1281,766){\usebox{\plotpoint}}
\sbox{\plotpoint}{\rule[-0.200pt]{0.400pt}{0.400pt}}%
\put(171.0,131.0){\rule[-0.200pt]{0.400pt}{175.375pt}}
\put(171.0,131.0){\rule[-0.200pt]{305.461pt}{0.400pt}}
\put(1439.0,131.0){\rule[-0.200pt]{0.400pt}{175.375pt}}
\put(171.0,859.0){\rule[-0.200pt]{305.461pt}{0.400pt}}
\end{picture}

\caption{Závislosť frekvencie $f$ od módu kmitania struny $n$, preložená lineárnou závislosťou
}  \label{G_1}
\end{figure}

\subsection{Quinckova trubice}

Namerané hodnoty polohy minim vlnenia, pre jednotlivé polohy, viď priložené pracovné papiere.
Z nich boli pre jednotlivé frekvencie vypočítané podľa \ref{SCH_1} priemer a chyba. viď Tab. \ref{T_2}.

Do grafu Obr. \ref{G_2} bola vynesená závislosť vlnovej dĺžky $\lambda\equiv 2\Delta$ od doby kmitu $T$. 
Zo smernice preloženej závislosť bola určená rýchlosť zvuku
\eq{
v = "\(348.7\pm6.7\) m\cdot s^{-1}"\,.
}

\begin{table}[h]
\begin{center}
\begin{tabular}{ | c | c |}
\hline
\popi{f}{kHz} & \popi{\mean{\Delta d_i}}{m} \\
\hline
$2$ & $0.080\pm0.004$\\
$2.5$ & $0.070\pm0.001$\\
$3$ & $0.058\pm0.002$\\
$3.5$ & $0.050\pm0.001$\\
$4$ & $0.041\pm0.005$\\
$4.5$ & $0.039\pm0.001$\\
$5$ & $0.034\pm0.008$\\
$5.5$ & $0.031\pm0.001$\\
$6$ & $0.029\pm0.001$\\
$4.25$& $0.040\pm0.001$\\
\hline
\end{tabular}
\caption{Vypočítané hodnoty priemernej dĺžky vzdialenosti minim $\Delta d_i$ od frekvencie $f$
} \label{T_2}
\end{center}
\end{table}



\begin{figure}
% GNUPLOT: LaTeX picture
\setlength{\unitlength}{0.240900pt}
\ifx\plotpoint\undefined\newsavebox{\plotpoint}\fi
\begin{picture}(1500,900)(0,0)
\sbox{\plotpoint}{\rule[-0.200pt]{0.400pt}{0.400pt}}%
\put(171.0,131.0){\rule[-0.200pt]{4.818pt}{0.400pt}}
\put(151,131){\makebox(0,0)[r]{ 50}}
\put(1419.0,131.0){\rule[-0.200pt]{4.818pt}{0.400pt}}
\put(171.0,252.0){\rule[-0.200pt]{4.818pt}{0.400pt}}
\put(151,252){\makebox(0,0)[r]{ 100}}
\put(1419.0,252.0){\rule[-0.200pt]{4.818pt}{0.400pt}}
\put(171.0,374.0){\rule[-0.200pt]{4.818pt}{0.400pt}}
\put(151,374){\makebox(0,0)[r]{ 150}}
\put(1419.0,374.0){\rule[-0.200pt]{4.818pt}{0.400pt}}
\put(171.0,495.0){\rule[-0.200pt]{4.818pt}{0.400pt}}
\put(151,495){\makebox(0,0)[r]{ 200}}
\put(1419.0,495.0){\rule[-0.200pt]{4.818pt}{0.400pt}}
\put(171.0,616.0){\rule[-0.200pt]{4.818pt}{0.400pt}}
\put(151,616){\makebox(0,0)[r]{ 250}}
\put(1419.0,616.0){\rule[-0.200pt]{4.818pt}{0.400pt}}
\put(171.0,738.0){\rule[-0.200pt]{4.818pt}{0.400pt}}
\put(151,738){\makebox(0,0)[r]{ 300}}
\put(1419.0,738.0){\rule[-0.200pt]{4.818pt}{0.400pt}}
\put(171.0,859.0){\rule[-0.200pt]{4.818pt}{0.400pt}}
\put(151,859){\makebox(0,0)[r]{ 350}}
\put(1419.0,859.0){\rule[-0.200pt]{4.818pt}{0.400pt}}
\put(171.0,131.0){\rule[-0.200pt]{0.400pt}{4.818pt}}
\put(171,90){\makebox(0,0){ 0.5}}
\put(171.0,839.0){\rule[-0.200pt]{0.400pt}{4.818pt}}
\put(382.0,131.0){\rule[-0.200pt]{0.400pt}{4.818pt}}
\put(382,90){\makebox(0,0){ 1}}
\put(382.0,839.0){\rule[-0.200pt]{0.400pt}{4.818pt}}
\put(594.0,131.0){\rule[-0.200pt]{0.400pt}{4.818pt}}
\put(594,90){\makebox(0,0){ 1.5}}
\put(594.0,839.0){\rule[-0.200pt]{0.400pt}{4.818pt}}
\put(805.0,131.0){\rule[-0.200pt]{0.400pt}{4.818pt}}
\put(805,90){\makebox(0,0){ 2}}
\put(805.0,839.0){\rule[-0.200pt]{0.400pt}{4.818pt}}
\put(1016.0,131.0){\rule[-0.200pt]{0.400pt}{4.818pt}}
\put(1016,90){\makebox(0,0){ 2.5}}
\put(1016.0,839.0){\rule[-0.200pt]{0.400pt}{4.818pt}}
\put(1228.0,131.0){\rule[-0.200pt]{0.400pt}{4.818pt}}
\put(1228,90){\makebox(0,0){ 3}}
\put(1228.0,839.0){\rule[-0.200pt]{0.400pt}{4.818pt}}
\put(1439.0,131.0){\rule[-0.200pt]{0.400pt}{4.818pt}}
\put(1439,90){\makebox(0,0){ 3.5}}
\put(1439.0,839.0){\rule[-0.200pt]{0.400pt}{4.818pt}}
\put(171.0,131.0){\rule[-0.200pt]{0.400pt}{175.375pt}}
\put(171.0,131.0){\rule[-0.200pt]{305.461pt}{0.400pt}}
\put(1439.0,131.0){\rule[-0.200pt]{0.400pt}{175.375pt}}
\put(171.0,859.0){\rule[-0.200pt]{305.461pt}{0.400pt}}
\put(30,495){\makebox(0,0){\popi{U}{V}}}
\put(805,29){\makebox(0,0){\popi{I^2}{A^2}}}
\put(1279,213){\makebox(0,0)[r]{namerané dáta}}
\put(1299.0,213.0){\rule[-0.200pt]{24.090pt}{0.400pt}}
\put(1299.0,203.0){\rule[-0.200pt]{0.400pt}{4.818pt}}
\put(1399.0,203.0){\rule[-0.200pt]{0.400pt}{4.818pt}}
\put(382.0,204.0){\rule[-0.200pt]{0.400pt}{23.367pt}}
\put(372.0,204.0){\rule[-0.200pt]{4.818pt}{0.400pt}}
\put(372.0,301.0){\rule[-0.200pt]{4.818pt}{0.400pt}}
\put(426.0,264.0){\rule[-0.200pt]{0.400pt}{23.608pt}}
\put(416.0,264.0){\rule[-0.200pt]{4.818pt}{0.400pt}}
\put(416.0,362.0){\rule[-0.200pt]{4.818pt}{0.400pt}}
\put(519.0,325.0){\rule[-0.200pt]{0.400pt}{23.367pt}}
\put(509.0,325.0){\rule[-0.200pt]{4.818pt}{0.400pt}}
\put(509.0,422.0){\rule[-0.200pt]{4.818pt}{0.400pt}}
\put(674.0,386.0){\rule[-0.200pt]{0.400pt}{23.367pt}}
\put(664.0,386.0){\rule[-0.200pt]{4.818pt}{0.400pt}}
\put(664.0,483.0){\rule[-0.200pt]{4.818pt}{0.400pt}}
\put(788.0,446.0){\rule[-0.200pt]{0.400pt}{23.608pt}}
\put(778.0,446.0){\rule[-0.200pt]{4.818pt}{0.400pt}}
\put(778.0,544.0){\rule[-0.200pt]{4.818pt}{0.400pt}}
\put(911.0,507.0){\rule[-0.200pt]{0.400pt}{23.367pt}}
\put(901.0,507.0){\rule[-0.200pt]{4.818pt}{0.400pt}}
\put(901.0,604.0){\rule[-0.200pt]{4.818pt}{0.400pt}}
\put(975.0,568.0){\rule[-0.200pt]{0.400pt}{23.367pt}}
\put(965.0,568.0){\rule[-0.200pt]{4.818pt}{0.400pt}}
\put(965.0,665.0){\rule[-0.200pt]{4.818pt}{0.400pt}}
\put(1110.0,628.0){\rule[-0.200pt]{0.400pt}{23.608pt}}
\put(1100.0,628.0){\rule[-0.200pt]{4.818pt}{0.400pt}}
\put(1100.0,726.0){\rule[-0.200pt]{4.818pt}{0.400pt}}
\put(1254.0,689.0){\rule[-0.200pt]{0.400pt}{23.367pt}}
\put(1244.0,689.0){\rule[-0.200pt]{4.818pt}{0.400pt}}
\put(1244.0,786.0){\rule[-0.200pt]{4.818pt}{0.400pt}}
\put(357.0,252.0){\rule[-0.200pt]{12.286pt}{0.400pt}}
\put(357.0,242.0){\rule[-0.200pt]{0.400pt}{4.818pt}}
\put(408.0,242.0){\rule[-0.200pt]{0.400pt}{4.818pt}}
\put(399.0,313.0){\rule[-0.200pt]{12.768pt}{0.400pt}}
\put(399.0,303.0){\rule[-0.200pt]{0.400pt}{4.818pt}}
\put(452.0,303.0){\rule[-0.200pt]{0.400pt}{4.818pt}}
\put(489.0,374.0){\rule[-0.200pt]{14.213pt}{0.400pt}}
\put(489.0,364.0){\rule[-0.200pt]{0.400pt}{4.818pt}}
\put(548.0,364.0){\rule[-0.200pt]{0.400pt}{4.818pt}}
\put(641.0,434.0){\rule[-0.200pt]{15.899pt}{0.400pt}}
\put(641.0,424.0){\rule[-0.200pt]{0.400pt}{4.818pt}}
\put(707.0,424.0){\rule[-0.200pt]{0.400pt}{4.818pt}}
\put(753.0,495.0){\rule[-0.200pt]{17.104pt}{0.400pt}}
\put(753.0,485.0){\rule[-0.200pt]{0.400pt}{4.818pt}}
\put(824.0,485.0){\rule[-0.200pt]{0.400pt}{4.818pt}}
\put(873.0,556.0){\rule[-0.200pt]{18.308pt}{0.400pt}}
\put(873.0,546.0){\rule[-0.200pt]{0.400pt}{4.818pt}}
\put(949.0,546.0){\rule[-0.200pt]{0.400pt}{4.818pt}}
\put(936.0,616.0){\rule[-0.200pt]{18.790pt}{0.400pt}}
\put(936.0,606.0){\rule[-0.200pt]{0.400pt}{4.818pt}}
\put(1014.0,606.0){\rule[-0.200pt]{0.400pt}{4.818pt}}
\put(1069.0,677.0){\rule[-0.200pt]{19.995pt}{0.400pt}}
\put(1069.0,667.0){\rule[-0.200pt]{0.400pt}{4.818pt}}
\put(1152.0,667.0){\rule[-0.200pt]{0.400pt}{4.818pt}}
\put(1210.0,738.0){\rule[-0.200pt]{21.199pt}{0.400pt}}
\put(1210.0,728.0){\rule[-0.200pt]{0.400pt}{4.818pt}}
\put(382,252){\makebox(0,0){$+$}}
\put(426,313){\makebox(0,0){$+$}}
\put(519,374){\makebox(0,0){$+$}}
\put(674,434){\makebox(0,0){$+$}}
\put(788,495){\makebox(0,0){$+$}}
\put(911,556){\makebox(0,0){$+$}}
\put(975,616){\makebox(0,0){$+$}}
\put(1110,677){\makebox(0,0){$+$}}
\put(1254,738){\makebox(0,0){$+$}}
\put(1349,213){\makebox(0,0){$+$}}
\put(1298.0,728.0){\rule[-0.200pt]{0.400pt}{4.818pt}}
\put(1279,172){\makebox(0,0)[r]{fit $y(x)="\(94.0\pm3.2\)x+17.1\pm6.7"$}}
\multiput(1299,172)(20.756,0.000){5}{\usebox{\plotpoint}}
\put(1399,172){\usebox{\plotpoint}}
\put(357,266){\usebox{\plotpoint}}
\put(357.00,266.00){\usebox{\plotpoint}}
\put(375.36,275.68){\usebox{\plotpoint}}
\put(393.74,285.30){\usebox{\plotpoint}}
\put(412.11,294.95){\usebox{\plotpoint}}
\put(430.06,305.37){\usebox{\plotpoint}}
\put(448.43,315.02){\usebox{\plotpoint}}
\put(466.80,324.67){\usebox{\plotpoint}}
\put(485.17,334.32){\usebox{\plotpoint}}
\put(503.12,344.73){\usebox{\plotpoint}}
\put(521.49,354.38){\usebox{\plotpoint}}
\put(539.86,364.03){\usebox{\plotpoint}}
\put(558.17,373.78){\usebox{\plotpoint}}
\put(576.14,384.08){\usebox{\plotpoint}}
\put(594.50,393.75){\usebox{\plotpoint}}
\put(612.86,403.43){\usebox{\plotpoint}}
\put(631.22,413.11){\usebox{\plotpoint}}
\put(649.10,423.55){\usebox{\plotpoint}}
\put(667.46,433.23){\usebox{\plotpoint}}
\put(685.82,442.91){\usebox{\plotpoint}}
\put(704.17,452.59){\usebox{\plotpoint}}
\put(722.06,463.03){\usebox{\plotpoint}}
\put(740.42,472.71){\usebox{\plotpoint}}
\put(758.77,482.39){\usebox{\plotpoint}}
\put(777.04,492.22){\usebox{\plotpoint}}
\put(795.05,502.53){\usebox{\plotpoint}}
\put(813.41,512.20){\usebox{\plotpoint}}
\put(831.77,521.87){\usebox{\plotpoint}}
\put(850.14,531.52){\usebox{\plotpoint}}
\put(868.25,541.63){\usebox{\plotpoint}}
\put(886.61,551.30){\usebox{\plotpoint}}
\put(904.96,560.98){\usebox{\plotpoint}}
\put(923.32,570.66){\usebox{\plotpoint}}
\put(941.21,581.10){\usebox{\plotpoint}}
\put(959.56,590.78){\usebox{\plotpoint}}
\put(977.92,600.46){\usebox{\plotpoint}}
\put(996.18,610.31){\usebox{\plotpoint}}
\put(1014.20,620.60){\usebox{\plotpoint}}
\put(1032.56,630.28){\usebox{\plotpoint}}
\put(1050.91,639.95){\usebox{\plotpoint}}
\put(1069.28,649.60){\usebox{\plotpoint}}
\put(1087.23,660.02){\usebox{\plotpoint}}
\put(1105.60,669.67){\usebox{\plotpoint}}
\put(1123.97,679.32){\usebox{\plotpoint}}
\put(1142.34,688.97){\usebox{\plotpoint}}
\put(1160.29,699.39){\usebox{\plotpoint}}
\put(1178.66,709.04){\usebox{\plotpoint}}
\put(1197.03,718.69){\usebox{\plotpoint}}
\put(1215.29,728.53){\usebox{\plotpoint}}
\put(1233.32,738.73){\usebox{\plotpoint}}
\put(1251.69,748.38){\usebox{\plotpoint}}
\put(1270.06,758.03){\usebox{\plotpoint}}
\put(1288.42,767.71){\usebox{\plotpoint}}
\put(1298,774){\usebox{\plotpoint}}
\put(171.0,131.0){\rule[-0.200pt]{0.400pt}{175.375pt}}
\put(171.0,131.0){\rule[-0.200pt]{305.461pt}{0.400pt}}
\put(1439.0,131.0){\rule[-0.200pt]{0.400pt}{175.375pt}}
\put(171.0,859.0){\rule[-0.200pt]{305.461pt}{0.400pt}}
\end{picture}

\caption{Závislosť vlnovej dĺžky $\lambda$ od doby kmitu $T$ preložená funkciou $\(348.7\pm6.7\)T - 0.001$.
}  \label{G_2}
\end{figure}

\subsection{Helmholtzovy rezonátory}
V tabuľke Tab \ref{T_3} sa nachádzajú, a sú vynesené do grafu Obr. \ref{G_3}.
Z neho bola preložením určená hodnota rýchlosti zvuku $v="\(293.2\pm9.0\) m/s"$.


\begin{table}[h]
\begin{center}
\begin{tabular}{ | c | c | c | c |}
\hline
\popi{f_n}{Hz} & \popi{V_v}{l} & \popi{V}{l} & \popi{f_v}{Hz}\\
\hline
$175.8 $ & $ 0.05 $ & $ 0.98 $ & $ 187.78 $\\
$175.8 $ & $ 0.1  $ & $ 0.93 $ & $ 192.76 $\\
$177.8 $ & $ 0.15 $ & $ 0.88 $ & $ 198.16 $\\
$179.7 $ & $ 0.2  $ & $ 0.83 $ & $ 204.04 $\\
$179.7 $ & $ 0.25 $ & $ 0.78 $ & $ 210.48 $\\
$179.7 $ & $ 0.3  $ & $ 0.73 $ & $ 217.57 $\\
$179.7 $ & $ 0.35 $ & $ 0.68 $ & $ 225.42 $\\
$179.7 $ & $ 0.4  $ & $ 0.63 $ & $ 234.20 $\\
$179.7 $ & $ 0.45 $ & $ 0.58 $ & $ 244.09 $\\
$179.8 $ & $ 0.5  $ & $ 0.53 $ & $ 255.34 $\\
\hline
\end{tabular}
\caption{Namerané hodnoty frekvencie rezonancie $f_n$, 
vypočítané hodnoty rezonancie $f_v$ podla \ref{R_30} pri objeme vody $V_v$ a z neho dopočítaný objem $V$.
} \label{T_3}
\end{center}
\end{table}




\begin{figure}
% GNUPLOT: LaTeX picture
\setlength{\unitlength}{0.240900pt}
\ifx\plotpoint\undefined\newsavebox{\plotpoint}\fi
\begin{picture}(1500,900)(0,0)
\sbox{\plotpoint}{\rule[-0.200pt]{0.400pt}{0.400pt}}%
\put(171.0,131.0){\rule[-0.200pt]{4.818pt}{0.400pt}}
\put(151,131){\makebox(0,0)[r]{ 140}}
\put(1419.0,131.0){\rule[-0.200pt]{4.818pt}{0.400pt}}
\put(171.0,252.0){\rule[-0.200pt]{4.818pt}{0.400pt}}
\put(151,252){\makebox(0,0)[r]{ 160}}
\put(1419.0,252.0){\rule[-0.200pt]{4.818pt}{0.400pt}}
\put(171.0,374.0){\rule[-0.200pt]{4.818pt}{0.400pt}}
\put(151,374){\makebox(0,0)[r]{ 180}}
\put(1419.0,374.0){\rule[-0.200pt]{4.818pt}{0.400pt}}
\put(171.0,495.0){\rule[-0.200pt]{4.818pt}{0.400pt}}
\put(151,495){\makebox(0,0)[r]{ 200}}
\put(1419.0,495.0){\rule[-0.200pt]{4.818pt}{0.400pt}}
\put(171.0,616.0){\rule[-0.200pt]{4.818pt}{0.400pt}}
\put(151,616){\makebox(0,0)[r]{ 220}}
\put(1419.0,616.0){\rule[-0.200pt]{4.818pt}{0.400pt}}
\put(171.0,738.0){\rule[-0.200pt]{4.818pt}{0.400pt}}
\put(151,738){\makebox(0,0)[r]{ 240}}
\put(1419.0,738.0){\rule[-0.200pt]{4.818pt}{0.400pt}}
\put(171.0,859.0){\rule[-0.200pt]{4.818pt}{0.400pt}}
\put(151,859){\makebox(0,0)[r]{ 260}}
\put(1419.0,859.0){\rule[-0.200pt]{4.818pt}{0.400pt}}
\put(171.0,131.0){\rule[-0.200pt]{0.400pt}{4.818pt}}
\put(171,90){\makebox(0,0){ 30}}
\put(171.0,839.0){\rule[-0.200pt]{0.400pt}{4.818pt}}
\put(352.0,131.0){\rule[-0.200pt]{0.400pt}{4.818pt}}
\put(352,90){\makebox(0,0){ 32}}
\put(352.0,839.0){\rule[-0.200pt]{0.400pt}{4.818pt}}
\put(533.0,131.0){\rule[-0.200pt]{0.400pt}{4.818pt}}
\put(533,90){\makebox(0,0){ 34}}
\put(533.0,839.0){\rule[-0.200pt]{0.400pt}{4.818pt}}
\put(714.0,131.0){\rule[-0.200pt]{0.400pt}{4.818pt}}
\put(714,90){\makebox(0,0){ 36}}
\put(714.0,839.0){\rule[-0.200pt]{0.400pt}{4.818pt}}
\put(896.0,131.0){\rule[-0.200pt]{0.400pt}{4.818pt}}
\put(896,90){\makebox(0,0){ 38}}
\put(896.0,839.0){\rule[-0.200pt]{0.400pt}{4.818pt}}
\put(1077.0,131.0){\rule[-0.200pt]{0.400pt}{4.818pt}}
\put(1077,90){\makebox(0,0){ 40}}
\put(1077.0,839.0){\rule[-0.200pt]{0.400pt}{4.818pt}}
\put(1258.0,131.0){\rule[-0.200pt]{0.400pt}{4.818pt}}
\put(1258,90){\makebox(0,0){ 42}}
\put(1258.0,839.0){\rule[-0.200pt]{0.400pt}{4.818pt}}
\put(1439.0,131.0){\rule[-0.200pt]{0.400pt}{4.818pt}}
\put(1439,90){\makebox(0,0){ 44}}
\put(1439.0,839.0){\rule[-0.200pt]{0.400pt}{4.818pt}}
\put(171.0,131.0){\rule[-0.200pt]{0.400pt}{175.375pt}}
\put(171.0,131.0){\rule[-0.200pt]{305.461pt}{0.400pt}}
\put(1439.0,131.0){\rule[-0.200pt]{0.400pt}{175.375pt}}
\put(171.0,859.0){\rule[-0.200pt]{305.461pt}{0.400pt}}
\put(30,495){\makebox(0,0){\popi{f}{Hz}}}
\put(805,29){\makebox(0,0){\popi{\frac{1}{\sqrt{V}}}{\sqrt{m^{-3}}}}}
\put(1279,254){\makebox(0,0)[r]{namerané hodnoty}}
\put(347,348){\makebox(0,0){$+$}}
\put(424,348){\makebox(0,0){$+$}}
\put(507,360){\makebox(0,0){$+$}}
\put(598,372){\makebox(0,0){$+$}}
\put(697,372){\makebox(0,0){$+$}}
\put(806,372){\makebox(0,0){$+$}}
\put(927,372){\makebox(0,0){$+$}}
\put(1062,372){\makebox(0,0){$+$}}
\put(1215,372){\makebox(0,0){$+$}}
\put(1388,372){\makebox(0,0){$+$}}
\put(1349,254){\makebox(0,0){$+$}}
\put(1279,213){\makebox(0,0)[r]{preložená závylosť}}
\multiput(1299,213)(20.756,0.000){5}{\usebox{\plotpoint}}
\put(1399,213){\usebox{\plotpoint}}
\put(347,212){\usebox{\plotpoint}}
\put(347.00,212.00){\usebox{\plotpoint}}
\put(366.96,217.69){\usebox{\plotpoint}}
\put(386.63,224.29){\usebox{\plotpoint}}
\put(406.30,230.89){\usebox{\plotpoint}}
\put(426.10,237.04){\usebox{\plotpoint}}
\put(445.90,243.17){\usebox{\plotpoint}}
\put(465.57,249.77){\usebox{\plotpoint}}
\put(485.24,256.37){\usebox{\plotpoint}}
\put(505.19,262.08){\usebox{\plotpoint}}
\put(524.83,268.68){\usebox{\plotpoint}}
\put(544.56,275.11){\usebox{\plotpoint}}
\put(564.25,281.64){\usebox{\plotpoint}}
\put(584.10,287.66){\usebox{\plotpoint}}
\put(603.74,294.29){\usebox{\plotpoint}}
\put(623.37,300.92){\usebox{\plotpoint}}
\put(643.26,306.82){\usebox{\plotpoint}}
\put(663.05,313.02){\usebox{\plotpoint}}
\put(682.75,319.54){\usebox{\plotpoint}}
\put(702.44,326.07){\usebox{\plotpoint}}
\put(722.32,332.00){\usebox{\plotpoint}}
\put(741.96,338.62){\usebox{\plotpoint}}
\put(761.78,344.74){\usebox{\plotpoint}}
\put(781.48,351.26){\usebox{\plotpoint}}
\put(801.23,357.61){\usebox{\plotpoint}}
\put(820.86,364.23){\usebox{\plotpoint}}
\put(840.81,369.94){\usebox{\plotpoint}}
\put(860.48,376.54){\usebox{\plotpoint}}
\put(880.15,383.15){\usebox{\plotpoint}}
\put(899.92,389.37){\usebox{\plotpoint}}
\put(919.61,395.84){\usebox{\plotpoint}}
\put(939.40,402.02){\usebox{\plotpoint}}
\put(959.07,408.62){\usebox{\plotpoint}}
\put(978.99,414.40){\usebox{\plotpoint}}
\put(998.67,420.91){\usebox{\plotpoint}}
\put(1018.30,427.54){\usebox{\plotpoint}}
\put(1038.05,433.82){\usebox{\plotpoint}}
\put(1057.91,439.77){\usebox{\plotpoint}}
\put(1077.58,446.38){\usebox{\plotpoint}}
\put(1097.25,452.98){\usebox{\plotpoint}}
\put(1117.14,458.86){\usebox{\plotpoint}}
\put(1136.86,465.26){\usebox{\plotpoint}}
\put(1156.53,471.83){\usebox{\plotpoint}}
\put(1176.23,478.36){\usebox{\plotpoint}}
\put(1196.13,484.22){\usebox{\plotpoint}}
\put(1215.77,490.85){\usebox{\plotpoint}}
\put(1235.40,497.47){\usebox{\plotpoint}}
\put(1255.24,503.54){\usebox{\plotpoint}}
\put(1275.01,509.82){\usebox{\plotpoint}}
\put(1294.64,516.45){\usebox{\plotpoint}}
\put(1314.28,523.08){\usebox{\plotpoint}}
\put(1334.23,528.77){\usebox{\plotpoint}}
\put(1353.90,535.37){\usebox{\plotpoint}}
\put(1373.73,541.45){\usebox{\plotpoint}}
\put(1388,546){\usebox{\plotpoint}}
\sbox{\plotpoint}{\rule[-0.400pt]{0.800pt}{0.800pt}}%
\sbox{\plotpoint}{\rule[-0.200pt]{0.400pt}{0.400pt}}%
\put(1279,172){\makebox(0,0)[r]{vypočítané hodnoty}}
\sbox{\plotpoint}{\rule[-0.400pt]{0.800pt}{0.800pt}}%
\put(347,421){\makebox(0,0){$\ast$}}
\put(424,451){\makebox(0,0){$\ast$}}
\put(507,484){\makebox(0,0){$\ast$}}
\put(598,520){\makebox(0,0){$\ast$}}
\put(697,559){\makebox(0,0){$\ast$}}
\put(806,602){\makebox(0,0){$\ast$}}
\put(927,649){\makebox(0,0){$\ast$}}
\put(1062,702){\makebox(0,0){$\ast$}}
\put(1215,762){\makebox(0,0){$\ast$}}
\put(1388,831){\makebox(0,0){$\ast$}}
\put(1349,172){\makebox(0,0){$\ast$}}
\sbox{\plotpoint}{\rule[-0.200pt]{0.400pt}{0.400pt}}%
\put(171.0,131.0){\rule[-0.200pt]{0.400pt}{175.375pt}}
\put(171.0,131.0){\rule[-0.200pt]{305.461pt}{0.400pt}}
\put(1439.0,131.0){\rule[-0.200pt]{0.400pt}{175.375pt}}
\put(171.0,859.0){\rule[-0.200pt]{305.461pt}{0.400pt}}
\end{picture}

\caption{Teoretická a nameraná závislosť rezonančnej frekvencie $f_n$ a $f_v$ od objemu $V$, preložená funkciou $f=\frac{293.2\pm9.0}{2\pi}\sqrt{\frac{\pi r^2}{l+1.4r}} \frac{1}{\sqrt{V}}$.
}  \label{G_3}
\end{figure}




