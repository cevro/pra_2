\section{Pracovní úkol}

\begin{enumerate}
\item DU: Spočítejte, jakou vlastní a vyšší harmonické frekvence má struna napjatá
zátěží 5 kg o délce 1 m, víte-li, že její lineární hustota je $\mu="0.0162 kg\cdot m^{-1}"$
\item Do vzorce z domácího úkolu dosadťe délku struny a spočtěte totéž. Naměřte pro prvních 10
rezonančních frekvencí. Z naměřených vyšších harmonických frekvencí zpětně dopočítejte
lineární hustotu a porovnejte s uvedenou konstantou. Dopočtěte rychlost šíření vlnění na
struně.
\item Pro 10 různých frekvencí v rozsahu 2 až 6 kHz hledejte interferenční minima (či maxima)
změnou délky Quinckovy trubice. Vyneste do grafu závislost vlnové délky zvuku na frekvenci.
Proložením dat s errorbary určete rychlost zvuku.
\item Najděte vlastní frekvence Helmholtzova rezonátoru. Vyneste do grafu závislost vlastní frekvence
na objemu rezonátoru. Pro hledání vlastní frekvence využijte Fourierovské frekvenční
analýzy. Z naměřených hodnot určete rychlost zvuku proložením naměřených hodnot vhodnou
funkcí.
\end{enumerate}

