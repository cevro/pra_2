\section{Výsledky merania}

\subsection{Měření kruhových otvorů}
Namerané hodnoty priemeru otvoru pomocou mikroskopu a difrakcie sú v tabuľke \ref{T_1}.
\begin{table}[h]
\begin{center}
\begin{tabular}{ |  c | c | c | c | }
\hline
\popi{n}{-} & \popi{d_m}{mm}& \popi{x}{cm} & \popi{d_i}{mm} \\
\hline
$"2.3\pm0.1"$ & $"0.58\pm0.03"$ & $"0.84\pm0.05"$ & $"0.54\pm0.08"$\\
$"4.0\pm0.1"$ & $"1.00\pm0.03"$ & $"0.50\pm0.05"$ & $"0.91\pm0.16"$\\
$"8.3\pm0.1"$ & $"2.08\pm0.03"$ & $"0.28\pm0.05"$ & $"1.63\pm0.38"$\\
\hline
\end{tabular}
\caption{$n$ je počet otáčok závitu mikroskopu z ktorých bol vypočítaný priemer otvoru $d_m$  a $d_i$ priemer otvoru vypočítaný podľa vzťahu \ref{R_D_O}, $x$ je vzdialenosť maxím v difrakčnom obrazci.
} \label{T_1}
\end{center}
\end{table}


\subsection{Měření štěrbiny}

Namerané hodnoty veľkosti štrbiny odčítanej z mikrometrického šróbu a vypočítané z difrakcie sú v tabuľke \ref{T_2}.


\begin{table}[h]
\begin{center}
\begin{tabular}{ |  c | c | c | }
\hline
\popi{d_m}{mm}& \popi{x}{cm} & \popi{d_i}{mm} \\
\hline
$"0.57\pm0.01"$ & $"1.30\pm0.10"$ & $"0.35\pm0.03"$\\
$"0.75\pm0.01"$ & $"0.80\pm0.10"$ & $"0.57\pm0.06"$\\
$"1.00\pm0.01"$ & $"0.60\pm0.10"$ & $"0.76\pm0.09"$\\
$"1.25\pm0.01"$ & $"0.55\pm0.10"$ & $"0.83\pm0.11"$\\
$"1.50\pm0.01"$ & $"0.35\pm0.10"$ & $"1.30\pm0.24"$\\
$"1.15\pm0.01"$ & $"0.50\pm0.10"$ & $"0.91\pm0.13"$\\
$"0.90\pm0.01"$ & $"0.60\pm0.10"$ & $"0.76\pm0.09"$\\
$"0.80\pm0.01"$ & $"0.75\pm0.10"$ & $"0.61\pm0.07"$\\
$"0.65\pm0.01"$ & $"1.00\pm0.10"$ & $"0.46\pm0.04"$\\
$"0.55\pm0.01"$ & $"2.50\pm0.10"$ & $"0.18\pm0.01"$\\
\hline
\end{tabular}
\caption{Odčítané veľkosti štrbiny $d_m$ na vzdialenosti maxím $x$ a vypočítaná mriežková konštanta $d_i$ podľa \ref{R_D_O}
} \label{T_2}
\end{center}
\end{table}


\subsection{Měření mřížkové konstanty}

Namerané hodnoty sú v tabuľke \ref{T_3} z nich za pomoci vzťahu \ref{SCH_1} a \ref{SCH_2} bola vypočítaná mriežková konštanta
\eq{
d = "\(1.54\pm0.054\)\cdot10^{-6} m"\,.
}

\begin{table}[h]
\begin{center}
\begin{tabular}{ |  c | c | c |   }
\hline
\popi{\Delta l}{cm}& \popi{x}{cm} & \popi{d_i}{mm} \\
\hline
$"20.5\pm0.1"$ & $"8.5\pm0.1"$ & $"1.53\pm0.3"$\\
$"15.0\pm0.1"$ & $"6.0\pm0.1"$ & $"1.58\pm0.4"$\\
$"10.0\pm0.1"$ & $"4.2\pm0.1"$ & $"1.51\pm0.5"$\\
\hline
\end{tabular}
\caption{Nameraná hodnoty vzdialenosti tienika od mriežky $\Delta l$ v závislosti na vzdialenosti maxím prvého rádu $x$ a z nich podľa vzťahu \ref{R_D_O} mriežková konštanta $d$.
} \label{T_3}
\end{center}
\end{table}


\subsection{Michelsonův interferometr}

Namerané hodnoty sú v tabuľke Tab. \ref{T_4}, 
z nich bol pomocou vzťahu \ref{SCH_1} a \ref{SCH_2} vypočítané hodnota vlnová dĺžka použitého laseru
\eq{
\lambda = "720\pm390 mn"\,. \lbl{R_V_4}
}
\begin{table}[h]
\begin{center}
\begin{tabular}{ |  c | c | c |   }
\hline
\popi{\Delta l}{cm}& \popi{n}{-} & \popi{d_i}{mm} \\
\hline
$"4.0\pm0.1"$ & $"12\pm2"$ & $"0.67\pm0.13"$\\
$"4.5\pm0.1"$ & $"12\pm2"$ & $"0.75\pm0.14"$\\
$"4.0\pm0.1"$ & $"10\pm2"$ & $"0.80\pm0.18"$\\
$"4.0\pm0.1"$ & $"12\pm2"$ & $"0.67\pm0.13"$\\
$"3.5\pm0.1"$ & $"10\pm2"$ & $"0.70\pm0.16"$\\
\hline
\end{tabular}
\caption{Zmeny polohy zrkadla $\Delta l$ na počte maxím prejdených určeným bodom $n$ a z nich podľa vzťahu \ref{R_MI} vypočítaná vlnová dĺžka $\lambda$.%
} \label{T_4}
\end{center}
\end{table}

\subsection{Meranie dráhy paprsku}

Dráha laserového paprsku bola určená na $l = "\(7.20\pm0.1\text{stat.}\pm0.2\text{sys.}\) m"$.

