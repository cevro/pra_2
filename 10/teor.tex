\section{Teória}

\subsection{Fraunhoferovy difrakce}
Pre Fraunhoferovy difrakce platí vzťah
\eq{
\sin \theta_{max} = \frac{m \lambda}{d}\,, \lbl{R_1}
}
kde $\lambda$ je vlnová dĺžka, $d$ je mriežková konštanta a $m$ je rád maxima.

Zároveň nám platí 
\eq{
\sin \theta_{max} = \frac{m x}{l}\,,\lbl{R_2}
}
kde $m$ je opäť rád maxima, $x$ je vzdialenosť maxima od stredu a $l$ je urazená dráha laseru.

z rovností vzťahov \ref{R_1} a \ref{R_2} môžeme uvažovať
\eq[m]{
\frac{m x}{l} &= \frac{m \lambda}{d} \,,\\
\frac{x}{l} &= \frac{\lambda}{d} \,, \lbl{R_D_O}
} 
ktorý nám hovorí vzťah medzi vzdialenosťami maxím vzhľadom k vlnovej dĺžke,  dráhy zväzku a mriežkovej konštanty.

\subsection{ Michelsonův interferometr}

Pre zmenu jednej dráhy o $\Delta x$ u Michelsonovho interferometru platí
\eq{
\lambda = \frac{2\Delta x}{n}\,, \lbl{R_MI}
} 
kde $\lambda$ je vlnová dĺžka a $n$ je počet minim/maxím, ktorý prejde pri zmene vzdialenosti cez daný bod.

%%%%%%%%%%%%%%%%%%%%%%%%%%%%%%%%%%%%%%%
\subsubsection{Spracovanie chýb merania}

Označme $\mean{t}$ aritmetický priemer nameraných hodnôt $t_i$, a $\Delta t$ hodnotu $\mean{t}-t$, pričom 
\eq{
\mean{t} = \frac{1}{n}\sum_{i=1}^n t_i \,, \lbl{SCH_1}
}  
a chybu aritmetického priemeru 
\eq{
  \sigma_0=\sqrt{\frac{\sum_{i=1}^n \(t_i - \mean{t}\)^2}{n\(n-1\)}}\,, \lbl{SCH_2}
}
pričom $n$ je počet meraní.



