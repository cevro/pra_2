\section{Pracovní úkol}
\begin{enumerate}
\item DÚ: Ve vztazích (9), (13) a (18) \cite{C_1} vyjádřete $\sin \theta$ pomocí polohy maxima/minima
od středu a uražené dráhy laserového paprsku.
\item  Změřte průměr tří nejmenších kruhových otvorů užitím Fraunhoferovy difrakce světla s
pomocí měřicího mikroskopu a výsledky srovnejte. Odhadněte chybu měření šířky štěrbiny
mikroskopem. Pro který průměr kruhového otvoru je přesnější měření interferencí a pro
který mikroskopem?
\item  Změřte 10 různých šířek štěrbiny užitím Fraunhoferovy difrakce světla a srovnejte s hodnotou
na mikrometrickém šroubu. Pro jaké šířky štěrbiny je výhodnější měření interferencí
a pro jaké mikrometrickým šroubem?
\item Změřte mřížkovou konstantu optické mřížky a srovnejte s hodnotou uvedenou na mřížce.
\item Sestavte Michelsonův interferometr a změřte vlnovou délku laserového svazku
\end{enumerate}

