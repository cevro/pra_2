\section{Pracovní úkol}
\begin{enumerate}
\item Pomocí rovnice (1)\cite{C_1} sestavte diferenciální rovnici a jejím řešením odvoďťe
zákon radioaktivního rozpadu (2)\cite{C_1}. S jeho pomocí dále podle definice odvoďte vztah
(3) pro poločas rozpadu.
\item Osciloskopem pozorujte spektrum 137Cs na výstupu z jednokanálového analyzátoru. Načrtněte
tvar spektra (závislost intenzity na energii záření) a přiložte k protokolu. (Osciloskop ukazuje
tvary a amplitudy jednotlivých pulzů. Počet pulzů je dán intenzitou čáry a energie výškou impulzu.)
\item Naměřte spektrum impulzů \ce{^{137}Cs} jednokanálovým analyzátorem pomocí manuálního měření.
Okno volte o šířce 100 mV (10 malých dílků). Spektrum graficky zpracujte.
\item Mnohokanálovým analyzátorem naměřte jednotlivá spektra přiložených zářičů (\ce{^{137}Cs}, \ce{^{60}Co},
\ce{^{241}Am} a \ce{^{133}Ba}). Určete výrazné píky a porovnejte je s tabulkovými hodnotami. (Každé spektrum
nabírejte 10 minut. Před zpracováním odečtěte pozadí - viz úkol 9. \cite{C_1})
\item Pomocí zářičů \ce{^{137}Cs} a \ce{^{60}Co} určete kalibrační křivku spektrometru a použijte ji při zpracování
všech spekter naměřených mnohokanálovým analyzátorem. (Spektrum nemusíte nabírat znovu,
použijte data z předchozího měření.)
\item S využitím všech naměřených spekter určete závislost rozlišení spektrometru na energii gama
záření. (Je definováno jako poměr šířky fotopíku v polovině jeho výšky k jeho energii - viz
poznámka.)
\item Z naměřeného spektra \ce{^{137}Cs} určete hodnotu píku zpětného rozptylu, Comptonovy hrany,
energii rentgenového píku a energii součtového píku.
\item Mnohokanálovým analyzátorem naměřte spektrum neznámého zářiče. Určete tento zářič, pozorujte
a zaznamenejte další jevy v jeho spektru. (Spektrum nabírejte 10 minut.)
\item Mnohokanálovým analyzátorem naměřte spektrum pozadí v místnosti (zářiče uschovejte do
trezoru). Najděte v pozadí přirozené zářiče a toto pozadí odečtěte od všech zaznamenaných
spekter ještě před jejich vyhodnocením. (Pozadí nabírejte 10 minut.)
\item Graficky určete závislost koeficientu útlumu olova na energii gama záření. (Použijte zářiče
\ce{^{137}Cs}, \ce{^{60}Co} a \ce{^{133}Ba} současně, jednotlivá spektra nabírejte 10 minut.)
\end{enumerate}

