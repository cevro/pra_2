\section{Výsledky merania}
Osciloskopom bolo pozorované spektrum viď Obr. \ref{G_cs}.

Z tohoto obrázku na základne jasnosti fotky v oblasti maxima bolo vykreslené spektrum viď Obr. \ref{G_gs}

Pre všetky merania bol počet impolzov prepočítaný na aktivitu podľa vzťahu \ref{R_akt}, kde $t = "10 min"$.


Namerané spektrá pre jednotlivé žiariace s fitnutými charakteristickými 
píkmi sú zobrazené v grafe Obr. \ref{G_ba} pre \ce{^{133}Ba}, Obr. \ref{G_cs} pre \ce{^{137}Cs} a pre \ce{^{60}Co} Obr. \ref{G_co}.


Pre tieto píky boli dohľadné tabuľkové hodnoty z \cite{C_2}. viď tabuľka Tab. \ref{T_1}, a z nich pomocou fitu určená kalibračná krivka viď graf Obr. \ref{G_kal} a určená kalibračná rovnica 
\eq{
E^\prime = E\_t\(0.174\pm0.004\)+ \(1.94\pm3.11\)\,, \lbl{R_kal}
} kde $E^\prime$ je energia nameraná našim detektorom a $E\_t$ je skutočná hodnota.

Následne boli podľa vzťahu \ref{R_FWHM} a \ref{R_kal} vypočítané hodnoty FWHM pre jednotlivé píky a vynesené v závislosti na energií do graf Obr. \ref{G_e}.

Comptonovo kontinum bolo určená v rozsahu $"\(220\pm20\)-\(440\pm20\) keV"$ a Comptonovu hranu v oblasti $"\(440\pm20\)-\(520\pm20\) keV"$.


\begin{table}[h]
\begin{center}
\begin{tabular}{| c |  c | c | }
\hline
- &\popi{E\_t}{keV}& \popi{E^\prime}{keV} \\
\hline
\ce{^{137}Cs} & $"661.657\pm0.003"$&$"110.41\pm0.10"$\\
\hline
\ce{^{60}Co} & $"1332.501\pm0.005"$&$"236.70\pm0.76"$\\
\ce{^{60}Co} & $"1173.237\pm0.004"$&$"204.79\pm0.48"$\\
\hline
\ce{^{133}Ba} & $"356.013\pm0.001"$&$"64.39\pm0.16"$\\
\ce{^{133}Ba} & $"302.851\pm0.001"$&$"53.50\pm0.22"$\\
\ce{^{133}Ba} & $"160.611\pm0.002"$&$"33.93\pm0.39"$\\
\hline
\end{tabular}
\caption{Energia $E^\prime$ píkov určená z fitu grafov Obr. \ref{G_cs},Obr. \ref{G_co} a Obr. \ref{G_ba}, v porovnaní s tabuľkovou hodnotou $E\_t$. \cite{C_2}
} \label{T_1}
\end{center}
\end{table}


Podľa vzťahu boli pre jednotlivé spektrá zkalibrované dáta.


Pre jednotlivé žiariče boli vynesené tienené a netienenú spektrá do grafov: 
Obr. \ref{G_ba-t-2} pre \ce{^{133}Ba}, 
Obr. \ref{G_cs-t-2} pre \ce{^{137}Cs} 
a pre \ce{^{60}Co} Obr. \ref{G_co-t-2}.

Následne bola do grafu vynesená závislosť koeficientu útlmu $I_0/I$ 
na energií $E$ pre jednotlivé žiariče: Obr. \ref{G_ba-t-1} pre \ce{^{133}Ba}, 
Obr. \ref{G_cs-t-1} pre \ce{^{137}Cs} 
a pre \ce{^{60}Co} Obr. \ref{G_co-t-1}.
Z grafov bol určený fitom koeficient útlu
\eq[m]{
\(\frac{I_0}{I}\)_\ce{^{133}Ba} &= "\(2.79\pm0.18\)"\,,\\
\(\frac{I_0}{I}\)_\ce{^{60}Co} &= "\(0.91\pm0.05\)"\,,\\
\(\frac{I_0}{I}\)_\ce{^{137}Cs} &= "\(1.45\pm0.03\)"\,,
}
a pomocou vzťahu \ref{SCH_2} bola vypočítaný priemerný útlmu
\eq{
\frac{I_0}{I} = "1.71\pm0.20"\,, \lbl{R_u}
}
kde $I_0$ je intenzita netieneneho žiariča a $I$ je intenzita tienená.


Nasledne bolo odmerané spektrum neznámenho vzorku a určené píky viď graf obr. \ref{G_nez}
Prvý pre $E_1="767.6\pm0.89 keV"$ a druhý $E_2="544.78\pm3.61 keV"$. 


Na záver pridaný graf pre \ce{^{241}Am}, ktoré žiari natoľko málo že nieje rozpoznateľné od pozdia viď graf Obr \ref{G_am}


